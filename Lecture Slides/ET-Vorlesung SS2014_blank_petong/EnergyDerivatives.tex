% !TEX root = EnergyTrading_ss14UDE.tex
\section{Energy Derivatives}
\subsection{Caps and  Floors}
\frame{\frametitle{Caps}

Buying a cap, the option holder has the right (but not the
obligation) to buy a certain amount of energy at stipulated times
$t_1,\ldots,t_N$ during the delivery period at a fixed strike
price $K$. It can be viewed as a strip of
independent call options, for each time $t_i$ the holder of the cap holds call options with maturity $t_i$ and Strike $K$. \\
The static factors describing the cap are:
\begin{itemize}
\item times $t_1,\ldots,t_N$ (how often? when?)
\item strike $K$ (price?)
\item amount of the underlying (how much?)
\end{itemize}



\frame{\frametitle{Cap - Payoff}
\begin{figure}
	\centering
		\includegraphics[width=.80\textwidth]{../../../pics/Cap2}
	\label{fig:Cap2}
\end{figure}


\frame{\frametitle{Caps - Pricing}

Whenever the price of the underlying exceeds the strike K at one of the dates $t_1,\ldots,t_N$, the seller of the cap pays the holder of the cap the difference between the price of the underlying and the strike K or - in case one agreed on physical delivery - the underlying is delivered for the price K.
Typically, the price of a cap is quoted as price per delivery hours to make
different delivery periods comparable. In this case we get a price
per MWh. The formula is
$$U_c(t)=\frac{1}{N}\sum_{i=1}^Ne^{-r(t_i-t)}\EX[\max(S(t_i)-K,0)].$$


\frame{\frametitle{Caps - Hedging}
The strike price $K$ secures a maximum price for which the option holder is able to buy energy. A cap is used to cover a short position in the underlying (energy) against
increasing market prices not only at a certain point in time but over the whole period covered by the exercising times $t_1,\ldots,t_N$.
On the other hand, the option holder is still able to profit from low energy prices as he has the right but not the obligation to exercise the option at each time point.


\frame{\frametitle{Caps - Example}
Assume you need 100 units of the underlying per day to run your business. Today it costs 100 Euro/unit. You can accept resource cost of up to 110 Euro/unit in order to beneficially run your business. You are afraid of rising prises and want to hedge against this risk but still have the chance to profit from low prices.\\
Thus, you ask for a cap with daily exercise up to the business horizon of 8 days with volume 100 units and, say, strike 108 Euro/unit which might cost 1600 Euro (2 Euro/unit). Then, your total cost is at most 108 Euro + 2 Euro = 110 Euro/unit but you still participate on low prices.


\frame{\frametitle{Caps - Example}
The table shows one possible result of the cap on the profit of the company.
\begin{tabular}{rrrr}
       Day & Underlying & Cost without Cap & Cost with Cap \\
\hline
         1 &        100 &        100 &        102 \\
         2 &        111 &        111 &        110 \\
         3 &        116 &        116 &        110 \\
         4 &        120 &        120 &        110 \\
         5 &        109 &        109 &        110 \\
         6 &         97 &         97 &         99 \\
         7 &         85 &         85 &         87 \\
         8 &         78 &         78 &         80 \\
\hline
   Average &            &        102 &        101 \\
   S.d.&&14.9&11.7
\end{tabular}


\frame{\frametitle{Floors}
Buying a floor, the option holder has the right (but not the
obligation) to sell a certain amount of energy at stipulated times
$t_1,\ldots,t_N$ during the delivery period at a fixed strike
price $K$. It can be viewed as a strip of
independent put options, for each time $t_i$ the holder of the floor holds put options with maturity $t_i$ and Strike $K$. \\
Similar to the case of a cap, the pricing formula is
$$U_f(t)=\frac{1}{N}\sum_{i=1}^Ne^{-r(t_i-t)}\EX[\max(K-S(t_i),0)].$$
As with the cap, the price is quoted in Euro/MWh.


\frame{\frametitle{Floor - Payoff}
\begin{figure}
	\centering
		\includegraphics[width=.80\textwidth]{../../../pics/Floor}
	\label{fig:Floor}
\end{figure}


\frame{\frametitle{Floors - Hedging}
The strike price $K$ secures a minimum price for which the option holder is able to sell energy. A floor is used to cover a long position in the underlying (energy) against decreasing market prices not only at a certain point in time but over the whole period covered by the exercising times $t_1,\ldots,t_N$.
On the other hand, the option holder is still able to profit from high energy prices as he has the right but not the obligation to exercise the option at each time point. \\
The holder of a short position might write a floor to produce liquidity upfront. The maximum gain from the short position is then limited to the strike $K$.


\frame{\frametitle{Example: Caps and Floors}
For a fixed premium, a buyer of a cap (call) is protected on the market price becoming stronger, while a buyer of a floor (put) is protected on the market price becoming weaker.\\
\begin{center}
\includegraphics[height=4.3cm]{../../../pics/cap}
\end{center}


\frame{\frametitle{Collars}
A collar is a combination of a cap and a floor such that variable prices are limited to a certain corridor. A long collar position consists of long one cap (with high strike $K_2$) and short one floor (with low strike $K_1$) - a short collar position is short one cap and long one floor. As long as the price of the underlying is between $K_1$ and $K_2$ at one of the dates $t_i$, no cash flows are exchanged. If the underlying is above $K_2$, the holder of the long collar position receives the difference of the actual price and $K_2$. If the underlying is below $K_1$, the short collar position receives the difference between $K_1$ and the actual price.


\frame{\frametitle{Collar - Payoff}
As a long collar position is a strip of call options minus a strip of put options, the payoff of a collar at each time point $t_i$ is the following:
\begin{figure}
	\centering
		\includegraphics[width=.80\textwidth]{../../../pics/Collar}
	\label{fig:Collar}
\end{figure}


\frame{\frametitle{Collar - Pricing}
Collars might be seen as a strip of bear/bull spreads, or as a strip of call options minus a strip of put options in the case of a long collar position. Consequently, the pricing formula is just the combination of the formulas for the cap and the floor:
\begin{align*}
	U^{K_1, K_2}_{collar}(t)&=U^{K_2}_{cap}(t)-U^{K_1}_{floor}(t)\\
	&=\frac{1}{N}\sum_{i=1}^Ne^{-r(t_i-t)}\EX[(S(t_i)-K_2)^+ - (K_1-S(t_i))^+]
\end{align*}
The price of a collar might be positive or negative - or even zero. In case the price is zero, the collar is called zero-cost collar.


\frame{\frametitle{Collars - Hedging}
The holder of a long position in a collar is protected against increases in the underlying price above $K_2$, but does not profit from falling underlying prices below $K_1$. Thus he is protected against rising prices with limited participation on downside prices. Having a short position in the underlying, a long collar ensures the ability to cover the short position for prices in the range of $[K_1, K_2]$.
A short collar protects against falling prices. At the same time, the ability to participate on rising prices is limited to $K_2$. Having a long position in the underlying, a short collar ensures that the position can be closed for prices in the range of $[K_1, K_2]$.


\frame{\frametitle{Collars - Example}
An energy consuming manufacturer bought the energy needed on the futures market. As its competitors did not, the manufacturer is now concerned about falling energy prices which would lead to a competitive disadvantage. Thus, the manufacturer tries to enter a short collar, protecting him against falling prices but leaving the risk of rising prices above $K_2$. This risk might be acceptable for the manufacturer as if prices rise too much, the manufacturer is able to stop its production and selling the energy already bought on the spot market - offsetting the losses of the collar.



\frame{\frametitle{Example: Jet Fuel Hedge by an Airline}
\vspace{-0.4cm}
$$\includegraphics[scale=0.7]{../../../pics/hedge1}$$
\vspace{-0.7cm}
\begin{itemize}
  \item An airline buys a fixed-price swap from a bank or trader against its jet-fuel price exposure.
  \item Buying a swap it must lock in its minimum net price receivable at the current perceived swap value.
\end{itemize}


\frame{\frametitle{Jet Fuel Hedge by an Airline: Collar}
\vspace{-0.4cm}
$$\includegraphics[scale=0.6]{../../../pics/hedge2}$$




\frame{\frametitle{Jet Fuel Hedge by an Airline: Collar}
\begin{itemize}
  \item Using a collar structure the airline can still protect itself from a price increase, but can keep its minimum net price receivable locked in at a lower rate than the current swap price.
  \item The purchase of the cap protects against jet-fuel prices rising above the strike of the cap.
  \item The sale of the floor reduces the cost of the premium in the purchase of the cap.
  \item A popular strategy is a zero-cost collar.
\end{itemize}


\frame{\frametitle{Collars - 3-way-collars}
A long collar is short one floor with strike $K_1$, long one cap with higher strike $K_2$. A possible extension is to include a short position in one cap with strike $K_3 >> K_2$ in order to reduce the cost of the collar. This extension is called 3-way-collar.
The price of a 3-way-collar is thus:
\begin{align*}
	U^{K_1, K_2, K_3}_{3-way}(t)&=U^{K_2}_{cap}(t)-U^{K_3}_{cap}(t)-U^{K_1}_{floor}(t)&\\
	&=\frac{1}{N}\sum_{i=1}^Ne^{-r(t_i-t)}\EX[(S(t_i)-K_2)^+ &\\
	 &- (S(t_i)-K_3)^+ - (K_1-S(t_i))^+]&
\end{align*}


\frame{\frametitle{3-Way-Collar - Payoff}
The holder of the 3-way-collar is protected against increases in the underlying price above $K_2$, but only till $K_3$. Afterwards, no protection exists anymore. This strategy might be a good choice if one wants to protect its buying costs but is able to stop its business if prices rally unexpectedly high (above $K_3$).
\begin{figure}
	\centering
		\includegraphics[width=.80\textwidth]{../../../pics/collar3way}
	\label{fig:collar3way}
\end{figure}


\subsection{Swing Options}
\frame{\frametitle{Swing Options}
A swing option is similar to a cap or floor except that we have
additional restrictions on the number of option exercises. Let
$\phi_i\in\{0,1\}$ be the decision whether to exercise
$(\phi_i=1)$ or not to exercise $(\phi_i=0)$ the option at time
$t_i$. The option's payoff at time $t_i$ is given by
$$\phi_i(S(t_i)-K)\quad\mbox{call resp.}\quad\phi_i(K-S(t_i))\quad\mbox{put}.$$
We now require that the number of exercises is between $E_{\min}$
and $E_{\max}$.

\frame{\frametitle{Swing Options}
To determine the swing option value, we have to find an optimal exercise
strategy $\Phi=(\phi_1,\ldots,\phi_N)$ maximising the expected
payoff
$$\sum_{i=1}^Ne^{-r(t_i-t)}\EX[{\phi_i(S(t_i)-K)}]\quad\rightarrow\max$$
subject to $$E_{\min}\leq\sum_{i=1}^N\phi_i\leq E_{\max}.$$

To calculate the option value various mathematical techniques are used.

\frame{\frametitle{Bounds for Swing Options}

\underline{Strategy}\\
For deterministic spot prices, we
\begin{itemize}
  \item  Calculate the discounted payoffs
  $P(t_i)=e^{-r(t_i-t)}(S(t_i)-K)$.
  \item  Sort the discounted payoffs $P(t_i)$ in descending order.
  \item Take the first $E_{\min}$ payoffs regardless of their
  value and subsequent payoffs up to $E_{\max}$ until their sign
  become negative.
\end{itemize}

\frame{\frametitle{Bounds for Swing Options}

For stochastic spot prices the MC-approach gives an upper bound,
since information on the whole path is used, but in reality only
information up to time $t$ is available when deciding at time $t$.

A lower bound is given by the intrinsic value
$$\sum_{i=1}^Ne^{-r(t_i-t)} \phi_i^F (F(t,t_i)-K) \quad\rightarrow\max$$
subject to $$E_{\min}\leq\sum_{i=1}^N\phi^F_i\leq E_{\max}$$
where $\phi_i^F=\IF_{\{F(t,t_i)>K\}}$, unless the restriction on $E_{\min}$ is in force.




\subsection{Spread Options}
\frame{\frametitle{Spread Options}
Some market participants are exposed to the difference of
commodity prices. Examples are
\begin{itemize}
  \item<1-> the dark spread between power and coal (model for a coal-fired power plant)
  \item<2-> the spark spread between power and gas (model for a gas-fired power plant)
  \item<3-> the crack spread between different refinements of oil (model for a refinement plant)
\end{itemize}


\frame{\frametitle{Spread Trading}
Spreads are used to describe power plants, refineries, storage facilities and transmission lines. Spread positions may be initiated in futures contracts
\begin{itemize}
  \item<1-> for different, but related commodities,
  \item<2-> for different delivery month of the same commodity,
  \item<3-> for same commodity traded on different exchanges.
\end{itemize}


\frame{\frametitle{Spread Trading}
Spread trading involves taking a long position in one futures contract and simultaneously taking a short position in another, related futures contract.
\begin{itemize}
  \item<1-> Spread position neutralizes price risk.
  \item<2-> A profit or loss results only if the relative prices of the two contracts change.
  \item<3-> If spreads are expected to narrow, buy the lower-priced contract and sell the higher-priced contract.
  \item<4-> If spreads are expected to widen, buy the higher-priced contract and sell the lower-priced contract.
\end{itemize}



\frame{\frametitle{Spark Spread}
\begin{itemize}
  \item<1-> Differential between the price of electricity (output) and the price of natural gas (input).
  \item<2-> Can be used to financially replicate the physical reality of a gas-fired power plant: Short position in fuels and long position in electricity.
  \item<3-> Spark spreads are traded OTC.
\end{itemize}


\frame{\frametitle{Spark Spread}

$$\text{Spark\_Spread}=\text{Power\_Price} - \text{Heat\_Rate}\cdot\text{Fuel\_Price}.$$
\vspace{0.2cm}
\begin{itemize}
  \item<1-> Heat rate provides a conversion factor between fuels used to generate power and the power itself.
  \item<2-> Heat rate is the number of Btus needed to make 1kWh of electricity.
  \item<3-> In the absence of any inefficiency it takes 3412Btu to produce 1kWh of electricity.
\end{itemize}



\frame{\frametitle{Example: Spark Spread}
The price of electricity is currently 42.69EUR/MWh, the price of natural gas is 4.86EUR/MMBtu and the heat rate is 8152Btu/kWh. The spark spread quoted in EUR/MWh is
$$Spread=42.69\text{EUR/MWh}-0.001\ast8152\text{Btu/kWh}\ast4.86\text{EUR/MMBtu}$$
$\,\qquad\quad\;\;=3.07\text{EUR/MWh}.$\\
\vspace{0.2cm}
The positive spark spread means that it is economical to run the plant (without taking into account additional generating costs).


%\subsection{Clean Spread}
\frame{\frametitle{Clean Spreads}
In countries covered by the European Union Emissions Trading Scheme, utilities have to consider also the cost of carbon dioxide emission allowances. Emission trading has started in the EU in January 2005.
\begin{itemize}
  \item Clean spark spread represents the net revenue a gas-fired power plant makes from selling power, having bought gas and the required number of carbon allowances.
  \item Clean dark spread represents the net revenue a coal-fired power plant makes from selling power, having bought coal and the required number of carbon allowances.
  \item The difference between the clean dark spread and the clean spark spread is known as the climate spread.
\end{itemize}



\frame{\frametitle{Clean Spark Spread}
Clean Spark Spread = Power Price - Heat Rate $\cdot$ Gas Price -\\- Gas Emission Intensity Factor $\cdot$ Carbon Price\\
  \vspace{0.6cm}
Clean Spark Spread reflects the cost of generating power from gas after taking into account gas and carbon allowance costs. A positive spread effectively means that it is profitable to generate electricity, while a negative spread means that generation would be a loss-making activity. However, it is important to note that the Clean Spark Spreads do not take into account additional generating charges beyond gas and carbon, such as operational costs.




\frame{\frametitle{Clean Dark Spread}
Clean Dark Spread = Power Price - Heat Rate $\cdot$ Coal Price -\\- Coal Emission Intensity Factor $\cdot$ Carbon Price\\
  \vspace{0.6cm}
Clean Dark Spread reflects the cost of generating power from coal after taking into account coal and carbon allowance costs. A positive spread effectively means that it is profitable to generate electricity for the period in question, while a negative spread means that generation would not be profitable. Clean Dark Spreads do not account for additional generating charges beyond coal and carbon.




\frame{\frametitle{Power Plant as a Clean Dark Spread}
A coal-fired power plant can be viewed as a call option on the clean dark spread with the variable cost of running the plant (beyond coal and carbon) being the strike and the payoff equal to\\
$$\Pi=max\{P-HR\cdot Coal - I\cdot Carbon - V\}.$$
P:\quad\;\;\;\;\; Power Price\\
HR:\quad\;\;\; Heat Rate\\
Coal:\quad\, Coal Price\\
I:\qquad\;\;\; Coal Emission Intensity Factor\\
Carbon: Carbon Price\\
V:\qquad\quad Variable cost of running the plant (beyond coal and\\
\qquad\quad\;\;\; carbon)


\frame{\frametitle{Power Plant as a Clean Dark Spread}
Indeed, the decision to run or not to run the power plant can be described as follows:
\begin{itemize}
  \item If $P-HR\cdot Coal - I\cdot Carbon - V\geq0$, then run the plant. In this case buying fuel and paying variable costs ($HR\cdot Coal + I\cdot Carbon + V$) to run the plant and then selling the generated power for P results in the positive gain.
  \item If $P-HR\cdot Coal - I\cdot Carbon - V<0$, then do not run the plant. In this case buying fuel and paying variable costs ($HR\cdot Coal + I\cdot Carbon + V$) to run the plant will not be compensated by sold power.
\end{itemize}



%\subsection{Climate Spread}
\frame{\frametitle{Climate Spread}
$$\text{Climate Spread = Clean Dark Spread - Clean Spark Spread}$$\\
  \vspace{0.6cm}
In a carbon constrained economy a power producer in a geographic area where coal is currently the preferred method by which electricity is generated may eventually encounter a negative climate spread  if carbon credit prices rise. This would mean that when taking into consideration the cost to produce (coal is on average 2.5 times as polluting as natural gas for the same MWh of electricity) the natural gas would be a better decision.



\frame{\frametitle{Example: Clean Spark Spread}
The price of electricity is currently 42.69EUR/MWh, the price of natural gas is 4.86EUR/MMBtu, the carbon price is 12EUR/t$CO_2$, the heat rate is 8152Btu/kWh and the gas emission intensity factor is 0.11t$CO_2$/MWh. The clean spark spread quoted in EUR/MWh is\\
$\text{Clean Spark Spread}=42.69\text{EUR/MWh}$\\
$\qquad\qquad\qquad\qquad\;\,-0.001\ast8152\text{Btu/kWh}\ast4.86\text{EUR/MMBtu}$\\
$\qquad\qquad\qquad\qquad\;\,-0.11tCO_2\text{/MWh}\ast12\text{EUR/t}CO_2$\\
$\qquad\qquad\qquad\qquad\;\,=1.75\text{EUR/MWh}.$\\
\vspace{0.2cm}
It is profitable to generate electricity, if additional generating charges beyond gas and carbon are lower than 1.75EUR/MWh.


\frame{\frametitle{Example: Clean Dark Spread}
The price of electricity is currently 42.69EUR/MWh, the coal price is 95.04EUR/t or 3.96EUR/MMBtu (with heat content of 24MMBtu/t), the carbon price is 12EUR/t$CO_2$, the heat rate is 9500Btu/kWh and the coal emission intensity factor is 0.26t$CO_2$/MWh. The clean dark spread quoted in EUR/MWh is\\
$\;\text{Clean Dark Spread}=42.69\text{EUR/MWh}$\\
$\qquad\qquad\qquad\qquad\;\,-0.001\ast9500\text{Btu/kWh}\ast3.96\text{EUR/MMBtu}$\\
$\qquad\qquad\qquad\qquad\;\,-0.26tCO_2\text{/MWh}\ast12\text{EUR/t}CO_2$\\
$\qquad\qquad\qquad\qquad\;\,=1.95\text{EUR/MWh}.$\\
\vspace{0.2cm}
It is profitable to generate electricity, if additional generating charges beyond coal and carbon are lower than 1.95EUR/MWh.



\frame{\frametitle{Example: Climate Spread}
Suppose that the price of carbon rises to 19.6EUR/t$CO_2$.\\
\vspace{0.25cm}
$\text{Climate Spread = Clean Dark Spread - Clean Spark Spread}$\\
$\qquad\qquad\qquad\;\;\,=-0.026\text{EUR/MWh}-0.915\text{EUR/MWh}$\\
$\qquad\qquad\qquad\;\;\,=-0.941\text{EUR/MWh}.$\\
\vspace{0.25cm}
The clean dark spread becomes negative (-0.026EUR/MWh), implying that electricity generation by a coal-fired power plant would be a loss-making activity, whereas the clean spark spread remains positive (0.915EUR/MWh), meaning that it is profitable to generate electricity by a gas-fired power plant, if additional generating charges beyond gas and carbon are lower than 0.915EUR/MWh.


\frame{\frametitle{Clean Spark Spread Forward}
\begin{figure}[htp]
\centering
\includegraphics[width=\textwidth]{../../../pics/Spark-Spread-2012.pdf}
\end{figure}



 %%%%%%%%%% Clean Dark Spread
\frame{\frametitle{Gas Power Plant}
\begin{figure}[htp]
\centering
\includegraphics[width=\textwidth]{../../../pics/GuD-Lingen}
\label{prices}
\end{figure}


\frame{\frametitle{Political Risk}
\begin{itemize}
\item<1-> Installed capacity: 876 MW
\item<2-> Variable cost ca. 60 EUR/MWh
\item<3-> Profitable hours per year
\begin{itemize}
\item  2010 (993),
\item 2011 (2309),
\item 2012 (737),
\end{itemize}
with average profit 6.9 EUR per MWh.
\item<4-> Typical assumption on investing
\begin{itemize}
\item 3500 profitable hours
\item 10 EUR per MWh profit
\end{itemize}
\item<5-> loss per year
\begin{itemize}
\item  2010: (3500-993)*876*10=21961320 EUR,
\item 2011:  (3500-2309)*876*10 = 10433160 EUR,
\item 2012: (3500-737)*876*10= 24203880 EUR.
\end{itemize}

\end{itemize}




\frame{\frametitle{A day in august}
\begin{figure}[htp]
\centering
\includegraphics[width=\textwidth]{../../../pics/day-profile-august}
%\caption{Wind, Sonne und Strompreise}
\end{figure}



\frame{\frametitle{Wind, sun and electricity}
\begin{figure}[htp]
\centering
\includegraphics[width=\textwidth]{../../../pics/week1-14Nov.png}
%\caption{Wind, Sonne und Strompreise}
\end{figure}


\subsection{Energy Swaps}

\frame{\frametitle{Standard Specification of Energy Swap}
\begin{itemize}
\item<1-> An Energy Swaps is a contract between two parties to exchange - or swap - cash flows, one of which is a fixed price normally agreed at execution; the other is based on the average of a floating price index during the contract period.
\item<2-> Thus the two counterparties of the deal, the buyer and the seller, exchange fixed cash-flows agreed at the contract time for unknown floating cash-flows in the future.
\item<3->
No physical delivery of the underlying energy takes place; there is only financial settlement.
\end{itemize}


\frame{\frametitle{Specification of a Swap Contract}
When traders are negotiating a swap contact (OTC deal) they have to define
\begin{itemize}
  \item<1-> the fixed price
  \item<2-> the floating-price reference
  \item<3-> the pricing period (e.g. one month, quarterly, calender year)
  \item<4-> the start date (effective date)
  \item<5-> the end date (termination)
  \item<6-> the payment-due date
\end{itemize}


\frame{\frametitle{Pricing Period}
\begin{itemize}
\item<1-> In energy and general commodity markets, OTC derivatives are priced monthly.
\item<2-> So even if a quarterly contract is traded, after each month during the pricing period, one-third of the volume will be priced out and a settlement will become due or a payment will be received by the organization.
\end{itemize}


\frame{\frametitle{Payment-due Date}
\begin{itemize}
\item<1-> For a swap priced against an American or European floating price reference, the payment due date is normally the 5. business day after the last pricing day of each pricing period.
\item<2->
For contracts priced against an Asian-based floating-price reference, payment for settlement is generally due 10 business days, sometimes up to 14 business days, after each pricing period.
\end{itemize}


%\subsection{Basic Swap Contracts}

\frame{\frametitle{Plain Vanilla Swap}
\begin{itemize}
  \item<1-> Simple monthly averaging swap in which a fixed price is exchanged against a floating price in the future.
  \item<2-> Extensively used in oil, LPG (liquified petrolium/propane gas), and LNG (liquified natural gas)-related trading and hedging.
  \item<3-> When executing the deal, the counterparties agree on the fixed price for that day, and which floating price reference they will use to calculate the settlement.
      \item<4->
$$\includegraphics[scale=0.3]{../../../pics/ex1}$$
\end{itemize}


\frame{\frametitle{Plain Vanilla Swap}
\textcolor[rgb]{0.00,0.25,0.50}{Cash-flow example of plain vanilla swap:}\\
\begin{itemize}
\item<1->
A buys fixed price$\qquad\qquad$\$15.00 (buys fixed, sells floating)
\item<2-> B sells fixed price$\qquad\qquad\;$\$15.00 (sells fixed, buys floating)
\item<3->
Floating price reference (e.g. Platts) average during the pricing period \$16.00
\item<4->
\vspace{0.3cm}
\textcolor[rgb]{0.00,0.25,0.50}{Net result:}$\quad$Counterparty A = +\$1.00 (floating-fixed)\\
$\qquad\qquad\quad\,$  Counterparty B = -\$1.00
\item<5->
$\textcolor[rgb]{0.00,0.25,0.50}{\Rightarrow}$ B pays A \$1.00. Only the difference is exchanged, NOT the \\
$\quad\;\,$principal national amount.
\end{itemize}


%\subsection{Differential Swap}

\frame{\frametitle{Differential Swap}
\begin{itemize}
  \item<1-> Instead of having one fixed price against a floating price, it is based on the difference between a fixed price in two products.
  \item<2-> In the oil sector, the most popular differential swap is the jet kero against gasoil (regrade swap).
  \item<3-> Use across the whole energy spectrum, e.g. spark spread, dark spread.
\item<4->
$$\includegraphics[scale=0.3]{../../../pics/ex2}$$
\end{itemize}



\frame{\frametitle{Differential Swap}
\textcolor[rgb]{0.00,0.25,0.50}{Cash-flow example of differential swap:}
\begin{itemize}
\item<1->
A buys fixed-price kero and sells fixed-price gasoil at a difference of \$0.50 per barrel kero premium
\item<2-> B sells fixed-price kero and buys fixed-price gasoil at a difference of \$0.50 per barrel kero premium
\item<3-> Floating price reference (Platts) kero and gasoil average difference during the pricing period $\;\,$ \$0.60 per barrel kero premium
\item<4->
\textcolor[rgb]{0.00,0.25,0.50}{Net result:}$\quad$Counterparty A = +\$0.10 per barrel (floating-fixed)\\
$\qquad\qquad\quad\,$  Counterparty B = -\$0.10 per barrel
\item<5->
$\textcolor[rgb]{0.00,0.25,0.50}{\Rightarrow}$ B pays A \$0.10 per barrel. Only the difference is exchanged,\\
$\quad\;\,$NOT the principal national amount.
\end{itemize}


%\subsection{Margin Swap}

\frame{\frametitle{Margin Swap}
\begin{itemize}
\item<1->
An organization can take its overall price risks from several energy inputs and outputs of the business process and get a complete swap structure that guarantees its profit margin.
\item<2-> Instead of managing many individual positions with several counterparties it can be more cost-efficient to enter into a margin swap with one counterparty that is willing to provide a contract that covers all the price risk.
\item<3-> Single swap contract protects the \textcolor{red}{net} overall margin.
\end{itemize}


\frame{\frametitle{Margin Swap}
Margin Swap for an oil refiner:
$$\includegraphics[scale=0.25]{../../../pics/ex4}$$


%\subsection{Jet Fuel SWAP in USD/mt}

\frame{\frametitle{Jet Fuel Swap}
\vspace{-0.4cm}
$$\includegraphics[scale=0.35]{../../../pics/hedge1}$$
\vspace{-0.7cm}
\begin{itemize}
  \item<1-> An airline buys a fixed-price swap from a bank or trader against its jet-fuel price exposure.
  \item<2-> Buying a swap it must lock in its minimum net price receivable at the current perceived swap value.
\end{itemize}


\frame{\frametitle{Example: Jet Fuel Swap}
An airline expects that the oil price will rise in the future. In order to fix the oil price it buys on 15. June 2010 a \textcolor[rgb]{1.00,0.00,0.00}{fixed-price swap}. $$\includegraphics[scale=0.5]{../../../pics/ex5}$$



\frame{\frametitle{Example: Jet Fuel Swap}
Contract specification/Fixed Jet Fuel Price:\\
\vspace{0.2cm}
$$\includegraphics[scale=0.63]{../../../pics/ex6}$$


\frame{\frametitle{Example: Jet Fuel Swap}
Contract specification/Floating Jet Fuel Price:\\
\vspace{0.2cm}
$$\includegraphics[scale=0.63]{../../../pics/ex7}$$


\frame{\frametitle{Example: Jet Fuel Swap}
Clearing:
\begin{itemize}
  \item<1-> The payment settlement date is the 1. bank business day (1. November 2011) after the last pricing day (31. October 2011) of the pricing period (1.-31. October 2011).
  \item<2-> The payment due date is the 5. bank business day (7. November 2011) after the last pricing day of the pricing period.
  \item<3-> Bank holidays according to \url{http://www.chicagofed.org/webpages/utilities/about_us/bank_holidays.cfm}.
\end{itemize}


\frame{\frametitle{Example: Jet Fuel Swap}
Payment:
\begin{itemize}
  \item Payment =(Float-Fix)$\times$Volume,$\quad$if Float$\geq$Fix.
  \item Payment=(Fix-Float)$\times$Volume,$\quad$if Float$<$Fix.
\end{itemize}
\vspace{0.3cm}
Payment in USD\\
Floating/fixed price in USD/mt\\
Volume in mt (metric tones)



\frame{\frametitle{Example: Jet Fuel Swap}
\begin{itemize}
  \item<1-> Suppose that the average Jet Fuel reference index for October 2011 is 700USD/mt. The airline buys the fuel for 600USD/mt (fixed price) and sells for 700USD/mt (floating price reference for October 2011). Thus the airline's profit is
      $$(700\text{USD/mt} - 600\text{USD/mt})\times2000\text{mt} = 200\;000\text{USD}.$$
  \item<2-> Suppose that the average Jet Fuel reference index for October 2011 is 550USD/mt. The airline buys the fuel for 600USD/mt (fixed price) and sells for 550USD/mt (floating price reference for October 2011). Thus the airline's loss is
      $$(600\text{USD/mt} - 550\text{USD/mt})\times2000\text{mt} = 100\;000\text{USD}.$$
\end{itemize}



\subsection{Options on Electricity Swaps}


\frame{\frametitle{Two-Factor GBM Specification}
The basic HJM model for the dynamics of the forward rates $f(t,T)$ is given by
$$
%\label{forward rate dynamics}
df(t,T) = \alpha(t,T) dt + \s(t,T) dW(t).
$$
We consider a two-factor model
$$
\frac{dF(t,T)}{F(t,T)}=\sigma_1(t,T)dW_1(t)+\sigma_2dW_2(t)
$$
where the volatilities are
$$
\sigma_1(t,T)=e^{-\kappa (T-t)}\sigma_1 \; \mbox{ and } \; \sigma_2>0
$$



\frame{ \frametitle{The Model Framework for Electricity Swaps}
\begin{itemize}
\item<1-> Use observable products, e.g. month futures as building blocks,
\item<2-> Under a risk-neutral measure month forward prices $F(t,T,T+m)=F(t,T)$ have to be martingales,
\item<3-> Assume the dynamics
$$dF(t,T)=\sigma(t,T)F(t,T)dW(t),$$
where $\sigma(t,T)$ is an adapted $d$-dimensional deterministic function and
$W(t)$ a $d$-dimensional Brownian motion.
\item<4-> Initial value of this SDE is the initial forward curve observed at the market.
\end{itemize}


\frame{ \frametitle{Options on Building Blocks}
A European call option on $F(t,T)$ with maturity $T_0$ and strike
$K$ can be easily evaluated by the Black-formula
\begin{eqnarray}\label{eq:month-option}
V^{option}(0)=e^{-rT_0}\left(F(0,T)\Phi(d_1)-K\Phi (d_2)\right),
\end{eqnarray}
where $\Phi$ denotes the normal distribution, $\Sigma(T_0,T)=\int_0^{T_0}||\sigma(s,T)||^2ds$ and
\begin{eqnarray*}
d_1& = & \frac{\log \frac{F(0,T)}{K}+\frac{1}{2}\Sigma(T_0,T)}{\sqrt{\Sigma(T_0,T)}}\\
d_2 & = & d_1 - \sqrt{\Sigma(T_0,T)}
\end{eqnarray*}



\frame{ \frametitle{The Model Framework -- $n$-period futures}
\begin{itemize}
\item<1-> Use observable products, e.g. month futures as building blocks,
\item<2-> Express an $n$-period delivery futures as
$$Y_{T_1, \ldots, T_n}(t)=\frac{\sum_{i=1}^n e^{-r(T_i-t)}F(t,T_i)}{\sum_{i=1}^n e^{-r(T_i-t)}}.$$
(Compare modelling of forward swap rates in terms of forward LIBOR rates)
\item<3-> In case of 1-year-futures, the swap rate is the forward price of the 1-year-futures,
which can be also observed in the market.
\end{itemize}


\frame{ \frametitle{Implied Prices of Futures}
\begin{center}
\includegraphics[height=6cm, width=10cm]{../../../pics/forwardcurve2}
\end{center}




\frame{ \frametitle{Options on $n$-period futures}
\begin{itemize}
\item<1-> We need to compute
$$e^{-rT_0} \EX \left[\left(Y(T_0)-K\right)^+\right],$$
where the distribution of $Y$ as a sum of lognormals is unknown.
\item<2->
We approximate $Y$ by a random variable $\hat{Y}$,
which is lognormal and matches $Y$ in mean and variance.
\item<3->
Then,
$$\log \hat{Y} \sim \Phi(m,s)$$
with $s^2$ depending on the choice of the volatility functions
$\sigma(t,T_i)$.
\item<4->
An analysis of the goodness of the approximation
can be found in Brigo-Mercurio (2003).
\end{itemize}



\frame{ \frametitle{Options on $n$-period futures}

Using this approximation, it
is possible to apply a Black-Option formula again to obtain the
option value as
\begin{eqnarray}\label{eq:approx-option}
V^{option} & = & e^{-rT_0} \EX \left[\left(Y(T_0)-K\right)^+\right] \nonumber \\
& \approx & e^{-rT_0} \EX \left[\left(\hat{Y}(T_0)-K\right)^+\right]\nonumber \\
& = &  e^{-rT_0} \left(Y(0)\Phi(d_1)-K\Phi(d_2)\right)
\end{eqnarray}
with
\begin{eqnarray*}
d_1& = & \frac{\log \frac{Y(0)}{K}+\frac{1}{2}s^2}{s}\\
d_2 & = & d_1 - s
\end{eqnarray*}



\frame{ \frametitle{Specific  Two-Factor Model}

\begin{itemize}
\item<1-> For a fixed delivery start $T$ and delivery period 1 month, let the dynamics of a Future $F_{t,T}$ be given by the two factor model:
\begin{eqnarray*}
F(t,T)& =&F(0,T)\exp\left\{\mu(t,T)  +\int_0^t\hat{\sigma_1}(s,T)dW_s^{(1)}+\sigma_2W_t^{(2)}\right\}
\end{eqnarray*}

\item<2-> $W^{(1)}$ and $W^{(2)}$ independent Brownian motions
\item<3-> $\hat{\sigma_1}(s,T)=\sigma_1e^{-\kappa(T-s)}$
\item<4-> $\sigma_1$, $\sigma_2$, $\kappa>0$ constants
\item<5-> $\mu(t,T)$ being the risk-neutral martingale drift
\end{itemize}


\frame{ \frametitle{Model Parameters}
$\sigma_1$ affects the level at the short end of the volatility curve

\begin{center}
\includegraphics[height=6cm, width=10cm]{../../../pics/sigma1}
\end{center}



\frame{ \frametitle{Model Parameters}
$\kappa$ affects the slope of the volatility curve at the short end

\begin{center}
\includegraphics[height=6cm, width=10cm]{../../../pics/kappa}
\end{center}



\frame{ \frametitle{Model Parameters}
$\sigma_2$ affects the level at the long end of the volatility curve

\begin{center}
\includegraphics[height=6cm, width=10cm]{../../../pics/sigma2}
\end{center}



\frame{ \frametitle{Pricing of Futures}
\begin{itemize}
\item<1-> In this model, all products are expressed using Month-Futures
\item<2-> Prices of quarterly and yearly Futures are given as an
average of the  $n$ corresponding monthly Futures.
\item<3-> $Y_{t,T_1, \ldots T_n}=Y=\frac{\sum e^{-rT_i}F_{t,T_i}}{\sum e^{-rT_i}}$
is the forward price of a $n$-month forward quoted in the market (cp. swap rate)
\end{itemize}

\frame{ \frametitle{Pricing of Options on Month-Futures}
\begin{itemize}
\item<1-> At time $t=0$, the price of a  Call-Option with strike $K$ and maturity $T_0$ on a Month-Future $F_{t,T}$ is given by
$$e^{-rT_0}\EX\left[\left(F_{T_0,T}-K\right)^+\right]$$
\item<2-> Within the model,  $F_{T_0,T}$ is log-normally distributed with known variance
$$
\Sigma(T_0,T) =   \frac{\sigma_1^2}{2\kappa}(e^{-2\kappa (T-T_0)}-e^{-2\kappa T})+\sigma_2^2T_0
$$
\item<3-> Thus the option's value is given by the formula (Black 76):
\begin{eqnarray*}
e^{-rT_0}\EX\left[\left(F_{T_0,T}-K\right)^+\right] & = &
e^{-rT_0}\left(F_{0,T}\Phi(d_1)-K\Phi(d_2)\right)
\end{eqnarray*}with $d_{1,2}$ depending on the parameters $\sigma_1, \sigma_2, \kappa$.
\end{itemize}

\frame{ \frametitle{Pricing of Options on quarterly and yearly Futures}
\begin{itemize}
\item<1-> At time $t=0$, the price of a  Call-Option with strike $K$ and maturity $T_0$ on a $n$-Month-Future
$Y$ is given by
$$
e^{-rT_0}\EX\left[\left(Y-K\right)^+\right]=e^{-rT_0}\EX
\left[\left(\frac{\sum e^{-rT_i}F_{t,T_i}}{\sum e^{-rT_i}}-K\right)^+\right]$$
\item<2-> The distribution of the sum is not known within the model. There is no explicit solution to this integral.
\item<3-> Approximate the random variable $Y$ by a log-normal random variable $\hat{Y}$
with same mean and variance (depending on the model parameters)
\end{itemize}

\frame{ \frametitle{Matching the Variance}
Using the moment-generating function of a normal random variable, we get
$$
\exp (s^2) = \frac{\textrm{Var}(Y)}{\left(\EX (Y)\right)^2} + 1  = \frac{\EX(Y^2)}{\EX(Y)^2}
$$
From the martingale property
$\EX (F_{T_0,T_i}) =  F_{0,T_i} $ and
$$ \EX (Y_{T_1, \ldots T_n}(T_0))  =
\frac{\sum e^{-r(T_i-T_0)} F_{0,T_i}}{\sum e^{-r(T_i-T_0)}}.$$

\frame{ \frametitle{Matching the Variance}

So
$$
\EX (Y_{T_1, \ldots T_n}(T_0)^2)
=  \frac{\sum_{i, j}e^{-r(T_i+T_j-2T_0)} F_{0,T_i}F_{0,T_j}\cdot \exp Cov_{ij}}{\left(\sum e^{-r(T_i-T_0)}\right)^2}
$$ with
$Cov_{ij} = \textrm{Cov}(\log F(T_0,T_i), \log F(T_0,T_j))$.

The covariance can be computed directly from the explicit solution of the SDE
$$\begin{array}{ll}
& \DSE \textrm{Cov}(\log F(T_0,T_i), \log F(T_0,T_j)) \\*[12pt]
 = & \DSE
 e^{-\kappa(T_i+T_j-2T_0)}\frac{\sigma_1^2}{2\kappa}(1-e^{-2\kappa T_0})+\sigma_2^2 T_0
\end{array}
$$

\frame{ \frametitle{Pricing of Options on quarterly and yearly Futures}
The option value can be computed by Black's formula
\begin{eqnarray*}
e^{-rT_0}\EX\left[\left(Y-K\right)^+\right]&\approx & e^{-rT_0}\EX\left[\left(\hat{Y}-K\right)^+\right]\\
& = &e^{-rT_0}\left(Y(0)\Phi(d_1) - K\Phi(d_2)\right)
\end{eqnarray*}
with $d_{1,2}$ depending on the parameters $\sigma_1, \sigma_2, \kappa$.



\frame{ \frametitle{Parameter Estimation}
\begin{itemize}
\item<1-> Use the approximating Black-formula
    \begin{eqnarray*}
        \textrm{Option value}&=&    e^{-rT_0}\left(Y(0)\Phi (d_1)-K\Phi(d_2) \right)
\\
        d_{1,2} & = & d_{1,2}\left(Y(0),K,Var(\log \hat{Y}(T_0))\right)
    \end{eqnarray*}
        Only the variance $Var(\log \hat{Y}(T_0))$ depends on the unknown parameters

\item<2-> Compute the variances $Var(\log\hat{Y}(T_0))$ for products observable in the market
\item<3-> Choose parameter $\sigma_1$, $\sigma_2$ and $\kappa$ to minimize the distance of the model variances
to the market variances in a given metric (in the least-square sense)
\end{itemize}


\frame{ \frametitle{Data}
{\small \begin{table}[btp]
\begin{center}
\begin{tabular}{ccrrrr}
Product     & Delivery Start    & Strike    &   Forward & Market Price  &Implied Vola \\
\hline
Month   &October 05     &48 &48.90&2.023&43.80\%\\
Month   &November 05        &49 &50.00&3.064&37.66\%\\
Month   &December 05        &49 &49.45&3.244&34.72\%\\
\hline
Quarter & October 05        &48 &49.44&2.086&35.15\%\\
Quarter & January 06        &47 &48.59&3.637&28.43\%\\
Quarter & April 06          &40 &40.71&3.421&26.84\%\\
Quarter & July 06           &42 &41.80&3.758&27.19\%\\
Quarter & October 06        &43 &43.71&4.566&25.35\%\\
\hline
Year    &January 06     &44 &43.68&1.521&20.19\%\\
Year    &January 07     &43 &42.62&3.228&19.14\%\\
Year    &January 08     &42 &42.70&4.286&17.46\%\\
\end{tabular}
\caption{ATM calls and implied Black-volatility, Sep 14}
\label{fig:data}
\end{center}
\end{table}
}
\frame{ \frametitle{Parameter Estimates}
 \begin{table}[btp]
 \begin{center}
\begin{tabular}{p{4cm}cccrrrrcc}
Method  &Constraints    &$\sigma_1$ &$\sigma_2$ &$\kappa$   &Time&\\
\hline
Function calls and numerical gradient       &yes            &0.37       &0.15       &1.40       &$<$1min&\\
Least Square Algorithm  &no&0.37&0.15&1.41      &$<$1min&\\
\end{tabular}
\caption{Parameter estimates with different optimizers, market data as of Sep 14}
\label{fig:estimates18}
\end{center}
\end{table}

Options, which are far away from maturity, will have a volatility of about $15\%$,
which can add up to more than $50\%$, when time to maturity decreases.

A $\kappa$ value of 1.40 indicates, that disturbances in the futures market
halve in $\frac{1}{\kappa}\cdot \log 2 \approx 0.69$ years.













