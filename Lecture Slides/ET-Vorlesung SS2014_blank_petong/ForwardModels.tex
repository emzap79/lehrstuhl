% !TEX root = EnergyTrading_ss14UDE.tex
\section{Forward Models}
\subsection{Basic Pricing Relations for Forwards/Futures}

% frametitle
{Storage, Inventory and Convenience Yield}


% begin itemize




	The theory of storage aims to explain the differences between spot
and Futures (Forward) prices by analyzing why agents hold inventories.


	Inventories
allow to meet unexpected demand, avoid the cost of frequent revisions in
the production schedule and eliminate manufacturing disruption.


	This
motivates the concept of convenience yield as a benefit, that accrues to the
owner of the physical commodity but not to the holder of a forward contract.


	Thus the convenience yield is comparable to the dividend yield for stocks.


	A modern view is to view storage (inventory) as a timing option, that
allows to put the commodity to the market when prices are high
and hold it when the prices are low.


% end itemize



% frametitle
{Spot-Forward Relationship in Commodity Markets }
Under the no-arbitrage assumption we have
\begin{equation}\label{SF-rel}
F(t,T)=S(t)e^{(r-y)(T-t)}
\end{equation}
where $r$ is the interest rate at time $t$ for maturity $T$ and $y$ is the convenience yield.

% frametitle
{Spot-Forward Relationship in Commodity Markets }

Observe that (\ref{SF-rel}) implies


% begin itemize




	spot and forward are redundant (one can replace the other) and form a
linear relationship (unlike options)


	with two forward prices we can derive the value of $S(t)$ and $y$


	knowledge of $S(t)$ and $y$ allows us to construct the whole forward curve


	for $r-y <0$ we have backwardation; for $r-y>0$ we have contango.


% end itemize



% frametitle
{Spot-Forward Relationship: Classical theory}


% begin itemize




	In a stochastic model we use
$$
F(t,T)=\EX_{\Q}[S(T)|\F_t]
$$
where $\F_t$ is the accumulated available market information (in most models the information generated by the spot price).


	$\Q$ is a risk-neutral probability


% begin itemize




	discounted spot price is a $\Q$-martingale


	fixed by calibration to market prices or a market price of risk argument


% end itemize




% end itemize



% frametitle
{Forward Prices and Expectation of Future Spot Prices}
The rational expectation hypothesis (REH) states that the current forward price $F(t,T)$ for a commodity with
delivery a time $T>t$ is the best estimator for the price $S(T)$ of the commodity.
In mathematical terms
\begin{equation}\label{REH}
F(t,T) = \EX_\prb[S(T) |\F_t].
\end{equation}
where $\F_t$ represents the information available at time $t$. The REH has been statistically
tested in many studies for a wide range of commodities.

% frametitle
{Futures Prices and Expectation of Future Spot Prices}
When equality in (\ref{REH}) does not hold forward prices are biased estimators of
future spot prices. If


% begin itemize




	holds, then $F(t,T)$ is an up-ward biased estimate, then risk-aversion
among market participants is such that buyers are willing to pay more than the expected
spot price in order to secure access to the commodity at time $T$ (political unrest);


	holds, then $F(t,T)$ is an down-ward biased estimate, this may reflect a
perception of excess supply in the future.


% end itemize



% frametitle
{Market Risk Premium}


% begin itemize




	The \emph{market risk premium} or \emph{forward bias} $\pi (t,T)$
relates forward and expected spot prices.


	It is defined as the difference, calculated at time $t$, between
the forward $F(t,T)$ at time $t$ with delivery at $T$ and expected
spot price:
\begin{equation}\label{forward risk premium}
\pi(t,T)= F(t,T)-\EX_\prb[S(T)| \F_t].
\end{equation}
Here $\EX_\prb$ is the expectation operator, under the
historical measure $\prb$, with information up until time $t$ and
$S(T)$ is the spot price at time $T$.


% end itemize



\subsection{HJM-type models}

% frametitle
{ Heath-Jarrow-Morton (HJM) model}

The Heath-Jarrow-Morton model uses the entire forward rate curve as
(infinite-dimensional) state variable. The dynamics of the forward rates $F(t,T)$ are {\it exogenously} given by
$$
%\label{forward rate dynamics}
dF(t,T) = \alpha(t,T) dt + \s(t,T) dW(t).
$$
For any fixed maturity $T$, the
initial condition of the stochastic differential equation
is determined by the current value
of the empirical (observed) forward rate for the future date $T$
which prevails at time $0$.

% frametitle
{One-Factor GBM Specification}
Here the volatility is
$$
\sigma_1(t,T)=e^{-\kappa (T-t)}\sigma
$$
and
$$
dF(t,T)=F(t,T)\sigma_1(t,T)dW(t)
$$

% frametitle
{Two-Factor GBM Specification}
Here the volatilities are
$$
\sigma_1(t,T)=e^{-\kappa (T-t)}\sigma_1 \; \mbox{ and } \; \sigma_2>0
$$
and
$$
\frac{dF(t,T)}{F(t,T)}=\sigma_1(t,T)dW_1(t)+\sigma_2dW_2(t)
$$

\subsection{A Market Model}

% frametitle
{Modelling Approach}


% begin itemize




	We use the HJM-framework to model the forward dynamics directly.


	We distinguish between forward contracts with a fixed time delivery and forward contracts with a delivery period, called \emph{swaps}.


	Since the HJM-framework cannot be applied to the swap dynamics literally, we differentiate between the \emph{decomposable} swaps and the \emph{atomic} swaps and apply the framework to the atomic swaps.


% end itemize



% frametitle
{Modelling Approach}
The dynamics of a decomposable swap with delivery period $[T_1,T_{N}]$ can than be obtained from $N-1$ atomic swaps by
\begin{align}
F(t,T_1,T_N)=\sum\limits_{i=1}^{N-1} \frac{T_{i+1}-T_i}{T_N-T_1} F(t,T_i,T_{i+1})\label{eqn: decomposbale swap}
\end{align}

% frametitle
{Modelling Approach}
We discuss several lognormal dynamics of the swap price,
\begin{equation}
dF(t,T_1,T_2)=\Sigma(t,T_1,T_2)F(t,T_1,T_2)\, dW(t). \label{eqn: lognormal dynamics}
\end{equation}
The only parameter in this model is the volatility function $\Sigma$ which has to capture all movements of the swap price and especially the time to maturity effect.

% frametitle
{Volatility Functions}
We assume that the swap price dynamics \emph{for all atomic swaps} is given by (\ref{eqn: lognormal dynamics})
where $\Sigma(t,T_1,T_2)$ is a continuously differentiable and positive function.

Starting out with a given volatility function for a fixed time forward contract we see that the volatility function $\Sigma$ for the swap contract is given by
\begin{equation}
\Sigma(t,T_1,T_2)=\int_{T_1}^{T_2} \hat{w}(u,T_1,T_2) \sigma(t,u) \, du. \label{eqn: swap volatility creation}
\end{equation}

% frametitle
{Schwartz}
For the volatility function of the forward we use
\begin{equation}\label{vol-schwartz}
\sigma(t,u)=a e^{-b(u-t)}
\end{equation}
where $a,b >0 $ are constant.

% frametitle
{Schwartz}
The time to maturity effect is modeled by a negative exponential function.


% begin itemize




	When the time to maturity tends to infinity the volatility function converges to zero.


	The exponential function causes that the volatility increases as the time to maturity decreases which leads to an increased volatility when the contract approaches the maturity.


% end itemize



% frametitle
{Schwartz}

Applying this forward volatility to (\ref{eqn: swap volatility creation}) the swap volatility is:
\begin{align}
\Sigma(t,T_1,T_2)&=a\,\varphi(T_1,T_2)
\end{align}
where
\begin{align}
\varphi(T_1,T_2)= \frac{e^{-b(T_1-t)}-e^{-b(T_2-t)}}{b(T_2-T_1)}
\label{volatility function varphi}
\end{align}
The Black-76 specification of the forward volatility can be obtained if $\varphi(T_1,T_2) =1$, that is $b=0$
in (\ref{vol-schwartz}).

The associated swap price volatility is then given by $\Sigma(t,T_1,T_2)=a$.

% frametitle
{Fackler-Tian}
\emph{Fackler and Tian}  replaced the factor $a$ by a seasonality function $a(t)$.
\begin{equation}
\sigma(t,u) = a(t)e^{-b(u-t)}
\end{equation}
where $a(t)$ is a seasonality function modeled by
\begin{equation}
a(t)=a+\sum_{j=1}^{7}(\alpha_j\, \cos(2\pi jt)+ \beta_j \, sin(2\pi jt)) \label{seasonality function}
\end{equation}
with $t$  measured in years. The parameters $\alpha_j$ and $\beta_j$ are real numbers and $a \ge 0, b > 0$.

% frametitle
{Fackler-Tian}

The volatility function of the swap price is therefore
\begin{equation}
\Sigma(t,T_1,T_2)=a(t)\, \varphi(T_1,T_2)
\end{equation}
where $\varphi(T_1,T_2)$ is given by (\ref{volatility function varphi}).\\

% frametitle
{Clewlow - Strickland}


% begin itemize




	Empirical observations show that in energy markets we can observe that the volatility increases strictly as the contract approaches maturity.


	Such a sharp raise can be modeled by a high value for $b$.


	The drawback of a high value for $b$ is that for contracts with long time to maturity the volatility decreases fast and thus becomes very small.


% end itemize



% frametitle
{Clewlow - Strickland}

For this reason \emph{Clewlow and Strickland} suggested
\begin{equation}
\sigma(t,u)=a((1-c)e^{-b(u-t)}+c) \label{Strickland model}
\end{equation}
where $a\ge 0, b > 0$ and $ 0 \le c\le 1$.

% frametitle
{Clewlow - Strickland}

The associated swap volatility model becomes
\begin{equation}
\Sigma(t,T_1,T_2)=a((1-c)\,\varphi(T_1,T_2)+c)
\end{equation}
where $\varphi(T_1,T_2)$ is defined as in (\ref{volatility function varphi}).\\

% frametitle
{Koekenbakker -- Lien}
\emph{Koekenbakker and Lien} suggested to use a seasonality function in the \emph{Strickland} model (\ref{Strickland model}) and proposed a volatility which is given by
\begin{equation}
\sigma(t,u)=a(t)\, ((1-c)e^{-b(u-t)}+c)
\end{equation}
where $a(t)$ is defined as in (\ref{seasonality function}) and with parameters $a \ge 0, b > 0\, ,\, 0 \le c \le 1, \,\alpha_j, \, \beta_j$ are real constants.

% frametitle
{Koekenbakker -- Lien}


% begin itemize




	Observe that $\lim \limits_{u \to t} \sigma(t,u)$ exists and $\sigma(t,t)=a(t)$.


	Hence the spot volatility is modeled by the seasonal function.


	As $u$ approaches infinity we obtain that $\sigma(t,u)=a(t)c$.


	Thus, the volatility is bounded within the interval $[a(t)c, a(t)]$, where $ca(t) \le a(t)$, since $0 \le c \le 1$.\\


% end itemize



% frametitle
{Koekenbakker -- Lien}
We obtain the swap volatility model
\begin{equation}
\Sigma(t,T_1,T_2)=a(t) ((1-c)\, \varphi(T_1,T_2)+c).
\end{equation}

% frametitle
{Benth -- Koekenbakker}
\emph{Benth and Koekebakker} suggested the following forward curve model
\begin{equation}
\sigma(t,u)=\hat{a} e^{-b(u-t)}+a(t)
\end{equation}
where $\hat{a} \ge 0, b>0$ and $a(t)$ is given by (\ref{seasonality function}).

In this model the seasonality effect is separated from the maturity effect. For the short term volatility we have $\hat{a}+a(t)$ and for the long term volatility we obtain $a(t)$.

% frametitle
{Benth -- Koekenbakker}

The associated swap volatility is given by
\begin{equation}
\Sigma(t,T_1,T_2)=\hat{a}\, \varphi(T_1,T_2) +a(t).
\end{equation}

% frametitle
{Model Summary}
A summary of the different models  is listed in Table \ref{tab: lognormal models}.
\begin{table}[h]
 \small \centering {\sf
\begin{tabular}[t]{l|c}
        Model & $\Sigma(t,T_1,T_2)$ \\ %\toprule
       Black-76 & $a$ \\
        Schwartz  &$a\, \varphi(T_1,T_2)$  \\
        Fackler and Tien  & $a(t)\, \varphi(T_1,T_2)$ \\
        Clewlow and Strickland & $a((1-c)\, \varphi(T_1,T_2)+c)$ \\
        Koekebakker and Lien  & $a(t)((1-c)\, \varphi(T_1,T_2)+c)$ \\
        Benth and Koekebakker& $\hat{a}\, \varphi(T_1,T_2) +a(t)$ \\

    \end{tabular}}
    \caption[Swap Volatility Models]
    {\small The associated swap volatility models generated by (\ref{eqn: swap volatility creation}) with $a \ge 0, b > 0$ and $0 \le c \le 1$ constants, $a(t)$ defined in (\ref{seasonality function}) and $\varphi(T_1,T_2)$ is given by (\ref{volatility function varphi}). }
    \label{tab: lognormal models}
\end{table}

