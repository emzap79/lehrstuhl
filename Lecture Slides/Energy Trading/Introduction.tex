% !TEX root = EnergyTrading_ss13UDE.tex


\section{Introduction}
\subsection{Energy Markets}

\frame{\frametitle{Liberalisation}
The German Electricity market went into Liberalization in April 1998.\\
The Pre - Liberalisation system was based on calculatory costs: the price was according to the 'cost-plus' rule
\begin{itemize}
\item<1-> Integrated value-chain: production, grid, distribution
\item<2-> Electricity production to secure supply within a regional monopole
\item<3-> Long-term supply contracts
\item<4-> No liquid market on the whole sale market
\item<5-> Regulated consumer prices, regulated investments
\end{itemize}
}
\frame{\frametitle{Liberalisation}
Post - Liberalisation system based on forces of market: higher volatility of prices, flexibility has value.
\begin{itemize}
\item<1-> Unbundling of value-chain
\item<2-> Power plants are used optimally -- no obligation to secure supply
\item<3-> New players and products
\item<4-> Trading in Long- and Short-positions on a liquid whole sale market
\item<5-> Investments based on market expectations
\end{itemize}

}
\frame{\frametitle{Markets}

Since the deregulation of electricity markets in the end of the
1990s, power can be traded at exchanges like the Nordpool, http://www.nordpoolspot.com/  or the
European Energy Exchange (EEX), http://www.eex.com/en. All exchanges have established
spot and futures markets.


}

\frame{\frametitle{Spot prices}
\begin{center}
\includegraphics[heigth=0.9 \textheight, width=0.9 \textwidth]{../../../pics/phelixBase2002_12.pdf}
\end{center}


}

\frame{\frametitle{Spot prices}
\begin{center}
\includegraphics[heigth=0.9 \textheight, width=0.9 \textwidth]{../../../pics/phelixBase2002_08.pdf}
\end{center}


}
\frame{\frametitle{Spot prices}
\begin{center}
\includegraphics[heigth=0.9 \textheight, width=0.9 \textwidth]{../../../pics/phelixBase2008_12.pdf}
\end{center}


}


\subsection{Options, Forwards and Swaps}
\frame{\frametitle{ Derivative Background}

A derivative security, or contingent claim, is a financial
contract whose value at expiration date $T$ (more briefly, expiry)
is determined exactly \index{contingent claim} by the price (or
prices within a prespecified time-interval) of the underlying
financial assets (or instruments) at time $T$ (within the time
interval $[0,T]$).


Derivative securities can be grouped under three general headings:
{\it Options, Forwards and Futures} and {\it Swaps}. During this
lectures we will encounter all this structures and further variants.

}

\frame{\frametitle{Underlying Securities}

\begin{itemize}
\item<1-> We will mainly use Commodities or Commodity Futures;
\item<2-> Fixed income instruments: T-Bonds, Interest Rates (LIBOR, EURIBOR);
\item<3-> Other classes are possible: (one or several) Stocks; Currencies (FX);
\item<4-> Also Derivatives may be used as underlying for compound derivatives (call on call).
\end{itemize}

}


\frame{\frametitle{ Modelling Assumptions (Financial Markets)}

We impose the following set of assumptions on the financial
markets:
\begin{itemize}
\item<1-> {\it No market frictions: } No transaction costs, no bid/ask
spread, no taxes,
 no margin requirements, no restrictions on short sales.
\item<2-> {\it No default risk:} Implying same interest for borrowing
and lending. \item<3-> {\it Competitive markets:}  Market participants
act as price takers. \item<4-> {\it Rational agents:} Market
participants prefer more to less.
\end{itemize}
}

\frame{\frametitle{Arbitrage}

The concept of arbitrage lies at
the centre of the relative pricing theory. All we need to assume additionally is
that economic agents prefer
more to less, or more precisely, an increase in consumption
without any costs will always be accepted.

The essence of the technical sense of arbitrage is that it should
not be possible to guarantee a profit without exposure to risk.
Were it is possible to do so, arbitrageurs (we use the French
spelling, as is customary) would do so, in unlimited quantity,
using the market as a \lq {money-pump}' to extract arbitrarily
large quantities of riskless profit.

{\it We assume that arbitrage opportunities do not exist!} }


\frame{\frametitle{ Options}

An option is a financial instrument giving one the {\it right but
not the obligation} to make a specified transaction at (or by) a
specified date at a specified price. {\it Call} options give one
the right to buy. {\it Put} options give one the right to sell.
{\it European} options give one the right to buy/sell on the
specified date, the expiry date, on which the option expires or
matures.

{\it American} options give one the right to buy/sell at any time
prior to or at expiry.


}

\frame{\frametitle{ Options}

The simplest call and put options are now so standard that they are
called {\it vanilla} options.

Many kinds of options now exist, including so-called {\it exotic}
options.  Types include: {\it Asian} options, which depend on the
{\it average} price over a period, {\it lookback} options, which
depend on the  {\it maximum} or {\it minimum} price over a period
and {\it barrier} options, which depend on some price level being
attained or not. }

\frame{\frametitle{ Terminology}

The asset to which the option refers is called the {\it underlying
asset} or the {\it underlying}. The price at which the transaction
to buy/sell the underlying, on/by the expiry date (if exercised),
is made, is called the {\it exercise price} or {\it strike price}.
We shall usually use $K$ for the strike price, time $t = 0$ for
the initial time (when the contract between the buyer and the
seller of the option is struck), time $t = T$ for the expiry or
final time.

Consider, say, a European call option, with strike price $K$;
write $S(t)$ for the value (or price) of the underlying at time
$t$.  If $S(t) > K$, the option is {\it in the money}, if $S(t) =
K$, the option is said to be {\it at the money} and if $S(t) < K$,
the option is {\it out of the money}.

}

\frame{\frametitle{ Payoff}

The payoff from the option is $$ S(T) - K \mbox{ if } S(T)
> K\A \mbox{ and }\A 0 \;\; \mbox{otherwise} $$ (more briefly
written as  $(S(T) - K)^+$).


Taking into account the initial payment of an investor one obtains
the profit diagram below.

%}
}

\frame{\frametitle{ Payoff}

%\frame{
Profit diagram for a European call
\begin{figure}\label{payoffeurocall}
\unitlength1cm \thicklines
\begin{picture}(10,7)
\put(1,2){\vector(1,0){7}} \put(8,1.5){$S(T)$} \put(4,2){$K$}
\put(2,1){\vector(0,1){5}} \put(1.4,6.5){profit}
\put(4,1.5){\line(1,1){4}} \put(2,1.5){\line(1,0){2}}
\end{picture}
\caption{Profit diagram for a European call}
\end{figure}

}

\frame{\frametitle{Example: Options}

A trader purchases a European call option maturing in 6 month for 100 Barrels crude oil with strike 82 USD/Barrel. He pays a premium of 2 USD/Barrel.\\
If the oil price rises to 87 USD/Barrel at maturity of the option, then the trader can exercise the call and buy 100 Barrels of crude oil for 82 USD/Barrel and sell them at 87 USD/Barrel in the market, making a profit of 300 USD.\\
If, however, the price of crude oil at maturity drops to 81 USD/Barrel below the strike price, then the trader would not exercise the option, making a loss (limited to the cost of the call premium) of 200 USD.

}





\frame{\frametitle{Arbitrage Relationship- Example}

We now use the principle of no-arbitrage to obtain bounds for
option prices. We focus on
European options (puts and calls) with identical underlying (say a
stock $S$), strike $K$ and expiry date $T$. Furthermore we assume
the existence of a risk-free bank account (bond) with constant
interest rate $r$ (continuously compounded) during the time
interval $[0,T]$. We start with a fundamental relationship:


We have the following  put-call parity between the prices of the
underlying asset $S$ and European call and put options on stocks
that pay no dividends:
\begin{equation}\label{Europutcall}
S_t + P_t - C_t = K e^{-r(T-t)}.
\end{equation}

}

\frame{\frametitle{Arbitrage Relationship - Example}


Consider a portfolio consisting of one stock, one put
and a short position in one call (the holder of the portfolio has
written the call); write $V(t)$ for the value of this portfolio.
Then
$$
V(t) = S(t) + P(t) - C(t)
$$
for all $t \in [0,T]$. At expiry we have
$$\begin{array}{lll}
V(T)&=&S(T)+(S(T)-K)^--(S(T)-K)^+\\*[12pt]
&=&S(T)+K-S(T)=K.
\end{array}
$$
This portfolio thus guarantees a payoff $K$ at time $T$. Using the
principle of no-arbitrage, the value of the portfolio must at any
time $t$ correspond to the value of a sure payoff $K$ at $T$, that
is $V(t)=K e^{-r(T-t)}$. \hfill \eb

}

\frame{\frametitle{European Call Price}

For a European call $X = (S(T)-K)^+$ and  we can evaluate the
above expected value

The Black-Scholes price
pro\-cess of a European call is given by
$$
\begin{array}{lll}
C(t) &=&\DSE S(t) \Phi(d_1(S(t), T-t))\\*[12pt]
&&- K e^{-r(T-t)} \Phi(d_2(S(t), T-t)).
\end{array}
$$
The functions $d_1(s,t)$ and $d_2(s,t)$ are given by
$$
\begin{array}{lll}
d_1(s,t) &=&\DSE \frac{\log(s/K) + (r +
\frac{\sigma^2}{2})t}{\sigma \sqrt{t}},\\*[12pt] d_2(s,t) &=&\DSE
 \frac{\log(s/K) + (r -
\frac{\sigma^2}{2})t}{\sigma \sqrt{t}}
\end{array}
$$
}



\frame{\frametitle{Forwards and Futures}

\begin{itemize}
\item<1->
A {\it forward contract}
is an agreement to buy or sell an asset $S$ at a certain future
date $T$ for a certain price $K$.
\item<2->
The agent who agrees to
buy the underlying asset is said to have a {\it long} position,
the other agent assumes a {\it short} position.
\item<3-> The settlement
date is called {\it delivery date} and the specified price is
referred to as {\it delivery price}.
\end{itemize}
}

\frame{\frametitle{Forwards}
\begin{itemize}
\item<1-> The {\it forward
price} $F(t,T)$ is the delivery price which would make the
contract have zero value at time $t$.
\item<2-> At the time the contract is set up, $t=0$,
the forward price therefore equals the delivery price, hence
$F(0,T) = K$.
\item<3->
The forward prices $F(t,T)$ need not (and will not)
necessarily be equal to the delivery price $K$ during the
life-time of the contract.
\end{itemize}
}

\frame{\frametitle{Forwards}
\begin{itemize}
\item<1->
The payoff from a long position in a forward contract on one unit
of an asset with price $S(T)$ at the maturity of the contract is
$$ S(T)-K.$$
\item<2-> Compared with a call option with the same maturity
and strike price $K$ we see that the investor now faces a downside
risk, too. He has the obligation to buy the asset for price $K$.
\end{itemize}
}
\frame{\frametitle{Futures}
\begin{itemize}
\item<1-> Futures can be defined as standardized forward contracts traded at exchanges where a clearing house acts as a central counterparty for all transactions.
\item<2-> Usually an initial margin is paid as a guarantee.
\item<3-> Each trading day a settlement price is determined and gains or losses are immediately realized at a margin account.
\item<4-> Thus credit risk is eliminated, but there is exposure to interest rate risk.

\end{itemize}
}

\frame{\frametitle{Black's Formula}

We use the same notation -
strike $K$, expiry $T$ as in the spot case, and write $\Phi$ for the
standard normal distribution function.
{\it
The arbitrage price $C$ of a European futures call option is
$$
C(t)= c(f(t), T-t),
$$
where $c(f,t)$ is given by Black's futures options formula:
$$
c(f,t) := e^{-rt} (f \Phi(\tilde{d}_1 (f,t)) - K \Phi(\tilde{d}_2 (f,t))),
$$
where
$$
\tilde{d}_{1,2} (f,t) := \frac{\log (f/K) \pm \frac{1}{2} {\sigma}^2 t}{
\sigma \sqrt{t}}.
$$
}
}





\frame{\frametitle{Swaps}

A {\it swap} is an agreement whereby two parties
undertake to exchange, at known dates in the future, various
financial assets (or cash flows) according to a prearranged
formula that depends on the value of one or more underlying
assets. Examples are currency swaps (exchange currencies) and
interest-rate swaps (exchange of fixed for floating set of
interest payments).
}

\frame{\frametitle{Spread Options}

Spread options can be used by owners of corresponding plants to
manage market risk.

The pay off of a typical spread is
$$C_{\mbox{spread}}^{(T)}=\max(S_1(T)-S_2(T)-K,0)$$ with $S_i$ the
underlyings, $K$ the strike.
}



\frame{\frametitle{Spread Options}

For $K=0$ (exchange option) there is an analytic formula due to
Margrabe (1978).
$$\begin{array}{lll}
 C_{\mbox{spread}}(t) & = & e^{-r(T-t)}(S_1(t)\Phi(d_1)-S_2(t)\Phi(d_2))
 \\*[12pt]
 %P_{\mbox{spread}}(t) & = & e^{-r(T-t)}(S_2(t)\Phi(-d_2)-S_1(t)\Phi(-d_1))
 %\\*[12pt]
 \mbox{where}\quad d_1 & = & \frac{\log(S_1(t)/S_2(t))+\sigma^{2}(T-t)/2}{\sqrt{\sigma^{2}(T-t)}}\quad ,d_2=d_1-\sqrt{\sigma^{2}(T-t)}
 \\*[12pt]
 \mbox{and}\quad \sigma & = & \sqrt{\sigma_1^2-2\rho\sigma_1\sigma_2+\sigma_2^2}
\end{array}$$
where $\rho$ is the correlation between the two underlyings.

For $K\neq 0$ no easy analytic formula is available.

}

