\documentclass[11pt,a4paper,titlepage]{article}
\usepackage{a4,latexsym,amsfonts}
\usepackage{epsfig}
\usepackage{amsmath}
\usepackage{amssymb}
\usepackage{eurosym}
\usepackage{rotating}

%Silbentrennung nach neuer deutscher Rechtschreibung
\usepackage[ngerman]{babel} % Neue Rechtschreibung

%TimesNewRoman Schrift,
%\usepackage{times}

%Zum Einbinden von Grafiken:
\usepackage{graphicx}

\pagestyle{headings}
\pagestyle{plain}

%F|r TeX Zeichen in beliebigen eingebundenen Grafiken:
%\usepackage{psfrag}
\setlength{\textheight}{26.0cm}
\setlength{\textwidth}{15.5cm}

\setlength{\parskip}{6pt}
\setlength{\parindent}{0pt}
\setlength{\oddsidemargin}{-0.0cm}
\setlength{\topmargin}{-1.5cm}
\parindent 0.0cm
\sloppy
\frenchspacing

\newcommand{\Var}{\text{Var}}
\newcommand{\sF}{\mathcal{F}}
\newcommand{\sH}{\mathcal{H}}
\newcommand{\sB}{\mathcal{B}}
\newcommand{\sD}{\mathcal{D}}
\newcommand{\sN}{\mathcal{N}}
\newcommand{\sO}{\mathcal{O}}
\newcommand{\mR}{\mathbb{R}}     % blackboard bold R
\newcommand{\mN}{\mathbb{N}}     % blackboard bold N
\newcommand{\mNO}{\mathbb{N}_0}
\newcommand{\mP}{\rm I\hspace{-0.7mm}P}
\newcommand{\mD}{\mathbb{D}}
\newcommand{\mF}{\mathbb{F}}
\newcommand{\mZ}{\mathbb{Z}}
\newcommand{\mQ}{\mathbb{Q}}
\newcommand{\bD}{\mathbf{D}}
\newcommand{\mE}{I\!\!E}
\newcommand{\si}{{(i)}}
\newcommand{\coloneqq}{\mathrel{\mathop:}=}
\newcommand{\eqqcolon}{=\mathrel{\mathop:}}
\newcommand{\norm}[1]{\left\lVert #1 \right\rVert}
\newcommand{\ip}[2]{\left< #1 , #2 \right>}
\newcommand{\Dom}{\text{Dom }}
\newcommand{\indicator}{\mathbf{1}}
\newcommand{\wh}[1]{\widehat{#1}}
% \newcommand{\cn}[1]{\citeasnoun{#1}}

\newcommand{\ind}{\hspace{0.2in}}
\newcommand{\ftext}[1]{\text{\ind\footnotesize #1}}
\newcommand{\comments}[1]{}

\newcommand{\answerbox}[1]{\\ ~\\
\renewcommand{\arraystretch}{#1}
\begin{tabular}{|p{12cm}|p{1cm}|}
 \hline
 &  \\
 \hline
\end{tabular} \\ }
\renewcommand{\answerbox}[1]{}

\makeatletter
 \def\Links{\tagsleft@true}\def\Rechts{\tagsleft@false}
\makeatother

\usepackage[latin1]{inputenc} % F�r Umlaute

%==============================================================================
\begin{document}

\setlength{\topmargin}{-2.5cm}

\parbox[t]{\textwidth}
%{ \includegraphics[height=0.8cm]{unilogo.png} }

        \bf Lehrstuhl f\"ur Energiehandel und Finanzdienstleistungen\\
        \sf Prof. Dr. R�diger Kiesel \\
		\sf Sascha Kollenberg  \\

\hfill
\parbox[t]{39mm}
          {
      \flushright
   \vspace{-2.35cm}
      \small
           \bf SS 2014\\
           \bf }
\hfill

\begin{center} {\LARGE \bf Einf�hrung in Optionen, Futures und derivative Finanzinstrumente (F\&O)} \\[5mm]
                {\large Klausur}
\end{center}

\vspace{0.3cm}

%\pagestyle{empty}

%-----------hier-beginnen-die-Aufgaben----------------------

%\newpage
%\begin{flushleft}
%\includegraphics[height=20cm]{TabelleStandNorm.pdf}
%\end{flushleft}
%\newpage



% begin enumerate



%%%%%%%%%%%%%%%%%%%%%%%%%%%%%%%%%%%%%%%%%%%%%%%%%%%%%%%%%%%%%%%%%%%%%%%%%%%%%%%%%
\vspace{0.3cm}

%\renewcommand{\arraystretch}{4}
%\begin{tabular}{|p{12cm}|p{1cm}|}
% \hline
% &  \\
% \hline
%\end{tabular} \\
%\renewcommand{\arraystretch}{10}
%\begin{tabular}{|p{12cm}|p{1cm}|}
% \hline
% &  \\
% \hline
%\end{tabular} \\



	{\bf Aufgabe 1 (Strategien) (20 Punkte)}
Eine H�ndlerin kaufe einen sogenannten Bear-Call-Spread, dessen Profit-Diagramm unten aufgetragen ist. Dabei ergibt sich der Profit aus dem Payoff abz�glich dem Preis des Portfolios. Diese Optionsstrategie wird mittels zwei europ�ischer Call-Optionen konstruiert. Seien $K_1$ und $K_2$ die beiden Strike-Preise und es gelte $K_1<K_2$.
%\begin{figure}[H]
\begin{center}
\includegraphics[scale=0.3]{bear.pdf}
\end{center}
%\end{figure}
\vspace*{0.1cm}



% begin enumerate





	Wie sieht das Portfolio f�r einen Bear-Call-Spread aus? Geben Sie auch die Werte der jeweiligen Strike-Preise der Optionen, sowie den Payoff jeder Einzelposition an. \answerbox{4}


	Was ist offensichtlich die Erwartung der H�ndlerin �ber den zuk�nftigen Verlauf des Preises dieses Underlyings, sofern sie ihren erwarteten Payoff maximieren m�chte? \answerbox{2}


	Wie wird ein Bull-Call-Spread konstruiert? \answerbox{2}


	Skizzieren Sie das \emph{Payoff}-Diagramm eines Bull-Spreads in obige Grafik.



% end enumerate



\newpage



	{\bf Aufgabe 2 (Arbitrage) (30 Punkte)} Begr�nden Sie Ihre Antworten f�r folgende Fragen jeweils mittels Argumentation �ber Arbitrage. Verwenden Sie hierzu Arbitragetabellen. Wir nehmen stetige Verzinsung mit konstantem Zinsatz $r$ an. Derivatepreise $\hat X(S,K,T)$ h�ngen vom Underlying $S$, dem Strike $K$ und der Maturit�t $T$ ab.
Zeitangaben sind in Jahren. Verwenden Sie Arbitragetabellen in folgendem Format: \\~\\
\begin{tabular}{|c|c|c||c|c|c|c|}
\hline
$t$ & Positionswert & Cashflow & $T$ & Payoff im Fall $1$ & Payoff im Fall $2$ & ... \\
\hline
Position 1 &&&&&& \\
\hline
Position 2 &&&&&& \\
\hline
... &&&&&& \\
\hline
$\sum$ &&&&&& \\
\hline
\end{tabular}
\vspace{0.5cm}



% begin enumerate





	Was ist der Gegenwartswert (Wert zum Zeitpunkt $t=0$) einer Zahlung in H�he von $K$ zum Zeitpunkt $T$? \answerbox{2}


	Nehmen Sie Arbitragefreiheit an. Leiten Sie die Put-Call-Parit�t her:
 \begin{equation}
  S(0) + \hat P_0(S,K,T) - \hat C_0(S,K,T) = K e^{-rT} \nonumber
\end{equation}  Verwenden Sie hierzu eine Arbitragetabelle. \answerbox{30}


	Man beobachte zum Zeitpunkt $0$ am Markt folgende Preise f�r europ�ische Calls:
\begin{equation}
 \quad \hat C_0(S,50,1)=2 \quad \hat C_0(S,50,2)=1 \nonumber
\end{equation}
Sind diese Preise mit der Annahme der Arbitragefreiheit vereinbar? Verwenden Sie eine bekannte Preisrelation. \answerbox{8}


	Betrachten Sie zwei europ�ische Calls mit identischer Maturit�t auf das gleiche Underlying jedoch zu unterschiedlichen Strikes $K_1<K_2$. Nehmen Sie Arbitragefreiheit an. Welcher der beiden Calls ist teurer? Begr�nden Sie Ihre Antwort mittels Arbitragetabelle. \answerbox{34}
% \newpage


	Ist der Wert eines Bull-Call-Spreads positiv oder negativ? Beweisen Sie Ihre Antwort mittels Arbitragetabelle. \answerbox{22}


	Es gelte $r=\ln(1,5)$. Man beobachte nun zum Zeitpunkt $0$ am Markt folgende Preise:
 \begin{align}
  \quad S_0=20 \quad \hat P_0(S,30,1)=3 \quad \hat C_0(S,30,1)=2 \nonumber
\end{align}
Sind diese Preise mit der Annahme der Arbitragefreiheit vereinbar?  Beweisen Sie ihre Antwort mittels einer bekannten Preisrelation. \answerbox{20}



% end enumerate



\vspace*{1cm}

\newpage



	{\bf Aufgabe 3 (Diskrete Modelle) (15 Punkte)}
Sei $S_t$ der Wert eines Underlyings zum Zeitpunkt $t$, wobei $t$ nur Werte in der diskreten Menge $\{1,...,N\}$ annehmen kann. Der jetzige Zeitpunkt sei $t=1$. Wir nehmen diskrete Verzinsung zum Zinssatz $r$ an.
Nehmen Sie zun�chst an, $S_{t+1}=(1+u)S_t$ mit Wahrscheinlichkeit $q$ und $S_{t+1}=(1+d)S_t$ mit Wahrscheinlichkeit $(1-q)$.



% begin enumerate





	Geben Sie den erwarteten Wert des Underlyings zum Zeitpunkt $2$ an. \answerbox{3}


	Berechnen Sie die arbitragefreien Wahrscheinlichkeiten $q$ und $1-q$. Begr�nden Sie Ihre Berechnung mittels Arbitragetabelle. \answerbox{40}


	Zeichnen Sie den Binomialbaum f�r $N=3$. \answerbox{20}


	Wie viele Pfade f�hren zum Preis $S_3=(1+u)(1+d)S_1$? \answerbox{4}


	Berechnen Sie den erwarteten Preis zum Zeitpunkt $3$.  \answerbox{4}
 %

	Sei $N$ wieder beliebig. Wie viele Pfade f�hren zum Preis $S_t=(1+u)^{t-2}(1+d)S_1$?


	Wie viele Pfade f�hren zum Preis $S_{n+1}=(1+u)^k(1+d)^{n-k}S_1$ mit $k<n<N$? \answerbox{4}


	Geben Sie eine Formel f�r den Erwartungswert $\mathbb E[S_{5}]$ an. \answerbox{10}



% end enumerate



\newpage



	{\bf Aufgabe 4 (Black-Scholes) (25 Punkte)}
Die ber�hmte Black-Scholes Formel f�r einen europ�ischen Call ist
\begin{align*}
C(t) = S(t) \Phi(d_1) - K e^{-r(T-t)} \Phi(d_2)
\end{align*}
wobei
\begin{align*}
d_1 = \frac{\log(\tfrac{S(t)}{K}) + (r + \tfrac{\sigma^2}{2}) (T-t)}{\sigma \sqrt{T-t}} ~~ , ~~ d_2 = \frac{\log(\tfrac{S(t)}{K}) + (r - \tfrac{\sigma^2}{2}) (T-t)}{\sigma \sqrt{T-t}} = d_1 - \sigma \sqrt{T-t}
\end{align*}
und $\Phi(\cdot)$ die Verteilungsfunktion der Standardnormalverteilung bezeichne.
F�r den Preis des Underlyings gilt (unter $\mathbb Q$)
\begin{equation}
 S_T=S_t \exp\left(\sigma \sqrt{T-t} X + (r-\frac{\sigma^2}{2})(T-t)\right), \quad X\sim\mathcal N(0,1), \quad S_t\geq 0. \nonumber
\end{equation}
Die Dichte der Standardnormalverteilung ist gegeben durch $f(y)=\frac{1}{\sqrt{2\pi}}e^{-\frac{y^2}{2}}$. Es gilt die Formel $S_t f(d_1) = K e^{-r(T-t)} f(d_2)$, welche unbewiesen verwendet werden darf.



% begin enumerate





	Leiten Sie eine Formel zur Berechnung des Rho ($\rho$) eines europ�ischen Calls her.  \answerbox{35}


	Steigt oder sinkt der Preis $C(t)$ eines europ�ischen Calls mit steigendem Zinssatz $r$? Beweisen Sie ihre Behauptung mathematisch. \answerbox{14}
%

	Leiten Sie eine Formel f�r die Berechnung des Vega eines europ�ischen Calls her. \answerbox{35}
%

	Steigt oder sinkt der Preis $C(t)$ des Calls bei steigender Volatilit�t? Beweisen Sie ihre Behauptung mathematisch \emph{oder} geben Sie eine knappe �konomische Begr�ndung. \answerbox{22}


	Steigt oder sinkt der Preis $S_T$ des Underlyings zum Zeitpunkt $T$ bei steigendem Zinssatz? Beweisen Sie ihre Behauptung mathematisch \emph{oder} geben Sie eine knappe �konomische Begr�ndung. \answerbox{22}
%

	Eine Bank verkaufe einen europ�ischen Call. Der momentane Kurs sei $S_t = 100$ und es werde eine Volatilit�t von $35\%$ angenommen. Der Kunde w�nsche den Strike $K = 110$ und Maturit�t in $24$ Monaten. Der risikofreie Zinssatz p.a. sei $r = 1.5\%$. \\
%Um sich abzusichern m�chte die Bank f�r die Option einen Delta-Hedge aufsetzen. Stellen Sie in $t=0$ das Delta-Portfolio zusammen.



% end enumerate





% end enumerate



%\newpage ~

\end{document}
