% !TEX root = FuturesOptions_ss14UDE.tex
%\section{Literature}

% frametitle
{}
\vspace{0cm}
\begin{center}
\color{beamer@blendedblue}{\large{Literatur}}
\end{center}
\end{frame}

% frametitle
{Literature}


% begin itemize




	\textbf{Main Reference:}


% begin itemize




	Hull, J. (2009). Options, Futures, and Other Derivatives, Pearson.
%

	Burger, M., B. Graeber, et al. (2008). Managing Energy Risk: An Integrated View on Power and Other Energy Markets, John Wiley \& Sons.


% end itemize




% end itemize



\section{Introduction -- The History of Derivatives}

% frametitle
{}
\vspace{0cm}
\begin{center}
\color{beamer@blendedblue}{\large{The History of Derivatives}}
\end{center}
\end{frame}

% frametitle
{Derivatives Quotes}
"We view them as time bombs both for the parties that deal in them and the economic system ... In our view ... derivatives are financial weapons of mass destruction, carrying dangers that, while now latent, are potentially lethal."
\begin{flushright}
  \begin{footnotesize}
  Warren Buffett in his Chairman's Letter in the Berkshire Hathaway 2002 Annual Report
 \end{footnotesize}
\end{flushright}

% frametitle
{Derivatives Quotes}
"The use of a growing array of derivatives and the related application of more-sophisticated approaches to measuring and managing risk are key factors underpinning the greater resili-ence of our largest financial institutions .... Derivatives have permitted the unbundling of financial risks."
\begin{flushright}
  \begin{footnotesize}
  Alan Greenspan, May 2005
 \end{footnotesize}
\end{flushright}

% frametitle
{Derivatives Quotes}
"Derivatives don't kill people, people kill people."
\begin{flushright}
  \begin{footnotesize}
 Clifford W. Smith, Jr., Professor, University of Rochester - Risk, March, 1994, p. 6
 \end{footnotesize}
\end{flushright}

% frametitle
{Derivatives Quotes}
"Blaming derivatives for financial losses is akin to blaming cars for drunk driving fatalities."
\begin{flushright}
  \begin{footnotesize}
 Christopher L. Culp; MediaNomics, April, 1995, p. 4
 \end{footnotesize}
\end{flushright}

% frametitle
{Derivatives Quotes}
"'Derivatives.' That's the 11-letter four-letter word."
\begin{flushright}
  \begin{footnotesize}
  Richard Syron, Chairman, American Stock Exchange; Fortune, March 20, 1995, p. 50
 \end{footnotesize}
\end{flushright}

% frametitle
{History of Derivatives}
\textbf{Late 1960s - Black-Scholes Formula}\\


% begin itemize




	Fischer Black and Myron Scholes tackle the problem of determining how much an option is worth. Robert Merton joins them in 1970.


% end itemize



\textbf{1993 - Metallgesellschaft bankrupt}


% begin itemize




	Metallgesellschaft AG, formely one of Germany's largest industrial conglomerates with over 20.000 employees, looses 1.3 billion dollars after speculating for a rise in the futures market. Oil prices dropped and left the company buying oil at a price substantially over the market price.


% end itemize



% frametitle
{History of Derivatives (2)}
\textbf{1995 - Barings Bank Disaster }


% begin itemize




	Nick Leeson loses $\$$1.4 billion, speculating that the Nikkei 225 index of leading Japanese company shares would not move materially from its normal trading range (short straddle). That assumption was shattered by the Kobe earthquake on the 17th January 1995 after which Leeson attempted to conceal his losses, which reached twice the bank's available trading capital. Barings was declared insolvent the 26th Feburary 1995


% end itemize


\textbf{1997 - Nobel Prize in Economics awarded to Robert Merton and Myron Scholes}


% begin itemize




	"for a new method to determine the value of derivatives."


% end itemize



% frametitle
{History of Derivatives (3)}
\textbf{1998 - Long Term Credit Management Bailout}


% begin itemize




	The hedge fund, among the board of directors Myron Samuel Scholes und Robert C. Merton, needs to be rescued at a cost of $\$$3.5 billion. The fund had speculated on spreads of govern-ment bonds, i.e. the interest spread between 30 and 10 years US treasury bonds.


% end itemize


\textbf{2001 - Enron goes Bankrupt}


% begin itemize




	The 7th largest company in the US and the world's largest energy trader made extensive use of energy and credit derivatives but becomes the biggest firm to go bankrupt in American history after systematically attempting to conceal huge losses.


% end itemize



% frametitle
{History of Derivatives (4)}
\textbf{2006 - Amaranth Advisors}


% begin itemize




	The multi strategy hedge fund loses more than $\$$ 6 billion from trading in natural gas futures. The hedge fund collapses.


% end itemize



% frametitle
{History of Derivatives(5)}
\textbf{2008 - Societe Generale trading scandal}


% begin itemize




	Jerume Kerviel, a French trader from Societe Generale , causes a loss of EUR 4.9 billion. Kerviel was assigned to arbitrage discrepancies between equity derivatives and cash equity prices, but he manipulated risk management systems and exceeded his authority to engage in un-authorized trades totaling as much as EUR 49.9 billion, which is higher than the bank's total market capitalization. When Societe Generale  tried to close out open positions built up by Kerviel, European stock markets suffered heavy losses of about 6 $\%$.


% end itemize


\textbf{2011 - UBS trading scandal}


% begin itemize




	A UBS trader causes losses in a dimension of EUR 2 billion through unauthorized trading of stock index futures.


% end itemize



% frametitle
{Financial Derivatives Timeline (6)}
\textbf{2012 - Whale of London}


% begin itemize




	Bruno Michael Iksli, nicknamed the London Whale, causes $\$$ 2 billion trading losses for JPMorgan with aggressive trading.


% end itemize


\textbf{2012 - Libor Manipulation}


% begin itemize




	In a series of fraudulent actions, the Libor (London Interbank Offered Rate), which underpins approximately $\$$350 trillion in derivatives, is manipulated by several parties.


% end itemize



