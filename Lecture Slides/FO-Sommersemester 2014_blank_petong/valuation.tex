% !TEX root = FuturesOptions_ss14UDE.tex
\section{Arbitrage and Valuation}

% frametitle
{}
\vspace{0cm}
\begin{center}
\color{beamer@blendedblue}{\large{Arbitrage and Valuation}}
\end{center}
\end{frame}

\subsection{Arbitrage Relations}
\subsection{Valuation Principles}

% frametitle
{Valuation by Replication}


% begin itemize




	We now want to find the fair price of a derivative, i.e. we want to find the option premium
\vspace{0.3cm}


	We consider a one-period model, i.e. we allow trading only at
$t=0$ and $t=T$(say).


	Our aim is to value at $t=0$  a European
derivative on a stock $S$ with maturity $T$.


% end itemize



% frametitle
{Valuation by Replication - Idea}


% begin itemize




	If it is possible to duplicate the payoff $H$ of a derivative
using a portfolio $V$ of underlying (basic) securities, the price of the
portfolio at $t=0$ must equal the price of the derivative at
$t=0$.


% end itemize



% frametitle
{Valuation by Replication - Example (1)}


% begin itemize




	We have a stock $S$ with $S(0)=20$ and a European Call option with strike $K=21$ and maturity $T$.


	In $T$, there are two possible states: an up- and a down state


% end itemize


%\begin{center}
%\includegraphics[height=3.5cm]{../../../pics/oneperiodtree1.pdf}
%\end{center}


% begin itemize




	up: $20 (1+u)=22 \Rightarrow u=0.1$


	down: $20 (1+d)=18 \Rightarrow d=-0.1$


% end itemize



% frametitle
{Valuation by Replication - Example (2)}


% begin itemize




	The key idea now is to try to find a portfolio combining bond and
stock, which synthesizes the cash flow of the option.


	If such a portfolio exists, holding this portfolio today would be equivalent
to holding the option -- they would produce the same cash flow in
the future.


	Therefore the price of the option $C(0)$ should be the same
as the price of constructing the portfolio $V(0)$, otherwise investors
could just restructure their holdings in the assets and obtain a
riskfree profit today.


% end itemize



% frametitle
{Valuation by Replication - Example (3)}
We construct the portfolio from


% begin itemize




	a riskfree bond (bank account) with $B(0)=1$ and $B(T)=B(t) \cdot (1+r) = 1.05$,
that is the interest rate $r=0.05$


	a the risky stock $S$ with $S(0) = 20$ and two
possible values at $t=T$,


% begin itemize




	$S_1(u)=22$


	$S_1(d)=18$


% end itemize




% end itemize



% frametitle
{Valuation by Replication - Example (4)}


% begin itemize




	we invest $\theta_0$ in the bank account and buy


	$\theta_1$ stocks


	the value of our portfolio today is given as
  % begin itemize


	$V(0) = \theta_0 + \theta_1 \cdot S(0)$.
  % end itemize


% end itemize



% frametitle
{Valuation by Replication - Example (5)}


% begin itemize




	at expiry date $t=T$, the value of the portfolio depends on the value of the stock $S(T)$, which can be 18 or 22 $\$$.
  % begin itemize


	$V(T) = \theta_0 \cdot (1+r) + \theta_1 \cdot S(T)$
  % end itemize


	the Payoff from the call option in $t=T$ is given with 1 or 0.
 % end itemize

% frametitle
{Valuation by Replication - Example (6)}


% begin itemize




	We can reproduce the call-option with the stock and the bank account if we find the values $\theta_0$ and $\theta_1$ which replicate the payoff from the call-option.


	The problem is a linear system with two equations and two unknown variables
    % begin itemize


	$\theta_0 \cdot 1.05 + \theta_1 \cdot 22 = 1$


	$\theta_0 \cdot 1.05 + \theta_1 \cdot 18 = 0$.
    % end itemize


	The solution of this equations is $\theta_0 = -4.286$ and $\theta_1 = 0.25$



% end itemize



% frametitle
{Valuation by Replication - Example (7)}


% begin itemize




	To replicate the option payoff in $t$, we need $-4.286+0.25 \cdot 20 = 0.714 \$$. This is the value of the call-option in $t=0$, where $C(0)=V(0)$


	V(0) is called the no-arbitrage price. Every other price allows a
riskless profit, since if the option is too cheap, buy it and
finance yourself by selling short the above portfolio (i.e. sell the
portfolio without possessing it and promise to deliver it at time
T = 1) this is riskfree because you own the option. If on the
other hand the option is too dear, write it (i.e. sell it in the
market) and cover yourself by setting up the above portfolio.


% end itemize



% frametitle
{No-Arbitrage and Expectation (1)}


% begin itemize




	Let us now assume the probability of the stock rising is $p$, the probability of the stock falling is $1-p$


	The probability $p$ is unkown to the market because the future development of the stock is unknown
  \vspace{0.3cm}


	Still, the market has a perception of the probability and it will  be called $q$


	And of course, under what market participants think, the market must be free of Arbitrage!


% end itemize



% frametitle
{No-Arbitrage and Expectation (2)}


% begin itemize




	This means that the following equation must hold:
  \begin{align*}
    q (1+u) S_0 + (1-q) (1+d) S_0 = S_0 (1+r)
  \end{align*}


	We solve this equation for $q$:
  \begin{align*}
    q (1+u) + (1-q) (1+d) & = 1+r \\
    q (1+u) + (1+d) -q (1+d) & = 1+r \\
    q (1+u - 1 -d) & = 1+r -1 - d\\
    q & = \frac{r-d}{u-d}
  \end{align*}


	This is the so-called Arbitrage-free Risk-neutral probability!


	In our example it takes the value $q = \frac{0.05 + 0.1}{0.1 + 0.1} = 0.75$ and $1-q=0.25$


% end itemize



% frametitle
{No-Arbitrage and Expectation - General Principle}


% begin itemize




	Using the risk-neutral probabilities we state the following general pricing principle:


	The Arbitrage-free price of a derivative $H$ on an asset $S$ in $t$ with maturity in $T$ is given by
  \begin{align*}
    H_t = \frac{B_t}{B_T} \mathbb{E}[H_T]
  \end{align*}


	This is the so-called Risk-Neutral Valuation Formula
  \vspace{0.3cm}


	We calculate the price of the example:
  \begin{align*}
    H_0 & = \frac{B_t}{B_T} \mathbb{E}[H_T] = \frac{1}{1+r} (q (1+u) H_0 + (1-q) (1+d) H_0) \\
    & = \frac{1}{1.05} (0.75 \cdot 1 + 0.25 \cdot 0) = 0.714
  \end{align*}


	The same price as before!


% end itemize



\subsection{The Cox-Ross-Rubinstein model}
\subsubsection{The Model}

% frametitle
{ The Cox-Ross-Rubinstein model}



% begin itemize




	Our model consists of two basic securities. Recall that the essence of the relative pricing theory
is to take the price processes of these basic securities as given
and price secondary securities in such a way that no arbitrage is
possible.


	Our time horizon is $T$ and the set of dates in our financial
market model is $t = 0,1, \ldots, T$. Assume that the first of our
given basic securities is a (riskless) bond or bank account $B$,
which yields a riskless rate of return $r >0$ in each time
interval $[t,t+1]$, i.e.


% begin itemize




	$B(t+1)=(1+r)B(t),\A B(0)=1.$


% end itemize




	So its price process is $ B(t) = (1+r)^t, \A t = 0,1, \ldots, T. $


% end itemize



% frametitle
{ The CRR Model}



% begin itemize




	Furthermore, we have a risky asset (stock) $S$ with price process


% begin itemize




	$$S(t+1) = \left\{
\begin{array}{lll}
(1+u) S(t) \A &\mbox{with prob}\A &q,\\
(1+d) S(t) \A &\mbox{with prob}\A &1 - q,
\end{array}
\right.
$$
with $-1 < d < u, S_0 \in {\setR}_0^+$ and
for $ t = 0,1,\ldots,T-1.$


% end itemize




% end itemize



% frametitle
{ The CRR Model}

\begin{figure}\label{one-step-tree}
\unitlength1cm \thicklines
\begin{center}
\begin{picture}(6,4)
\put(0,2){$S(0)$} \put(0.8,2.1){\line(4,1){4}} \put(2.5,3.3){$q$}
\put(4.8,3){$S(1)=(1+u)S(0)$} \put(0.8,2.1){\line(4,-1){4}}
\put(1.6,0.9){$1-q$} \put(4.8,1){$S(1)=(1+d)S(0)$}
\end{picture}
\caption{One-step tree diagram}
\end{center}
\end{figure}

% frametitle
{ The CRR Model}



% begin itemize




	This construction emphasises again that a multi-period model can
be viewed as a sequence of single-period models. Indeed, in the
Cox-Ross-Rubinstein case we use identical and independent
single-period models.


	As we will see in the sequel this will make
the construction of equivalent martingale measures relatively
easy. Unfortunately we can hardly defend the assumption of
independent and identically distributed price movements at each
time period in practical applications.


% end itemize



%\section{Derivative Pricing}

% frametitle
{Pricing in the CRR Model}



% begin itemize




	We now turn to the pricing of derivative assets in the
Cox-Ross-Rubinstein market model. To do so we first have to
discuss whether the Cox-Ross-Rubinstein model is arbitrage-free
and complete.


	To answer these questions we have, according to our fundamental
theorems, to
understand the structure of equivalent martingale measures in the
Cox-Ross-Rubinstein model. In trying to do this we use (as is
quite natural and customary) the bond price process $B(t)$ as
num\'{e}raire.


% end itemize



% frametitle
{Pricing measure}

(i) A pricing probabilities for the discounted stock price
$\tilde{S}$ exists if and only if
\begin{equation}\label{noarbcrreq}
d < r < u.
\end{equation}
(ii) If inequality (\ref{noarbcrreq}) holds true, then these probabilities are given by
\begin{equation}\label{crrprob}
q=\frac{r-d}{u-d}.
\end{equation}

% frametitle
{Pricing Formula}



% begin itemize




	We can now use the risk-neutral valuation formula to price {\it
every} contingent claim in the Cox-Ross-Rubinstein model.


	The arbitrage price process of a contingent claim $X$ in the
Cox-Ross-Rubinstein model is given by
$$
\pi_X(0) = B(t) \EX^\mathbb{Q}\left( X/B(T)\right)
$$
where $\EX^\mathbb{Q}$ is the expectation with respect to the
unique probabilities
$q= (r-d)/(u-d).$


% end itemize



% frametitle
{Pricing Formula}



% begin itemize




	We now give simple formulas for pricing (and hedging) of European
contingent claims $X=f(S_T)$ for suitable functions $f$ (in this
simple framework all functions $f:\setR \rightarrow \setR)$. We
use the notation
\begin{eqnarray}\label{BinomialF}
&& F_\tau(x,q)\\\nonumber
&:=&\DSE \sum_{j=0}^\tau {\tau \choose j} q^j (1-q)^{\tau-j}
f\left(x(1+u)^j(1+d)^{\tau-j}\right)
\end{eqnarray}


	Observe that this is just an evaluation of $f(S(j))$ along the
probability-weighted paths of the price process. Accordingly, $j$,
$\tau-j$ are the numbers of times $Z(i)$ takes the two possible
values $d, u$.


% end itemize



% frametitle
{European claims}


% begin itemize




	Consider a European contingent claim with expiry $T$ given by
$X=f(S_T)$. The arbitrage price process $\pi_X(t), \; t=0, 1,
\ldots, T$ of the contingent claim is given by (set $\tau=T-t$)
\begin{equation}\label{crrcontclaimprice}
\pi_X(t) = (1+r)^{-\tau} F_\tau(S_t,q).
\end{equation}


% end itemize



% frametitle
{European call}


% begin itemize




	Consider a European call option with expiry $T$ and strike price
$K$ written on (one share of) the stock $S$. The arbitrage price
process $\Pi_C(t), \; t=0, 1, \ldots, T$ of the option is given by
(set $\tau=T-t$)
\begin{eqnarray}\label{crrcallprice}
&& \Pi_C(t)\\\nonumber
&=&\DSE (1+r)^{-\tau} \sum_{j=0}^{\tau} {{\tau} \choose j}
{q}^j (1 - q)^{\tau-j}\\\nonumber
&&(S(t) (1+u)^j (1+d)^{\tau-j} - K)^+.
\end{eqnarray}


	For a European put option, we can either argue similarly or use
put-call parity.


% end itemize



% frametitle
{ A Three-period Example (1)}



% begin itemize




	We consider a continuous-time model with one-year risk-free interest rate (continuously compounded) $\rho=0.06$ and the volatility of the stock is 20\%, so $\s = 0.2$.


	The corresponding 3-step CRR-quantities are $\Delta =1/3$


% begin itemize




	
the up and down movements of the stock price
$$
1+u=e^{\sigma \sqrt{\Delta}} = 1.1224, \; \;
1+d =
(1+u)^{-1} = e^{-\s \sqrt{\Delta}} = 0.8910,
$$


	the discrete interest rate follows from
$$
(1+r)^3= e^\rho.
$$


	the risk-neutral
probabilities
$$
q=\frac{r-d}{u-d}= 0.5584.
$$


% end itemize




% end itemize



% frametitle
{ A Three-period Example (2)}


% begin itemize




	We assume that $S(0)=100$.
Prices of the stock and the call  with strike $K=100$ are given below.


% end itemize


\begin{figure}[hbtp]
\unitlength1cm \thicklines
\begin{center}
\begin{picture}(12,6)
\put(0,0){time $t=0$} \put(0,3.2){$S=100$} \put(0,2.8){$c=11.56 $}
\put(1.5,3){\line(3,2){1}} \put(3,3.8){$S=112.24$}
\put(3,3.4){$c=18.21$} \put(1.5,3){\line(3,-2){1}}
\put(3,2.4){$S=89.10$} \put(3,2){$c=3.67$} \put(3.5,0){$t=1$}
%%%%%%%%%top branch -step 1
\put(4.6,3.7){\line(2,1){1}} \put(6,4.5){$S=125.98$}
\put(6,4.1){$c=27.96$} \put(4.6,3.7){\line(2,-1){1}}
\put(6,3.0){$S=100$} \put(6,2.6){$c=6.70$}
%%%%%%%%%bottom branch -step 1
\put(4.6,2.3){\line(2,1){1}} \put(4.6,2.3){\line(2,-1){1}}
\put(6,1.8){$S=79.38$} \put(6,1.4){$c=0$} \put(6.5,0){$t=2$}
%%%%%%%%%bottom branch -step 2
\put(7.7,1.7){\line(2,1){1}} \put(7.7,1.7){\line(2,-1){1}}
\put(9.2,2.5){$S=89.10$} \put(9.2,2.1){$c=0$}
\put(9.2,1.3){$S=70.72$} \put(9.2,0.9){$c=0$}
%%%%%%%%%middle branch -step 2
\put(7.7,2.9){\line(2,1){1}} \put(9.2,3.7){$S=112.24$}
\put(9.2,3.3){$c=12.24$} \put(7.7,2.9){\line(2,-1){1}}
%%%%%%%%%top branch - step 2
\put(7.7,4.4){\line(2,1){1}} \put(9.2,4.9){$S=141.40$}
\put(9.2,4.5){$c=41.40$} \put(7.7,4.4){\line(2,-1){1}}
\put(9.7,0){$t=3$} \unitlength1cm \thicklines
\end{picture}
\end{center}
\caption{Stock and European call prices}
\end{figure}
%It is interesting to compare the approxinative Cox-Ross-Rubinstein
%prices $c_n$ (discrete model with $n$ time steps, see
%(\ref{crrcallprice}), (\ref{crrpricen}) to the Black-Scholes price.

% frametitle
{ A Three-period Example (3)}


% begin itemize




	To price a European call option with maturity one year ($N=3$) and
strike $K=100)$ we can either use the explicit valuation formula or work our way backwards through the tree.


	One can implement the simple evaluation formulae for the CRR- and
the BS-models and compare the values. The figure (\ref{BSCRR}) below is for
$S=100, K=90, \rho=0.06, \s=0.2, T=1$.


% end itemize



% frametitle
{Delta  of an option}


% begin itemize




	The delta $\Delta$ of an option is the ration of the change in the price of the option to the
change of the price of the underlying.


	It is the number of units of the underlying we should hold for each option short in order to
create a riskfree portfolio -- the delta hedge.


	The delta of a call is positive, whereas the delta of a put is negative.



% end itemize



% frametitle
{Example of the Delta}


% begin itemize




	
$$
\Delta_{0,1}=\frac{112.24-89.10}{18.21-3.67}=1.5915
$$


	
$$
\Delta^u_{1,2}=\frac{125.89-100}{27.96-6.70}=1.2178
$$


	
$$
\Delta^d_{1,2}=\frac{100-79.38}{6.70-0}=3.0776
$$


	Observe that the delta is time and state dependent.


% end itemize



\subsubsection{Binomial approximation}

% frametitle
{Binomial Approximations (1)}



% begin itemize




	Suppose we observe financial assets during a continuous time
period $[0,T]$.


	To construct a stochastic model of the price
processes of these assets (to, e.g. value contingent claims) one
basically has two choices:


% begin itemize




	one could model the processes as
continuous-time stochastic processes (for which the theory of
stochastic calculus is needed)


	one could construct a sequence of discrete-time models in which
the continuous-time price processes are approximated by
discrete-time stochastic processes in a suitable sense.


% end itemize




	We follow the second approach and obtain the asymptotics
 of a sequence of Cox-Ross-Rubinstein models.


% end itemize



% frametitle
{Binomial Approximations (2)}



% begin itemize




	
 With $B^{n, p}$ the Binomial cumulative distribution function of
 $\bar{B}^{n, p} = 1- B^{n, p}$, we find in the $n$th
Cox-Ross-Rubinstein model for the price of a European call at time
$t=0$ the following formula
\begin{eqnarray}\label{crrpricen}
\Pi_C^{(n)}(0) &=&\DSE S_n(0) \bar{B}^{k_n, \hat{p}_n}(a_n)\\*[12pt]\nonumber
&&\DSE- K
(1+r_n)^{-k_n} \bar{B}^{k_n, p^*_n}(a_n).
\end{eqnarray}


	We have the following limit relation:
$$
\lim_{n \ra \infty} \Pi_C^{(n)}(0) = \Pi_C^{BS}(0)
$$
with $\Pi_C^{BS}(0)$ given by the Black-Scholes formula.


% end itemize



% frametitle
{The Black-Scholes Formula}



% begin itemize




	The Black-Schole Formula for the price $\Pi_C^{BS}(0)$ of a European call  (we use
$S=S(0)$ to ease the notation) is
\begin{equation}\label{BScallprice4}
\Pi_C^{BS}(0) = S \Phi(d_1(S, T)) - K e^{-rT} \Phi(d_2(S, T)).
\end{equation}
with $\Phi(.)$ the standard Normal cumulative distribution function.


	The functions $d_1(s,t)$ and $d_2(s,t)$ are given by
$$
\begin{array}{lll}
d_1(s,t) &=&\DSE \frac{\log(s/K) + (r +
\frac{\sigma^2}{2})t}{\sigma \sqrt{t}},\\*[12pt] d_2(s,t) &=&\DSE
d_1(s,t) -\sigma \sqrt{t}\\*[12pt]
&=&\DSE \frac{\log(s/K) + (r -
\frac{\sigma^2}{2})t}{\sigma \sqrt{t}}
\end{array}
$$


% end itemize



\subsubsection{American Options}

% frametitle
{ American Options in the CRR model (1)}


% begin itemize




	We now consider how to evaluate an American put option in a
standard CRR model.


	We assume that the time interval $[0,T]$ is
divided into $N$ equal subintervals of length $\Delta $ say.


	Assuming the risk-free rate of interest $r$ (over [0,T]) as given,
we have $1+r = e^{\rho \Delta}$ (where we denote the risk-free
rate of interest in each subinterval by $r$).


% end itemize



% frametitle
{ American Options in the CRR model (2)}


% begin itemize




	The remaining
degrees of freedom are resolved by choosing $u$ and $d$ as
follows:
$$
1+u= e^{\sigma \sqrt{\Delta}}, \A \mbox{and} \A 1+d = (1+u)^{-1} =
e^{-\sigma \sqrt{\Delta}}.
$$


	
The risk-neutral probabilities for
the corresponding single period models are given by
$$
p^*= \frac{r-d}{u-d} = \frac{e^{\rho \Delta}-e^{-\sigma
\sqrt{\Delta}}} {e^{\sigma \sqrt{\Delta}}-e^{-\sigma
\sqrt{\Delta}}}.
$$


% end itemize



% frametitle
{ American Options in the CRR model (3)}


% begin itemize




	Thus the stock with initial value $S = S(0)$ is worth $S (1+u)^i
(1+d)^j$ after $i$ steps up and $j$ steps down.


	Consequently,
after $N$ steps, there are $N+1$ possible prices, $S (1+u)^i
(1+d)^{N-i}$ ($i = 0, \ldots, N$). There are $2^N$ possible paths
through the tree.


% end itemize



% frametitle
{ American Options in the CRR model (4)}


% begin itemize




	It is common to take $N$ of the order of 30, for
two reasons:


% begin itemize




	(i) typical lengths of time to expiry of
options are measured in months (9 months, say); this
gives a time step around the corresponding number of days,


	(ii) $2^{30}$ paths is about the order of magnitude that can be
comfortably handled by computers (recall that $2^{10} = 1,024$, so
$2^{30}$ is somewhat over a billion).


% end itemize




% end itemize



% frametitle
{ American Options in the CRR model (5)}



% begin itemize




	We can now calculate both the value of an American put option and
the optimal exercise strategy by working backwards through the
tree


	This method of backward recursion in time is a form of the
dynamic programming (DP) technique, \label{dynamic programming}
due to Richard Bellman, which is important in many areas of
optimisation and Operational Research.


% end itemize



% frametitle
{American Options in the CRR model - Step (1)}



% begin enumerate




	Draw a binary
tree showing the initial stock value and having the
right number, $N$, of time intervals.


	Fill in the stock prices: after one time interval, these are
$S(1+u)$ (upper) and $S(1+d)$ (lower); after two time intervals,
$S(1+u)^2$, $S$ and $ S(1+d)^2 = S/(1+u)^2$; after $i$ time
intervals, these are $S(1+u)^j (1+d)^{i-j} = S (1+u)^{2j- i}$ at
the node with
$j$ `up' steps and $i-j$ `down' steps (the `$(i,j)$' node).


	Using the strike price $K$ and the prices at the terminal
nodes, fill in the payoffs $f^A_{N,j} = \max\{K - S (1+u)^j
(1+d)^{N-j}, 0\}$ from the option
at the terminal nodes underneath the terminal prices.


% end enumerate



% frametitle
{American Options in the CRR model - Step (2)}


% begin enumerate


\setcounter{enumi}{3}


	Work back down the tree, from right to left. The no-exercise
values $f_{ij}$ of the option at the $(i,j)$ node are given in
terms of those of its upper and lower right neighbours in the
usual way, as discounted expected values under the risk-neutral
measure:
$$
f_{ij} = e^{-\rho \Delta } [q f^A_{i+1,j+1} + (1 - q)
f^A_{i+1,j}].
$$
The intrinsic (or early-exercise) value of the American put at the
$(i,j)$
 node -- the value there if it is exercised early -- is
$$
K - S (1+u)^j (1+d)^{i-j}
$$
(when this is non-negative, and so has any value).


% end enumerate



% frametitle
{American Options in the CRR model - Step (3)}


% begin itemize




	The value of
the American put is the higher of these:
$$
\begin{array}{ll}
&f^A_{ij} \\*[12pt]=&\DSE \max\{f_{ij}, K - S (1+u)^j (1+d)^{i-j}\}\\
=&\DSE \max\left\{e^{-\rho \Delta} (q f^A_{i+1,j+1} + (1 - q)
f^A_{i+1,j})\right.,\\*[12pt]
&\hspace{3cm} \left. K - S (1+u)^j (1+d)^{i-j}\right\}\!.
\end{array}
$$


% end itemize



% frametitle
{American Options in the CRR model - Step (4)}


% begin enumerate


\setcounter{enumi}{4}


	The initial value of the option is the value $f^A_0$ filled in
at the
root of the tree.


	At each node, it is optimal to exercise early if the
early-exercise value there exceeds the value $f_{ij}$ there of
expected discounted future payoff.


% end enumerate



% frametitle
{ A Three-period Example (4)}

\begin{figure}[htb]
\begin{center}
  \includegraphics[width=10cm, height=8cm]{../../../pics/BSCRR}
\caption{Approximation of Black-Scholes price by Binomial models}
\end{center}\label{BSCRR}
\end{figure}

%$C_{BS}=10.9895$ (according to (\ref{BScallprice4})). We have
%$$
%\begin{array}{lc}
%{\boldsymbol n}& \mbox{ c(n)}\\ 5 & 11.33\\10 & 10.79\\ 50 & 10.95\\
%100 & 10.97\\ 200 & 10.98\\ 500 & 10.99 \end{array} $$

% frametitle
{ A Three-period Example (5)}



% begin itemize




	To price a European put, with price process denoted by $p(t)$, and
an American put, $P(t)$, (maturity $N=3$, strike $100$), we can
for the European put either use the put-call parity, the risk-neutral pricing formula, or work
backwards through the tree. For the prices of the American put we
use the technique outlined above.


	We indicate the early exercise times of the American put in bold
type. Recall that the discrete-time rule is to exercise if the
intrinsic value $K-S(t)$ is larger than the value of the
corresponding European put.


% end itemize



% frametitle
{ A Three-period Example (6)}

\begin{figure}[hbtp]
\unitlength1cm \thicklines
\begin{center}
\begin{picture}(12,6)
\put(0,0){time $t=0$} \put(0,3.2){$p=5.82$} \put(0,2.8){$P=6.18$}
\put(1.5,3){\line(3,2){1}} \put(3,3.8){$p=2.08$}
\put(3,3.4){$P=2.08$} \put(1.5,3){\line(3,-2){1}}
\put(3,2.4){$p=10.65$} \put(3,2){$\boldsymbol P=11.59$}
\put(3.5,0){$t=1$}
%%%%%%%%%top branch -step 1
\put(4.6,3.7){\line(2,1){1}} \put(6,4.5){$p=0$} \put(6,4.1){$P=0$}
\put(4.6,3.7){\line(2,-1){1}} \put(6,3.0){$p=4.76$}
\put(6,2.6){$P=4.76$}
%%%%%%%%%bottom branch -step 1
\put(4.6,2.3){\line(2,1){1}} \put(4.6,2.3){\line(2,-1){1}}
\put(6,1.8){$p=18.71$} \put(6,1.4){$\boldsymbol P=20.62$}
\put(6.5,0){$t=2$}
%%%%%%%%%bottom branch -step 2
\put(7.7,1.7){\line(2,1){1}} \put(7.7,1.7){\line(2,-1){1}}
\put(9.2,2.5){$p=10.90$} \put(9.2,2.1){$P=10.90$}
\put(9.2,1.3){$p=29.28$} \put(9.2,0.9){$P=29.28$}
%%%%%%%%%middle branch -step 2
\put(7.7,2.9){\line(2,1){1}} \put(9.2,3.7){$p=0$}
\put(9.2,3.3){$P=0$} \put(7.7,2.9){\line(2,-1){1}}
%%%%%%%%%top branch - step 2
\put(7.7,4.4){\line(2,1){1}} \put(9.2,4.9){$p=0$}
\put(9.2,4.5){$P=0$} \put(7.7,4.4){\line(2,-1){1}}
\put(9.7,0){$t=3$} \unitlength1cm \thicklines
\end{picture}
\end{center}
\caption{European $p(.)$ and American $P(.)$ put prices}
\end{figure}

