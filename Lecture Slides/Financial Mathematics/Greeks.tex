\section{Parameters of the Black-Scholes Model}
\subsection{The Greeks}
\frame{\frametitle{Greeks}
 We will now analyse the impact of the
underlying parameters in the standard Black-Scholes model on the
prices of call and put options. The Black-Scholes option values
depend on the (current) stock price, the volatility, the time to
maturity, the interest rate and the strike price. The
sensitivities of the option price with respect to the first four
parameters are called the {\it Greeks} and are widely used for
hedging purposes. We can determine the impact of these parameters
by taking partial derivatives. } \frame{\frametitle{BS-formula}
 Recall the Black-Scholes formula
for a European call:
$$
\pi^{call}(0) = C(S,T,K,r,\s) = S N(d_1(S, T)) - K e^{-rT}
N(d_2(S, T)),
$$
with the functions $d_1(s,t)$ and $d_2(s,t)$ given by
$$
\begin{array}{lll}
d_1(s,t) &=&\DSE \frac{\log(s/K) + (r +
\frac{\sigma^2}{2})t}{\sigma \sqrt{t}},\\*[12pt] d_2(s,t) &=&\DSE
d_1(s,t) -\sigma \sqrt{t}= \frac{\log(s/K) + (r -
\frac{\sigma^2}{2})t}{\sigma \sqrt{t}}.
\end{array}
$$
} \frame{\frametitle{Greeks}

$$
\begin{array}{lllll}
\Delta &:=&\DSE \frac{\partial C}{\partial S} &=&\DSE N(d_1) >
0,\\*[12pt] {\cal V} &:=&\DSE \frac{\partial C}{\partial \s} &=&
\DSE S \sqrt{T} n(d_1) >0,\\*[12pt] \Theta &:=&\DSE \frac{\partial
C}{\partial T} &=& \DSE \frac{S \s}{2 \sqrt{T}} n(d_1) + K r
e^{-rT} N(d_2) >0,\\*[12pt] \rho &:=&\DSE \frac{\partial
C}{\partial r} &=& \DSE T K e^{-rT} N(d_2) >0,\\*[12pt] \Gamma
&:=&\DSE \frac{\partial^2 C}{\partial S^2} &=& \DSE
\frac{n(d_1)}{S \s \sqrt{T}} >0.\\*[12pt]
\end{array}
$$
} \frame{\frametitle{Greeks} From the definitions it is clear that
$\Delta$ -- delta -- measures the change in the value of the
option compared with the change in the value of the underlying
asset, ${\cal V}$ -- vega -- measures the change of the option
compared with the change in the volatility of the underlying, and
similar statements hold for $\Theta$ -- theta -- and $\rho$ -- rho
}
\begin{frame}[fragile]
\frametitle{The Greeks}
\begin{itemize}
\item In order to quantify the risk associated with an instrument, one looks at
\emph{ how much the price of the instrument changes if one of the underlying
risk drivers changes} its value. Those risk measures are often called
Greeks.
\item Mathematically speaking, the Greeks are just the derivative of the price
of the instrument with respect to the value of the risk driver.
\item Once those Greeks are known for a portfolio, one can easily calculate how
much the value of an option or a portfolio changes \emph{marginally}, if one of
the variables changes \emph{marginally}, all others remain fixed.
\item The most important Greeks are Delta, Gamma, Vega, Rho, and Theta.
\end{itemize}
\end{frame}

\begin{frame}[fragile]
\frametitle{The Delta}
\begin{itemize}
\item Delta is the derivative of the instrument price with respect to the price
of the underlying.
\item The delta of the underlying security is one.
\item If the payoff is not linear (for example if the instrument is an option),
Delta is not constant.
\item As Delta is the derivative with respect to the underlying, it tells how
much the value of the instrument changes if the value of the underlying changes
marginally.
\item Thus in a continuous time model, Delta is the amount of the underlying
needed to be sold in order to offset the price change of the instrument. The
ability to neutralize the trading book with respect to price changes in the
underlying makes Delta the most important Greek.
\end{itemize}
\end{frame}

\begin{frame}[fragile]
\frametitle{Delta Hedging I}
\begin{itemize}
  \item Assume that you have an options position with delta $\Delta$. This means
  that if the price of the underlying moves by a (very) small amount $\epsilon$, the
  price of the option position moves approximately by $\Delta \epsilon$.
  \item A \emph{delta hedge} consists of selling $\Delta$ units of the
  underlying (buying if $\Delta <0$) and gives a portfolio that does not
  change its value if the price of the underlying changes by a marginal amount.
  The portfolio has a delta of zero and is called \emph{delta neutral}.
\end{itemize}
\end{frame}

\begin{frame}[fragile]
\frametitle{Delta Hedging II}
\begin{itemize}
  \item BUT: Delta is not constant, so the portfolio has to be rebalanced.
  \item In the Black-Scholes model, a delta hedge with continuous rebalancing is
  a perfect hedge.
  \item In practice, continuous rebalancing is not possible so that the
  portfolio is not protected against larger market movements. Transaction
  costs and bid-ask spread also result in losses.
\end{itemize}
\end{frame}



\begin{frame}[fragile]
\frametitle{Delta in the Black-Scholes Model}
In the Black-Scholes model, the Greeks can be computed explicitly for
European call and put options. We have:
\begin{align*}
  \Delta = \frac{\partial}{\partial S}Call_{BS}(S,K,\sigma,r,t,T) = N(d_1)
\end{align*}
where, as usual, $N$ denotes the c.d.f. of the standard normal distribution and
\begin{align*}
  d_1 = \frac{\log \left( \frac{S}{K} \right) + \left( r + \frac{\sigma^2}{2}
  \right)(T-t)}{\sigma \sqrt{T-t}}.
\end{align*}
\end{frame}

\begin{frame}[fragile]
\frametitle{Delta of a Call Option in the Black-Scholes Model}
%\usepackage{graphics} is needed for \includegraphics
\begin{figure}[htp]
\begin{center}
  \includegraphics[width=0.8\textwidth]{../../../pics/delta}
  \caption{Delta for a European call in the BS model, $T=1$, $r=1\%$,
  $\sigma=20\%$, $K=100$.}
  \label{fig:deltaBS}
\end{center}
\end{figure}
\end{frame}

\begin{frame}[fragile]
\frametitle{Example: Delta Hedge}
You are long USD $1,000$ in the $104$ call. Interest rate is $5\%$,
stock price today is $99$, time to maturity $1$ month, and implied volatility is
$15.7\%$.
\begin{itemize}
  \item How can you make your portfolio delta neutral by investing in the stock?
  \item You set up the delta neutral portfolio and the stock price jumps to
  USD$100$ immediately. What is your P/L for the portfolio?
\end{itemize}

\end{frame}

\begin{frame}[fragile]
\frametitle{Example: Delta Hedge}
\begin{itemize}
  \item Compute the price of the call with the BS formula:
  \begin{align*}
    Call_{BS} = 0.3858 \text{USD}.
  \end{align*}
  \item The position consists of $N=1,000/Call_{BS}=2592$ call options.
  \item The delta of each option is
  \begin{align*}
    \Delta &= N(d_1) =N(\frac{\log \left( S/K \right) + (r+\sigma^2/2)(T-t)
    }{\sigma\sqrt{T-t}}) \\
    	&= N(\frac{\log \left( 99/104 \right) + (0.05+0.157^2/2)(1/12)
    }{0.157\sqrt{1/12}})\\
     	&= 0.1654.
  \end{align*}
\end{itemize}
\end{frame}


\begin{frame}[fragile]
\frametitle{Example: Delta Hedge}
\begin{itemize}
  \item The delta of the position is long $\Delta_P=N\cdot \Delta=428.70$.
  \item The stock has $\Delta=1$.
  \item To make the position delta neutral, you have to enter a short position
  of $428.70$ shares.
\end{itemize}
\end{frame}

\begin{frame}[fragile]
\frametitle{Example: Delta Hedge}
\begin{itemize}
  \item The loss from the short position in the stock is $428.70 \cdot
  1=428.70$.
  \item To compute the gain from the long options position, we have to compute
  the option price for $S=100$. Using the BS formula, we obtain
  \begin{align*}
    Call_{BS}(S=100) = 0.5808.
  \end{align*}
  \item The gain from the options position is $2592\cdot(0.5808-0.3858)=505.37$.
  \item Our profit is $505.37-428.70=76.67$.
\end{itemize}
\end{frame}

\begin{frame}[fragile]
\frametitle{The Gamma}
\begin{itemize}
  \item Gamma is the second derivative of the instrument price with respect to
  the price of the underlying.
  \item If the instrument has a linear payoff, Gamma is zero.
  \item As delta is the first derivative of the option price with respect to the
  underlying, gamma is the derivative of delta with respect to the underlying
  and thus measures, how much Delta changes if the underlying changes.
  \item This is an important information in risk management as it tells the
  trader how much of the underlying he has to buy or sell if the underlying
  itself changes price.
  \item Geometrically, gamma might be seen as the slope of Delta.
\end{itemize}
\end{frame}

\begin{frame}[fragile]
\frametitle{Gamma Neutral Portfolios}
As the delta of a portfolio changes with the price of the underlying, a delta
hedge has to be rebalanced frequently.
\begin{itemize}
  \item A delta hedge for a portfolio with a high (absolute) gamma has to be
  monitored more carefully than a delta hedge of a portfolio with gamma close to
  zero.
  \item To decrease the hedging error, a trader might want to make a delta
  neutral portfolio gamma neutral.
  \item The gamma cannot be changed by investing in the underlying because its
  gamma is zero.
  \item Strategy: Make the portfolio gamma neutral by investing in an option,
  then make it delta neutral by investing in the underlying.
\end{itemize}
\end{frame}


\begin{frame}[fragile]
\frametitle{Gamma in the Black-Scholes Model}
The gamma of a European call option in the Black-Scholes model is given by
\begin{align*}
  \Gamma = \frac{\partial^2}{\partial S^2}Call_{BS}(S,K,\sigma,r,t,T) =
  \frac{\varphi(d_1)}{S\sigma \sqrt{T-t}}.
\end{align*}
Here, $\varphi(x)$ denotes the p.d.f. of the standard normal distribution.
\end{frame}

\begin{frame}[fragile]
\frametitle{Gamma of a Call Option in the Black-Scholes Model}
\begin{figure}[htp]
\begin{center}
  \includegraphics[width=0.8\textwidth]{../../../pics/gamma}
  \caption{Gamma for a European call in the BS model, $T=1$, $r=1\%$,
  $\sigma=20\%$, $K=100$.}
  \label{fig:gammaBS}
\end{center}
\end{figure}
\end{frame}

\begin{frame}[fragile]
\frametitle{Gamma of a Call Option in the Black-Scholes Model}
\begin{figure}[htp]
\begin{center}
  \includegraphics[width=0.8\textwidth]{../../../pics/gamma2}
  \caption{Gamma for a European call in the BS model, $T=10$, $r=1\%$,
  $\sigma=20\%$, $K=100$.}
  \label{fig:gamma2BS}
\end{center}
\end{figure}
\end{frame}

\begin{frame}[fragile]
\frametitle{Example: Gamma Neutral Portfolio}
You are in the same position as in the previous example. Additionally, you can
trade in the $97$ call.
\begin{itemize}
  \item Put together a portfolio that is delta and gamma neutral.
  \item What is your P/L if the stock price jumps to $100$?
\end{itemize}
\end{frame}

\begin{frame}[fragile]
\frametitle{Example: Gamma Neutral Portfolio}
\begin{itemize}
  \item The gamma of the $104$ call is
  \begin{align*}
  \Gamma_{104C} = \frac{\varphi(d_1)}{S\sigma \sqrt{T-t}} = 0.0554.
  \end{align*}
  \item The gamma of the $97$ call is
  \begin{align*}
  \Gamma_{97C} = \frac{\varphi(d_1)}{S\sigma \sqrt{T-t}} = 0.0758.
  \end{align*}
  \item To make the portfolio gamma neutral, we have to build up a position of
  $n$ $97$ calls so that $2592\cdot 0.0554 +n\cdot 0.0758 = 0$.
  \item We find that $n=-1894.74$ makes the portfolio gamma neutral, i.e., we
  sell $1894.74$ units of the $97$ call.
\end{itemize}
\end{frame}

\begin{frame}[fragile]
\frametitle{Example: Gamma Neutral Portfolio}
\begin{itemize}
  \item We compute the delta of the portfolio consisting of the two positions in
  the calls.
  \item The $97$ call has a delta of $\Delta_{97C}=0.7139$.
  \item The delta of the position is
  \begin{align*}
    \Delta_{P'} = 2592 \cdot 0.1654 - 1894.74 \cdot 0.7139 = -924.02
  \end{align*}
  \item We have to buy $924.02$ units of the underlying to make the portfolio
  delta neutral. By buying the underlying, we do not change the gamma of the
  portfolio, it remains zero.
\end{itemize}
\end{frame}

\begin{frame}[fragile]
\frametitle{Example: Gamma Neutral Portfolio}
\begin{itemize}
  \item The price of the $97$ call for $S=99$ is $Call_{BS,97}(S=99)=3.2235$.
  \item The price of the $97$ call for $S=100$ is $Call_{BS,97}(S=100)=3.9735$.
  \item The P/L from the position in the $97$ call is
  $-1894.97*(3.9735-3.2235)=-1421.11$.
  \item The P/L from the position in the stock is $924.02$.
  \item The portfolio P/L is $924.02-1421.11+505.37=-8.28.$
\end{itemize}
\end{frame}

\begin{frame}[fragile]
\frametitle{The Vega}
\begin{itemize}
  \item Vega is the derivative of the instrument price with respect to
  implied volatility.
  \item Thus, vega indicates how much the option price changes if the implied
  volatility changes.
  \item Vega is not a Greek letter.
\end{itemize}
\end{frame}

\begin{frame}[fragile]
\frametitle{Vega in the Black-Scholes Model}
The vega of a European call option in the Black-Scholes model is given by
\begin{align*}
  \nu = \frac{\partial}{\partial \sigma}Call_{BS}(S,K,\sigma,r,t,T) =
  S\sqrt{T-t} \varphi(d_1).
\end{align*}
Therefore, we have
\begin{align*}
  \nu = S^2 \sigma (T-t) \Gamma.
\end{align*}
\end{frame}

\begin{frame}[fragile]
\frametitle{Vega of a Call Option in the Black-Scholes Model II}
\begin{figure}[htp]
\begin{center}
  \includegraphics[width=0.8\textwidth]{../../../pics/vega}
  \caption{Vega for a European call in the BS model, $T=1$, $r=1\%$,
  $\sigma=20\%$, $K=100$.}
  \label{fig:vegaBS}
\end{center}
\end{figure}
\end{frame}

\begin{frame}[fragile]
\frametitle{Vega of a Call Option in the Black-Scholes Model III}
\begin{figure}[htp]
\begin{center}
  \includegraphics[width=0.8\textwidth]{../../../pics/vega_impliedvol}
  \caption{Vega for a European call in the BS model, $T=1$, $r=1\%$, $K=100$.}
  \label{fig:vega2BS}
\end{center}
\end{figure}
\end{frame}

\begin{frame}[fragile]
\frametitle{The Theta}
\begin{itemize}
  \item Theta is the derivative of the instrument price with respect to
  time to maturity.
  \item Thus, theta indicates how much the option price changes as time moves
  closer to maturity.
  \item Theta is usually negative and therefore is also called \emph{time decay}
  or \emph{rent}.
  \item Note: The passage of time is deterministic. It does not make sense to
  hedge against these losses.
\end{itemize}
\end{frame}

\begin{frame}[fragile]
\frametitle{Theta of a Call Option in the Black-Scholes Model}
\begin{figure}[htp]
\begin{center}
  \includegraphics[width=0.8\textwidth]{../../../pics/theta}
  \caption{Theta for a European call in the BS model, $\sigma=20\%$, $r=1\%$,
  $K=100$.}
  \label{fig:thetaBS}
\end{center}
\end{figure}
\end{frame}

\begin{frame}[fragile]
\frametitle{The Rho}
\begin{itemize}
  \item Rho is the derivative of the instrument price with respect to the risk
  free interest rate.
  \item It measures, how the price of the instrument changes if the interest
  rate changes.
  \item Rho is particular important for fixed-income portfolios. If the hedging
  portfolio of the instrument consists of a large portion of debt and only a
  small amount of initial capital, the influence of changes in the risk free
  rate might become quite big.
\end{itemize}
\end{frame}

\begin{frame}[fragile]
\frametitle{Rho of a Call Option in the Black-Scholes Model}
\begin{figure}[htp]
\begin{center}
  \includegraphics[width=0.8\textwidth]{../../../pics/rho}
  \caption{Rho for a European call in the BS model, $\sigma=20\%$, $T=1$,
  $K=100$.}
  \label{fig:rhoBS}
\end{center}
\end{figure}
\end{frame}

\subsection{Volatility}
\frame{\frametitle{Vega}

One of the main issues raised by the Black-Scholes formula is the
question of modelling the volatility $\s$. Before we can implement
the Black-Scholes formula to price options, we have to estimate
$\s$.

Because the formula is explicit, we can, determine the ${\cal V}$
-- the partial derivative
$$
{\cal V} = \partial C/\partial \s,
$$
finding
$$
{\cal V} = S \sqrt{T} n (d_1).
$$
The important thing to note here is that vega is always positive.

} \frame{\frametitle{Implied Volatility}

Next, since vega is positive, $C$ is a continuous -- indeed,
differentiable -- strictly increasing function of $\s$.  Turning
this round, $\s$ is a continuous (differentiable) strictly
increasing function of $C$; indeed,
$$
{\cal V} = \frac{\partial C}{\partial \s}, \A \mbox{so} \A
\frac{1}{\cal V} = \frac{\partial \s}{\partial C}.
$$

Thus the value $\s = \s(C)$ corresponding to the actual value $C =
C(\s)$ at which call options are observed to be traded in the
market can be read off.  The value of $\s$ obtained in this way is
called the {\it implied volatility}. }

\section{Variants}

\frame{\frametitle{Dividend-Paying Assets}

Let $S_t$ be a dividend-paying stock with continuous-dividend rate
$\rho$. To price a derivative with expiry $T$ we set
$$
X_t=e^{-\rho(T-t)}S_t.
$$
Then $X_t$ is a non-dividend paying asset and must have the
dynamic
$$
dX_t= rX_tdt+\sigma X_t dW_t.
$$
We can thus compute the dynamics of $S_t$ using It{\^o}'s formula
(or the product rule)
$$
dS_t=(r-\rho)S_tdt+\sigma S_tdW_t
$$
} \frame{\frametitle{Dividend-Paying Assets}

 The usual calculation give the European call option price
$$
\begin{array}{lll}
C(t) &=&\DSE S(t)e^{-\rho(T-t)} N(d_1(S(t), T-t))\\*[12pt] &&- K
e^{-r(T-t)} N(d_2(S(t), T-t)).
\end{array}
$$
The functions $d_1(s,t)$ and $d_2(s,t)$ are given by
$$
\begin{array}{lll}
d_1(s,t) &=&\DSE \frac{\log(s/K) + (r -\rho +
\frac{\sigma^2}{2})t}{\sigma \sqrt{t}},\\*[12pt] d_2(s,t) &=&\DSE
 \frac{\log(s/K) + (r - \rho-
\frac{\sigma^2}{2})t}{\sigma \sqrt{t}}
\end{array}
$$
} \frame{\frametitle{Time-dependent Volatility}

Assume
$$
dS_t=r S_t dt + \sigma(t) S_t dW_t
$$
We can solve this SDE an find
$$
S_t=S_0\exp\left\{ \int_0^t\sigma(s)dW_s + \left(rt +
\frac{1}{2}\int_0^t\sigma^2(s)ds \right) \right\}.
$$
Now
$$
\int_0^t\sigma(s)dW_s \sim N\left(0,
\sqrt{\int_0^t\sigma^2(s)ds}\right)
$$
and we set
$$
\bar{\sigma}(t,T)=\sqrt{\frac{1}{T-t}\int_t^T\sigma^2(s)ds}.
$$
} \frame{\frametitle{Time-dependent Volatility} The usual
calculation give the European call option price
$$
\begin{array}{lll}
C(t) &=&\DSE S(t) N(d_1(S(t), T-t))\\*[12pt] &&- K e^{-r(T-t)}
N(d_2(S(t), T-t)).
\end{array}
$$
The functions $d_1(s,t)$ and $d_2(s,t)$ are given by (with
appropriate time parameter)
$$
\begin{array}{lll}
d_1(s,t) &=&\DSE \frac{\log(s/K) + (r +
\frac{\bar{\sigma}^2}{2})t}{ \bar{\sigma}\sqrt{t}},\\*[12pt]
d_2(s,t) &=&\DSE
 \frac{\log(s/K) + (r -
\frac{\bar{\sigma}^2}{2})t}{\bar{\sigma} \sqrt{t}}
\end{array}
$$
}

