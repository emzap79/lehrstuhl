% !TEX root = FinancialMathematics_ws1314UDE.tex
\part{Interest Rates}
\section{Basic Interest Rates -- Discrete Version}
\frame{\frametitle{Development Bond Market}
\begin{figure}
	\centering
		\includegraphics[width=.80\textwidth]{../../../pics/USbonds}
	\label{fig:Risk_map}
\end{figure}
}

\frame{\frametitle{Basic Interest Rates}
\begin{itemize}
\item<1-> Economic agents have to be rewarded for postponing consumption; in addition, there is a risk premium for the uncertainty of the size of future consumption.
\item<2-> Investors, Firms, banks pay compensation for the willingness to postpone
\item<3-> A common interest rate (equilibrium) emerges which allows to fulfill the aggregate liquidity demand.
\end{itemize}
}
\frame{\frametitle{Basic Interest Rates}
Interest rates change with different maturities because
\begin{itemize}
\item<1-> {\it market segmentation hypothesis} different agents have different preferences for borrowing and lending in different segments of the yield curve. Agents can change their segment if the compensation for switching is high enough.
\item<2-> {\it expectation hypothesis} today's interest rates (spot rates) are determined on the basis of the expected future rates plus a risk premium.
\item<3-> {\it liquidity preference hypothesis } on average agents prefer to invest in short-term assets. Here Liquidity refers to 'closeness to maturity'. Agents are compensated through a risk premium for investing in longer dated assets.
\end{itemize}
}


\frame{\frametitle{Yield Curves}

\includegraphics[height=6cm]{../../../pics/TS-curve.pdf}

}

\frame{\frametitle{Yield Curves}

\includegraphics[height=6cm]{../../../pics/German-TS-curve.pdf}

{\tiny Daten: Staatsanleihen mit 1,2,3,4,5,7 und 10 Jahren Laufzeit, monatlich,
30.9.72 bis 28.2.06.}

}

\frame{\frametitle{10y yield US}

\includegraphics[height=6cm]{../../../pics/history_10y}

%{\tiny Daten: Staatsanleihen mit 1,2,3,4,5,7 und 10 Jahren Laufzeit, monatlich,
%30.9.72 bis 28.2.06.}

}



\section{Basic Interest Rates -- Continuous Version}
\subsection{Continuously-Compounded Rates}
\frame{ \frametitle{Notation}

$p(t,T)$ denotes the price of a risk-free zero-coupon bond at time
$t$ that pays one unit of currency at time $T$.

We will use continuous compounding, i.e. a zero bond with interest
rate $r(t,T)$ maturing at $T$ will have the price
$$p(t,T)=e^{-r(t,T)(T-t)}.$$
}

\frame{ \frametitle{Forward Rates}
Given three dates $t < T_1 <
T_2$ the basic question is: what is the risk-free rate of return,
determined at the contract time $t$, over the interval $[T_1,T_2]$
of an investment of $1$ at time $T_1$?\\

\begin{table}[htbp]
\begin{center}
\begin{tabular}{|c|c|c|c|}
\hline
{\rule[-3mm]{0mm}{8mm} Time }& $t$ & $T_1$ & $T_2$\\
\hline & & & \\*[-2mm]
& Sell $T_1$ bond & Pay out $1$ & \\
& Buy $\frac{p(t,T_1)}{p(t,T_2)}\;$ $T_2$ bonds & & Receive
$\frac{p(t,T_1)}{p(t,T_2)}$\\*[2mm] \hline {\rule[-3mm]{0mm}{8mm}
Net investment} & $0$ & $-1$ &
$+\frac{p(t,T_1)}{p(t,T_2)}$\\*[2mm] \hline
\end{tabular}
\end{center}
\caption{Arbitrage table for forward rates}
\end{table}

}

\frame{ \frametitle{Forward Rates}

To exclude arbitrage opportunities, the equivalent constant rate
of interest $R$ over this period (we pay out $1$ at time $T_1$ and
receive $e^{R(T_2-T_1)}$ at $T_2$) has thus to be given by
$$
e^{R(T_2-T_1)} = \frac{p(t, T_1)}{p(t, T_2)}.
$$
}
\frame{ \frametitle{Various Interest Rates}

\begin{itemize}
\item <1-> The forward rate at time $t$ for time period
$[T_1,T_2]$ is defined as
\[
R(t,T_1,T_2)=\frac{\log(p(t,T_1))-\log(p(t,T_2))}{T_2-T_1}
\]
\item <2->The spot rate for the time period $[T_1,T_2]$ is defined
as
\[
R(T_1,T_2)=R(T_1,T_1,T_2)
\]
\item<3-> The instantaneous forward rate is
\[
f(t,T)=-\frac{\partial \log(p(t,T))}{\partial T}
\]
\item <4-> The instantaneous spot rate is
\[
r(t)=f(t,t)
\]
\end{itemize}

} \frame{ \frametitle{Rates}
\begin{itemize}
\item<1-> The forward rate is the interest rate at which parties
at time $t$ agree to exchange $K$ units of currency at time $T_1$
and give back $Ke^{R(t,T_1,T_2)(T_2-T_1)}$ units at time $T_2$.
This means, one can lock in an interest rate for a future time
period today. \item<2->The spot rate $R(t,T_1)$ is the interest
rate (continuous compounding) at which one can borrow money today
and has to pay it back at $T_1$. \item<3->The instantaneous
forward and spot rate are the corresponding interest rates at
which one can borrow money for an infinitesimal short period of
time.
\item<4-> The spot rate $R(t,T)$ as a function of $T$ is referred to as the
yield curve at time $t$.
\end{itemize}
}

\frame{\frametitle{Simple Relations}

The money account process is defined by
$$
B(t) = \exp\left\{ \int_0^t r(s) ds\right\}\!.
$$

The interpretation of the money market account is a strategy of
instantaneously reinvesting at the current short rate.

For $t \leq s \leq T$ we have
$$
p(t,T) = p(t,s) \exp\left\{-\int_s^T f(t,u) du\right\}\!,
$$
and in particular
$$
p(t,T) = \exp\left\{-\int_t^T f(t,s) ds\right\}\!.
$$

}



\subsection{Simply-Compounded Forward Interest Rates}
\frame{\frametitle{Simple Spot Rate}
The simply-compounded spot interest rate prevailing at time $t$ for
the maturity $T$ is denoted by $L(t,T)$ and is the constant rate
at which an investment has to be made to produce an amount of one
unit of currency at maturity, starting from $p(t,T)$ units of
currency at time $t$, when accruing occurs proportionally to the
investment time.
\begin{equation}\label{LIBOR-spot}
L(t,T)=\frac{1-p(t,T)}{\tau(t,T)p(t,T)}
\end{equation}
Here $\tau(t,T)$ is the
daycount for the period $[t,T]$ (typically $T-t$).
}

\frame{\frametitle{Simple Spot Rate}
\begin{itemize}
\item<1-> The bond price can be expressed as
$$
p(t,T)=\frac{1}{1+L(t,T)\tau(t,T)}.
$$
Other 'daycounts' denoted  by $\tau(t,T)$ are possible.
\item<2-> Notation is motivated by LIBOR rates (London InterBank Offered
Rates).
\end{itemize}
}


\frame{\frametitle{Forward Rate Agreements}
In order to introduce simply-compounded forward interest rates we
consider forward-rate agreements (FRA). A FRA involves the current
time $t$, the expiry time $T>t$ and the maturity time $S>T$. The
contract gives its holder an interest-rate payment for the period
between $T$ and $S$. At maturity $S$, a fixed payment based on a
fixed rate $K$ is exchanged against a floating payment based on
the spot rate $L(T,S)$ resetting in $T$ with maturity $S$.
}
\frame{\frametitle{Forward Rate Agreements}

Formally, at time $S$ one receives $\tau(T,S)K\cdot N$ units of
currency and pays the amount $\tau(T,S)L(T,S)\cdot N$, where $N$
is the contract nominal value. The value of the contract is
therefore at $S$
\begin{equation}\label{FRA-1}
N\tau(T,S)(K-L(T,S)).
\end{equation}
We write this in terms of bond prices as
$$
N\tau(T,S)\left(K-\frac{1-p(T,S)}{\tau(T,S)p(T,S)}\right)=N\left(K\tau(T,S)-\frac{1}{p(T,S)}+1\right).
$$
}
\frame{\frametitle{Forward Rate Agreements}

Now we discount to obtain the value of this time $S$ cashflow at
$t$
$$
\begin{array}{ll}
& FRA(t,T,S,\tau(T,S),N,K) \\*[12pt] = & Np(t,S)\left(K\tau(T,S)-\frac{p(t,T)}{p(t,T)p(T,S)}+1\right) \\*[12pt]
  = &N(K p(t,S)\tau(T,S)-p(t,T)+p(t,S)).
\end{array}
$$
There is only one value of $K$ that renders the contract value $0$
at $t$. The resulting rate defines the simply-compounded forward
rate.
}
\frame{\frametitle{Simply-Compounded Forward Interest Rate}

The simply-compounded forward interest rate prevailing at time $t$
for the expiry $T>t$ and maturity $S>T$ is denoted by $F(t;T,S)$
and is defined by
\begin{equation}
F(t;T,S):=\frac{1}{\tau(T,S)} \left[\frac{p(t,T)}{p(t,S)}-1\right].
\end{equation}
}
\frame{\frametitle{Simply-Compounded Forward Interest Rate}

\begin{itemize}
\item<1-> $FRA(\ldots)=Np(t,S)\tau(T,S)(K-F(t; T,S))$ is an
equivalent definition.
\item<2-> To value a FRA (typically with a different $K$) replace the LIBOR rate in (\ref{FRA-1}) by
the corresponding forward rate $F(t;T,S)$ and take the present
value of the resulting quantity.
\end{itemize}
}


\subsection{Interest-Rate Swaps}
%{\bf Proof (i)}
%$$
%Np(t,S)\tau(T,S)(K-F(t;T,S))=N(p(t,S)\tau(T,S)\cdot K-p(t,T)+p(t,S)).
%$$
\frame{\frametitle{Interest-Rate Swap}

A generalisation
of the FRA is the Interest-Rate Swap (IRS). A Payer (Forward-start)
Interest-Rate Swap (PFS) is a contract that exchanges payments
between two differently indexed legs, starting from a future time
instant. At every instant $T_i$ in a prespecified set of dates
$T_{\alpha+1},\ldots ,T_{\beta}$ the fixed leg pays out the amount
$$
N\tau_i\cdot K
$$
corresponding to a fixed interest rate $K$, a
nominal value $N$, and a year fraction $\tau_i$ between $T_{i-1}$
and $T_i$, whereas the floating leg pays the amount
$$
N\tau_i L(T_{i-1},T_i).
$$
Corresponding to the interest rate
$L(T_{i-1},T)$ resetting at the previous instant $T_{i-1}$ for the
maturity given by the current payment instant $T_i$, with
$T_{\alpha}$ a given date.
}
%%%%%%%%%% Skizze %%%%%%%%%%%%%
\frame{\frametitle{Interest-Rate Swap}

Set
$${\cal T}:=\{T_{\alpha},\ldots ,T_{\beta}\}\quad\mbox{and}\quad
\tau:=\{\tau_{\alpha+1},\ldots,\tau_{\beta}\}.
$$
Payers IRS(PFS):
fixed leg is paid and floating leg is received \\
Receiver IRS (RFS): fixed leg is received and floating leg is
paid.

The discounted payoff at time $t<T_{\alpha}$ of a PFS is
$$
\sum_{i=\alpha+1}^{\beta}D(t,T_i)N\tau_i(L(T_{i-1},T_i)-K)
$$
with $D(t,T)$ the discount factor (typically from bank account).
For a RFS we have
$$
\sum_{i=\alpha+1}^{\beta}D(t,T_i)N\tau_i(K-L(T_{i-1},T_i)).
$$
}
\frame{\frametitle{Interest-Rate Swap}
We
can view the last contract as a portfolio of FRAs and find
$$
\begin{array}{ll}
& \DSE RFS(t,{\cal T},\tau,N,K) \\*[12pt] = & \DSE \sum_{i=\alpha+1}^{\beta}FRA(t,T_{i-1}T_i,\tau_i,N,K) \\*[12pt]
= & \DSE N\sum_{i=\alpha+1}^{\beta}\tau_i p(t,T_i)(K-F(t,T_{i-1},T_i))\\*[12pt]
= & -N   p(t,T_{\alpha})+Np(t,T_{\beta})+N\sum_{i=\alpha+1}^{\beta}\tau_i Kp(t,T_i).
\end{array}
$$
The two legs of an IRS can be viewed as
coupon-bearing bond (fixed leg) and floating rate note (floating
leg).
}
\frame{\frametitle{Interest-Rate Swap}
A floating-rate note is a
contract ensuring the payment at future times
$T_{\alpha+1},\ldots,T_{\beta}$ of the LIBOR rates that reset at the previous instants
$T_{\alpha},\ldots,T_{\beta-1}$. Moreover, the note pays a last
cash flow consisting of the reimbursement of the notational value
of the note at the final time $T_{\beta}$.
}
\frame{\frametitle{Interest-Rate Swap}
We can value the note by changing sign and setting $K=0$ in the
RFS formula and adding it to $Np(t,T_{\beta})$, the present value
of the cash flow $N$ at $T_{\beta}$. So we see
$$
\underbrace{-RFS(t,T,\tau,N,0)+Np(t,T_{\beta})}_{\mbox{value of note}}=
\underbrace{Np(t,T_{\alpha})}_{\mbox{from RFS formula}}.
$$
This implies that the note is always equivalent to $N$ units at
its first reset date $T_{\alpha}$ (the floating note trades at
par). 
}
\frame{\frametitle{Interest-Rate Swap}

\begin{itemize}
\item<1-> We require the IRS to be fair at time $t$ to obtain the
forward swap rate.
\item<2->
The forward swap rate $S_{\alpha,\beta}(t)$ at time $t$ for the
sets of time $\cal T$ and year fractions $\tau$ is the rate in the
fixed leg of the above IRS that makes the IRS a fair contract at
the present time, i.e. it is the $K$ for which
$RFS(t,T,\tau,N,K)=0$.
\item<3->
 We obtain
\begin{equation}\label{FSR-1}
S_{\alpha,\beta}(t)=\frac{p(t,T_{\alpha})-p(t,T_{\beta})}{\sum_{i=\alpha+1}^{\beta}\tau_ip(t,T_i)}.
\end{equation}
\end{itemize}
}
\frame{\frametitle{Interest-Rate Swap}

We write (\ref{FSR-1}) in terms of forward rates. First divide the numerator
and the denominator by $p(t,T_{\alpha})$ and observe that
$$
\frac{p(t,T_k)}{p(t,T_{\alpha})}=
\prod_{j=\alpha+1}^k\frac{p(t,T_j)}{p(t,T_{j-1})}=
\prod_{j=\alpha+1}^k\frac{1}{1+\tau_jF_j(t)}$$
with $F_j(t):=F(t,T_{j-1};T_j)$. So (\ref{FSR-1}) can be written as
\begin{equation}
S_{\alpha,\beta}(t)=
\frac{1-\prod_{j=\alpha+1}^{\beta}\frac{1}{1+\tau_jF_j(t)}}
{\sum_{i=\alpha+1}^{\beta}\tau_i\prod_{j=\alpha+1}^{i}\frac{1}{1+\tau_jF_j(t)}}.
\end{equation}
}



\section{Interest Rate Derivatives}
\subsection{Caps and Floors}
\frame{\frametitle{Caps}
\begin{itemize}
\item<1->
A cap is a contract where the seller of the contract promises to
pay a certain amount of cash to the holder of the contract if the
interest rate exceeds a certain predetermined level (the cap rate)
at a set of future dates.
\item<2->It can be viewed as a payer IRS where
each exchange payment is executed only if it has positive value.
\item<3->The cap discounted payoff is
$$
\sum_{i=\alpha+1}^{\beta}D(t,T_i)N\tau_i(L(T_{i-1},T_i)-K)^+.
$$
\item<4->Each individual term is a caplet.
\end{itemize}
}
\frame{\frametitle{Floors}
\begin{itemize}
\item<1->A floor is
equivalent to a receiver IRS where each exchange is executed only
if it has positive value.
\item<2->
The floor discounted payoff is
$$
\sum_{i=\alpha+1}^{\beta}D(t,T_i)N\tau_i(K-L(T_{i-1},T_i))^+.
$$
\item<3->Each individual term is a floorlet.
\end{itemize}
}
%{\bf Motivation.} Protection against LIBOR increase.

\frame{\frametitle{Black's Formula for Caps}
{\small Pricing is done via Black's formula
$$
Cap^{\mbox{Black}}(0,{\mathcal T},\tau,N,K,\sigma_{\alpha,\beta})=
N\sum_{i=\alpha+1}^{\beta}p(0,T_i)\tau_i Bl(K,F(0,T_{i-1},T_i),v_i,1),
$$
where $$\begin{array}{lll}
Bl(K,F,v,\omega) & = &\DSE  F\omega \Phi(\omega d_i(K,F,v))-K \omega \Phi(\omega d_2(K,F,v)) \\*[12pt]
d_1(K,F,v) & = & \DSE \frac{\log(F/K)+v^2/2}{v} \\*[12pt]
d_2(K,F,v) & = & \DSE \frac{\log(F/K)-v^2/2}{v}\\*[12pt]
v_i & = & \DSE \sigma_{\alpha,\beta}\sqrt{T_{i-1}}
\end{array}
$$
with the volatility parameter $\sigma_{\alpha,\beta}$
retrieved from market quotes.}
}
\frame{\frametitle{Black's Formula for Floors}

The corresponding floor is priced according to
$$\begin{array}{ll}
&\displaystyle
Flr^{\mbox{Black}}(0,{\mathcal T},\tau,N,K,\sigma_{\alpha,\beta})\\*[12pt]
=&\displaystyle
N\sum_{i=\alpha+1}^{\beta}p(0,T_i)\tau_i Bl(K,F(0,T_{i-1},T_i),v_i,-1).
\end{array}
$$
}
\frame{\frametitle{Simple Properties}
A cap
(floor) is said to be at-the-money (ATM) if and only if
$$
K=K_{ATM}:=S_{\alpha,\beta}(0)=\frac{p(0,T_{\alpha})-p(0,T_{\beta})}{\sum_{i=\alpha+1}^{\beta}\tau_ip(0,T_i)}.
$$
The cap is instead said to be in-the-money (ITM) if $K<K_{ATM}$,
and out-of-the-money (OTM) if $K>K_{ATM}$, with the converse
holding for a floor.
}

\frame{\frametitle{Simple Properties Cap}
\begin{itemize}
\item<1-> Simple protection against rising interest rates, but requires the payment of a premium.
\item<2-> Strike is the maximal interest to be paid.
\item<3-> Advantageous only if market expectation becomes true.
\end{itemize}
}

\subsection{Swaptions}
\frame{\frametitle{Swaptions}
\begin{itemize}
\item<1->Swap options or more commonly swaptions are options on an IRS. A
European payer swaption is an option giving the right (and not the
obligation ) to enter a payer IRS at a given future time, the
swaption maturity. Usually the swaption maturity coincides with
the first reset date of the underlying IRS.
\item<2->
The underlying-IRS
length $(T_{\beta}-T_{\alpha})$ is called the tenor of the swap.
\item<3-> The discounted payoff of a payer swaption can be written by
considering the value of the underlying payer IRS at its first
reset date $T_{\alpha}$ (also the maturity of the swaption)
$$
N\sum_{i=\alpha+1}^{\beta} p(T_{\alpha},T_i)\tau_i(F(T_{\alpha};T_{i-1},T_i)-K).
$$

\end{itemize}
}
\frame{\frametitle{Swaptions Payoff}
The option will be exercised only if this
value is positive. So the current value is
$$
ND(t,T_{\alpha})\left(\sum_{i=\alpha+1}^{\beta}p(T_{\alpha},T_i)\tau_i
(F(T_{\alpha};T_{i+1},T_i)-K)\right)^+.
$$
}
\frame{\frametitle{Swaptions Payoff}

Since the positive part operator is a
piece-wise linear and convex function we have
$$\begin{array}{ll}
&\displaystyle
\left(\sum_{i=\alpha+1}^{\beta}p(T_{\alpha},T_i)\tau_i(F(T_{\alpha};T_{i-1},T_i)-K)\right)^+\\*[12pt]
\leq&\displaystyle \sum_{i=\alpha+1}^{\beta}p(T_{\alpha},T_i)\tau_i(F(T_{\alpha};T_{i-1},T_i)-K)^+
\end{array}
$$
with strict inequality in general. Thus an additive decomposition is not
feasible.
}
%\underline{Excercise:} Compare value of payer swaption and
%corresponding cap.
\frame{\frametitle{Black's Formula for Swaptions}
Swaptions are also valued with a Black-like
formula
$$
\begin{array}{ll}
&\displaystyle
PS^{\mbox{Black}}(0,{\mathcal T},\tau,N,K,\sigma_{\alpha,\beta})\\*[12pt]
=& \displaystyle NBl(K,S_{\alpha,\beta}(0),\sigma_{\alpha,\beta}\sqrt{T_{\alpha}},1)\sum_{i=\alpha+1}^{\beta}\tau_ip(0,T_i)
\end{array}
$$
where $\sigma_{\alpha,\beta}$ is now a volatility parameter quoted
in the market different from the corresponding
$\sigma_{\alpha,\beta}$ in the cap/floor case.
}
\frame{\frametitle{Black's Formula for Swaptions}
A receiver swaption
gives the holder the right to enter at time $T_{\alpha}$ a
receiver IRS with payment date in $\cal T$. Its Black-type
valuation formula is
$$
\begin{array}{ll}
&\displaystyle
RS^{\mbox{Black}}(0,\cal T,\tau,N,K,\sigma_{\alpha,\beta})\\*[12pt]
=&\displaystyle
NBL(K,S_{\alpha,\beta}(0),\sigma_{\alpha,\beta}\sqrt{T_{\alpha}},-1)\sum_{i=\alpha+1}^{\beta}\tau_ip(0,T_i).
\end{array}
$$
}
\frame{\frametitle{Swaptions Payoff}
A swaption (either payer or receiver) is said to be at-the-money
(ATM) if and only if
$$
K=K_{ATM}=S_{\alpha,\beta}(0)=\frac{p(0,T_{\alpha})-p(0,T_{\beta})}{\sum_{i=\alpha+1}^{\beta}\tau_ip(0,T_i)}.
$$
The payer swaption is instead said to be in-the-money (ITM) if
$K<K_{ATM}$, and out-of-the-money (OTM) if $K>K_{ATM}$. The
receiver swaption is ITM if $K>K_{ATM}$, and OTM if $K<K_{ATM}$.
}








\section{Models for Pricing Interest Rate Derivatives}
\subsection{Valuation Framework}
\frame{\frametitle{Derivatives}
\begin{itemize}
\item<1-> Derivatives (contingent claims, options) are viewed a random variables, whose value depends on some
underlying, i.e. $X=f(S)$.
\item<2-> The (no-arbitrage) price process of a derivative is given by the risk-neutral valuation
formula
$$
\Pi_X(t) = D(0)
\EX^*\left[\frac{X}{D(T)}\right],
$$
where $D(.)$ is some discount-factor (the num{\'e}raire.
\end{itemize}
}

\frame{\frametitle{Example: The money market account as num\'{e}raire.}
Assuming that a
riskless asset (e.g. bank account) exists, it is natural to take it as
a num\'{e}raire. Assume
$$
B(t) = \exp\left\{\int_0^t r(u) du \right\}\!.
$$
The discounted price process of a security with respect to the
num\'{e}raire $B(t)$ is simply
$$
\bar{S}(t) = \exp\left\{- \int_0^t r(u) du \right\} S(t).
$$
\lq {Historically}'
$B(t)$ was used as the
num\'{e}raire $S_0(t)$ and then $\Q_B = \prb^*$.
}

\subsection{Change of Num{\'e}raire}

\frame{\frametitle{Change of num{\'e}raire technique -- Concept}
\begin{itemize}
\item<1-> With a given num\'{e}raire we use an equivalent
martingale measure $\prb^*$ such that the discounted basic security price processes are $\prb^*$-martingales.
We then calculate derivative prices as expectations under $\prb^*$ probabilities.
\item<2-> However, there may be situations where a different discount factor is more useful. Then we can change
the num{\'e}raire and calculate prices of derivatives as expectations under new probabilities.
\item<3-> So, prices can be calculated under any num{\'e}raire pair $X(t), \Q_X$, i.e. a process and a corresponding
probability measure such the discounted basic security price processes are $\Q_X$-martingales.
\end{itemize}
}


\frame{\frametitle{Example: Zero-coupon bonds as num\'{e}raire}
\begin{itemize}
\item<1-> A zero-coupon bond is
the natural choice of num\'{e}raire if one looks at the time $0$
price of an asset giving the right to a single cash-flow at a
well-defined future time $T$. The simplest such asset is a
zero-coupon bond with cash-flow $1$ at time $T$. We denote its
time $t$ price by $p(t,T)$. We assume that the money-market account $B(t)$ (as
defined above) was used to define $\prb^*$.
\item<2-> The corresponding probabilities are
$$
\frac{d\Q^T}{d\prb^*} = \frac{1}{p(0,T) B(t)} = \frac{1} {p(0,T)}
\exp\left\{- \int_0^t r(u) du \right\}\!.
$$
\end{itemize}
}

\frame{\frametitle{Example: Zero-coupon bonds as num\'{e}raire}
The relative price process of a basic asset $Z$ with respect to
$p(t,T)$, i.e. $\bar{Z} = Z(t)/p(t,T)$, is called the forward price
$F_Z(t)$ of the security $Z$, and given by
$$
F_Z(t) = \EX_{\Q^T} \left[Z(T)|\F_t\right],
$$
implying that $F_Z$ is a $\Q^T$ martingale.
}

\subsection{European Bond Options}
\frame{\frametitle{Gaussian Forward Rates}
\begin{itemize}
\item<1-> The dynamics of the forward rate are given under a
risk-neutral martingale measure $\prob^*$ by
$$
df(t,T) = \alpha(t,T) dt + \s(t,T) dW(t)
$$
with deterministic forward rate volatility.
\item<2-> Then the short-rate $r(t)$ as well as the forward rates
$f(t,T)$ have Gaussian probability laws.
\end{itemize}
}
\frame{\frametitle{Zero-Coupon Bond Prices}
Now $p(t,T)$ satisfies
$$
dp(t,T) = p(t,T) \left\{\left(r(t) + A(t,T) + \frac{1}{2}
S(t,T)^2 \right)dt + S(t,T) dW(t)\right\},
$$
where
$$
A(t,T) =\DSE -\int_t^T \alpha(t,s) ds, \A S(t,T) =\DSE -\int_t^T
\s(t,s) ds.
$$
So zero-coupon bond prices have a lognormal distribution.
}

\frame{\frametitle{Options on Bonds}

Consider a European call $C$ on a
$T^*$-bond with maturity $T \leq T^*$ and strike $K$. So we
consider the $T$-contingent claim
$$
X = \left(p(T,T^*)-K\right)^+\!\!.
$$
Its price at time $t=0$ is
$$
C(0) = p(0,T^*) \Q^*(A) -K p(0,T) \Q^T(A),
$$
with $A= \left\{p(T,T^*) >K\right\}$ and $\Q^T$ resp.
$\Q^*$ the $T$- resp. $T^*$-forward risk-neutral measure.

}

\frame{\frametitle{ Options on Bonds}
$$
\td{Z}(t,T) = \frac{p(t,T^*)}{p(t,T)}
$$
has $\Q$-dynamics
$$
d \td{Z} = \td{Z} \left\{ S(S-S^*)  dt -(S-S^*) dW
(t)\right\}\!,
$$
so a deterministic variance coefficient. Now
$$\begin{array}{ll}
&\DSE \Q^*(p(T,T^*) \geq K)\\*[12pt]
=&\DSE \Q^*\left(\frac{p(T,T^*)}{p(T,T)} \geq
K\right)\\*[18pt]
=&\DSE \Q^*(\td{Z}(T,T) \geq K).
\end{array}
$$
}

\frame{\frametitle{ Options on Bonds}

Since $\td{Z}(t,T)$ is a $\Q^T$-martingale with $\Q^T$-dynamics
$$
d \td{Z}(t,T) = - \td{Z}(t,T)  (S(t,T)-S(t,T^*)) d W^T(t),
$$
we find that under $\Q^T$
$$
\begin{array}{lll}
\td{Z}(T,T) &=&\DSE  \frac{p(0,T^*)}{p(0,T)} \exp\left\{-\int_0^T (S-S^*)
dW^T_t \right\} \\*[12pt]
&&\DSE \times\exp\left\{- \frac{1}{2} \int_0^T (S-S^*)^2 d t\right\}
\end{array}
$$
The stochastic integral in
the exponential is Gaussian with zero mean and variance
$$
\Sigma^2(T) = \int_0^T (S(t,T)-S(t,T^*))^2 dt.
$$
}

\frame{\frametitle{Options on Bonds}
So
$$\begin{array}{ll}
&\DSE \Q^T(p(T,T^*) \geq K) \\*[12pt]
=&\DSE \Q^T(\td{Z}(T,T) \geq K) = \Phi(d_2)
\end{array}
$$
with
$$
d_2 = \frac{\log\left(\frac{p(0,T)}{K p(0,T^*)}\right) -
\frac{1}{2} \Sigma^2(T)}{\sqrt{\Sigma^2(T)}}.
$$
We need to repeat the argument to get the price of the call option.

}

\frame{\frametitle{Price of Call Option}
The price of the call option is
given by
$$
C(0) = p(0,T^*)\Phi(d_2) - K p(0,T)\Phi(d_1),
$$
with parameters given as above  and
$$
d_1 = \frac{\log\left(\frac{p(0,T)}{K p(0,T^*)}\right) +
\frac{1}{2} \Sigma^2(T)}{\sqrt{\Sigma^2(T)}}.
$$
}

\frame{\frametitle{Options on Bonds - Simplified Formula}
\begin{itemize}
\item<1->We write the call option pricing formula as
$$
C(0) = p(0,T) (F_B \Phi(d_2) - K \Phi(d_1),
$$
where
$$
F_B=\frac{p(0,T^*)}{p(0,T)}
$$
is the forward price of the $T^*$-bond.
\item<2-> To use the formula for options on coupon-bonds also, we simply calculate the forward bond price and the forward bond price volatility $\sigma_B$
and use it in the formula.
\end{itemize}
}
\frame{\frametitle{Call Option on Bonds - Simplified Formula}
\begin{itemize}
\item<1-> $F_B$ can be calculated using the formula
$$
F_B=\frac{B_c(0)-I}{p(0,T)}
$$
with $B_c(0)$ the bond price at time zero and $I$ is the present value of the coupons that will be paid during
the life of the options.
\item<2->
$$
d_{1/2} = \frac{\log\left(\frac{F_B}{K}\right) \pm
\frac{1}{2} \sigma_B^2}{\sigma^2_B\sqrt{(T)}}.
$$
\end{itemize}
}
\subsection{Swaps, Caps and Bonds}
\frame{\frametitle{ Swaps}
Consider the
case of a {\it forward swap settled in arrears} characterized by:
\begin{itemize}
\item a fixed time $t$, the  contract time,
\item dates $T_0 < T_1, \ldots < T_n$, equally distanced $T_{i+1}-T_i =
\delta$,
\item $R$,  a prespecified fixed rate of interest,
\item $K$,  a nominal amount.
\end{itemize}
}



\frame{\frametitle{ Swaps}

A swap contract $S$ with $K$ and $R$ fixed for the period $T_0,
\ldots T_n$ is a sequence of payments, where the amount of money
paid out at $T_{i+1}, \; i=0, \ldots, n-1$ is defined by
$$
X_{i+1} = K \delta (L(T_i,T_{i+1})-R).
$$

The floating rate over $[T_i, T_{i+1}]$ observed at $T_i$ is a
simple rate defined as
$$
p(T_i, T_{i+1}) = \frac{1}{1 + \delta L(T_i,T_{i+1})}.
$$

}



\frame{\frametitle{Pricing Formula for  Swaps}

Using the risk-neutral pricing  formula we obtain (use $K=1$),
{\tiny
$$
\begin{array}{llll}
\Pi(t,S) &=& \DSE \sum_{i=1}^n &\DSE \EX_{\Q}\left[\left.e^{-
\int_t^{T_i} r(s)ds} \delta(L(T_{i-1},T_{i})-R)\right|\F_t\right]\\*[12pt]
&=&\DSE \sum_{i=1}^n &\DSE \EX_{\Q}\left[\EX_{\Q}\left[\left.e^{-
\int_{T_{i-1}}^{T_i}
r(s)ds}\right|\F_{T_{i-1}}\right]\right.\\*[12pt] & & &\DSE \times
\left. \left.e^{- \int_t^{T_{i-1}} r(s)ds}
\left(\frac{1}{p(T_{i-1},T_i)}-(1+ \delta
R)\right)\right|\F_t\right]\\*[12pt] &=&\DSE \sum_{i=1}^n &\DSE
\left( p(t,T_{i-1})-(1 + \delta R) p(t,T_i)\right) \\*[12pt]
&=&&\DSE p(t,T_0) -
\sum_{i=1}^n c_i p(t,T_i),
\end{array}
$$
}
with $c_i=\delta R, i=1, \ldots, n-1$ and $c_n = 1+\delta R$. So we obtain the
swap price as a linear combination of zero-coupon bond prices.
}



\frame{\frametitle{ Caps}
\begin{itemize}
\item<1->
An interest cap is a contract where the seller of the
contract promises to pay a certain amount of cash to the holder of
the contract if the interest rate exceeds a certain predetermined
level (the cap rate) at some future date. A cap can be broken down
in a series of caplets.
\item<2->
A caplet is a contract written at $t$,
in force between $[T_0, T_1],\, \delta = T_1-T_0$,
the nominal amount is $K$, the cap rate is denoted
by $R$. The relevant interest rate (LIBOR, for instance) is
observed in $T_0$ and defined by
$$
p(T_0,T_1) = \frac{1}{1 + \delta L(T_0,T_1)}.
$$
\end{itemize}
}



\frame{\frametitle{ Caplets}

A caplet $C$ is a $T_1$-contingent claim with payoff
$$
X = K \delta
(L(T_0,T_1)-R)^+.$$
Writing $L = L(T_0,T_1), p=p(T_0,T_1), R^*=1 +
\delta R$, we have $L=(1-p)/(\delta p)$, (assuming $K=1$) and
$$
\begin{array}{lll}
X &=&\DSE \delta (L-R)^+ =\DSE \delta \left(\frac{1-p}{\delta p}
-R\right)^+\\
&=&\DSE \left(\frac{1}{p} - (1+ \delta R) \right)^+ = \DSE
\left(\frac{1}{p} - R^* \right)^+\!\!.
\end{array}
$$
}



\frame{\frametitle{Risk-Neutral Formula for Caplets}
{\tiny
$$
\begin{array}{lll}
\Pi_C(t) &=& \EX_{\Q}\left[ \left. e^{- \int_t^{T_1} r(s)ds}
\left(\frac{1}{p} - R^* \right)^+\right|\F_t\right]
\\*[18pt]
&=&\DSE \EX_{\Q}\left[ \EX_{\Q}\left[\left.e^{- \int_{T_{0}}^{T_1}
r(s)ds}\right|\F_{T_{0}}\right]\right. \left. \left.e^{-
\int_t^{T_0} r(s)ds} \left(\frac{1}{p} - R^*
\right)^+\right|\F_t\right]\\*[18pt] &=&\DSE \EX_{\Q}\left[ \DSE
p(T_{0},T_1) \left.e^{- \int_t^{T_0} r(s)ds} \left(\frac{1}{p} -
R^* \right)^+\right|\F_t\right]\\*[18pt] &=&\DSE \EX_{\Q}\left[
\DSE \left.e^{- \int_t^{T_0} r(s)ds} \left(1 - p R^*
\right)^+\right|\F_t\right]\\*[18pt] &=&\DSE R^*\EX_{\Q}\left[
\DSE \left.e^{- \int_t^{T_0} r(s)ds} \left(\frac{1}{R^*} - p
\right)^+\right|\F_t\right]\!.
\end{array}
$$}
}

\frame{\frametitle{Risk-Neutral Formula for Caps}
\begin{itemize}
\item<1-> This shows that a caplet is equivalent to $R^*$ put options on a $T_1$-bond
with maturity $T_0$ and strike $1/R^*$.
\item<2-> Unfortunately, for pricing a cap we can not simply extend this formula.
\end{itemize}
}



\section{Market Models}
\subsection{LIBOR-Rate Market Models}
\frame{\frametitle{Introduction}
\begin{itemize}
\item<1-> Market models of LIBOR and Swap-rates are consistent
with the market practice of pricing caps,
floors and swaptions by means of the Black-formula.
\item<2-> They provide a consistent and coherent framework for the
joint modelling of a whole set of forward rates.
\item<3-> They use the discretely
compounded forward LIBOR and forward swap rates -- both directly
observable in the market -- as fundamental quantities in the
modelling process.
\end{itemize}
}

\frame{\frametitle{Forward LIBOR-Rates}
\begin{itemize}
\item<1-> Define the {\em tenor structure} $\mathcal T=\{T_0,
\dots, T_n\}$ as a set of maturities $T_i$ with
$0=T_0<T_1<\dots<T_n$, where $T_n$ is the time horizon of our
economy.
\item<2-> A given tenor structure $\mathcal T$ is associated with a
set of $\{\tau_1,\dots,\tau_n\}$ of year fractions, where
$\tau_i=T_i-T_{i-1}, \, i=1,\dots,n.$
\item<3-> We assume that in the
financial market under consideration, there exist zero-coupon
bonds  $p(\cdot,T_i)$ of all maturities $T_i,\, i=1,\dots,n$.
\item<4-> The
discretely compounded {\em forward LIBOR rate} prevailing at time
$t$ over the future period from $T_{i-1}$ to $T_i$ is defined by
$$
L(t,T_{i-1})=\frac{ p(t,T_{i-1})-p(t,T_i) }{\tau_i p(t,T_i)},
\quad 0 \leq t \leq T_{i-1}.
$$
% This corresponds to the forward rate agreement
\end{itemize}
}

\frame{\frametitle{LIBOR Dynamics Under the FLM}
\begin{itemize}
\item<1->
We call a $\prb$-equivalent probability
measure $\mathbb Q^{T_k}$ the {\it forward LIBOR measure (FLM) for the
maturity $T_k,$} or more briefly {\it $T_k$ forward measure,} if
all bond price processes
$$
\left( \frac{ p(t,T_i) }{ p(t,T_k) } \right)_{t \in
[0,\min\{T_i,T_k\}]}, \quad i=1,\hdots,n,
$$
relative to the num\'eraire $p(\cdot,T_k)$ are (local) martingales under $\mathbb Q^{T_k}$.
\item<2-> So  $(L(t,T_{k-1}))_{t \in
[0,T_{k-1}]}$ of the forward LIBOR over the period from $T_{k-1}$
to $T_k$ is a martingale under $\mathbb Q^{T_k}$.
\end{itemize}
}

\frame{\frametitle{LIBOR Dynamics Under the FLM}
\begin{itemize}

\item<1-> In a diffusion setting, we can posit, for every $k
\in \{1,\dots,n\},$ the following driftless dynamics under the
respective forward LIBOR measure $\mathbb Q^{T_k}$:
\begin{eqnarray}\label{LIBORSDE}
dL(t,T_{k-1}) & = & L(t,T_{k-1}) \sigma(t,T_{k-1}) \cdot dW^{T_k}(t) \nonumber \\
& = & L(t,T_{k-1}) \sum_{i=1}^d \sigma_i(t,T_{k-1}) dW_i^{T_k}(t),
\end{eqnarray}
where $W^{T_k}$ is a $d$-dimensional Brownian motion with respect
to $\mathbb F$ under the measure $\mathbb Q^{T_k}$ and has the
instantaneous covariance matrix $\rho=(\rho_{ij})_{i,j=1,\hdots,d}
\in \mathbb R^{d\times d}$.
\item<2-> For simplicity, we assume that $\sigma: [0,T_{n-1}] \times
\mathcal \{ T_1,\hdots,T_{n-1} \} \rightarrow \mathbb R^d_+$ is a
bounded and deterministic function, with
$\sigma(\cdot,\cdot)=(\sigma_1(\cdot,\cdot),\hdots,\sigma_d(\cdot,\cdot))$
a row vector.
\end{itemize}
}


\frame{\frametitle{LIBOR Dynamics Under a FLM}
Take $T_k \in \{T_1,\hdots,T_n\}$ as fixed.  The following relations for the {\bf LIBOR dynamics under the
forward measure} $\mathbb Q^{T_i},$ $T_i \in \mathcal T,$ hold:
{\tiny
$$
\begin{array}{llll}
i<k &:& dL(t,T_{k-1}) = & \DSE  L(t,T_{k-1})
\sigma(t,T_{k-1})\\*[18pt] &&&\cdot \DSE \left( \sum_{j=i+1}^k
\frac{ \tau_j L(t,T_{j-1})}{ 1+ \tau_j L(0,T_{j-1}) } \rho
\sigma(t, T_{j-1})' dt  + dW^{T_i}(t) \right), \\*[18pt] i>k &:&
dL(t,T_{k-1})= & \DSE  L(t,T_{k-1}) \sigma(t,T_{k-1})\\*[18pt] &&&
\cdot \DSE \left( -\sum_{j=k+1}^i \frac{ \tau_j L(t,T_{j-1})}{ 1+
\tau_j L(0,T_{j-1}) } \rho \sigma(t, T_{j-1})' dt  + dW^{T_i}(t)
\right),
\end{array}
$$
where $0 \leq t \leq \min\{T_i,T_{k-1}\}$ and $W^{T_i}$ is a
$d$-dimensional $(\mathbb F , \mathbb Q^{T_i})$-Brownian motion
with instantaneous covariance matrix $\rho$.
}}

\frame{\frametitle{LIBOR Dynamics Under a FLM}
\begin{itemize}
\item<1->
So a forward LIBOR process $L(\cdot,T_{k-1})$ is a lognormal martingale only
under its respective forward measure $\mathbb Q^{T_k}.$
\item<2-> Put
differently, there exists no measure under which all LIBORs are
simultaneously lognormal.
\end{itemize}
}
\frame{\frametitle{Simulation of LIBOR Dynamics}
\begin{itemize}
\item<1-> One has to resort
to numerical methods such as Monte Carlo simulation when pricing
certain complex derivatives that depend on the simultaneous
realization of several LIBOR rates in the above setup.
\item<2-> For simulation purposes
{\em one} measure $\mathbb Q^{T_k}$ has to be chosen, under which
all forward LIBOR rates have to be evolved simultaneously.
\item<3-> The following
relation between the Brownian motions $W^{T_k}$ and $W^{T_{k-1}}$
under the respective measures $\mathbb Q^{T_k}$ and $\mathbb
Q^{T_{k-1}}$ has to be used

\begin{eqnarray*}
dW^{T_{k-1}}(t) & = & dW^{T_k}(t) - \frac{\tau_k
L(t,T_{k-1})}{1+\tau_k L(0,T_{k-1})} \rho \sigma(t,T_k)' \, dt.
\end{eqnarray*}
\end{itemize}
}

\frame{\frametitle{Valuation of Caplets in the LMM}
\begin{itemize}
\item<1-> Recall that a  {\em caplet with reset date $T_k$ and maturity $T_{k+1}$ and
strike rate $K$}, or briefly a {\em $T_k$-caplet with strike $K$},
is a derivative  that pays the holder
$$
\tau_{k+1}(L(T_k,T_k)-K)^+
$$
at time $T_{k+1}$.
\item<2-> So a caplet  can be regarded as a
call option on a LIBOR rate.
\item<3-> We
work under the $\mathbb Q^{T_{k+1}}$-measure, then
\begin{eqnarray*}
L(t,T_k) & = & L(0,T_k) e^{ \left( \int_0^t \sigma(s,T_k) \cdot
dW^{T_k}(s) - \frac{1}{2} \int^t_0 \|\sigma(s,T_k) \| ^2 \, ds
\right)}
\end{eqnarray*}
for $0 \leq t \leq T_k$.
\end{itemize}
}
\frame{\frametitle{Valuation of Caplets in the LMM}
So
\begin{eqnarray*}
\log L(T_k,T_k) \sim N \left( m , s^2 \right),
\end{eqnarray*}
with
$$
m= \log L(0,T_k) - \frac{1}{2} \int^{T_k}_0 \|\sigma(s,T_k) \| ^2
\, ds
$$
and
$$
s^2=\int^{T_k}_0  \sigma(s,T_k) \rho \sigma(s,T_k)'   \, ds.
$$
}

\frame{\frametitle{Valuation of Caplets in the LMM}
The caplet price, denoted by $C(0,T_K,K),$ is:
\begin{eqnarray*}
C(0,T_K,K) & = & p(0,T_k) \EX_{\mathbb Q_{T_k}} \left(  \frac{\tau_{k+1}(L(T_k,T_k)-K)^+}{p(T_k,T_K)}  \right) \\
& = & \tau_{k+1} p(0,T_k) \EX_{\mathbb Q_{T_k}} \left( (L(T_k,T_k)-K)^+  \right) \\
& = & \tau_{k+1} p(0,T_k) \left( L(0,T_k) N(d_1) - K N(d_2)
\right),
\end{eqnarray*}
with
$$
d_1 = \frac{ \log (L(0,T_k)/K) + s^2/2 }{s}
$$
and
$$
d_2 = \frac{ \log (L(0,T_k)/K) - s^2/2 }{s}.
$$\index{Black formula!caplets}

This is the {\em Black formula} for caplets.
}



\subsection{Swap Rate Market Models}
\frame{\frametitle{Interest Rate Swap}
\begin{itemize}
\item<1-> Recall that an interest rate swap (IRS) is a contract to exchange fixed
against floating payments, where the floating payments typically
depend on LIBOR rates.\index{swap}
\item<2-> An IRS is specified by its {\em
reset-dates} $T_\alpha,T_{\alpha+1},\dots,T_{\beta-1}$, its {\em
payment-dates} $T_{\alpha+1},\dots,T_{\beta},$ and the fixed rate
$K.$
\item<3-> At every $T_j \in \{T_{\alpha+1},\dots,T_\beta\},$ the fixed
%leg
payment is $\tau_j K$ with $\tau_j=T_j-T_{j-1},$ while the
floating
%leg
payment is $\tau_j L(T_{j-1},T_{j-1})$.
\end{itemize}
}
\frame{\frametitle{Interest Rate Swap}
The value
of a swap in $t \leq T_\alpha$ can be determined without making
any distributional assumptions on the LIBOR rates as
\begin{equation}\label{IRS}
\sum_{i=\alpha+1}^\beta p(t,T_i) \tau_i (L(t,T_{i-1})-K),
\end{equation}
or alternatively
\begin{equation}
\label{IRS2}
p(t,T_\alpha) - p(t,T_\beta) - \sum_{i=\alpha+1}^\beta p(t,T_i) \tau_i K,
\end{equation}
}
\frame{\frametitle{Forward Swap Rate}
\begin{itemize}
\item<1->
The {\em forward swap rate} (FSR) at time $t$ of the above IRS,
which we denote by $S_{\alpha,\beta}(t),$ is the value for the
fixed rate $K$ that makes the $t$-value of the IRS zero.
\item<2-> $S_{\alpha,\beta}(t)$ can thus be obtained by equating Expression
$(\ref{IRS})$ to zero and solving for $K,$ which gives
\begin{eqnarray}
\label{FSR} S_{\alpha,\beta}(t) & = &  \sum_{i=\alpha+1}^\beta
w_i(t) L(t,T_{i-1})
\end{eqnarray}
with
\begin{eqnarray*}
w_i(t) & = & \frac{ \tau_i p(t,T_i) }{\sum_{j=\alpha+1}^\beta
\tau_j p(t,T_j)},
\end{eqnarray*}
\end{itemize}
}
\frame{\frametitle{Forward Swap Rate}
Equivalently equate $(\ref{IRS2})$ to zero, which gives
\begin{eqnarray*}
S_{\alpha,\beta}(t)  &=&  \frac{ p(t,T_\alpha) - p(t, T_\beta) }{
\sum_{i=\alpha+1}^\beta \tau_i p(t,T_i) }.
\end{eqnarray*}
Equation (\ref{FSR}) shows that the FSR can be expressed as a
suitably weighted average of the spanning forward LIBORs.
}


\frame{\frametitle{Swaptions}
\begin{itemize}
\item<1->  A European
{\em payer swaption} gives the holder the right to enter a swap as
fixed-rate payer at a fixed rate $K$ (the {\em swaption strike})
at a future date that normally coincides with the first reset date
$T_\alpha$ of the underlying swap.
\item<2-> Similarly, a European {\em
receiver swaption} gives the holder the right to enter a swap as
fixed-rate receiver.\index{swaption}
%\footnote{evtl. noch die Begriffe payer und receiver swap einf"uhren.}
\item<3-> The following relation holds for the $t$-value of a payer-swap: for $ 0 \leq t
\leq T_\alpha$,
%{\tiny
\begin{eqnarray}
\label{PSO} && \sum_{i=\alpha+1}^\beta p(t,T_i) \tau_i
(L(t,T_{i-1})-K)\\\nonumber & = &  (S_{\alpha,\beta}(t) -K)
\sum_{i=\alpha+1}^\beta \tau_i p(t,T_i).
\end{eqnarray}
%}
\end{itemize}
}

\frame{\frametitle{Swaptions}
\begin{itemize}
\item<1->
The advantage of the expression on the right-hand side of
Equation (\ref{PSO}) over the expression on the left-hand side
(which is our Formula (\ref{IRS})) is that one can instantly tell
from $S_{\alpha,\beta}(t)$ if the $t$-value of the payer-swap is
positive or negative.
\item<2-> At the maturity date $T_\alpha,$ a payer swaption is
exercised if and only if the value of the underlying swap is
positive, which is the case if and only if
$S_{\alpha,\beta}(T_\alpha) -K>0$ holds.
\item<3-> The
payer-swaption-value in $T_\alpha$ is
\begin{eqnarray*}
(S_{\alpha,\beta}(T_\alpha) -K)^+ \sum_{i=\alpha+1}^\beta \tau_i
p(T_\alpha,T_i),
\end{eqnarray*}
and the receiver-swaption-value is
\begin{eqnarray*}
( K - S_{\alpha,\beta}(T_\alpha))^+ \sum_{i=\alpha+1}^\beta \tau_i
p(T_\alpha,T_i).
\end{eqnarray*}
\end{itemize}
}

\frame{\frametitle{Swap Rate Dynamics}
\begin{itemize}
\item<1->
Observe that
\begin{eqnarray*}
A_{\alpha,\beta}(t) = \sum_{i=\alpha+1}^\beta \tau_i p(t,T_i)
\end{eqnarray*}
is the $t$-price of a a portfolio of bonds (i.e. a traded asset).
\item<2-> Thus, $A_{\alpha,\beta}(t),$ which if known as {\em
accrual factor} or {\em present value of a basis point,} can be
used as num\'eraire.
\item<3-> Now note that
\begin{eqnarray*}
S_{\alpha,\beta}(t) =
\frac{p(t,T_\alpha)-p(t,T_\beta)}{A_{\alpha,\beta}(t)},
\end{eqnarray*}
where the numerator can be regarded as the price of a traded asset
as well.
\end{itemize}
}

\frame{\frametitle{Swap Rate Dynamics}
\begin{itemize}
\item<1->
In order for our model to be
arbitrage-free,
 the swap rate $S_{\alpha,\beta}(\cdot)$ has to be a martingale
under the num\'eraire pair $(\mathbb
Q^{\alpha,\beta},A_{\alpha,\beta}(\cdot)).$
\item<2-> $\mathbb
Q^{\alpha,\beta}$ is the so-called {\em forward swap measure}.
\end{itemize}
}

\frame{\frametitle{Swap Rate Dynamics}
\begin{itemize}
\item<1->
We assume that $S_{\alpha,\beta}(\cdot)$ follows a
lognormal martingale:
\begin{eqnarray*}
dS_{\alpha,\beta}(t) = \sigma (t) S_{\alpha,\beta}(t)
dW_{\alpha,\beta}(t),
\end{eqnarray*}
where $\sigma$ is a deterministic function and
$W_{\alpha,\beta}(\cdot)$ is a standard $\mathbb
Q^{\alpha,\beta}$-Brownian motion.
\item<2-> The fact that the forward swap
rate $S_{\alpha,\beta}(t)$ is lognormally distributed under
$\mathbb Q^{\alpha,\beta}$ motivates the name {\em lognormal
forward swap model.}
\end{itemize}
}

\frame{\frametitle{Black's Swaption Pricing Formula}
The price of a payer swaption as specified
above in a lognormal forward swap model is consistent with Black's
formula for swaptions and is given by
\begin{eqnarray*}
PS(0,\{T_\alpha,{T_\alpha,\dots,T_\beta}\},K) & = &
A_{\alpha,\beta}(0) \left( S_{\alpha,\beta}(0) N(d_1)-K N(d_2)
\right)
\end{eqnarray*}
with
\begin{eqnarray*}
d_1=\frac{ \log (S_{\alpha,\beta}(0)/K
)+\frac{1}{2}\Sigma^2(T_\alpha)} {\Sigma(T_\alpha)}
\end{eqnarray*}
and
\begin{eqnarray*}
d_2=d_1-\Sigma(T_\alpha) & \mbox{ with } & \Sigma^2(T_\alpha) =
\int_0^{T_\alpha} \sigma(s)^2 \, ds.
\end{eqnarray*}
The price of a receiver-swaption is given by
\begin{eqnarray*}
RS(0,\{T_\alpha,{T_\alpha,\dots,T_\beta}\},K) & = &
A_{\alpha,\beta}(0) \left(K N(-d_2) - S_{\alpha,\beta}(0) N(-d_1)
\right).
\end{eqnarray*}
}

\frame{\frametitle{Swap Market Models}
\begin{itemize}
\item<1->
The pricing of more complicated derivatives
whose value depends on more than one swap-rate necessitates the
modelling of the {\em simultaneous} evolution of a set of swap
rates under one probability-measure.
\item<2-> This leads us to the class of
{\em Swap Market Models (SMMs),} which are arbitrage-free models
of the joint evolution of a set of swap rates under a common
probability measure.
\end{itemize}
}

\frame{\frametitle{Swap Market Models}
Examples  are:
\begin{enumerate}
\item [1.]The set of swap-rates $ \left\{
S_{\alpha,\alpha+1}(t),S_{\alpha+1,\alpha+2}(t),\dots,S_{\beta-1,\beta}(t),
\right\}.$ As this is exactly the set of LIBOR rates
$$ \left\{ L(t,T_\alpha),
L(t,T_{\alpha+1}), \dots, L(t,T_{\beta-1}) \right\},
$$ this choice leads us back to the LMM framework.
\item [2.] The set of swap-rates $ \left\{
S_{\alpha,\alpha+1}(t),S_{\alpha,\alpha+2}(t),\dots,S_{\alpha,\beta}(t)
\right\}.$ \item [3.] The set of swap-rates $ \left\{
S_{\alpha,\beta}(t),S_{\alpha+1,\beta}(t),\dots,S_{\beta-1,\beta}(t)
\right\}.$
\end{enumerate}
}
\frame{\frametitle{The Relation Between LMM and SMM}
\begin{itemize}
\item<1-> All products that can be priced in a
LMM-framework can in principle also be priced in a SMM-framework
and vice versa.
\item<2-> This is due to the fact that swap rates can be
expressed as weighted sums of LIBOR rates, and LIBOR rates can be
expressed as functions of swap-rates.
\item<3-> However, lognormal LMMs do not yield
swaption-prices that agree with Black-prices, and
lognormal SMMs do not give Black-consistent caplet prices.
\item<4-> The incompatibility is mostly of a theoretical nature, as it can
be shown (for example by simulation studies) that swap rates in
the lognormal LMM are ``almost'' lognormal, and therefore
swaption-prices in the lognormal LMM
are very similar to those one would obtain by the Black formula.
\end{itemize}
}
