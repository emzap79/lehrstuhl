\part{Cox-Ross-Rubinstein Model}

\section{Option Pricing}
\subsection{Binomial Tree Models}
 A fundamental example}
We consider a one-period model, i.e. we allow trading only at
$t=0$ and $t=T=1$(say).

   Our aim is to value at $t=0$  a European contingent claim on a stock $S$
  with maturity $T$.
   The payoff $H$ of the instrument is a function $f$ of the stock price at
  $T$. For a European call option with strike $K$, we would have
  $H=f(S_T)=(S_T-K)^+$.
   We assume that stocks do not pay dividends.
   Investors can borrow and lend at the risk-free rate $r$.

}

[fragile]
Replication: The Black-Scholes-Merton Approach}
If we can find a \emph{replicating portfolio} of instruments with known prices
that has the same payoff as the contingent claim $H$ in every state of the world, the value of the
two positions has to be equal by the no arbitrage assumption.


[fragile]
An Example I}
We calculate the price of a European call option on a stock $S$ with strike
$K=50$ and maturity in $T=1$ in a one-period model. The stock price today is
$50$ and can move up or down by $10\%$.

\center
\includegraphics[height=3cm]{../../../pics/COpt1}


[fragile]
An Example II}
In this example, we use $r=1\%$. We try to replicate the option with
investments of $x$ in the stock and $y$ in the risk-free bank account.
\begin{align*}
  5 &= x \cdot 55 + y \cdot 1.01 \\
  0 &= x \cdot 45 + y \cdot 1.01 \\
  \Rightarrow  x &= 0.5, y= -22.277
\end{align*}
The value of our investment today, and, by no arbitrage, the value of the option
today, is
\begin{align*}
  V_0 = 0.5 \cdot 50 -22.277 \cdot 1 = 2.72
\end{align*}


[fragile]
Risk-Neutral Valuation}

   In the example above, we did not use any assumption about the risk
  preferences of investors and we did not need the probability with which the
  stock moves up or down.
   Idea: Pretend that investors are indifferent about risk. Then, every
  asset has to earn an expected return equal to the return of the risk-free
  investment (if there is no arbitrage). Find the probabilities for up- and
  down-movements of the stock in such a world and use them to find the price of
  the option.



[fragile]
An Example III}
In the setup from our example, the risk-free rate of return is $1\%$. Therefore,
\begin{align*}
  1.01 &\stackrel{!}{=} \text{prob}_{up} \cdot 1.1 + \text{prob}_{down} \cdot 0.9 \\
  \Leftrightarrow 1.01 &\stackrel{!}{=} \text{prob}_{up} \cdot 1.1 + (1-\text{prob}_{up})
  \cdot 0.9 \\
  \Leftrightarrow \text{prob}_{up} &= 55\%
\end{align*}
We get the price of the option
\begin{align*}
  V_0 = (0.55 \cdot 5 + (1-0.55)\cdot 0)/1.01 = 2.72
\end{align*}


[fragile]
Risk-Neutral Valuation for Options}
The probabilities found in this way are called ``risk-neutral'' probabilities.
They define a probability measure, the ``risk-neutral'' or ``pricing''
measure.\\
\vspace{0.5cm}

The price $V(t)$ at time $t$ of an option with payoff $P(S_T)$ at time $T$ is

   the expectation
   under the risk-neutral measure $\Q$
   of the discounted
   payoff.

\begin{block}{}
\begin{align*}
  V(t) = \uncover<2->{\E\uncover<3->{^\Q}\left[ \uncover<4->{e^{-r(T-t)}}
  \uncover<5->{P(S_T)}\right]}
\end{align*}
\end{block}


[fragile]
The Cox-Ross-Rubinstein (CRR) Model}
The Cox-Ross-Rubinstein model is a binomial tree model. The figure below shows a
two-period tree, but it can easily be extended to multiple periods.
%\usepackage{graphics} is needed for \includegraphics
\begin{figure}[htp]
\begin{center}
\includegraphics[height=3.5cm]{../../../pics/CRR_tree}
\end{center}
\end{figure}



[fragile]
CRR Model: Choice of Parameters}
The parameters $u$ and $d$ that control the movements of the stock in a binomial
model can be chosen to match the volatility $\sigma$ of the stock. In the CRR
model, we have:

   $u=e^{\sigma\sqrt{\Delta t}}$ and $d=e^{-\sigma\sqrt{\Delta t}}$
   To match the expected return $\mu$ in the real world, the probability of
  an up movement has to be $p^*=\frac{e^{\mu \Delta t} -d}{u-d}$.
   For pricing, we need the risk neutral probabilities that can be derived
  as in the example. For an up movement, we get: $p=\frac{e^{r \Delta t}
  -d}{u-d}$.




[fragile]
Convergence}
In a one-period binomial model, we only consider two points in time, $t=0$ and
$t=T$. This approximation is rather rough as it allows only two states of the
world in $t=T$. We can add realism by dividing the interval $[0,T]$ into more
steps. If we add more and more steps, the results get closer and closer to the
results from the Black-Scholes model that we consider next. Mathematically
speaking, the CRR model converges to the Black-Scholes model as the number of
time steps in the interval increases to infinity.


\subsection{The Black-Scholes-Merton Model}
[fragile]
The Black-Scholes-Merton Model}
In 1973, Black and Scholes (1973) and Merton (1973)
developed the Black-Scholes (or Black-Scholes-Merton) model which was a major breakthrough in
option valuation. It can be derived in different ways:

   As the limit of the CRR model;
   Via the Black-Scholes partial differential equation (PDE);
   Via risk-neutral pricing.

For the second and third method, we need the additional assumption that stock
prices follow a geometric Brownian motion with constant drift and volatility.
This is not needed for the first method because there is an implicit distributional assumption in the CRR model.


[fragile]
The Black-Scholes Formula}
The time-$t$ price $c(t)$ of a European call option with strike $K$ and
maturity $T$ on a non-dividend-paying stock $S$ with volatility $\sigma$ in the
Black-Scholes model with risk-free interest rate $r$ is
\begin{align*}
  c(t) &= S_t N(d_1) - e^{-r(T-t)}KN(d_2)\\
  d_1 &= \frac{\ln \left(\frac{S_t}{K} \right) +
  \left(r+\frac{\sigma^2}{2}\right)(T-t)}{\sigma\sqrt{T-t}}\\
  d_2 &= d_1 - \sigma \sqrt{T-t}
\end{align*}
where $N(.)$ denotes the cumulative distribution function (cdf) of the standard
normal distribution.


[fragile]
Example: Option Prices in the Black Scholes Model}
Find the price of a European call option in the BS model with strike $K=100$,
time to maturity $1$ year, on a stock with time-$0$ price $100$. The risk free
rate is $5\%$ p.a., the volatility of the stock is $20\%$ p.a.
\begin{align*}
  d_1&=\frac{\ln \left(\frac{100}{100} \right) +
  \left(0.05+\frac{0.2^2}{2}\right)(1-0)}{0.2 \sqrt{1-0}}\\&=0.35\\
  d_2&=0.35-0.2*\sqrt{1-0} \\&= 0.15\\
  c(0)&=100 N(0.35) - e^{-0.05\cdot 1}100N(0.15)\\&=10.45.
\end{align*}