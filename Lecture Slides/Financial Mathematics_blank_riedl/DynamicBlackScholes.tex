% !TEX root = FinancialMathematics_ws1314UDE.tex

\part{Dynamic Financial Models and Black-Scholes}
\section{The Financial Market Model}
\subsection{The Model}

Financial Market Model
	$T>0$ is a fixed a planning horizon.
  
	Uncertainty in the financial market is modelled by a probability
	space $(\Omega, {\cal F},\prb)$ and an information set  (filtration) $\fil =({\cal
	F}_t)_{0\leq t\leq T}$.
		
	There are $d+1$ primary traded assets, whose price processes are
	given by stochastic processes $S_0, \ldots, S_d$, which represent
	the prices of some traded assets.
		
	A num\'{e}raire is a price process $X(t)$ almost surely strictly
	positive for each $t \in [0,T]$.
		
	\lq {Historically}' the money market account $B(t)=e^{rt}$ with a
	positive constant $r$ was used as a
	num\'{e}raire.


Trading Strategies
	We call an $\setR^{d+1}$-valued process
		$$
		\varphi(t) = (\varphi_0(t), \ldots, \varphi_d(t)), \A t \in [0,T]
		$$
	a trading strategy (or dynamic portfolio process).
  
	Here $\varphi_i(t)$ denotes the number of shares of asset $i$ held
	in the portfolio at time $t$ - to be determined on the basis of
	information available {\it before} time $t$; i.e. the investor
	selects his time $t$ portfolio after observing the prices $S(t-)$.

	The value of the portfolio $\varphi$ at time $t$ is given by
		$$
		V_\varphi(t) :=  \varphi(t) \cdot S(t) = \sum_{i=0}^d \varphi_i(t)
		S_i(t).
		$$
	$V_\varphi(t)$ is called the value process, or wealth process, of
	the trading strategy $\varphi$.\  The gains process $G_\varphi(t)$ is
		$$
		G_\varphi(t) := \sum_{i=0}^d \int_0^t \varphi_i(u) dS_i(u).
		$$
	A trading strategy $\varphi$ is called self-financing if the wealth process $V_\varphi(t)$ satisfies
		$$
		V_\varphi(t) = V_\varphi(0) + G_\varphi(t)\A \mbox{for all}\; t\in
		[0,T].
		$$


Discounted Processes
	The discounted price process is
		$$
		\td{S}(t) := \frac{S(t)}{S_0(t)} = (1, \td{S}_1(t), \ldots
		\td{S}_d(t))
		$$
	with $\td{S}_i(t) = S_i(t)/S_0(t),\; i=1,2, \ldots, d$. The
	discounted wealth process $\td{V}_\varphi(t)$ is
		$$
		\td{V}_\varphi(t):= \frac{V_\varphi(t)}{S_0(t)} = \varphi_0(t) +
		\sum_{i=1}^d \varphi_i(t) \td{S}_i(t)
		$$
	and the discounted gains process $\td{G}_\varphi(t)$ is
		$$
		\td{G}_\varphi(t) := \sum_{i=1}^d \int_0^t \varphi_i(t)
		d\td{S}_i(t).
		$$


Self-Financing
	$\varphi$ is self-financing if and only if
		$$
		\td{V}_\varphi(t) = \td{V}_\varphi(0) + \td{G}_\varphi(t).
		$$

	Thus a self-financing strategy is completely determined by its
	initial value and the components $\varphi_1, \ldots, \varphi_d$.
	Any set of processes $\varphi_1, \ldots, \varphi_d$
	such that the stochastic integrals $\int \varphi_i d\td{S}_i$
	exist can be uniquely extended to a self-financing strategy
	$\varphi$ with specified initial value $\td{V}_\varphi(0)= v$ by
	setting the cash holding as
		$$
		\varphi_0(t) = v + \sum_{i=1}^d \int_0^t \varphi_i(u) d\td{S}_i(u)
		- \sum_{i=1}^d \varphi_i(t) \td{S}_i.
		$$


\subsection{Arbitrage}
Arbitrage Opportunities
	A self-financing trading strategy $\varphi$ is called an arbitrage
	opportunity if the wealth process $V_\varphi$ satisfies the
	following set of conditions:
		$$
		V_\varphi(0)=0,\A \prb(V_\varphi(T) \geq 0) =1,
		$$
	and
		$$
		\prb(V_\varphi(T) > 0) >0.
		$$


\subsection{Equivalent Martingale Measures}

Martingale Measure
	A probability measure $\Q$ defined on $(\Om,\F)$ is an equivalent
	martingale measure (EMM) if:\\*[6pt]
	(i) $\Q$ is equivalent to $\prb$,\\
	(ii) the discounted price process $\td{S}$ is a $\Q$ martingale.


	Assume $S_0(t) = B(t) = e^{rt}$, then $\Q \sim \prb$ is a
	martingale measure if and only if every asset price process $S_i$
	has price dynamics under $\Q$ of the form
	$$
	dS_i(t) = r S_i(t) dt + dM_i(t),
	$$
	where $M_i$ is a $\Q$-martingale.



EMMs and Arbitrage
	Assume $\Q$ is an EMM. Then the market model contains no arbitrage
	opportunities.

	{\it Proof.} Under $\Q$ we have that $\td{V}_\varphi(t)$ is a
	martingale. That is,
		$$
		\EX_{\Q}\left(\td{V}_\varphi(t)|\F_u\right) = \td{V}_\varphi(u),\;
		\mbox{ for all }\A u \leq t \leq T.
		$$
	For $\varphi \in \Phi$ to be an arbitrage opportunity we must have
	$\td{V}_\varphi(0) =V_\varphi(0)= 0$.  Now
		$$
		\EX_{\Q}\left(\td{V}_\varphi(t)\right) = 0,\; \mbox{ for all } 0
		\leq t \leq T.
		$$
	Now $\td{V}_\varphi(t)$ is a martingale, so
		$$
		\EX_{\Q}\left(\td{V}_\varphi(t)\right) = 0,\; 0 \leq t \leq T,
		$$
	in particular $ \EX_{\Q}\left(\td{V}_\varphi(T)\right) = 0. $

	For an arbitrage opportunity $\varphi$ we have $
	\prb\left(V_\varphi(T) \geq 0\right) =1, $ and since $\Q \sim
	\prb$, this means $ \Q\left(V_\varphi(T) \geq 0\right) =1. $

	Both together yield
		$$
		\Q\left(V_\varphi(T) > 0\right) = \prb\left(V_\varphi(T) >
		0\right) =0,
		$$
	and hence the result follows.\hfill \eb


Admissible Strategies
	A SF strategy $\varphi$ is called ($\prb^*$-) admissible if
		$$
		\td{G}_\varphi(t) = \int_0^t \varphi(u) d\td{S}(u)
		$$
	is a ($\prb^*$-) martingale.
  
	By definition $\td{S}$ is a martingale, and $\td{G}$ is the
	stochastic integral with respect to $\td{S}$.
  
	The financial market model $\cal M$ contains no arbitrage
	opportunities wrt admissible strategies.



\section{The Black-Scholes Model}
\subsection{Valuation Principles}

Contingent Claims
	A contingent claims $X$ is a random variable with existing expected value. 
		A contingent claim $X$ is called attainable if there
		exists at least one admissible trading strategy such that
			$$
			V_\varphi(T) = X.
			$$
		We call such a  trading strategy $\varphi$ a replicating strategy
		for $X$.  The financial market model ${\cal M}$ is said
		to be complete if any contingent claim is attainable.


No-Arbitrage Price
	If a contingent claim $X$ is attainable, $X$ can be replicated by
	a portfolio $\varphi \in \Phi(\prb^*)$. This means that holding
	the portfolio and holding the contingent claim are equivalent from
	a financial point of view. In the absence of arbitrage the
	(arbitrage) price process $\Pi_X(t)$ of the contingent claim must
	therefore satisfy
		$$
		\Pi_X(t) = V_\varphi(t).
		$$


Risk-Neutral Valuation
	The arbitrage price process of
	any attainable claim is given by the risk-neutral valuation
	formula
		$$
		\Pi_X(t) = S_0(t)
		\EX_{\prb^*}\left[\left.\frac{X}{S_0(T)}\right|\F_t\right].
		$$

	Thus, for any two replicating portfolios $\varphi, \psi \in
	\Phi(\prb^*)$
		$$
		V_\varphi(t) = V_\psi(t).
		$$

	{\it Proof.} Since $X$ is
	attainable, there exists a replicating strategy $\varphi \in
	\Phi(\prb^*)$ such that $V_\varphi(T) = X$ and $\Pi_X(t) =
	V_\varphi(t)$. Since $\varphi \in \Phi(\prb^*)$ the discounted
	value process $\td{V}_\varphi(t)$ is a martingale, and hence
		$$
		\begin{array}{lll}
		\Pi_X(t) &=&\DSE V_\varphi(t) = S_0(t)\td{V}_\varphi(t) \\*[12pt]
		&=&\DSE S_0(t) \EX_{\prb^*}
		\left[\left.\td{V}_\varphi(T)\right|\F_t\right]\\*[18pt]
		&=&\DSE S_0(t)
		\EX_{\prb^*}
		\left[\left.\frac{V_\varphi(T)}{S_0(T)}\right|\F_t\right]\\*[18pt]
		&=&\DSE S_0(t) \EX_{\prb^*}
		\left[\left.\frac{X}{S_0(T)}\right|\F_t\right]\!.
		\end{array}
		$$
	\hfill \eb


\subsection{BS-Model: Properties}

Black-Scholes Model
	The classical Black-Scholes model is
		$$
		\begin{array}{llll}
		dB(t) &=&\DSE r B(t) dt, \A &B(0)= 1,\\ dS(t) &=&\DSE S(t) \left(
		b dt + \sigma dW(t) \right)\!, \A &S(0) = p,
		\end{array}
		$$
	with constant coefficients $b \in \setR,\; r,\s \in \setR_+$.

	The Black-Scholes price pro\-cess of a European call is given by
		$$
		\begin{array}{lll}
		C(t) &=&\DSE S(t) \Phi(d_1(S(t), T-t))\\*[12pt] &&- K e^{-r(T-t)}
		\Phi(d_2(S(t), T-t)).
		\end{array}
		$$
	The functions $d_1(s,t)$ and $d_2(s,t)$ are given by
		$$
		\begin{array}{lll}
		d_1(s,t) &=&\DSE \frac{\log(s/K) + (r +
		\frac{\sigma^2}{2})t}{\sigma \sqrt{t}},\\*[12pt] d_2(s,t) &=&\DSE
		 \frac{\log(s/K) + (r -
		\frac{\sigma^2}{2})t}{\sigma \sqrt{t}}
		\end{array}
		$$
	The classical Black-Scholes model is
		$$
		\begin{array}{llll}
		dB(t) &=&\DSE r B(t) dt, \A &B(0)= 1,\\ dS(t) &=&\DSE S(t) \left(
		b dt + \sigma dW(t) \right)\!, \A &S(0) = p,
		\end{array}
		$$
	with constant coefficients $b \in \setR,\; r,\s \in \setR_+$.

	We use the bank account being the natural num\'{e}raire and get
	from It\^{o}'s formula
		$$
		d\td{S}(t) = \td{S}(t)\left\{ (b-r) dt + \s dW(t) \right\}\!.
		$$


EMM in BS-model
	Any EMM is a Girsanov pair
		$$
		\left.\frac{d\Q}{d\prb}\right|_{\F_t} =L(t)
		$$
	with
		$$
		L(t) = \exp\left\{- \int_0^t \gamma(s) dW(s) - \frac{1}{2}
		\int_0^t \gamma(s)^2 ds\right\}.
		$$
	By Girsanov's theorem
		$$
		dW(t) =  d\td{W}(t) - \gamma(t) dt,
		$$
	where $\td{W}$ is a $\Q$-BM. Thus the $\Q$-dynamics for $\td{S}$ are
		$$
		d\td{S}(t) = \td{S}(t)\left\{ (b-r - \s \gamma(t)) dt + \s
		d\td{W}(t) \right\}\!.
		$$
	Since $\td{S}$ has to be a martingale under $\Q$ we must have
		$$
		b-r - \s \gamma(t)= 0\A t \in [0,T],
		$$
	and so we must choose
		$$
		\gamma(t) \equiv \gamma = \frac{b-r}{\s},
		$$
	this argument leads to a unique martingale measure. The $\Q$-dynamics of $S$ are
		$$
		dS(t) = S(t) \left\{ r dt + \s d\td{W} \right\}\!.
		$$


\subsection{Pricing and Hedging Contingent Claims}

Pricing Contingent Claims
	By the risk-neutral valuation principle
		$$
		\Pi_X(t) = e^{\left\{ -r (T-t)\right\}} \EX^*\left[\left.X
		\right|\F_t\right]\!,
		$$
	with $\EX^*$ given via the Girsanov density
		$$
		L(t)=\exp\left\{ -\left(\frac{b-r}{\s}\right) W(t) - \frac{1}{2}
		\left(\frac{b-r}{\s}\right)^2 t \right\}\!.
		$$


Pricing Contingent Claims
	For a European call $X = (S(T)-K)^+$ and  we can evaluate the above expected value

	The Black-Scholes price pro\-cess of a European call is given by
		$$
		\begin{array}{lll}
		C(t) &=&\DSE S(t) \Phi(d_1(S(t), T-t))\\*[12pt]
		&&- K e^{-r(T-t)} \Phi(d_2(S(t), T-t)).
		\end{array}
		$$
	The functions $d_1(s,t)$ and $d_2(s,t)$ are given by
		$$
		\begin{array}{lll}
		d_1(s,t) &=&\DSE \frac{\log(s/K) + (r +
		\frac{\sigma^2}{2})t}{\sigma \sqrt{t}},\\*[12pt] d_2(s,t) &=&\DSE
		 \frac{\log(s/K) + (r -
		\frac{\sigma^2}{2})t}{\sigma \sqrt{t}}
		\end{array}
		$$


BS by Arbitrage
	Consider a self-financing portfolio which has dynamics
		$$
		\begin{array}{lll}
		dV_\varphi(t)&=&\DSE \varphi_0(t)dB(t)+\varphi_1(t)dS(t)\\*[12pt]
		&=&\DSE (\varphi_0 r B+\varphi_1b S)dt + \varphi_1\s SdW.
		\end{array}
		$$
	Assume that
		$
		V_\varphi(t)=V(t)=f(t,S(t)).
		$
		Then by It{\^o}'s formula
		$$
		\begin{array}{lll}
		dV &=& \DSE (f_t+f_xSb+\frac{1}{2}S^2\s^2f_{xx})dt\\*[12pt] &&\DSE
		+ f_x\s SdW.
		\end{array}
		$$
	We match coefficients and find
		$$
		\varphi_1=f_x \mbox{ and }
		\varphi_0=\frac{1}{rB}(f_t+\frac{1}{2}\s^2S_t^2f_{xx}).
		$$
	So $f(t,x)$ must satisfy the Black-Scholes  PDE
		$$
		f_t+rxf_x+\frac{1}{2}\s^2x^2f_{xx}-rf=0
		$$
	and initial condition $f(T,x)=(x-K)^+$. 


Option Dynamics
	Let us now compute the dynamics of the call option's price $C(t)$
	under the risk-neutral martingale measure $\prb^*$. We find
		$$
		dC(t) = r C(t) dt + \s \Phi(d_1(S(t),T-t)) S(t) d \td{W}(t).
		$$
	Defining the {\it elasticity coefficient} of the option's price as
		$$
		\eta^c(t) = \frac{\Delta(S(t),T-t) S(t)}{C(t)} =
		\frac{\Phi(d_1(S(t),T-t))}{C(t)}
		$$
	we can rewrite the dynamics as
		$$
		dC(t) = r C(t) dt + \s \eta^c(t) C(t) d \td{W}(t).
		$$
	So, as expected in the risk-neutral world, the appreciation rate
	of the call option equals the risk-free rate $r$. The volatility
	coefficient is $\s \eta^c$, and hence stochastic.


\section{Parameters of the Black-Scholes Model}
\subsection{The Greeks}

Greeks
	We will now analyse the impact of the
	underlying parameters in the standard Black-Scholes model on the
	prices of call and put options. The Black-Scholes option values
	depend on the (current) stock price, the volatility, the time to
	maturity, the interest rate and the strike price. The
	sensitivities of the option price with respect to the first four
	parameters are called the {\it Greeks} and are widely used for
	hedging purposes. We can determine the impact of these parameters
	by taking partial derivatives. } BS-formula}
	Recall the Black-Scholes formula
	for a European call:
		$$
		\pi^{call}(0) = C(S,T,K,r,\s) = S \Phi(d_1(S, T)) - K e^{-rT}
		\Phi(d_2(S, T)),
		$$
	with the functions $d_1(s,t)$ and $d_2(s,t)$ given by
		$$
		\begin{array}{lll}
		d_1(s,t) &=&\DSE \frac{\log(s/K) + (r +
		\frac{\sigma^2}{2})t}{\sigma \sqrt{t}},\\*[12pt] d_2(s,t) &=&\DSE
		d_1(s,t) -\sigma \sqrt{t}= \frac{\log(s/K) + (r -
		\frac{\sigma^2}{2})t}{\sigma \sqrt{t}}.
		\end{array}
		$$

		$$
		\begin{array}{lllll}
		\Delta &:=&\DSE \frac{\partial C}{\partial S} &=&\DSE \Phi(d_1) >
		0,\\*[12pt] 
		{\cal V} &:=&\DSE \frac{\partial C}{\partial \s} &=&
		\DSE S \sqrt{T} \varphi(d_1) >0,\\*[12pt]
		 \Theta &:=&\DSE \frac{\partial C}{\partial T} &=& 
		 \DSE \frac{S \s}{2 \sqrt{T}} \varphi(d_1) + K r e^{-rT} \Phi(d_2) >0,\\*[12pt] 
		\rho &:=&\DSE \frac{\partial C}{\partial r} &=& \DSE T K e^{-rT} \Phi(d_2) >0,\\*[12pt] 
		\Gamma
		&:=&\DSE \frac{\partial^2 C}{\partial S^2} &=& \DSE
		\frac{\varphi(d_1)}{S \s \sqrt{T}} >0.\\*[12pt]
		\end{array}
		$$
	
	From the definitions it is clear that $\Delta$ -- delta -- measures the change in the value of the
	option compared with the change in the value of the underlying
	asset, ${\cal V}$ -- vega -- measures the change of the option
	compared with the change in the volatility of the underlying, and
	similar statements hold for $\Theta$ -- theta -- and $\rho$ -- rho

	Likewise, in order to quantify the risk associated with an instrument, one looks at
	\emph{ how much the price of the instrument changes if one of the underlying
	risk drivers changes} its value. Those risk measures are also called
	Greeks.
	
	Mathematically speaking, the Greeks are just the derivative of the price
	of the instrument with respect to the value of the risk driver.
	Once those Greeks are known for a portfolio, one can easily calculate how
	much the value of an option or a portfolio changes \emph{marginally}, if one of
	the variables changes \emph{marginally}, all others remain fixed.
	
	Again, the most important Greeks are Delta, Gamma, Vega, Rho, and Theta.


The Delta
	Delta is the derivative of the instrument price with respect to the price
	of the underlying.
	
	The delta of the underlying security is one.
	
	If the payoff is not linear (for example if the instrument is an option),
	Delta is not constant.
	
	As Delta is the derivative with respect to the underlying, it tells how
	much the value of the instrument changes if the value of the underlying changes
	marginally.
	
	Thus in a continuous time model, Delta is the amount of the underlying
	needed to be sold in order to offset the price change of the instrument. The
	ability to neutralize the trading book with respect to price changes in the
	underlying makes Delta the most important Greek.


Delta Hedging I
	Assume that you have an options position with delta $\Delta$. This means
  that if the price of the underlying moves by a (very) small amount $\epsilon$, the
  price of the option position moves approximately by $\Delta \epsilon$.
  
	A \emph{delta hedge} consists of selling $\Delta$ units of the
  underlying (buying if $\Delta <0$) and gives a portfolio that does not
  change its value if the price of the underlying changes by a marginal amount.
  The portfolio has a delta of zero and is called \emph{delta neutral}.


Delta Hedging II
	BUT: Delta is not constant, so the portfolio has to be rebalanced.
  
	In the Black-Scholes model, a delta hedge with continuous rebalancing is
  a perfect hedge.
  
	In practice, continuous rebalancing is not possible so that the
  portfolio is not protected against larger market movements. Transaction
  costs and bid-ask spread also result in losses.


Delta in the Black-Scholes Model
	In the Black-Scholes model, we have computed the Greeks explicitly for
	European call options. We have:
		\begin{align*}
		\Delta = \frac{\partial}{\partial S}Call_{BS}(S,K,\sigma,r,t,T) = \Phi(d_1)
		\end{align*}
	where, as usual, $\Phi$ denotes the c.d.f. of the standard normal distribution and
		\begin{align*}
		d_1 = \frac{\log \left( \frac{S}{K} \right) + \left( r + \frac{\sigma^2}{2}
		\right)(T-t)}{\sigma \sqrt{T-t}}.
		\end{align*}


Delta of a Call Option in the Black-Scholes Model}
%\usepackage{graphics} is needed for \includegraphics
\begin{figure}[htp]
\begin{center}
  \includegraphics[width=0.8\textwidth]{../../../pics/delta}
  \caption{Delta for a European call in the BS model, $T=1$, $r=1\%$,
  $\sigma=20\%$, $K=100$.}
  \label{fig:deltaBS}
\end{center}
\end{figure}


Example: Delta Hedge
	You are long USD $1,000$ in the $104$ call. Interest rate is $5\%$,
	stock price today is $99$, time to maturity $1$ month, and implied volatility is
	$15.7\%$.
		How can you make your portfolio delta neutral by investing in the stock?
		
		You set up the delta neutral portfolio and the stock price jumps to
		
		USD$100$ immediately. What is your P/L for the portfolio?

		Compute the price of the call with the BS formula:
			\begin{align*}
			Call_{BS} = 0.3858 \text{USD}.
			\end{align*}
   The position consists of $N=1,000/Call_{BS}=2592$ call options.
   The delta of each option is
			\begin{align*}
				\Delta &= \Phi(d_1) =\Phi\left(\frac{\log \left( S/K \right) + (r+\sigma^2/2)(T-t)
				}{\sigma\sqrt{T-t}}\right) \\
					&= \Phi\left(\frac{\log \left( 99/104 \right) + (0.05+0.157^2/2)(1/12)
				}{0.157\sqrt{1/12}}\right)\\
					&= 0.1654.
			\end{align*}
   
	The delta of the position is long $\Delta_P=N\cdot \Delta=428.70$.
  
	The stock has $\Delta=1$.
   
	To make the position delta neutral, you have to enter a short position
  of $428.70$ shares.

	The loss from the short position in the stock is $428.70 \cdot
  1=428.70$.
  
	To compute the gain from the long options position, we have to compute
  the option price for $S=100$. Using the BS formula, we obtain
		\begin{align*}
			Call_{BS}(S=100) = 0.5808.
		\end{align*}
   
	The gain from the options position is $2592\cdot(0.5808-0.3858)=505.37$.
  
	Our profit is $505.37-428.70=76.67$.


The Gamma
	Gamma is the second derivative of the instrument price with respect to
  the price of the underlying.
  
	If the instrument has a linear payoff, Gamma is zero.
  
	As delta is the first derivative of the option price with respect to the
  underlying, gamma is the derivative of delta with respect to the underlying
  and thus measures, how much Delta changes if the underlying changes.
  
	This is an important information in risk management as it tells the
  trader how much of the underlying he has to buy or sell if the underlying
  itself changes price.
  
	Geometrically, gamma might be seen as the slope of Delta.


Gamma Neutral Portfolios
As the delta of a portfolio changes with the price of the underlying, a delta
hedge has to be rebalanced frequently.
	A delta hedge for a portfolio with a high (absolute) gamma has to be
  monitored more carefully than a delta hedge of a portfolio with gamma close to
  zero.
  
	To decrease the hedging error, a trader might want to make a delta
  neutral portfolio gamma neutral.
  
	The gamma cannot be changed by investing in the underlying because its
  gamma is zero.
  
	Strategy: Make the portfolio gamma neutral by investing in an option,
  then make it delta neutral by investing in the underlying.


Gamma in the Black-Scholes Model
	The gamma of a European call option in the Black-Scholes model is given by
	\begin{align*}
		\Gamma = \frac{\partial^2}{\partial S^2}Call_{BS}(S,K,\sigma,r,t,T) =
		\frac{\varphi(d_1)}{S\sigma \sqrt{T-t}}.
	\end{align*}
	Here, $\varphi(x)$ denotes the p.d.f. of the standard normal distribution.


Gamma of a Call Option in the Black-Scholes Model
	\begin{figure}[htp]
	\begin{center}
		\includegraphics[width=0.8\textwidth]{../../../pics/gamma}
		\caption{Gamma for a European call in the BS model, $T=1$, $r=1\%$,
		$\sigma=20\%$, $K=100$.}
		\label{fig:gammaBS}
	\end{center}
	\end{figure}


Gamma of a Call Option in the Black-Scholes Model
	\begin{figure}[htp]
	\begin{center}
		\includegraphics[width=0.8\textwidth]{../../../pics/gamma2}
		\caption{Gamma for a European call in the BS model, $T=10$, $r=1\%$,
		$\sigma=20\%$, $K=100$.}
		\label{fig:gamma2BS}
	\end{center}
	\end{figure}


Example: Gamma Neutral Portfolio
	You are in the same position as in the previous example. Additionally, you can
	trade in the $97$ call.
		Put together a portfolio that is delta and gamma neutral.
		
		What is your P/L if the stock price jumps to $100$?

		The gamma of the $104$ call is
			\begin{align*}
			\Gamma_{104C} = \frac{\varphi(d_1)}{S\sigma \sqrt{T-t}} = 0.0554.
			\end{align*}
		The gamma of the $97$ call is
			\begin{align*}
			\Gamma_{97C} = \frac{\varphi(d_1)}{S\sigma \sqrt{T-t}} = 0.0758.
			\end{align*}
		To make the portfolio gamma neutral, we have to build up a position of
		$n$ $97$ calls so that $2592\cdot 0.0554 +n\cdot 0.0758 = 0$.
		
		We find that $n=-1894.74$ makes the portfolio gamma neutral, i.e., we
		sell $1894.74$ units of the $97$ call.

		We compute the delta of the portfolio consisting of the two positions in
		the calls.
   
		The $97$ call has a delta of $\Delta_{97C}=0.7139$.
		
		The delta of the position is
			\begin{align*}
				\Delta_{P'} = 2592 \cdot 0.1654 - 1894.74 \cdot 0.7139 = -924.02
			\end{align*}
		We have to buy $924.02$ units of the underlying to make the portfolio
		delta neutral. By buying the underlying, we do not change the gamma of the
		portfolio, it remains zero.

		The price of the $97$ call for $S=99$ is $Call_{BS,97}(S=99)=3.2235$.
		
		The price of the $97$ call for $S=100$ is $Call_{BS,97}(S=100)=3.9735$.
		
		The P/L from the position in the $97$ call is
		$-1894.97*(3.9735-3.2235)=-1421.11$.
		
		The P/L from the position in the stock is $924.02$.
		
		The portfolio P/L is $924.02-1421.11+505.37=-8.28.$


The Vega
	Vega is the derivative of the instrument price with respect to
  implied volatility.
  
	Thus, vega indicates how much the option price changes if the implied
  volatility changes.
  
	Vega is not a Greek letter.


Vega in the Black-Scholes Model
The vega of a European call option in the Black-Scholes model is given by
	\begin{align*}
		\nu = \frac{\partial}{\partial \sigma}Call_{BS}(S,K,\sigma,r,t,T) =
		S\sqrt{T-t} \varphi(d_1).
	\end{align*}
	Therefore, we have
	\begin{align*}
		\nu = S^2 \sigma (T-t) \Gamma.
	\end{align*}


Vega of a Call Option in the Black-Scholes Model II
\begin{figure}[htp]
\begin{center}
  \includegraphics[width=0.8\textwidth]{../../../pics/vega}
  \caption{Vega for a European call in the BS model, $T=1$, $r=1\%$,
  $\sigma=20\%$, $K=100$.}
  \label{fig:vegaBS}
\end{center}
\end{figure}


Vega of a Call Option in the Black-Scholes Model III}
\begin{figure}[htp]
\begin{center}
  \includegraphics[width=0.8\textwidth]{../../../pics/vega_impliedvol}
  \caption{Vega for a European call in the BS model, $T=1$, $r=1\%$, $K=100$.}
  \label{fig:vega2BS}
\end{center}
\end{figure}


The Theta
	Theta is the derivative of the instrument price with respect to
  time to maturity.
  
	Thus, theta indicates how much the option price changes as time moves
  closer to maturity.
  
	Theta is usually negative and therefore is also called \emph{time decay}
  or \emph{rent}.
  
	Note: The passage of time is deterministic. It does not make sense to
  hedge against these losses.



Theta of a Call Option in the Black-Scholes Model
\begin{figure}[htp]
\begin{center}
  \includegraphics[width=0.8\textwidth]{../../../pics/theta}
  \caption{Theta for a European call in the BS model, $\sigma=20\%$, $r=1\%$,
  $K=100$.}
  \label{fig:thetaBS}
\end{center}
\end{figure}



The Rho
	Rho is the derivative of the instrument price with respect to the risk
	free interest rate.
	
	It measures, how the price of the instrument changes if the interest
  rate changes.
  
	Rho is particular important for fixed-income portfolios. If the hedging
  portfolio of the instrument consists of a large portion of debt and only a
  small amount of initial capital, the influence of changes in the risk free
  rate might become quite big.


Rho of a Call Option in the Black-Scholes Model
\begin{figure}[htp]
\begin{center}
  \includegraphics[width=0.8\textwidth]{../../../pics/rho}
  \caption{Rho for a European call in the BS model, $\sigma=20\%$, $T=1$,
  $K=100$.}
  \label{fig:rhoBS}
\end{center}
\end{figure}


\subsection{Volatility}
Vega
	One of the main issues raised by the Black-Scholes formula is the
	question of modelling the volatility $\s$. Before we can implement
	the Black-Scholes formula to price options, we have to estimate $\s$.

	Because the formula is explicit, we can, determine the ${\cal V}$
	-- the partial derivative
		$$
		{\cal V} = \partial C/\partial \s,
		$$
	finding
		$$
		{\cal V} = S \sqrt{T} \varphi(d_1).
		$$
	The important thing to note here is that vega is always positive.

Implied Volatility
	Next, since vega is positive, $C$ is a continuous -- indeed,
	differentiable -- strictly increasing function of $\s$.  Turning
	this round, $\s$ is a continuous (differentiable) strictly
	increasing function of $C$; indeed,
		$$
		{\cal V} = \frac{\partial C}{\partial \s}, \A \mbox{so} \A
		\frac{1}{\cal V} = \frac{\partial \s}{\partial C}.
		$$

	Thus the value $\s = \s(C)$ corresponding to the actual value $C =
	C(\s)$ at which call options are observed to be traded in the
	market can be read off.  The value of $\s$ obtained in this way is
	called the {\it implied volatility}. }


\section{Variants}

Dividend-Paying Assets
	Let $S_t$ be a dividend-paying stock with continuous-dividend rate
	$\rho$. To price a derivative with expiry $T$ we set
		$$
		X_t=e^{-\rho(T-t)}S_t.
		$$
	Then $X_t$ is a non-dividend paying asset and must have the
	dynamic
		$$
		dX_t= rX_tdt+\sigma X_t dW_t.
		$$
	We can thus compute the dynamics of $S_t$ using It{\^o}'s formula
	(or the product rule)
		$$
		dS_t=(r-\rho)S_tdt+\sigma S_tdW_t
		$$
	The usual calculation give the European call option price
		$$
		\begin{array}{lll}
		C(t) &=&\DSE S(t)e^{-\rho(T-t)} \Phi(d_1(S(t), T-t))\\*[12pt] &&- K
		e^{-r(T-t)} \Phi(d_2(S(t), T-t)).
		\end{array}
		$$
	The functions $d_1(s,t)$ and $d_2(s,t)$ are given by
		$$
		\begin{array}{lll}
		d_1(s,t) &=&\DSE \frac{\log(s/K) + (r -\rho +
		\frac{\sigma^2}{2})t}{\sigma \sqrt{t}},\\*[12pt] d_2(s,t) &=&\DSE
		 \frac{\log(s/K) + (r - \rho-
		\frac{\sigma^2}{2})t}{\sigma \sqrt{t}}
		\end{array}
		$$
		
		
Time-dependent Volatility
	Assume
		$$
		dS_t=r S_t dt + \sigma(t) S_t dW_t
		$$
	We can solve this SDE an find
		$$
		S_t=S_0\exp\left\{ \int_0^t\sigma(s)dW_s + \left(rt +
		\frac{1}{2}\int_0^t\sigma^2(s)ds \right) \right\}.
		$$
	Now
		$$
		\int_0^t\sigma(s)dW_s \sim N\left(0,
		\sqrt{\int_0^t\sigma^2(s)ds}\right)
		$$
	and we set
		$$
		\bar{\sigma}(t,T)=\sqrt{\frac{1}{T-t}\int_t^T\sigma^2(s)ds}.
		$$

	The usual calculation give the European call option price
		$$
		\begin{array}{lll}
		C(t) &=&\DSE S(t) \Phi(d_1(S(t), T-t))\\*[12pt] &&- K e^{-r(T-t)}
		\Phi(d_2(S(t), T-t)).
		\end{array}
		$$
	The functions $d_1(s,t)$ and $d_2(s,t)$ are given by (with
	appropriate time parameter)
		$$
		\begin{array}{lll}
		d_1(s,t) &=&\DSE \frac{\log(s/K) + (r +
		\frac{\bar{\sigma}^2}{2})t}{ \bar{\sigma}\sqrt{t}},\\*[12pt]
		d_2(s,t) &=&\DSE
		\frac{\log(s/K) + (r -
		\frac{\bar{\sigma}^2}{2})t}{\bar{\sigma} \sqrt{t}}
		\end{array}
		$$
