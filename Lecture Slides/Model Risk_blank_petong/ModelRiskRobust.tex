% !TEX root = ModelRisk_spring14UDE.tex

\part{Model Validation and Model Comparison}                          % 2.te Folie Inhaltsverzeichnis

%Analytic examples of Model Risk in Credit Modelling

\section{Model Comparison}

% frametitle
{How to compare models?}
We may have two "equally sound" but different models, which give considerably different prices to derivatives (see Schoutens).



% begin itemize




	Reasonableness: the models should capture the relevant factors which are relevant for the payoff, model assumptions should be realistic.


	Calibration: use liquid markets which contain relevant information for pricing the payoff (Schoutens: plain vanilla options). Models should produce results consistent with market practice.


% end itemize



% frametitle
{Comparing equally sound models}
What are the crucial risk factors for the payoff to be priced (these are more important than the relevant factors)?


% begin itemize




	Reality Check: analyse the crucial risk factors to decide which ones will be important for future price movements. This is not so important in case we do not have a liquid market and intend to keep a derivative until maturity.


	Market Intelligence: perform reverse engineering on prices from other market participants we observe in order to judge the market opinion.


% end itemize



% frametitle
{Example: Leveraged Credit Default Note}


% begin itemize




	The banks sells credit default protection to a client (notational 1).


	For a leveraged credit default note the banks sells protection for a notational $l_v>1$.


	Bank pays $l_v \times S_T(0)$, with $S_T(0)$ the spread for the reference entity for maturity at $T$.


	At maturity clients receives notational in case of no default. Otherwise, the client will receive the notational minus the leveraged default loss (nothing if this is below 0).


% end itemize



% frametitle
{Gap Risk}


% begin itemize




	Which fee should the client pay to the bank?


	In case of default the clients loss is not larger than the notational $1$, but the loss may be higher $I_v \times l_d$, with $l_d$ the loss-given-default.


	The bank has to cover a gap $l_v \times l_d -1$.


	Pricing the gap risk by risk neutral valuation leads to
$$
\EX\left[\IF_{\{\tau_d < T\}} p(0,\tau_d)(l_v \times l_d -1)^+\right],
$$
with $\tau_d$ the default time.



% end itemize



% frametitle
{Specifying a Trigger}


% begin itemize




	To mitigate the gap risk and thus reduce the price of the leveraged note a trigger can be introduced: the contract is terminated early if $S_T(t)$, the spread of the CDS of the reference entity touches a certain level  $s_{tr}$.


	We set $\tau_{tr}= \inf\{t: S_T(t) \geq s_{tr}\}$, then the leveraged note is unwind at $\tau_{term}= \min\{\tau_{tr}, \tau_{d} \}$.


	The bank can try to set $s_{tr}$ in such a way that the cost of unwinding the leveraged note is just the notational, i.e. there should be no gap risk.


% end itemize



\section{Credit Risk Models}
\subsection{Credit Default Swap (CDS)}

% frametitle
{Definition}
A credit default swap is an exchange of a periodic payment against
a one-off contingent payment if some credit event occurs on a
reference asset.

\begin{table}[htb]
\begin{center}
\begin{tabular}{ccc}
& {\small contingent payment} & \\
Protection& $\longleftarrow $ & Protection\\
Buyer & $\longrightarrow $& Seller\\
& {\small periodic fee} &
\end{tabular}
\end{center}
\end{table}

% frametitle
{Structure}

The ingredients of the basic structure are the specification of


% begin itemize




	maturity $T$: usually from one to ten years,


	
underlying: corporate or sovereign,

	<3-> credit event:
default, bankruptcy, downgrade.


% end itemize



% frametitle
{Valuation}



% begin itemize




	Assume a deterministic term structure of interest rates with short rate $r>0$.


	Let $S_T(0)$ be the fixed coupon that the protection buyer pays at coupon dates
$t_i, i=1,\ldots, n; t_n=T$.


	The
payment continues until either default or maturity. In case of
default, assume that the payment from the protection seller to the
protection buyer is equal to the difference between the notional
amount of the bond and the recovery value $\delta$, i.e. the loss given default is paid.


	The fixed side
of the payment is set so that contract value is zero at
initiation.


% end itemize



% frametitle
{Valuation}

Thus, since the cash flow at coupon date $t_i$ for the protection
buyer is $S_T(0)\IF_{\{\tau>t_i\}}$ and the payment for the protection
seller at time of default $\tau$ is $(1-\delta)\IF_{\{\tau\leq
T\}}$, we obtain
$$
(1-\delta)\EX^*\left(e^{-r\tau}\IF_{\{\tau\leq
T\}}\right) = \DSE\sum_{i=1}^n e^{-rt_i} S_T(0)\EX^*\left(\IF_{\{\tau
>t_i\}}\right),
$$
and so
$$
S_T(0)=\frac{\DSE (1-\delta)\EX^*\left(e^{-r\tau}\IF_{\{\tau\leq
T\}}\right)} {\DSE\sum_{i=1}^n e^{-rt_i} \EX^*\left(\IF_{\{\tau
>t_i\}}\right)}.
$$

% frametitle
{Spread Calculation}



% begin itemize




	At a time $t_{j-1} < t \leq t_j$ we have
$$
S_T(t)=\frac{\DSE (1-\delta)\EX^*\left(e^{-r\tau}\IF_{\{\tau\leq
T\} }  | \{\tau >t\}\right)} {\DSE\sum_{i=j}^n e^{-rt_i} \EX^*\left(\IF_{\{\tau
>t_i\}}| \{\tau >t\}\right)}.
$$



	We need to obtain
$$\prb^*\left(\tau\leq T\ | \tau >t \right)$$ and
$$\prb^*\left(\tau >t_i | \tau >t\right)$$ within our models.


	So, to evaluate the gap risk these quantities are crucial.  % end itemize

\subsection{Structural Models}

% frametitle
{Basic Merton Model}

Merton (1974)
assumes that a firm is financed by
equity and a single zero-coupon bond with notational amount (face
value) $F$ and maturity $T$.

The firm's value is given by
$$
dV(t) = (r-\delta) V(t)  dt + \sigma V(t) dW(t)
$$
under an equivalent martingale (pricing) measure $\mathbb{P^*},$ with $r,
\sigma$ constant, $W$ Brownian motion and constant payout
(dividend) rate $\delta,$ which may be negative (i.e. pay-in).

% frametitle
{Basic Merton Model}

Default is only possible at maturity. There are two possibilities:
$$
V_T \geq F, \quad \mbox{thus} \quad p^d(T,T)=F
$$
or
$$
V_T < F, \quad \mbox{thus} \quad p^d(T,T)=V_T.
$$

% frametitle
{Equity Owners}

For equity owners the payoff  is
$$
S_T=\max\{V_T-F,0\}
$$
thus stocks can be viewed as call options on the value of the firm
with ($\bar{V}_t = e^{-\delta(T-t)}V_t$)
$$
S_t=C_E(\bar{V}_t,F)=e^{-\delta(T-t)}V_t\Phi(d_1)-Fe^{-r(T-t)}\Phi(d_2)
$$
and
$$
d_1=\frac{\log(\bar{V}_t/F)+(r-\delta+\sigma^2/2)(T-t)}{\sigma\sqrt{T-t}}=d_2+\sigma\sqrt{T-t}.
$$

% frametitle
{Bond Owners}

For bond owners the payoff is
$$
p^d(T,T)=F-\max\{F-V_T,0\}.
$$
This can be viewed as the difference
of a risk-free payment and a put option on the value of the firm
with
$$
p^d(t,T)=Fe^{-r(T-t)}-P_E(\bar{V}_t,F),
$$
where
$$
P_E(\bar{V}_t,F)=-V_t e^{-\delta(T-t)}
\Phi(-d_1(\bar{V}_t,T-t))+Fe^{-r(T-t)}\Phi(-d_2(\bar{V}_t,T-t)).
$$
So
$$
p^d(t,T)= V_t e^{-\delta(T-t)} \Phi(-d_1(\bar{V}_t,T-t))+
Fe^{-r(T-t)}\Phi(d_2(\bar{V}_t,T-t)).
$$

% frametitle
{Bond Credit Spreads}

For spreads  $S(0,T)$ we use $p^d(0,T)=e^{-(r+S(0,T))T}$ and
find
$$
S(0,T)=\frac{1}{T}\log\left\{\frac{1}{l_0}\Phi(-d_1)+\Phi(d_2)\right\},
$$
with
$$
l_t=\frac{F e^{-rt}}{V_t}
$$
the leverage ratio.

% frametitle
{Calculated Bond Credit Spreads  }

\includegraphics<1>[height=6cm]{../../../pics/merton_spreads.pdf}%

\subsection{Reduced Form  Models}

% frametitle
{Counting Process}

Define the stopping time (= time of default) $$\tau = \inf \{ t :
X(t) = D \}
$$
and the counting process (Poisson)
$$
N(t) = \begin{cases} 0 \;\;\; \mbox{if $\tau >t$}\\ 1 \;\;\; \mbox{if $\tau \leq t$}\end{cases}
$$
$N(t)$ is the number of defaults up to and including time $t$.

% frametitle
{Calculation in Standard Model}

Let $r_t$ be the (stochastic) short rate, $\delta$ the (stochastic, independent) recovery rate.
The valuation formula for a corporate bond is
$$
p^d (t,T) =  \EX^* \left[ \left. e^{-\int_t^T r_sds} \left(
\IF_{\{\tau >T\}} + \delta \IF_{\{\tau \leq T\}}\right)\right| \F_t\right]
$$
($\F_t$ is the financial market
filtration with all relevant information)
$$
=\EX^* \left[ \left. e^{-\int_t^T r_sds} \left(
\delta+ (1-\delta) \IF_{\{\tau >T\}} \right)\right| \F_t\right]
$$
using independence and the properties of the counting process
$$
=\EX^* \left[ \left.  \delta e^{-\int_t^T r_sds} \right| \F_t\right]
+\EX^* \left[ \left. (1-\delta) e^{-\int_t^T (r_s+ \lambda_s) ds} \right| \F_t\right]
$$

% frametitle
{Calculation in Standard Model II}

Now $\delta=0$ i.e. zero recovery
$$
p^d (t,T) =  \EX^* \left[ \left.e^{-\int_t^T (r_s+ \lambda_s) ds} \right| \F_t\right]
$$
with $r_s+\lambda_s$ a default-adjusted rate.

Constant (or independent) interest rate gives

$$
p^d (t,T) =  p(t,T) \EX^* \left[ \left.e^{-\int_t^T \lambda_s ds} \right| \F_t\right].
$$

% frametitle
{Survival Probability}

The standard "memory-less" property of exponential random variables gives for deterministic intensity
$$
\prb(\tau > T | \tau >s) =   e^{-\int_s^T \lambda_udu}.
$$
This coincides with the forward survival probabilities at time $t=0$
$$
\frac{\prb(\tau > T)}{\prb(\tau>s)}=  e^{-\int_s^T \lambda_udu}.
$$

\section{Assessing the Gap Risk}

% frametitle
{Structural Model}



% begin itemize




	In Merton's model the survival probability is dependent on the firm's value process and the barrier level (induced by the trigger).
$$
V(t) \A \downarrow \A \Rightarrow \prob(\tau > s |V(t)) \A \downarrow.
$$


	So $S_T(t)$ increases as
$$
S_T(t)=\frac{\DSE l_d e^{-r\tau} \prb^*\left(\tau\leq T  | \tau >t\right)} {\DSE\sum_{i=1}^n e^{-rt_i} \prb^*\left(\tau >t_i| \tau >t\right)}
=\frac{\DSE l_d e^{-r\tau} \prb^*\left(\tau\leq T  | V_t \right)} {\DSE\sum_{i=1}^n e^{-rt_i} \prb^*\left(\tau >t_i| V_t\right)}.
$$


	No gap risk


% end itemize



% frametitle
{Reduced Form Model}



% begin itemize




	
For deterministic intensity we know the behaviour of the spread  in advance!


	
Even with stochastic intensity the default event and the spread level show little correlation.


	So, in any case  -- maximal gap risk!


% end itemize



