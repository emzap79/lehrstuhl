% !TEX root = QCF_ss14UDE.tex
\section{Capital Market and Renewable Energy Projects}
\subsection{Carbon Bonds}

% frametitle
{Carbon Revenue Bonds}


% begin itemize


\item<1-> To finance high initial cost for Renewable Energy (RE) projects future returns of the project are securitized
\item<2->


% begin itemize


\item Sell future electricity from RE project
\item Sell environmental credits from RE project


% end itemize


\item<3-> Only revenue from environmental credits is used for bond
\item<4-> Rigorous forecast analysis of revenues, sensitivity tests and risk analysis is required.


% end itemize



% frametitle
{Structure of Bond}


% begin itemize


\item<1-> pass-through: all revenues are directly passed trough to the owner of the bond
\item<2->


% begin itemize


\item maturity: T
\item revenues in year 1: $c_i$
\item rate of return: $r$
\item initial price: $x$


% end itemize


\item<3-> Fair price
$$
x= \sum_{i=1}^T \frac{c_i}{(1+r)^i}
$$


% end itemize



% frametitle
{EUA Time Series}
\begin{figure}[h!]
\centering
\includegraphics[width=0.9\textwidth, height=0.7\textheight]{../../../pics/EUA-TimeSeries.pdf}
%\caption{EUA Time Series}
\label{fig:EUA-TS}
\end{figure}

% frametitle
{QQ-Plots for EUA fits}
\begin{figure}[h!]
\centering
\includegraphics[width=0.9\textwidth, height=0.7\textheight]{../../../pics/QQ-CarbonFit.pdf}
%\caption{EUA fits}
\label{fig:EUA-fits}
\end{figure}

% frametitle
{Carbon Bond Histogram}
\begin{figure}[h!]
\centering
\includegraphics[width=0.9\textwidth, height=0.7\textheight]{../../../pics/carbon-bond-histogram.pdf}
%\caption{Carbon Bond Histogram}
\label{fig:Carbon-Bond-Histogram}
\end{figure}

\subsection{Impact of Carbon Markets on Investment Markets}

% frametitle
{Financial Implications of Carbon Policies}


% begin itemize


\item<1-> Modern risk management has to include the consequences of climate change.
\item<2-> Regulatory risk (reduction targets for carbon emissions) transforms into financial risk  for several asset classes.
\item<3-> Carbon inefficient firms tend to have a higher credit spread and higher refinancing costs.


% end itemize



% frametitle
{Factors Affecting Carbon Risk}


% begin itemize


\item<1-> Energy intensity and fuel mix -- firms that are dependent on fossil fuels face higher costs.
\item<2-> Direct, indirect and embedded emissions of a firm's product affect market position.
\item<3-> Marginal abatement costs.
\item<4-> Technology trajectory -- progress in adapting low-carbon emission technologies.


% end itemize



% frametitle
{Motivation for Investors to Invest in Carbon-related Assets}


% begin itemize


\item<1-> Financial Motivation


% begin itemize


\item Portfolio diversification
\item Potential fundamental price appreciation of carbon
\item Hedging financial risk due to carbon price increases


% end itemize


\item<2-> Green Motivation


% begin itemize


\item Compliance with UN principles of responsible investment (UN PRI)
\item Public opinion, behaviour as corporate citizen
\item Incentivizing the corporate sector by taking carbon credits from the market


% end itemize




% end itemize



% frametitle
{Risk In Renewable Energy Companies}


% begin itemize


\item<1-> Costs: as the costs of producing RE come down while the costs of producing fossil fuels rise, a
substitution will occur.
\item<2-> Capital: Government and private capita
\item<3-> Competition: between governments as they try to build greener economies
\item<4-> China: huges efforts to establish a green economy
\item<5-> Consumers: demand products with less impact on the economy
\item<6-> Climate Change: will be tackled by investment in greener technologies.


% end itemize



% frametitle
{CAPM Analysis of RE Companies}


% begin itemize


\item<1-> Empirical evidence shows that RE companies have a $\beta$ close to two.
\item<2-> Model: $i$ firm, $t$ time
$$
R_{it}= \alpha_i + \beta_{it} R_{mt}+\epsilon_{it}
$$
$R$ returns, $\alpha$ component that is independent of the market.
\item<3-> Higher beta values indicate a higher equity cost of capital.Investors must then be compensated
through higher expected returns in order to take on the higher risk. A higher equity cost of capital can affect borrowing costs and
it can also affect the discount rate used in net present value calculations.


% end itemize



% frametitle
{CAPM Analysis of RE Companies II}


% begin itemize


\item<1-> Which factors affect systematic risk (the $\beta$)?
\item<2-> Empirical analysis shows:


% begin itemize


\item Increases in sales growth reduce market risk
\item Increases in oil price returns increases systematic risk


% end itemize


\item<3-> In order to foster RE companies governments can implement policies to increase sales of such companies (e.g. PV-industry and feed-in tariffs).


% end itemize



