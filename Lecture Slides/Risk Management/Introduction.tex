% !TEX root = riskmanagement_ws13UDE.tex
\part{Introduction}
\section{The Need For Risk Management?}
\subsection{Illustrative Examples}


\frame{\frametitle{Nick Leeson - Barings Bank}
\begin{itemize}
\item<1-> 
London's 232-year-old merchant bank Baring Brothers fails February 26 after its
Singapore trader Nicholas W. Leeson, 27, loses upwards of \$1 billion. Leeson
was supposed to have arbitraged the difference between the Nikkei 225 futures
prices traded on the Singapore, Osaka and Tokyo exchanges.
\item<2-> He has risked
twice the bank's net worth in speculations on the Tokyo exchange and lost almost
\$1.4 billion.
\item<3-> In the aftermath of Leeson's activity, Barings collapsed and was
purchased by the Dutch bank/insurance company ING for the nominal sum of \pounds
1.
\end{itemize}

}





\frame{\frametitle{What did we learn?}
\begin{center}
\includegraphics[scale=0.45]{../../../pics/kerviel1.pdf}
%\includegraphics[height=\textheight, width=7cm]{../pics/kerviel1.pdf}
\end{center}
}

\frame{\frametitle{J{\'e}r{\^o}me Kerviel - Soci{\'e}t{\'e} G{\'e}n{\'e}rale}
\begin{itemize}
\item<1-> Kerviel violated for more than two years  \textcolor{red}{his risk limits }.
\item<2->  Soci{\'e}t{\'e} G{\'e}n{\'e}rale made in Q4 2007 a profit of 1,4 Mrd. Euro on his trade.
\item<3-> To hedge his portfolio Kerviel placed in December 2007 eight big transactions with a total volume of 50 Mrd. Euro. His position were discovered on January,  18th 2008  and closed. Between  January 21rst and 23rd  Soci{\'e}t{\'e} G{\'e}n{\'e}rale made a loss of 4,9 Mrd. on this trades.
\end{itemize}
}

\frame{\frametitle{Dangers }
Major reasons of the collapse of risk management systems:
\vspace{0.2cm}
\begin{itemize}
  \item<1-> Lack of internal checks and balances
  \item<2-> Lack of understanding of the business
  \item<3-> Poor supervision of employees
  \item<4-> Lack of a clear reporting line
\item<5->$\Rrightarrow$ It is important \textcolor{red}{to define unambiguous risk limits and monitor carefully that these limits are adhered to}.
\end{itemize}

}


\frame{\frametitle{ Quotes on Risk Management (tools) }
{\it Derivatives are weapons of financial mass destruction.}\\*[6pt]
Warren Buffet\\*[12pt]

{\it Instruments that are more complex and less transparent -- such as credit-default swaps, collateralized debt obligations, and credit-linked notes -- have been developed and their use has grown very rapidly in recent years. The result? Improved credit-risk management together with more and better risk-management tools appear to have significantly reduced loan concentrations in telecommunications and, indeed, other areas and the associated stress on banks and other financial institutions.}\\*[6pt]

Alan Greenspan



}

\subsection{Classification of Risk}
\frame{\frametitle{Risk Types}
\begin{figure}
	\centering
		\includegraphics[width=0.8\textwidth]{../../../pics/risk_SST}
	\label{fig:Risk_map1}
\end{figure}
}

\frame{\frametitle{Risk Types}
\begin{figure}
	\centering
		\includegraphics[width=0.8\textwidth]{../../../pics/DeutscheBank-RiskTypes}
	\label{fig:Risk_map2}
\end{figure}
}

\frame{\frametitle{Sources of risk for a utility}
For the risk management process it is important to identify all main sources of risk. Those risks might be graphically illustrated in a risk map.
\begin{figure}
	\centering
		\includegraphics[width=0.8\textwidth]{../../../pics/Risk_map}
	\label{fig:Risk_map3}
\end{figure}
}

\frame{\frametitle{Example: Credit Risk Model}
%\usepackage{graphics} is needed for \includegraphics
\begin{figure}[htp]
\begin{center}
  \includegraphics[height= 6cm]{../../../pics/credit-bild2}
\end{center}
\end{figure}

}

\frame{\frametitle{Example: Credit Risk Model}
%\usepackage{graphics} is needed for \includegraphics
\begin{figure}[htp]
\begin{center}
  \includegraphics[width=0.9\textwidth]{../../../pics/credit-bild1}
\end{center}
\end{figure}

}


\frame{\frametitle{Basel II and Solvency II}
\begin{itemize}
\item<1-> Risk based supervision (qualitative)
\item<2-> Adequate capital requirements according to individual risk profile (chance of using internal models)
\item<3-> More emphasis on risk management, internal controls \& Corporate Governance
\item<4-> Integration of models in risk management and internal control systems
\item<5-> Developing frameworks which enable comparison, transparency and consistency
\item<6-> Competition
\end{itemize}
}

\frame{\frametitle{Pillars of Basel II}
\begin{figure}
	\centering
		\includegraphics[width=0.8\textwidth]{../../../pics/baselll-pillars}
	\label{fig:Risk_map6}
\end{figure}
}

\frame{\frametitle{Pillars of Solvency II}
\begin{figure}
	\centering
		\includegraphics[width=0.8\textwidth]{../../../pics/solv-pillars-kpmg}
	\label{fig:Risk_map}
\end{figure}
}

\subsection{Price Processes of Financial Assets}



\frame{\frametitle{Stocks -- Dax, Dax log-Returns}

\includegraphics[width=\textwidth, height=3.7cm]{../../../pics/DAX}\\
\includegraphics[width=\textwidth, height=3.7cm]{../../../pics/DAXlog}
}

\frame{\frametitle{Stocks -- Dow Jones, Dow Jones log-Returns}

%\\ Dow Jones Industrials Total Return Index (DJITR)
\includegraphics[width=\textwidth, height=3.7cm]{../../../pics/DJI}\\
\includegraphics[width=\textwidth, height=3.7cm]{../../../pics/DJIlog}
}


\frame{\frametitle{Gold Futures}
\includegraphics[width=\textwidth, height=3.7cm]{../../../pics/Gold}\\
\includegraphics[width=\textwidth, height=3.7cm]{../../../pics/Goldlog}
}

\frame{\frametitle{Oil  Futures }
\includegraphics[width=\textwidth, height=3.7cm]{../../../pics/WTI}\\
\includegraphics[width=\textwidth, height=3.7cm]{../../../pics/WTIlog}
}


\frame{\frametitle{Term Structures of Interest Rates}
\includegraphics[width=\textwidth, height=6.7cm]{../../../pics/Zinsstruktur}
}


\section{Outlook}
\frame{\frametitle{Questions}
\begin{itemize}
\item<1-> How to measure risk? Measures: Variance? Value-at-Risk? Expected Shortfall?
\item<2-> How to manage risk? Portfolio Risk -- Diversification? Use of Derivatives?
\end{itemize}
}