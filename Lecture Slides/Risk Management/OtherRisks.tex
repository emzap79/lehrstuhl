% !TEX root = riskmanagement_ws13UDE.tex
\part{Other Types of Risks}
\section{Operational Risk}
\subsection{Motivation and Introduction }

\frame{\frametitle{Examples}
\begin{itemize}
\item<1-> Internal fraud: Barings Bank, Societe General
\item<2-> Terror attack (11.September 2001), adverse weather
\item<3-> Software/Hardware problems (Salamon Brothers lost 303 Million USD during a switch of computer technology)
\end{itemize}
}
\frame{\frametitle{Definition of Operational Risk}
\begin{itemize}
\item<1-> Basel Committee January 2001:\\
Operational risk is the risk of loss resulting from inadequate or failed internal processes, people, and systems, or from external events.
\item<2-> This includes people risks, technology and processing risks, physical risks, legal risks, etc,
\item<3-> but excludes reputation risk and strategic risk
\end{itemize}
}
\frame{\frametitle{Types of operational risks}
\begin{itemize}
\item<1-> Internal fraud
\item<2-> External fraud
\item<3-> Employment practices and workplace safety
\item<4-> Clients, products and business practices
\item<5-> Damage to physical assets
\item<6-> Business disruption and system failures
\item<7> Execution, delivery and process management
\end{itemize}
}
\subsection{Calculation of Operational Risk}
\frame{\frametitle{Regulatory Framework}
There are three frameworks to calculate the Operational Risk
\begin{itemize}
\item<1-> Basic Indicator (15\% of annual gross income averaged over the last three years)
\item<2-> Standardized (different percentage for each business line)
\item<3-> Advanced Measurement Approach (AMA)
\end{itemize}
}

\frame{\frametitle{Standardized Approach}
\begin{itemize}
\item<1-> Activities of a bank are separated in 8 business lines with a beta factor specified for each
\item<2-> the annual gross income averaged over the last three years is multiplied by the beta factor
\item<3-> the regulatory capital is obtained by summing over the business line contributions
\end{itemize}
}

\frame{\frametitle{Standardized Approach - Business Lines}
\begin{tabular}{ll}
Corporate finance & 18\%\\
Trading and sales & 18\%\\
Retail banking & 12\%\\
Commercial banking & 15\%\\
Payment and settlement & 18\%\\
Agency services & 15\%\\
Asset management & 12\%\\
Retail brokerage & 12\%\\
\end{tabular}
}

\frame{\frametitle{Advanced Measurement Approach}
\begin{itemize}
\item<1-> Upon satisfying certain qualitative and quantitative criteria a bank is allowed to use the Standardized approach or the AMA
\item<2-> Using AMA a bank tries to find a probability distribution of the losses from Operational Risks in order to calculate a risk measure (typically VaR).
\item<3-> Banks need to estimate their exposure to each combination of type of risk and business line. Ideally this will lead to $7 \times 8=56$ VaR measures that can be combined into an overall VaR measure.
\end{itemize}
}
\frame{\frametitle{Tasks in Calculating AMA}
\begin{itemize}
\item<1-> The loss distribution from Operational Risks is generated by (discrete) loss events together with the size of each loss (loss severity).
\item<2-> We need a counting distribution for the number N of events during a time period (typically $T=1$ year). Typically the Poisson distribution is used where
$$
\prob(N=k) = e^{-\lambda T}\frac{(\lambda T)^k}{k!}
$$
$\lambda$ is the loss intensity and can be estimated using the average number of loss events during a time period.
\item<3-> Loss severity can be based on internal and external historical data. One possibility is to assume a parametric distribution so that we need only estimate the mean and standard deviation of losses.
\end{itemize}
}

\frame{\frametitle{Monte Carlo Simulation (from Hull)}
	\begin{center}
	\includegraphics[height=6.5cm]{../../../pics/OR-MonteCarlo-p1.pdf}
	\end{center}
	
}

\frame{\frametitle{Monte Carlo Algorithm}
\begin{itemize}
\item<1-> Sample from frequency distribution to determine the number of loss events (=k)
\item<2-> Sample k times from the loss severity distribution to determine the loss severity for each loss event $L_1, \ldots, L_k$
\item<3-> Sum loss severities to determine total loss $L^{(i)}= L_1+\ldots+L_k$
\item<4-> Repeat the above M times to get $L^{(i)}$ for $i=1, ...M$ and thus a loss distribution
\end{itemize}
}

\frame{\frametitle{Data Issues}
\begin{itemize}
\item<1-> Use own (internal) data as much as possible, but typically only limited amount of data is available.
\item<2-> For external data two possibilities exist
\begin{itemize}
\item data sharing
\item data vendors
\end{itemize}
\item<3-> Data from vendors is based on publicly available information and therefore is biased towards large losses, so it can only be used to estimate the relative size of the mean losses and standard deviation  of losses for different risk categories.
\item<4-> Furthermore, a scaling approach has to be used
$$
L_A^{est}=L_B^{obs}\times \left(\frac{R_A}{R_B}\right)^\alpha
$$
where $L_{A,B}$ are losses, $R_{A,B}$ are revenues for banks A and B and $\alpha=0.23$ is an estimated non-linearity constant.

\end{itemize}
}
\section{Liquidity Risk}
\subsection{Liquidity Trading Risk}
\frame{\frametitle{Liquidity Trading Risk}
\begin{itemize}
\item<1-> The price received for an asset depends on
\begin{itemize}
\item the mid-market price
\item how much is to be sold
\item how quickly it is to be sold
\item the economic environment
\end{itemize}
\item<2-> The subprime crisis showed that transparency is factor that affects liquidity massively
\end{itemize}
}
\frame{\frametitle{Bid-offer spread}
\begin{itemize}
\item<1-> The proportional bid-offer spread $s$ is defined as
$$
s= \frac{\mbox{offer price - bid price}}{\mbox{mid-market price}}
$$
\item<2-> The cost of liquidation in normal markets is
$$
\sum _{i=1}^{n} \frac{1}{2} s_i \alpha_i
$$
with
\begin{itemize}
\item $n$ the number of positions
\item $\alpha_i$ the position in the ith instrument
\item $s_i$ the proportional bid-offer spread
\end{itemize}
\end{itemize}
}

\frame{\frametitle{Liquidation in a stressed market}
\begin{itemize}
\item<1-> To account for changing market conditions we can view the spread $s$ as a random variable.
\item<2-> The standard approach is to assume a normal distribution. Then the 
cost of liquidation in stressed markets is
$$
\sum _{i=1}^{n} \frac{1}{2} (\mu_i+\sigma_i \lambda) \alpha_i
$$
with
\begin{itemize}
\item $\mu_i$ and $\sigma_i$ expectation and standard deviation of the ith spread
\item $\lambda$ factor to obtain the required confidence level
\end{itemize}
\end{itemize}
}
\frame{\frametitle{Liquidity-adjusted VaR}
\begin{itemize}
\item<1-> We an now simply add the liquidation costs to the standard VaR to obtain liquidity adjusted VaRs.
\item<2-> The liquidity-adjusted VaR is
$$
VaR+ \sum _{i=1}^{n} \frac{1}{2} s_i \alpha_i
$$
\item<2-> The liquidity-adjusted VaR under stress is
$$
VaR+ \sum _{i=1}^{n} \frac{1}{2}  (\mu_i + \lambda \sigma_i) \alpha_i
$$
\end{itemize}
}

\frame{\frametitle{Optimal unwinding of a position}
\begin{itemize}
\item<1-> Typically, the bid-ask spread increases with the size of a position. So a trader needs to develop an optimal strategy taking into account that the market might move during a longer time period. 
\item<2-> Suppose the dollar bid-offer spread as a function of units traded is $p(q)$, that the standard deviation of mid-market price changes per day is $\sigma$, that $q_i$ is amount traded on day i and $x_i$ is amount held on day i (so $x_i = x_{i-1}-q_i$)
\item<3-> The trader's objective might be to choose the $q_i$ to minimize the liquidity-adjusted VaR. This is
$$
\lambda \sqrt{\sum_{i=1}^n \sigma^2x_i^2} + \sum _{i=1}^{n} \frac{1}{2} q_i p(q_i)
$$
such that  $\sum q_i =V$ (the size of the position).
\end{itemize}
}
\subsection{Liquidity Funding Risk}
\frame{\frametitle{Sources of Liquidity}
\begin{itemize}
\item liquid assets
\item ability to liquidate trading positions
\item wholesale and retail deposits
\item lines of credit and the ability to borrow at short notice
\item securitization
\item central bank borrowing
\end{itemize}
}
\frame{\frametitle{Liquidity Black Holes}
\begin{itemize}
\item<1-> A liquidity black hole occurs when most market participants want to take one side of the market and liquidity dries up
\item<2-> Examples: Crash of 1987, Long Term Capital Management
\item<3-> Such black holes may be generated (or accentuated) by positive feedback trading where a trader buys after a price increase and sells after a price decrease.
\item<4-> Reasons for positive feedback trading maybe: Computer models incorporating stop-loss trading, dynamic hedging a short option position, margin calls
\end{itemize}

}

\section{Model Risk}
\frame{\frametitle{Marking Prices of an Instrument to Market}
\begin{itemize}
\item<1->
Use price quoted by market maker (usually financial institutions mark to mid of bid and offer)
\item<2-> Use price at which financial institution has traded product
\item<3-> Use interdealer broker prices
\item<4-> Use interdealer price indications
\item<5-> Use model (marking to model)
\end{itemize}
}

\frame{\frametitle{Model Risk can lead to ..}
\begin{itemize}
\item<1->
Incorrect price at time product is bought or sold
\item<2-> Incorrect hedging

\end{itemize}
}\frame{\frametitle{In Finance Models are used to }
\begin{itemize}
\item<1-> Observe model prices for similar instruments that trade
\item<2-> Imply model parameters and interpolate as appropriate
\item<3-> Value new instrument


\end{itemize}
}\frame{\frametitle{Products}
\begin{itemize}
\item<1-> Linear products: Very little uncertainty about the right model, but mistakes do happen
\item<2-> Standard Products
\begin{itemize}
\item We do not need usually a model to know the price of an actively traded product. The market tells us the price.
\item The model is a communication tool (e.g., implied volatilities are quoted for options)
\item It is also an interpolation tool (e.g., a tool for interpolating between strike prices and maturities)
\end{itemize}
\end{itemize}
}\frame{\frametitle{Models for Non-Standard Products}
\begin{itemize}
\item<1-> In the case of nonstandard models play a key role in both pricing and hedging
\item<2-> It is a good idea to use more that one model whenever possible


\end{itemize}
}\frame{\frametitle{Dangers in Model Building}
\begin{itemize}
\item<1-> Overfitting
\item<2-> Overparametrization
\end{itemize}
}

\frame{\frametitle{Detecting Model Problems}
\begin{itemize}
\item<1-> Monitor types of trading a financial institution is doing with other financial institutions
\item<2-> Monitor profits being recorded from trading of different products
\item<3-> Use Model Audit Group
\begin{itemize}
\item Check that a model has been implemented correctly
\item Examine whether there is a sound rationale for the model
\item Compare the model with others that can accomplish the same task
\item Specify limitations of model
\item Assess uncertainties in prices and hedge parameters given by model


\end{itemize}
\end{itemize}
}


