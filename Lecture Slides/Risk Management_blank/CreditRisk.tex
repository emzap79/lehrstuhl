\part{Credit Risk}
\section{Modelling Challenge}

What is Credit Risk?
Distinguish between \textcolor{blue}{individual risk elements}
	{\bf Default Probability / Credit Migration} The probability that the obligor
	or counterparty will default on its contractual obligations to
	repay its debt or the probability that its credit quality changes. Important is the time of default $\tau$.
	
	{\bf Recovery Rates $\delta$ (R).} The extent to which the face value of
	an obligation can be recovered once the obligor has defaulted.
 
	{\bf Exposure at Default.} The amount the creditor is owed at the time of default.


What is Credit Risk?
and \textcolor{blue}{portfolio risk elements}
	{\bf Default and Credit Quality Correlation.} The degree to
	which the default or credit quality of one obligor is related to
	the default or credit quality of another.
	{\bf Risk Contribution and Credit Concentration.} The extent
	to which an individual instrument or the presence of an obligor in
	the portfolio contributes to the totality of risk in the overall
	portfolio.


Empirical Issues
%\usepackage{graphics} is needed for \includegraphics
\begin{figure}[htp]
\begin{center}
  \includegraphics[height= 6cm]{../pics/credit-bild2}
\end{center}
\end{figure}


Modelling Approaches
%\usepackage{graphics} is needed for \includegraphics
\begin{figure}[htp]
\begin{center}
  \includegraphics[width=0.9\textwidth]{../pics/credit-bild1}
\end{center}
\end{figure}



\section{Objects and Notation}
\subsection{Definitions and Notation}
%-------------------------------------------------------------------------------------------------------------
Some Definitions
	\textbf{Maturity}: Expiry $T$ of contract.
	
	\textbf{Default Time}: $\tau$, modelled as a random variable with values in $[0,\infty]$.
	
	\textbf{Face Value}: (or Nominal Value) ($N\geq 0$) of a bond is the promised payment at maturity. We often simply use $N=1$.
	
	\textbf{Recovery-Rate}: In case of default a fraction $\delta$ (or $R$) of the nominal is recovered. Often we model it as random variable with values in $[0,1]$.
	
	\textbf{Loss given default}: The  (percentage loss) in case of
	default, i.e. $LGD=(1-\delta)=(1-R)$.
	
	\textbf{Exposure at default} The amount $EAD$ the creditor is owed at
	the time of default.
	
	\textbf{Probability of default} Probability $PD$ that the obligor
	defaults on its obligations.
		

Expected Loss
The central quantity for credit risk is the \emph{expected loss} $EL$:
\begin{align}
	\text{EL} = \text{PD} \times \text{EAD} \times \text{LGD}.
\end{align}


Credit Events I  
According the International Swaps and Derivatives Association (ISDA)
	\textbf{Bankruptcy} Insolvency, or a legally-recognised situation where a debtor is unable to pay off his debts in full.
  
	\textbf{Obligation Acceleration} covers the situation, other than a Failure to Pay, where the relevant obligation becomes
	due and payable as a result of a default by the reference entity before the time when such obligation would otherwise have
	been due and payable.
	
	\textbf{Obligation Default} covers the situation, other than a Failure to Pay, where the relevant obligation becomes capable 
	of being declared due and payable as a result of a default by the reference entity before the time when such obligation would 
	otherwise have been capable of being so declared.


Credit Events II}
	\textbf{Failure to Pay} is defined to be a failure of the reference entity to make, when and where due, any payments under one or more obligations. 
	Grace periods for payment are taken into account.

	\textbf{Repudiation/Moratorium} deals with the situation where the reference entity or a governmental authority disaffirms, disclaims or otherwise 
	challenges the validity of the relevant obligation.
	
	\textbf{Restructuring} covers events as a result of which the terms, as agreed by the reference entity or governmental authority and the holders of 
	the relevant obligation, governing the relevant obligation have become less favourable to the holders that they would otherwise have been.
	

\subsection{Credit Ratings}
Credit Ratings
	Rating agencies such as Standard \& Poor (S\&P) and Moody's rate the creditworthiness of financial assets such as
	sovereign and corporate bonds or structured products.
	
	In the S\&P rating system, AAA is the best rating, followed by  AA, A, BBB, BB, B, CCC, CC, and C
	
	The corresponding Moodyӳ ratings are Aaa, Aa, A, Baa, Ba, B, Caa, Ca, and C
	
	Bonds with ratings of BBB (or Baa) and above are considered to be ԩnvestment gradeԍ


Historical Default Probabilities
\begin{center}
	\includegraphics[height=7.cm, width=\textwidth]{../pics/TransitionMatrix.pdf}
\end{center}


Historical Recovery Rates
	In case of default creditors may get some of their investments back.
	Some of the investors have priority and their claims are met more fully.
	\begin{center}
	\begin{tabular}{l|l}
	class & mean in \% \\*[6pt]\hline
	Senior Secured	& 54.44\\
	Senior Unsecured & 38.39\\
	Senior Subordinated	& 32.85\\
	Subordinated & 31.61\\
	Junior Subordinated & 24.47\\
	\end{tabular}
	\end{center}


\subsection{Products}
Typical Products and Instruments
	\textbf{Individual Products (on single firms; single-name)}
			Bonds: Coupon Bonds, Zero-Coupon Bonds, etc.
			Derivatives: Credit Default Swaps (CDS), Credit Spread Options (CSO), etc.
			Mixed Forms: Asset Swaps
		
	\textbf{Portfolio Products (typically derivatives; multi-name):}
			Index (Portfolio) CDS.
			Collateralized Debt Obligations (CDO).
			Basket Credit Derivatives ($n$-th to default contracts).
		

% \section{Single-Name Credit-Risky Assets}
\subsection{Corporate Bonds}

Mathematical Notation
		% We call a probability space equipped with a filtration a stochastic basis  $\Fps$.
		We write $p^d(0,T)$ for the time zero price of a zero-coupon bond with maturity $T$ and nominal $N=1$.
		
		We assume a constant risk-free rate $r>0$, and use $\exp({-rt})$ as
		discount factor for $t$ years.
		
		We assume that the default probabilities are given by
		$$\prob(\tau\leq t),\quad \forall\,t\in[0,T],$$
		for each year $t$, where $\prob$ is a given valuation measure.
		
		The recovery rate $\delta$ is typically known, recovery payments are done at time of default $\tau$.
	

Corporate Bonds 
	Corporate bonds (=defaultable bonds) are debt instruments issued by corporations. The payment
	structure of a coupon-paying corporate bond is
	\begin{figure}[hbtp]
	\unitlength1cm \thicklines
	\begin{picture}(10,4)
	\put(1,2){\vector(1,0){8}} \put(9.5,2){time}
	\put(1,1.95){\line(0,1){0.1}} \put(0.5,1.5){$t_0=0$}
	\put(2,1.95){\line(0,1){0.1}} \put(1.8,1.5){$t_1$}
	\put(3,1.95){\line(0,1){0.1}} \put(2.8,1.5){$t_2$}
	\put(4,1.95){\line(0,1){0.1}} \put(5,1.95){\line(0,1){0.1}}
	\put(6,1.95){\line(0,1){0.1}} \put(7,1.95){\line(0,1){0.1}}
	\put(8,1.95){\line(0,1){0.1}} \put(7.5,1.5){$t_n=T$ maturity}
	\put(7.5,2.5){N face value} \put(5,1.7){$\ldots$}
	\put(1,0.8){\vector(0,1){0.5}} \put(2,0.8){\vector(0,1){0.5}}
	\put(3,0.8){\vector(0,1){0.5}} \put(8,0.8){\vector(0,1){0.5}}
	\put(3.5,0.2){coupon payments $c$}

	%\put(2,1){\vector(0,1){5}} \put(1.4,6.5){profit}
	%\put(4,1.5){\line(1,1){4}} \put(2,1.5){\line(1,0){2}}
	\end{picture}
	\end{figure}


Pricing a Zero-Coupon Corporate Bond
	The general valuation formula for a corporate bond is
		$$
		p^d (0,T) =  \EX \left[ e^{- r(T-t)} \IF_{\{\tau >T\}} + e^{-(\tau-t)} \delta \IF_{\{\tau \leq T\}}\right]\,.
		$$
	This is
		\begin{eqnarray*}
		p^d(0,T)	&=& e^{-rT} \prob(\tau > T ) +  \EX\left[e^{-r\tau} \delta \IF_{\{0\leq \tau \leq T\}}\right]\\
					&=& e^{-rT}\prob(\tau>T)+\delta\int_0^T e^{-rt}d\prob(\tau\leq t).
		\end{eqnarray*}
	To do calculations we need a model which allows to specify (compute) $\prob(\tau\leq t)$.


\subsection{Credit Default Swaps}

Credit Default Swaps
	A credit default swap is an exchange of a periodic payment against
	a one-off contingent payment if some credit event occurs on a
	reference asset.

	\begin{table}[htb]
	\begin{center}
	\begin{tabular}{ccc}
	& {\small contingent payment} & \\
	Protection& $\longleftarrow $ & Protection\\
	Buyer & $\longrightarrow $& Seller\\
	& {\small periodic fee, called spread $s$} &
	\end{tabular}
	\end{center}
	\end{table}


Market for Credit Derivatives}
	\begin{figure}[htp]
	\begin{center}
		\includegraphics[height=6cm]{../pics/CDS_Volume}
		\caption{Source: BIS}
		\label{figureLabel}
	\end{center}
	\end{figure}


Pricing a CDS
	Since the cash flow at coupon date $t_i$ for the protection
	buyer is $s\IF_{\{\tau>t_i\}}$ and the payment for the protection
	seller at time of default $\tau$ is $(1-\delta)\IF_{\{\tau\leq
	T\}}$, we obtain
	$$
	\EX\left(e^{-r\tau}(1-\delta)\IF_{\{\tau\leq
	T\}}\right) = \DSE\sum_{i=1}^n e^{-rt_i} s\EX\left(\IF_{\{\tau
	>t_i\}}\right),
	$$
	and so
	$$
	s=\frac{\DSE \EX\left(e^{-r\tau}(1-\delta)\IF_{\{\tau\leq
	T\}}\right)} {\DSE\sum_{i=1}^n e^{-rt_i} \EX\left(\IF_{\{\tau
	>t_i\}}\right)}.
	$$


\section{Firm Value (Structural) Models}
\subsection{Basic Concepts for Structural Models}

Idea
	A firm defaults, if \textbf{the value of its assets falls below the value of its liabilities.}
	
	\begin{center}\includegraphics[height=3.5cm]{../pics/Default_4.pdf}\end{center}
	
	Thus we distinguish these models according to the model assumptions for the firm value process, the liabilities, and the actual default trigger. 


Definitions and notation
	The process describing the firm value is $V=\{V_t\}_{t\geq 0}$.

	The default boundary is $d$.
	

\subsection{Merton Model}

Basic Merton Model
	Merton (1974) assumes that a firm is financed by
	equity and a single zero-coupon bond with notational amount (face
	value) $F$ and maturity $T$.

	The firm's value is given by
	$$
	dV(t) = (r-\gamma) V(t)  dt + \sigma V(t) dW(t)
	$$
	under an equivalent martingale (pricing) measure $\prob,$ with $r,
	\sigma$ constant, $W$ Brownian motion and constant payout
	(dividend) rate $\gamma,$ which may be negative (i.e. pay-in).


Basic Merton Model
	Default is only possible at maturity. There are two possibilities:
	$$
	V_T \geq F, \quad \mbox{thus} \quad p^d(T,T)=F
	$$
	or
	$$
	V_T < F, \quad \mbox{thus} \quad p^d(T,T)=V_T.
	$$


Equity Owners
	For equity owners the payoff  is
		$$
		S_T=\max\{V_T-F,0\}
		$$
	thus stocks can be viewed as call options on the value of the firm
	with ($\bar{V}_t = e^{-\gamma(T-t)}V_t$)
		$$
		S_t=C_E(\bar{V}_t,F)=e^{-\gamma(T-t)}V_t\Phi(d_1)-Fe^{-r(T-t)}\Phi(d_2)
		$$
	and
		$$
		d_1=\frac{\log(V_t/F)+(r-\gamma+\sigma^2/2)(T-t)}{\sigma\sqrt{T-t}}=d_2+\sigma\sqrt{T-t}.
		$$


Bond Owners
	For bond owners the payoff is
		$$
		p^d(T,T)=F-\max\{F-V_T,0\}.
		$$
	This can be viewed as the difference of a risk-free payment and a put option on the value of the firm with
		$$
		p^d(t,T)=Fe^{-r(T-t)}-P_E(\bar{V}_t,F),
		$$
	where
		$$
		P_E(\bar{V}_t,F)=-V_t e^{-\gamma(T-t)}
		\Phi(-d_1)+Fe^{-r(T-t)}\Phi(-d_2).
		$$
	So
		$$
		p^d(t,T)= V_t e^{-\gamma(T-t)} \Phi(-d_1)+
		Fe^{-r(T-t)}\Phi(d_2).
		$$


Calibration 
	Typically, we do not observe the value of a firm. Thus in order to
	calculate the parameters, i.e. $V_0$ and $\sigma$,  relevant for
	the dynamics of the firm's value, we have to use the identities
		\begin{eqnarray*}
		S_0 &=& V_0e^{-\gamma T}\Phi(d_1)-Fe^{-rT}\Phi(d_2),\\*[12pt] \sigma_s &=&
		\frac{\Phi(d_1)V_0\sigma}{S_0}
		\end{eqnarray*}
	to obtain risk-neutral default probabilities
		$$
		\prob(V_T<F).
		$$
	This approach has been used by Moody's KMV, named after
	the founders Kealhofer, McQuown and Vasicek, in their portfolio model of credit-risky assets.


Credit Spreads
	For spreads  $S(t,T)$ we use $p^d(t,T)=e^{-(r+S(t,T))(T-t)}$ and
	find (assuming F=1)
		$$
		S(t,T)=\frac{1}{T-t}\log\left\{\frac{1}{l_t}e^{-\gamma(T-t)}\Phi(-d_1)+\Phi(d_2)\right\},
		$$
	with
		$$
		l_t=\frac{F e^{-r(T-t)}}{V_t} = \frac{ e^{-r(T-t)}}{V_t}
		$$
	the leverage ratio.

	We find
		$$
		\lim_{t\rightarrow T} S(t,T)= \left\{\begin{array}{ll}
		+\infty, &\mbox{ on}\; \{V_T<F\}\\*[6pt]
		0, &\mbox{ on}\; \{V_T>F\}
		\end{array} \right.
		$$

% Credit Spreads 

%In terms of $l_t$ the valuation formula is
%$$
%\frac{p^d(t,T)}{F_t} = l_t^{-1} e^{-\delta(T-t)} \Phi
%(-h_1(l_t,T-t)) + \Phi (h_2(l_t,T-t)),
%$$
%where
%$$
%h_{1,2}(l_t,T-t) = \frac{ -\log(l_t)-\delta(T-t) \pm
%\sigma^2(T-t)/2 }{ \sigma \sqrt{T-t} }.
%$$

Credit Spreads 
	Merton's model doesn't generate the levels of yield spreads which can be
	observed in the market. Rather, this model is unable to generate yield spreads 
	in excess of approximately 120 basis points, whereas over the 1926--1986 period, 
	the yield spreads of AAA-rated corporates ranged from 15 to 215 basis points.


Calculated Credit Spreads  
	\includegraphics<1>[height=6cm]{../pics/merton_spreads.pdf}%


Real Credit Spreads 
	\includegraphics<1>[height=6cm]{../pics/defaultable-bonds-ts.pdf}%
	Source:  Chen and Poor 2003.


\section{Portfolio Models}
\subsection{Motivation}
Empirical Evidence for Dependent Defaults
	\vspace{-1cm}
		\begin{center}
		\includegraphics[height=6.5cm]{../pics/SPdefaultrates.pdf}
		\end{center}
	\vspace{-1cm}
		The first graph shows historical default rates from 1981 on. The second one shows simulated defaults from
		5.000 \textbf{independent} firms.


Modelling Approach
	Typically, we try to obtain the Portfolio Loss distribution from a portfolio model. The portfolio loss distribution 
	is needed for portfolio derivatives (CDOs), and also risk management (VaR) questions.
	
	Portfolio models try to capture empirical observations (stylized facts)
		Fluctuation of the default probabilities with the economic cycle. Typically some macro-variables are used in the model to achieve this.
		Default cluster (contageous defaults), defaults of firms can trigger defaults of further firms.
	
	%Portfolio models try to explain the dependence among several companies.
		
	Often individual default probabilities are assumed to be known and only the dependence structure is modelled.


\subsection{Vasicek Single Factor Model}
Vasicek Model
	We now want to model a group of firms based on the Merton model;
	
	The different firm-value processes are assumed to evolve according to correlated geometric
   
	Brownian motions, default is tested only at maturity.

	The model for each individual firm-value process is given by:
    $$dV_t^i=V_t^i(\gamma^i dt+\sigma^idW^i_t),\quad v_0>0.$$
	
	Using It\^{o}'s formula we find:
    $$V_T^i=V_t^i\exp\left\{\left(\gamma^i-\frac{(\sigma^i)^2}{2}\right)(T-t)+\sigma^i\sqrt{T-t}X_{t}^i\right\},$$
	where $X_{t}^i=(W^i_T-W^i_t)/\sqrt{T-t}$ follows a standard normal distribution.
  
	The default threshold is $d^i$.


Market Factor
	To incorporate correlation among different companies, we
	partially explain $X_{t}^i$ by the common market factor $M_t$
	and idiosyncratic risk factors $\epsilon_t^i$.
 
	So $$X_t^i=\rho M_t+\sqrt{1-\rho^2}\epsilon^i_t,\quad \rho\in[0,1],$$
  where $M_t, \epsilon_t^1,\dots,\epsilon_t^I$ are i.i.d. with $\mathcal{N}(0,1)$ distribution
	
	So $\mbox{Cor}(X^i_t,X^k_t)=\rho$ for $i\neq k$.
	
	Each $X_t^i$ is again distributed according to the standard normal law.


Conditional Distribution
	By conditioning on the common market factor default probabilities of different firms are independent.  
	
	We denote those conditioned default probability by $p^i(M_t)$ and obtain
    $$p^i(M_t)=\prob(\tau^i<t|M_t)=\Phi\left(\frac{k^i_t-\rho M_t}{\sqrt{1-\rho^2}}\right),$$
  where
    $$k^i_t=\frac{\log{\frac{d^i}{V_0^i}}-\left(\gamma^i-\frac{(\sigma^i)^2}{2}\right)t}{\sigma^i\sqrt{t}}.$$


Default Probabilities
	We assume the term structure of individual default
	probabilities as given and set $\Phi^{-1}(p^i_t)=k^i_t$. 

	Furthermore, we assume all companies to have the same default
	probabilities, the same portfolio weight and the same recovery 
	rate. That is we model a homogeneous portfolio.
 
	Therefore, we have
    $$p^i(M_t)=p(M_t)=\Phi\left(\frac{\Phi^{-1}(p_t)-\rho M_t}{\sqrt{1-\rho^2}}\right).$$


Loss Distribution
	We now define $\Omega_t$ as the random variable that
	describes the fraction of defaults in the portfolio
	up to time $t$.
	
	The distribution of $\Omega_t$ depends on two
	parameters: The individual default probabilities $p_t$
	and the correlation $\rho$ among different companies.
	
	This distribution will be denoted by
    $$F_{p_t,\rho}(x)=\prob(\Omega_t\leq x).$$


Large Portfolio Approximation
	We assume the number of companies within the portfolio
	is very large (tends to infinity).  
	
	By the law of large
	numbers we can approximate $\prob(\Omega_t\leq x)$ by $p(M_t)$.
  
	We show that
    $$F_{p_t,\rho}(x)\approx \Phi\left(\frac{\sqrt{1-\rho^2}\Phi^{-1}(x)-\Phi^{-1}(p_t)}{\rho}\right).$$
	
	This approximation is continuous and strictly increasing
	in $x$. As it further maps the unit interval
  onto itself it is a distribution function, too.


Large Portfolio Approximation -- Calculation
	We have
		\begin{eqnarray*}
		\prob(\Omega_t\leq x)	&=& \int_{-\infty}^{\infty}\prob(\Omega_t\leq x|M_t=y)d\prob(M_t\leq y)\\
							&=& \int_{-\infty}^{\infty}\IF_{p(y)}(x)d\prob(M_t\leq y) = \Phi(y^*),
		\end{eqnarray*}
	where $p(-y^*):=x$.
	
	So,
		$$\Phi\left(\frac{\Phi^{-1}(p_t)+\rho y^*}{\sqrt{1-\rho^2}}\right)=x\quad
		\Leftrightarrow\quad y^*=\frac{\sqrt{1-\rho^2}\Phi^{-1}(x)-\Phi^{-1}(p_t)}{\rho}.$$
	
	This shows
		$$F_{p_t,\rho}(x)\approx \Phi\left(\frac{\sqrt{1-\rho^2}\Phi^{-1}(x)-\Phi^{-1}(p_t)}{\rho}\right).$$
	

Correlation Sensitivity of Vasiceks Model
	\vspace{-0.5cm}
	\begin{center}
	\includegraphics[height=4.75cm, width=3cm]{../pics/PLD_r005_p01.pdf}
	\includegraphics[height=4.75cm, width=3cm]{../pics/PLD_r04_p01.pdf}
	\includegraphics[height=4.75cm, width=3cm]{../pics/PLD_r08_p01.pdf}
	\end{center}
	
	For small $\rho$ (left picture), implies nearly independent firms and (strong law) a portfolio loss of around  $10\%$.
	
	For larger $\rho$ the probability for \textit{many defaults} and \textit{few defaults} increases.
	

\subsection{General Factor Models}

Gaussian Model
	$$X^i=\rho M+\sqrt{1-\rho^2}\epsilon^i,\quad \rho\in[0,1],$$
  where $M, \epsilon^1,\dots,\epsilon^I$ are i.i.d. with $\mathcal{N}(0,1)$ distribution
	
	Therefore, we have
    $$p^i(M)=\Phi\left(\frac{\Phi^{-1}(F_i(t))-\rho M}{\sqrt{1-\rho^2}}\right).$$
	
	No tail dependence.


Double-t model
	Now
		$$X^i=\rho \left(\frac{\nu-2}{\nu}\right)^\frac{1}{2} M+ \sqrt{1-\rho^2}\left(\frac{\bar{\nu}-2}{\bar{\nu}}\right)^\frac{1}{2}T^i$$
		and 
		$M \sim t_\nu; T^i\sim t_{\bar{\nu}}$.
	
	Joint distribution is obtained by simulation.



\section{CreditMetrics as a Risk Management Tool}

CreditMetrics
	CreditMetrics is a framework based on VaR and Vasicek's model to calculate risk measures for credit risky portfolios  
	
	For the technical Document see the webpage  defaultrisk.com 




 