% !TEX root = riskmanagement_ws13UDE.tex
\part{Introduction}
\section{The Need For Risk Management?}
\subsection{Illustrative Examples}


Nick Leeson - Barings Bank
	London's 232-year-old merchant bank Baring Brothers fails February 26 after its
	Singapore trader Nicholas W. Leeson, 27, loses upwards of \$1 billion. Leeson
	was supposed to have arbitraged the difference between the Nikkei 225 futures
	prices traded on the Singapore, Osaka and Tokyo exchanges.
	 
	He has risked twice the bank's net worth in speculations on the Tokyo exchange and lost almost
	\$1.4 billion.
	
	In the aftermath of Leeson's activity, Barings collapsed and was
	purchased by the Dutch bank/insurance company ING for the nominal sum of \pounds 1.


What did we learn?
	\begin{center}
	\includegraphics[scale=0.45]{../../../pics/kerviel1.pdf}
	%\includegraphics[height=\textheight, width=7cm]{../pics/kerviel1.pdf}
	\end{center}


J{\'e}r{\^o}me Kerviel - Soci{\'e}t{\'e} G{\'e}n{\'e}rale}
	Kerviel violated for more than two years  \textcolor{red}{his risk limits }.
  
	Soci{\'e}t{\'e} G{\'e}n{\'e}rale made in Q4 2007 a profit of 1,4 Mrd. Euro on his trade.
	
	To hedge his portfolio Kerviel placed in December 2007 eight big transactions with a total volume of 50 Mrd. Euro. 
	His position were discovered on January,  18th 2008  and closed. Between  January 21rst and 23rd  
	Soci{\'e}t{\'e} G{\'e}n{\'e}rale made a loss of 4,9 Mrd. on this trades.


Dangers 
	Major reasons of the collapse of risk management systems:
		Lack of internal checks and balances
		Lack of understanding of the business
		Poor supervision of employees
		Lack of a clear reporting line
		$\Rrightarrow$ It is important \textcolor{red}{to define unambiguous risk limits and monitor carefully that these limits are adhered to}.



Quotes on Risk Management (tools) }
	{\it Derivatives are weapons of financial mass destruction.}\\*[6pt]
	Warren Buffet\\*[12pt]

	{\it Instruments that are more complex and less transparent -- such as credit-default swaps, collateralized 
	debt obligations, and credit-linked notes -- have been developed and their use has grown very rapidly in recent
	years. The result? Improved credit-risk management together with more and better risk-management tools appear 
	to have significantly reduced loan concentrations in telecommunications and, indeed, other areas and the associated 
	stress on banks and other financial institutions.}\\*[6pt]
	Alan Greenspan


\subsection{Classification of Risk}

Risk Types
\begin{figure}
	\centering
		\includegraphics[width=0.8\textwidth]{../../../pics/risk_SST}
	\label{fig:Risk_map1}
\end{figure}


Risk Types
\begin{figure}
	\centering
		\includegraphics[width=0.8\textwidth]{../../../pics/DeutscheBank-RiskTypes}
	\label{fig:Risk_map2}
\end{figure}


Sources of risk for a utility
	For the risk management process it is important to identify all main sources of risk. Those risks might be graphically illustrated in a risk map.
	\begin{figure}
		\centering
			\includegraphics[width=0.8\textwidth]{../../../pics/Risk_map}
		\label{fig:Risk_map3}
	\end{figure}


Example: Credit Risk Model
%\usepackage{graphics} is needed for \includegraphics
\begin{figure}[htp]
\begin{center}
  \includegraphics[height= 6cm]{../../../pics/credit-bild2}
\end{center}
\end{figure}


Example: Credit Risk Model
%\usepackage{graphics} is needed for \includegraphics
\begin{figure}[htp]
\begin{center}
  \includegraphics[width=0.9\textwidth]{../../../pics/credit-bild1}
\end{center}
\end{figure}


Basel II and Solvency II
	Risk based supervision (qualitative)
	
	Adequate capital requirements according to individual risk profile (chance of using internal models)
	
	More emphasis on risk management, internal controls \& Corporate Governance
	
	Integration of models in risk management and internal control systems
	
	Developing frameworks which enable comparison, transparency and consistency
	
	Competition


Pillars of Basel II
\begin{figure}
	\centering
		\includegraphics[width=0.8\textwidth]{../../../pics/baselll-pillars}
	\label{fig:Risk_map6}
\end{figure}


Pillars of Solvency II
\begin{figure}
	\centering
		\includegraphics[width=0.8\textwidth]{../../../pics/solv-pillars-kpmg}
	\label{fig:Risk_map}
\end{figure}


\subsection{Price Processes of Financial Assets}

Stocks -- Dax, Dax log-Returns
\includegraphics[width=\textwidth, height=3.7cm]{../../../pics/DAX}\\
\includegraphics[width=\textwidth, height=3.7cm]{../../../pics/DAXlog}


Stocks -- Dow Jones, Dow Jones log-Returns
%\\ Dow Jones Industrials Total Return Index (DJITR)
\includegraphics[width=\textwidth, height=3.7cm]{../../../pics/DJI}\\
\includegraphics[width=\textwidth, height=3.7cm]{../../../pics/DJIlog}


Gold Futures
\includegraphics[width=\textwidth, height=3.7cm]{../../../pics/Gold}\\
\includegraphics[width=\textwidth, height=3.7cm]{../../../pics/Goldlog}


Oil  Futures 
\includegraphics[width=\textwidth, height=3.7cm]{../../../pics/WTI}\\
\includegraphics[width=\textwidth, height=3.7cm]{../../../pics/WTIlog}


Term Structures of Interest Rates
\includegraphics[width=\textwidth, height=6.7cm]{../../../pics/Zinsstruktur}


\section{Outlook}
Questions
	How to measure risk? Measures: Variance? Value-at-Risk? Expected Shortfall?
	How to manage risk? Portfolio Risk -- Diversification? Use of Derivatives?