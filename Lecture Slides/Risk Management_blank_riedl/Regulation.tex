\part{Regulation}
\section{Solvency II}
\subsection{Overview of Solvency II}

Aims of Solvency II
Solvency II is one of the most important projects of the EU regulatory authorities. Its aim is to modernize the solvability requirements of
insurance companies towards a risk-based system. Insurance companies are required to improve (or even set up) their own risk management processes.
Currently, it does not seem possible that the deadline for implementation of
Solvency II---which is January, 1st 2013---can be met. Some parts of the rules
might not come into effect until January 2015.


Solvency I
Solvency I, currently in place, features simple rules for minimal capital requirements and solvency margins (16-18 \% of premium (non-life), 4 \% of
technical provisions (life).

It is volume-based and not explicitly risk-based, i.e. no difference
is made between investment mixes or asset/liability profiles.


Aims Solvency II
	Risk based supervision (qualitative)

	Adequate capital requirements according to individual risk profile (chance of using internal models)

	More emphasis on risk management, internal controls \& Corporate Governance

	Integration of models in risk management and internal control systems

	Developing frameworks which enable comparison, transparency and consistency

	Competition



Pillars of Solvency II}
\begin{figure}
	\centering
		\includegraphics[width=.80\textwidth]{../pics/solvpillars}
	\label{fig:Solv_Pillars}
\end{figure}


Market-consistent valuation
There are three classes of assets which are treated differently.
	Marking-to-market if liquid market exist. This includes government
	bonds, liquid shares, ...

	Derived from prices of similar tracked instruments (mix of
	marking-to-market and marking-to-model). Here  illiquid bonds, real estate, ...
	
	Finally, marking-to-model for participation, private equity, ...


\subsection{Levels of Capital requirement}
Solvency Margin
	The concept of a solvency margin can be seen as a buffer of free assets covering
	the liabilities.
 
	This buffer should consist of good-quality assets
	which allow timely liquidation within a given time horizon (e.g. 1
	year).

	We assume a "best estimate" of the liabilities of the
	insurance company as given (e.g. by a fair value principle). Also,
	the technical provisions should be calculated as best estimate.


Solvency Capital
	On top of liabilities we put a prudent solvency margin.
	
	This should achieve a minimum capital requirement (MCR) and a solvency
	capital requirement (SCR).
	
	Both MCR and SCR are risk charges and computed using an appropriate risk measure


Solvency Capital

\begin{figure}[hbtp]
\begin{center}
\setlength{\unitlength}{1cm}
\begin{picture}(10,7)
\put(1,6.5){Liability Side}
\put(6,6.5){Asset Side}
\put(1,1){\framebox(4,2.5){Technical Provisions}}
\put(6,1){\framebox(4,2.5){}}
\put(6,3.5){\framebox(4,1){Minimum Capital}}
\put(6,4.5){\framebox(4,1.5){Solvency Capital}}
\end{picture}
\end{center}
\caption{Capital Requirement}\label{pic:capital}
\end{figure}


Solvency Capital
	[\underline{SCR:}] this is a target level with no
	immediate intervention by the regulator. However, the insurer
	needs to convince the regulator of appropriate action to be
	taken (plan on how to restore the capital level)

	[\underline{MCR}] safety net; this is a hard solvency
	margin, with the objective of defining the level at which the
	management of the company is taken over by the supervisory
	authority.
	 
	MCR and SCR are calculated in terms of an unknown distribution
	function (giving the value of the SCR).


Shape of Solvency Capital distribution}
\begin{figure}
	\centering
		\includegraphics[width=.80\textwidth]{../pics/dichte-solvency2}
	\label{fig:SCR_distr}
\end{figure}
}


\subsection{Solvency Capital Calculation}
% \subsection{Classification and Aggregation of Risk}
Risk Types}
\begin{figure}
	\centering
		\includegraphics[width=1\textwidth]{../pics/risk_SST}
	\label{fig:Risk_types}
\end{figure}
}

Risk Levels
	{\bf Level 0:} between risk exposures, for example volatility and non-volatility within a product
	line or an asset category;
	
	{\bf Level 1:} between sub-portfolios within a risk
	category of a business unit for example, line-of-business or asset category (bonds, equity, cash,...)
	
	{\bf Level 2:} between main risk categories and sub-risk classes (see below)
	
	{\bf Level 3:} between business units to the group (consisting of several insurance companies).


Risk Categories
	Main risk categories (level 2) are Underwriting Risk (insurance risk), Credit Risk, Market Risk.

	Operational Risk: risk of losses due to inadequate or
	failing processes people and systems (e.g. fraud, risk from
	external events)
	
	Liquidity Risk: risk of loss due to the fact that
	insufficient liquid assets are available to meet cash flow
	requirements.


Risk Aggregation
\begin{figure}
	\centering
		\includegraphics[width=.80\textwidth]{../pics/riskmodules-solvency2}
	\label{fig:Risk_Aggr}
\end{figure}


% \subsection{Stochastic Model}
Standard Model}
	Our general framework will be a risk variable
	$$X=\sum_{i=1}^nX_i$$ where $X_i$ are variables describing the
	main risk categories. We need to take the distribution of $X_i$
	and dependency between the $X_i'$s into account.


Model Parameters}
	We assume $X=\sum_{i=1}^nX_i$ with $\mathbb{E}(X)=\mu_x$,
	$\var(X)=\sigma_x^2$ and $\mathbb{E}(X_i)=\mu_i$,
	$\var(X_i)=\sigma_i^2$, $\Cor(X_i,X_j)=\rho_{ij}$.


The usual formulae apply
	\begin{align*}
	\mathbb{E}(X)&=&\sum_{i=1}^n\mathbb{E}(X_i)\\
	\var(X)&=&\sum_{i=1}^n\sum_{j=1}^n\rho_{ij}\sigma_i\sigma_j
	\end{align*}


The Case of Normality
	Assume $X_i\sim {\cal N}(\mu_i,\sigma_i^2)$, then $X\sim
	{\cal N}(\mu,\sigma^2)$
 
	$VaR_{\alpha}(X)=\mu+q_{1-\alpha}\sigma$ with
	$q_{1-\alpha}$ the $(1-\alpha)$-Quantile of the $N(0,1)$
	distribution.


Correlation and Dependencies
	Typical risk categories are\\
	\begin{tabular}{lll}
	% after \\: \hline or \cline{col1-col2} \cline{col3-col4} ...
	$C_1$ & : & asset risk \\
	$C_2$ & :& technical risk \\
	$C_3$ &: & interest risk\\
	$C_4$ & : & business risk \\
	\end{tabular}\\
	with correlations
	$$\begin{array}{ll}
	\rho(C_1,C_3)=1, &\rho(C_1,C_2)=0,\\*[12pt]
	\rho(C_3,C_2)=0, & \rho(C_1,C_4)=\rho(C_2,C_4)=\rho(C_3,C_4)=1
	\end{array}
	$$
