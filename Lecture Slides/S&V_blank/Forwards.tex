% !TEX root = StructuringValuation_ws1314UDE.tex
\section{Spot-Forward Relationsship}

\subsection{Bessembinder - Lemon model}

{Bessembinder- Lemon Model specification}
	 One-period model
	 Power companies are able to forecast demand in the immediate future with precision
	 $N_P$ identical producers; $N_R$ identical retailers that buy power in the wholesale market and sell it to final consumers at fixed unit price
	 $P_R$ fixed unit price that consumers pay
	 $Q_{R_i}$ an exogenous random variable that denotes the realized demand for retailer $i$


{The cost function}
	Each producer $i$ has cost function
		$$
		TC_i=F+\frac{a}{c}(Q_{P_i})^c,
		$$
	where $F$ are fixed costs, $Q_{P_i}$ is the output of producer $i$, and $c\geq 2$.
	
	The cost function implies that the marginal production costs increase with output.
	
	If $c>2$ marginal costs increase at an increasing rate with output.
	
	Moreover, the distribution of power prices will be positively skewed even when the distribution of power demand is symmetric.


{Clearing prices}
 First, assume that forward prices are gives
 Obtain optimal behaviour in the spot market
 Work back and find optimal positions in the forward market.


{The wholesale spot market}
	Producers sell to retailers who in turn distribute to power consumers
	
	$P_W$ denotes the wholesale spot price, $Q_{P_i}^W$ quantity sold by producer $i$ in the wholesale spot market, $Q_{P_i}^F$ quantity
	that producer $i$ has agreed to deliver (purchase if negative) in the forward market at the fixed forward price $P_F$.
	
	The ex-post profit of producer $i$ is given by
		$$ \pi_{P_i}=P_W Q_{P_i}^W + P_FQ_{P_i}^F-F-\frac{a}{c}(Q_{P_i})^c,$$
		where each producer's physical production, $Q_{P_i}$, is the sum of its spot and forward sales $Q_{P_i}^W+Q_{P_i}^F$.


{The wholesale spot market}
 Retailers buy in the real-time wholesale market the difference between realised retail demand and their forward positions
 $Q_{R_j}^F$ quantity sold (purchased if negative) forward by retailer $j$, $P_R$ fixed retail price per unit
 The ex-post profit for each retailer is
$$
\pi_{P_j}=P_R Q_{R_j} + P_FQ_{R_j}^F - P_W (Q_{R_j}+Q_{R_j}^F),$$
 The profit maximising quantity for producer $i$ is (FOC wrt $Q_{P_i}^W$)
$$Q_{P_i}^W=\left(\frac{P_W}{a}\right)^{x}-Q_{P_i}^F$$
with $x=1/(c-1)$

{The wholesale spot market}

 The equilibrium total retail demand is equal to total production and forward contracts are in zero net supply
 Hence we must have that summing over all producers production must equal total demand from retailers
$$
N_P\left(\frac{P_W}{a}\right)^{x}=\sum_{i=1}^{N_R}Q_{R_i}^F
$$


{The wholesale spot market}

 Therefore the market-clearing wholesale price is
$$
P_W=a \left(\frac{Q^D}{N_P}\right)^{c-1},$$
where $Q^D=\sum_{j=1}^{N_R}Q_{R_j}$ is total system demand. We see that when $c>2$ an increase in demand has a disproportionate effect on power prices.
 Each producers sale in the wholesale market is
$$
Q_{P_i}^W= \frac{Q^D}{N_P}-Q_{P_i}^F.
$$


{Demand for forward positions}

 Producers profit (with no forwards) is
$$
\rho_{P_i}=P_W\frac{Q^D}{N_P}-F-\frac{a}{c}\left(\frac{Q^D}{N_P}\right)^{c}.$$
 Retailers profit (with no forwards) is
$$
\rho_{R_j}=P_RQ_{R_j}-P_WQ_{R_j}.
$$



{Mean-Variance Analysis for optimal forward position}
Assume that market players
$$
\max_{Q^F_{\{P_i,R_j\}}}\EX[\pi_{\{P_i,R_j\}}]-\frac{A}{2}\var[\pi_{\{P_i,R_j\}}]
$$
where, for example, producers have the profit function
$$
\pi_{P_i}=\rho_{P_i}+ P^FQ^F-P_WQ^F.
$$
FOCs imply
$$
Q^F_{\{P_i,R_j\}}= \frac{P^F-\EX[P_W]}{A\var[P_W]}+\frac{\Cov[\rho_{\{P_i,R_j\}},P_W]}{\var[P_W]}.
$$

{Mean-Variance Analysis for optimal forward position}

 The optimal forward position contains two components

 The first term reflects the position taken in response to the bias $P^F-\EX[P_W]$
 The second term is the quantity sold or bought forward to minimize the variance of profits

 Forward hedging can reduce risk precisely because the covariance term is generally non-zero.




{The equilibrium forward price}

 One can show that
$$
P_F=\EX[P_W]-\frac{N_P}{Nca^x}\left[cP_R\Cov[P_W^{x}, P_W]-\Cov[P_W^{x+1}, P_W]\right],
$$
where $N=(N_R+N_P)/A$ reflects the number of firms in the industry and the degree to which they are concerned with risk.
 The forward price will be less than the expected wholesale price, if the first term in brackets, which reflects retail risk, is
larger than the second term, which reflects production cost risk.
% The equilibrium forward price will depend only on the risks borne by the industry as a whole: variability in retail revenue and in %production costs


\subsection{A Dynamic Equilibrium Approach}

{Equilibrium Approach -- Players}



The main
motivation for players to engage in forward contracts is that of
risk diversification.

Producers have made large investments with the
aim of recouping them over a long period of time as well as making a
return on them.

Retailers (which might be intermediaries and/or use the commodity in
their production process) also have an incentive to hedge their
positions in the market by contracting forwards that help diversify
their risks.

Exposure to the market will differ both between producers and
retailers as well as within their own group.
So the need for risk-diversification has a temporal dimension.



{Market Risk Premium}


 These differences in the
desire to hedge positions are employed to explain the market risk premium and
its sign.
 Retailers are less incentivized to contract commodity forwards
the further out we look into the market. We associate situations where
$\pi(t,T)>0$ with the fact that retailers' desire to cover their
positions `outweighs' those of the producers, resulting in a
positive market risk premium.
 In contrast, on the producers' side the need to hedge in the long-term
does not fade away as quickly. Now the producers' desire to hedge their positions outweighs that of the retailers resulting in a negative market risk premium.





{Representative Agents}


 We describe producers' and retailers'
preferences via the utility function of two representative agents.
 Agents
must decide how to manage their exposure to the spot and forward
markets for every future date $T$.

A key question for the producer
is how much of his future production, which cannot be predicted with
total certainty, will he wish to sell on the forward market or, when
the time comes, sell it on the spot market.
 Similarly, the retailer
must decide how much of her future needs, which cannot be predicted
with full certainty either, will be acquired via the forward markets
and how much on the spot.


{Representative Agents}
We approach this financial decision and
equilibrium price formation in two steps.


 First, we determine the
forward price that makes the agents indifferent between the forward
and spot market.
 Second, we discuss how the relative willingness
of producers and retailers to hedge their exposures determines
market clearing prices.




{Representative Agents}

We assume that the risk preferences of the representative agents are
expressed in terms of an exponential utility function parameterized
by the risk aversion constant $\gamma>0$;
$$
U(x)=1-\exp(-\gamma x)\,.
$$
We let $\gamma:=\gamma_p$ for the producer and $\gamma:=\gamma_c$
for the consumer.



{The Model}
%Let $(\Omega,\mathcal{F},P)$ be a probability space equipped with a
%filtration $\mathcal{F}_t$.

We assume that the electricity spot price follows a
mean-reverting multi-factor additive process
\begin{equation}\label{equation for additive stock price}
S_t=\Lambda(t)+\sum_{i=1}^mX_i(t)+\sum_{j=1}^nY_j(t)
\end{equation}
where $\Lambda(t)$ is the deterministic seasonal spot price level,
while $X_i(t)$ and $Y_j(t)$ are the solutions to the stochastic
differential equations
\begin{equation}
dX_i(t)=-\alpha_i X_i(t)\,dt+\sigma_i(t)\,dB_i(t)
\end{equation}
and
\begin{equation}
dY_j(t)=-\beta_j Y_j(t)\,dt+dL_j(t).
\end{equation}
$B_i(t)$, $i=1,\ldots,m$, are standard independent Brownian
motions, $\sigma_i(t)$ are deterministic volatility functions
and $L_j(t)$, $j=1,\ldots,n$ are independent Jump (L\'evy)
processes.


{The Model}

The processes $Y_j(t)$ are
zero-mean reverting processes responsible for the spikes or large
deviations which revert at a fast rate $\beta_j>0$.\\*[12pt]

$X_i(t)$
are zero-mean reverting processes that account for the normal
variations in the spot price evolution with lower degree of
mean-reversion $\alpha_i>0$.

%It is thus natural from a market point
%of view to assume that $\max_i\alpha_i<\min_j\beta_j$, although this
%is not necessary in what follows.




{Indifference Prices}

Assume that the producer will deliver the spot over the time
interval $[T_1,T_2]$.\\*[12pt]

He has the choice to deliver the production in
the spot market, where he faces uncertainty in the prices over the
delivery period, or to sell a forward contract with delivery over
the same period.\\*[12pt]

The producer takes this decision at time $t\leq
T_1$.


{Indifference Prices}
We determine the forward price that makes the producer indifferent
between the two alternatives: denote by $F_{\hbox{pr}}(t,T_1,T_2)$
the forward price derived from the equation
$$
\begin{array}{ll}


& 1-\E^P\left[\exp\left(-\gamma_p\int_{T_1}^{T_2}S(u)\,du\right)\,|\,
\mathcal{F}_t\right]\\*[12pt]
= & 1-\E^P\left[\exp\left(-\gamma_p(T_2-T_1)F_{\hbox{pr}}(t,T_1,T_2)\right)\,|\,
\mathcal{F}_t\right]
\end{array}
$$

{Indifference Prices}
Equivalently,
\begin{equation}
\label{def-producer}
F_{\hbox{pr}}(t,T_1,T_2)=-\frac1{\gamma_p}\frac1{T_2-T_1}\ln\E^P
\left[\exp\left(-\gamma_p\int_{T_1}^{T_2}S(u)\,du\right)\,|\,
\mathcal{F}_t\right]\,,
\end{equation}
where for simplicity we have assumed that the risk-free interest
rate is zero.

$\int_{T_1}^{T_2}S(u)\,du$ is what the
producer collects from selling the commodity on the spot market
over the delivery period $[T_1,T_2]$, while he receives
$(T_2-T_1)F_{\hbox{pr}}(t,T_1,T_2)$ from selling it on the forward
market.

{Notation}
For $i=1,\ldots,m$ and $j=1,\ldots,n$,
\begin{equation}\label{alpha bar}
\bar{\alpha}_i(s,T_1,T_2)=\left\{\begin{array}{lll}
\frac1{\alpha_i}\left(\e^{-\alpha_i(T_1-s)}-\e^{-\alpha_i(T_2-s)}\right) &
, &
s\leq T_1\,, \\
\frac1{\alpha_i}\left(1-\e^{-\alpha_i(T_2-s)}\right) & , & s\geq T_1\,.
\end{array}\right.
\end{equation}
and
\begin{equation}\label{beta bar}
\bar{\beta}_j(s,T_1,T_2)=\left\{\begin{array}{lll}
\frac1{\beta_j}\left(\e^{-\beta_j(T_1-s)}-\e^{-\beta_j(T_2-s)}\right) & ,
&
s\leq T_1\,, \\
\frac1{\beta_j}\left(1-\e^{-\beta_j(T_2-s)}\right) & , & s\geq T_1\,.
\end{array}\right.
\end{equation}



{Indifference Prices}
The price for which the producer is indifferent between the forward
and spot market is given by
\begin{eqnarray*}
F_{\text{pr}}(t,T_1,T_2)&=&\frac{1}{T_2-T_1}\int_{T_1}^{T_2}\Lambda(u)\,du\\
&&+\sum_{i=1}^m\frac{\bar{\alpha}_i(t,T_1,T_2)}{T_2-T_1}X_i(t)+\sum_{j=1}^n
\frac{\bar{\beta}_j(t,T_1,T_2)}{T_2-T_1}Y_j(t) \\
&&-\frac{\gamma_p}{2(T_2-T_1)}\int_t^{T_2}\sum_{i=1}^m
\sigma_i^2(s)\bar{\alpha}_i^2(s,T_1,T_2)\,ds
\\&&-\frac1{\gamma_p}\frac1{T_2-T_1}\int_t^{T_2}\sum_{j=1}^n
\phi_j\left(-\gamma_p\bar{\beta}_j(s,T_1,T_2)\right)\,ds\,,
\end{eqnarray*}
where $ \bar{\alpha}_i$ and $\bar{\beta}_j$ are given by
\eqref{alpha bar} and \eqref{beta bar} respectively.



{Indifference Price -- Consumer}

The consumer will derive the indifference price from the incurred expenses
in the spot or forward market, which entails
$$
\begin{array}{ll}
&1-\E^P\left[\exp\left(-\gamma_c\left(-\int_{T_1}^{T_2}S(u)\,du\right)\right)
\,|\,\mathcal{F}_t\right]\\
= &1-\E^P\left[\exp\left(-\gamma_c(-(T_2-T_1)F_{\text{c}}(t,T_1,T_2)\right))\,|\,
\mathcal{F}_t\right]\,,
\end{array}
$$
or,
\begin{equation}
F_{\text{c}}(t,T_1,T_2)=\frac1{\gamma_c}\frac1{T_2-T_1}\ln\E^P\left[
\exp\left(\gamma_c\int_{T_1}^{T_2}S(u)\,du\right)\,|\,\mathcal{F}_t\right]\,.
\end{equation}

{Indifference Price -- Consumer}
The price that makes the consumer indifferent between the forward
and the spot market is given by
\begin{align*}
F_{\text{c}}(t,T_1,T_2)&=\frac1{T_2-T_1}\int_{T_1}^{T_2}\Lambda(u)\,du
+\sum_{i=1}^m\frac{\bar{\alpha}_i(t,T_1,T_2)}{T_2-T_1}X_i(t)\\
&\qquad+\sum_{j=1}^n\frac{\bar{\beta}_j(t,T_1,T_2)}{T_2-T_1}Y_j(t) \\
&\qquad+\frac{\gamma_c}{2(T_2-T_1)}\int_t^{T_2}
\sum_{i=1}^m\sigma_i^2(s)\bar{\alpha}_i^2(s,T_1,T_2)\,ds \\
&\qquad+\frac1{\gamma_c}\frac1{T_2-T_1}\int_t^{T_2}\sum_{j=1}^n
\phi_j\left(\gamma_c\bar{\beta}_j(s,T_1,T_2)\right)\,ds\,.
\end{align*}

{Indifference Price -- Bounds}
Note that the producer prefers to sell his production in the forward
market as long as the market forward price $F(t,T_1,T_2)$ is higher
than $F_{\text{pr}}(t,T_1,T_2)$. On the other hand, the consumer
prefers the spot market if the market forward price is more
expensive than his indifference price $F_{\text{c}}(t,T_1,T_2)$.
Thus, we have the bounds
\begin{equation}\label{bounds for forward}
F_{\text{pr}}(t,T_1,T_2)\leq F(t,T_1,T_2)\leq
F_{\text{c}}(t,T_1,T_2)\,.
\end{equation}

{Market Power}

 We introduce the deterministic function $p(t,T_1,T_2)\in[0,1]$
describing the \emph{market power of the representative producer}.
 For $p(t,T_1,T_2)=1$ the
producer has full market power and can charge the maximum price possible in the forward market (short-term positions),
namely $F_{\text{c}}(t,T_1,T_2)$.
 If the
consumer has full power, ie $p(t,T_1,T_2)=0$ (long-term positions), she will drive the
forward price as far down as possible which corresponds to
$F_{\text{pr}}(t,T_1,T_2)$.







{Market Power}

For any market power $0<p(t,T_1,T_2)<1$,\\
the forward price $F^p(t,T_1,T_2)$ is defined to be
\begin{eqnarray}
\nonumber
F^p(t,T_1,T_2)&=&p(t,T_1,T_2)F_{\text{c}}(t,T_1,T_2)\\*[12pt]
&&+(1-p(t,T_1,T_2))
F_{\text{pr}}(t,T_1,T_2).
\end{eqnarray}

{Market Power}
For $0\leq t\leq T_1<T_2$ the forward prices are
$$\begin{array}{ll}
& F^p(t,T_1,T_2)\\
&=\frac1{T_2-T_1}\int_{T_1}^{T_2}\Lambda(u)\,du
+\sum_{i=1}^m\frac{\bar{\alpha}_i(t,T_1,T_2)}{T_2-T_1}X_i(t)+
\sum_{j=1}^n\frac{\bar{\beta}_j(t,T_1,T_2)}{T_2-T_1}Y_j(t) \\*[12pt]
&\qquad+\frac{p(t,T_1,T_2)(\gamma_{\text{pr}}+
\gamma_{\text{c}})-\gamma_{\text{pr}}}{2(T_2-T_1)}\int_t^{T_2}
\sum_{i=1}^m\sigma_i^2(s)\bar{\alpha}_i^2(s,T_1,T_2)\,ds \\*[12pt]
&\qquad+\frac{p(t,T_1,T_2)}{\gamma_{\text{c}}(T_2-T_1)}\int_t^{T_2}
\sum_{j=1}^n\phi_j(\gamma_{\text{c}}\bar{\beta}_j(s,T_1,T_2))\,ds \\*[12pt]
&\qquad-\frac{1-p(t,T_1,T_2)}{\gamma_{\text{pr}}(T_2-T_1)}
\int_t^{T_2}\sum_{j=1}^n\phi_j(-\gamma_{\text{pr}}\bar{\beta}_j(s,T_1,T_2))
\,ds\,,
\end{array}
$$





{Producer's market power and market risk
premium, 18 monthly contracts with $t=$ January 2 2002}

\begin{figure}[htbp]
\includegraphics[width=10cm,height=6cm]{../../../pics/picFmonth1}
%\caption{Producer's market power and market risk premium, 18
%monthly contracts with $t=$ January 2 2002} \label{figure market
%power monthly forwards 2002}
\end{figure}


{Producer's market power and market risk premium, 7 quarterly contracts with $t=$ second quarter 2002}

\begin{figure}
\includegraphics[width=10cm,height=6cm]{../../../pics/picFquarter1}
%\caption{Producer's market power and market risk premium, 7
%quarterly contracts with $t=$ second quarter 2002} \label{figure
%market power quarterly forwards 2002}
\end{figure}

{Producer's market power and market risk premium, 3 yearly contracts with $t=$ 2002}


\begin{figure}[htbp]
\includegraphics[width=10cm,height=6cm]{../../../pics/picFyear1}
%\caption{Producer's market power and market risk premium, 3 yearly
%contracts with $t=$ 2002} \label{figure market power yearly
%forwards 2002}
\end{figure}


\subsection{Information Approach}


{Information Approach}

 As electricity is non-storable future predictions about the market will not affect the current spot price, but will affect forward prices.
 Stylized example: planned outage of a power plant in one month
 Market example: in 2007 the market knew that in 2008 CO$_2$ emission costs will be introduced; this had a clearly observable effect on the forward prices!
 German moratorium 2011: shut-down 7 nuclear power plants for 3 months with possible complete shut-down.



\begin{figure}[htbp]
  \includegraphics[width=0.8\textwidth]{../../../pics/spotdata.pdf}
    \caption{EEX spot prices}
\end{figure}




  {German Moratorium II}
%\vspace{-0.5cm}

\begin{figure}[htbp]
  \includegraphics[width=0.8\textwidth]{../../../pics/Mai2011graph.pdf}
    \caption{EEX forward prices delivery May 2011}
\end{figure}



  {German Moratorium III}
%\vspace{-0.5cm}

\begin{figure}[htbp]
  \includegraphics[width=0.8\textwidth]{../../../pics/August2011graph.pdf}
    \caption{EEX forward prices delivery August 2011}
\end{figure}


\vspace{-0.5cm}

\begin{figure}[htbp]
  \centering
  \subfigure{
    \includegraphics[width=0.47\textwidth]{../../../pics/forward3.pdf}
  }
  \subfigure{
    \includegraphics[width=0.47\textwidth]{../../../pics/forward2.pdf}
  }
  \caption{EEX Forward prices observed on 01/10/06 (left) and 01/10/07 (right)}
\end{figure}


 Typical winter and bank holidays behaviour in both graphs
 General upward shift in 2008 \\ \vspace{0.2cm}
\textcolor{red}{$~~~~~~ \Rightarrow$ 2nd phase of $CO_2$ certificates}





 Future information is incorporated in the forward price
 ... but not necessarily in the spot price due to \textcolor{red}{non-storability}
 ... buy-and-hold strategy does not work
% Thus:

%\pause
%\begin{block}{Efficient Markets Hypothesis (semi-strong)}
%The price (spot) now reflects all publicly available information.
%\end{block}
%
% ... is not valid on electricity markets!
%




  {Information Approach}
\vspace{-0.5cm}


 The usual pricing relation between spot and forward:
\begin{align*}
F(t,T)=\Ef^\mathbb{Q}[S_T|\mathcal{F}_t]
\end{align*}
 Not sufficient: natural filtration $\mathcal{F}_t = \sigma(S_s, s\leq t)$
\vspace{0.5cm}
\pause
 Idea: \textcolor{red}{enlarge the information set!}
\vspace{0.5cm}
\pause
 ... by information about the spot at some future time $T_\Upsilon$
 Info could be that spot will be in certain interval...
 ... or the value of a correlated process (temperature)





  {The Information Premium}
\vspace{-0.5cm}


 Quantify the influence of future information using:

\begin{block}{Information Premium}
The information premium is defined to be
\begin{align*}
I(t,T) = \Ef[S_T | \mathcal{G}_t] - \Ef[S_T | \mathcal{F}_t]
\end{align*}
i.e. the difference between the prices of the forward under $\mathcal{G}$ and $\mathcal{F}$.
\end{block}














