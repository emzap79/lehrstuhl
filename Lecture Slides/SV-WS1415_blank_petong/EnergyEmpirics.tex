% !TEX root = StructuringValuation_ws1314UDE.tex
\section{Empirical Analysis of Commodity Returns}
\subsection{Marginal Distributions}
\frame{\frametitle{Marginal Distributions}
%\vspace{-1cm}
The analysis consists of following commodities:\\
Power (Baseload), Brent Crude Oil, Coal, Natural Gas, CO$_2$
allowances

We highlight important features using some of them.

The commodities have a wide variety of different statistical
properties, e. g.
\begin{center}
\begin{figure}
\includegraphics[width=0.45\textwidth]{../../../pics/qqoil.pdf}
\includegraphics[width=0.45\textwidth]{../../../pics/qqpower.pdf}
\end{figure}
\end{center}

 \frame{\frametitle{Marginal Distributions -- Basic Statistics}
{\small
\begin{table}[ht]
            \vspace{0.5cm}
            \begin{center}
                \begin{tabular}{c|c|c|c|c}
                    &Power2007&Brent2007&Coal2007&Carbon2007\\\hline
                    Length&\multicolumn{4}{c}{- 314 -}\\
                    Minimum&-0.070490&-0.037960&-0.034940&-0.336500\\
                    Maximum&0.088390&0.059270&0.025550&0.569600\\
                    Mean&0.001237&0.001497&0.000180&0.000094\\
                    Median&0.001185&0.001427&0.000658&0.002574\\
                    SE Mean&0.000721&0.000796&0.000551&0.003025\\
                    Variance&0.000163&0.000199&0.000095&0.002873\\
                    Skewness&-0.185179&0.212642&-0.332898&1.967920\\
                    Excess kurtosis&11.680823&0.584996&0.452510&48.843905\\
                \end{tabular}
                \caption{Basic statistics of the 2007 log-return series}
            \end{center}
        \end{table}
}




\frame{\frametitle{Stylized facts}
%\vspace{-1.5cm}
\begin{center}
How non-normal are power prices?\\
\end{center}

\frame{\frametitle{Stylized facts}
%\vspace{-1.5cm}
\begin{center}
How non-normal are power prices?\\
Skewness: 0.07 \hspace{2cm} Excess Kurtosis: 2.31\\
\includegraphics[width=0.7\textwidth,  height=0.7\textheight]{../../../pics/DChistogram.pdf}
\end{center}

\frame{\frametitle{Stylized facts} %\vspace{-1.5cm}
\begin{center}
How non-normal are power prices?\\
Skewness: 1.95 \hspace{2cm} Excess Kurtosis: 25.93\\
\includegraphics[width=0.7\textwidth,height=0.7\textheight]{../../../pics/powerhistogram.pdf}
\end{center}






\subsection{Multivariate Parametric Modelling}
\subsubsection{Requirements for a Joint Distribution}
\frame{\frametitle{Towards a multivariate distribution --
Requirements} %\vspace{-1.5cm}
\begin{itemize}
\item Distribution must reflect statistical properties, at least:
    \begin{itemize}
    \item non-normality
    \item peaked center, heavy tails
    \item skewness and excess kurtosis
    \end{itemize}
\item Distribution must be tractable, i. e. it should allow for:
    \begin{itemize}
    \item Conclusion of portfolio distribution from joint distribution (stable under convolution)
    \item Computation of value-at-risk/expected shortfall
    \end{itemize}
\end{itemize}


\frame{\frametitle{Towards a multivariate distribution --
Requirements}

Univariate analyses support the use of NIG (applied to spot power
by Benth \& Saltyte-Benth) and hyperbolic distributions (applied
to power futures by Eberlein \& Stahl, to oil and natural gas by
Kat \& Oomen), Multivariate distributional analysis are not
available thus far.\\

Suggestion: Use a multivariate extension of the distributions
above, i. e. multivariate generalized hyperbolic distributions
(GH)

 \subsubsection{Generalized Hyperbolic Distributions}

\frame{\frametitle{GH-distributions} GH-distribution has a
representation as Normal mean-variance mixture, i. e. $X\sim GH$
(multivariate) if
$$X\stackrel{d}{=}\mu+W\gamma+\sqrt{W}AZ$$
where $Z$ is a $k$-dimensional standard Normal, $W$ Generalized
Inverse Gaussian random variable (mixture variable) independent of
$Z$, $A\in\mathbb{R}^{d\times k}$ and $\mu, \gamma \in
\mathbb{R}^d$.

\frame{\frametitle{GH-distributions} %\vspace{-2cm}
$$X\stackrel{d}{=}\mu+W\gamma+\sqrt{W}AZ$$
%\vspace{-1cm}
\begin{itemize}
\item Moments of $X$:
    \begin{itemize}
    \item $\E[X]=\mu+\gamma\E[W]$
    \item $\Var[X]=AA'\E[W]+\gamma\gamma'\Var[W]$\\[-1.5cm]
    \end{itemize}
\item $\mu$ controls mean value\\[-1.5cm]
\item $\gamma$ controls skewness ($\gamma=0$: symmetric)\\[-1.5cm]
\item $\Sigma:=AA'$ controls covariance\\[-1.5cm]
\item $W\sim GIG(\lambda,\chi,\psi)$ controls tail-behaviour
\end{itemize}

\frame{\frametitle{GH-distributions} %\vspace{-1.5cm}
Further important properties
are: %\vspace{-0.5cm}
\begin{itemize}
\item GH-distribution has heavy tails. \item Density is explicitly
known $\rightarrow$ Maximum-Likelihood estimation. \item
Distribution is closed under linear combinations $\rightarrow$
Portfolio is GH-distributed. \item GH-distribution is infinitely
divisible $\rightarrow$ Levy-processes
\end{itemize}


\frame{\frametitle{GH-distributions}

Well-known special cases are:
\begin{itemize}
\item NIG ($\lambda=-1/2$, successful in power markets)\\[-1.5cm]
\item Hyperbolic ($\lambda=1$, successful in power, gas and oil markets)\\[-1.5cm]
\item $t$-distribution ($\lambda=-1/2\nu, \chi=\nu, \psi=0,
\gamma=0, \nu>0$)
\end{itemize}

\frame{\frametitle{} %\vspace{-2cm}
\begin{center}
\begin{figure}
\includegraphics[width=0.9\textwidth]{../../../pics/densitiespower.pdf}
%\includegraphics[width=0.49\textwidth]{pics/logdensitiespower.pdf}
\end{figure}
\end{center}

\frame{\frametitle{} %\vspace{-2cm}
\begin{center}
\begin{figure}
%\includegraphics[width=\textwidth]{pics/densitiespower.pdf}
\includegraphics[width=0.9\textwidth]{../../../pics/logdensitiespower.pdf}
\end{figure}
\end{center}

 \frame{\frametitle{Further properties of GH-distributions --
Computation of VaR/ES} %\vspace{-1.5cm}
Once the joint distribution is specified, we want to analyze
energy portfolios, e. g. compute risk measures. We use the
following setting:
\begin{itemize}
\item $X_{t+1}^{(i)}:=\frac{S_{t+1}^{(i)}-S_t^{(i)}}{S_t^{(i)}}$
denotes relative returns of commodity $i$. \item
$X:=-\sum\omega_iS_t^{(i)}X_{t+1}^{(i)}$ denotes loss of a
portfolio consisting of $\omega_i$ shares of commodity $i$ during
the period $[t,t+1]$.
\end{itemize}

\frame{\frametitle{Further properties of GH-distributions --
Computation of VaR/ES}
\begin{itemize}
\item If $X_t^{(i)}$ are jointly GH-distributed, then $X$ is
univariate GH-distributed \item Compute value-at-risk
(numerically) via $ VaR_{\alpha}(X)=F^{-1}_X(\alpha) $ \item
Expected shortfall is defined (for absolute continuous
distributions) via $ ES_{\alpha}(X)
=\frac{1}{1-\alpha}\int_{\alpha}^1VaR_u(X)du$. In case of
GH-distributions this can be computed explicitly:
\begin{eqnarray*}
ES_{\alpha}(X)&=&\mu+\gamma\BE(W)+\sigma\frac{\varphi(\Phi^{-1}(\alpha))}{1-\alpha}\BE(\sqrt{W})
\end{eqnarray*}
\end{itemize}


\subsubsection{Risk Management Application}
\frame{\frametitle{Some statistical results} %\vspace{-1.5cm}
Results are reported for commodity futures/swaps (power, oil, gas,
CO$_2$) with delivery during 2007. We worked on time series with
1.5 years of daily prices.

Results are: %\vspace{-0.5cm}
\begin{itemize}
\item Mardia's tests reject the use of multivariate normal distribution.\\[-1.3cm]
\item Likelihood ratio tests prefer any of the alternative distributions to the Normal\\[-1.3cm]
\item Among all the tested alternatives, we recommend the NIG.
\end{itemize}


\frame{\frametitle{Coal-fired Power Plant} We consider a portfolio
that can be viewed roughly as the year production/consumption of a
coal-fired power plant that has a total output capacity of 1TW:
\begin{itemize}
\item $10^6$ power contracts (1 contract delivers 1MW during the entire year), long \\[-1.5cm]
\item 330,000 t coal, short\\[-1.5cm]
\item 900,000 t CO$_2$, short
\end{itemize}

\frame{\frametitle{Coal-fired
Power Plant} %\vspace{-2cm}
\begin{figure}
\begin{center}
%\includegraphics[width=0.49\textwidth]{pics/EScoal09ret.pdf}
\includegraphics[width=0.85\textwidth]{../../../pics/VARcoal09ret.pdf}
\end{center}
\end{figure}

\frame{\frametitle{Coal-fired
Power Plant} %\vspace{-2cm}
\begin{figure}
\begin{center}
\includegraphics[width=0.85\textwidth]{../../../pics/EScoal09ret.pdf}
%\includegraphics[width=0.49\textwidth]{pics/VARcoal09ret.pdf}
\end{center}
\end{figure}

\frame{\frametitle{Coal-fired
Power Plant} %\vspace{-1cm}
\begin{center}
Normal VaR $>$ NIG VaR?
\end{center}
%%\vspace{-2cm}
95\%-quantile: \hspace{2cm} 0.034 (Normal) \hspace{1cm}\\
\begin{center}
%\vspace{-1cm}
\includegraphics[width=0.7\textwidth, height= 0.7\textheight]{../../../pics/DChistogram.pdf}
\end{center}

\frame{\frametitle{Coal-fired
Power Plant} %\vspace{-1cm}
\begin{center}
Normal VaR $>$ NIG VaR?
\end{center}
%%\vspace{-2cm}
95\%-quantile: \hspace{1cm} 0.034 (Normal)  \hspace{1cm}$<$ \hspace{1cm} 0.041 (NIG)\\
%\vspace{-1cm}
\begin{center}
\includegraphics[width=0.7\textwidth, height= 0.7\textheight]{../../../pics/DCNIG.pdf}
\end{center}

\frame{\frametitle{Coal-fired
Power Plant} %\vspace{-1cm}
\begin{center}
Normal VaR $>$ NIG VaR?
\end{center}
95\%-quantile: \hspace{1cm} 0.073 (Normal) \hspace{1cm} \\
\begin{center}
%\vspace{-1cm}
\includegraphics[width=0.7\textwidth, , height= 0.7\textheight]{../../../pics/powerhistogram.pdf}
\end{center}

\frame{\frametitle{Coal-fired
Power Plant} %\vspace{-1cm}
\begin{center}
Normal VaR $>$ NIG VaR?
\end{center}
95\%-quantile: \hspace{1cm} 0.073 (Normal) \hspace{1cm}$>$ \hspace{1cm} 0.053 (NIG) \\
\begin{center}
%\vspace{-1cm}
\includegraphics[width=0.7\textwidth, , height= 0.7\textheight]{../../../pics/powerNIG.pdf}
\end{center}

 \subsubsection{Discussion of multivariate GH-model}
%\subsection{Marginals}
\frame{\frametitle{Shortcomings -- Marginals} %\vspace{-1.5cm}
\begin{center}
QQplot using a fitted joint NIG
\begin{figure}
\begin{center}
\includegraphics[width=0.4\textwidth]{../../../pics/marginalpower.pdf}
\includegraphics[width=0.4\textwidth]{../../../pics/marginalcoal.pdf}
\end{center}
\end{figure}
\end{center}


\frame{\frametitle{Shortcomings -- Marginals} %\vspace{-1.5cm}
\begin{center}
QQplot using a fitted joint NIG
\begin{figure}
\begin{center}
\includegraphics[width=0.4\textwidth]{../../../pics/marginaloil.pdf}
\includegraphics[width=0.4\textwidth]{../../../pics/marginalcarbon.pdf}
\end{center}
\end{figure}
\end{center}



\frame{\frametitle{Shortcomings -- Marginals} %\vspace{-1.5cm}
\begin{itemize}
%\item Unterschiedliches Tail-Verhalten macht auch die implizite Kalibrierung an Optionen mit unterschiedlichen Lieferperioden/Underlyings instabil.
\item Same tail-behaviour in all components of
$$X\stackrel{d}{=}\mu+W\gamma+\sqrt{W}AZ$$
since $W$ is the determinant for tail - and common to all
marginals. \item If $X$ and $Y$ are two GH-distributed random
variables driven by different $W$s, we loose the convolution
property ($X+Y$ is not GH-distributed). \item Exception: NIG
\end{itemize}


\subsection{Copula-based Modelling}
\frame{\frametitle{Copula Approach Outline} %\vspace{-1cm}
\begin{itemize}
\item Estimation of marginal  distribution of several commodity
returns \item Estimation of several Copulas to obtain joint
distributions \item Applications: \textbf{Risk Management} (VaR,
ES, Portfolio Analysis)
\end{itemize}



\subsubsection{Marginal Distributions}
\frame{\frametitle{Marginal Distributions}

For each model
\begin{itemize}
\item the estimated parameters \item the Akaike Information
Criterion (AIC), which is defined to be
    \[AIC(M_j)=-2log(L_j(\hat{\theta}_j; X))+2k_j\ ,j\in\{1,\ldots,n\}\]
\item the log-likelihood \item the p-value of a likelihood ratio
test against the asymmetric generalized hyperbolic model (which is
the most general model we consider)
\end{itemize}
are provided.




\frame{\frametitle{Marginal Distributions - Example: Power}

{\tiny

\begin{table}[ht]
            \vspace{0.5cm}
            %\begin{center}
                \begin{tabular}{c|c|c|c|c|c|c|c|c}
                    model&$\lambda$&$\overline{\alpha}$&$\mu$&$\sigma$&$\gamma$&AIC&log-likelihood&p-value\\\hline
                    t&-1.227624&0&0.001529&0.015459&0&-1976.120&991.0599&0.793012\\
                    NIG-s&-0.5&0.316160&0.001429&0.012340&0&-1974.458&990.2291&0.345514\\
                    sk- t&-1.239572&0&0.001713&0.015189&-0.000614&-1974.355&991.1773&0.632315\\
                    ghyp-s&-1.119172&0.129836&0.001511&0.012967&00&-1974.338&991.1689&0.620085\\
                    NIG&-0.5&0.319912&0.001582&0.012314&-0.000346&-1972.594&990.2971&0.158394\\
                    ghyp&-1.131927&0.130704&0.001714&0.012900&-0.000477&-1972.584&991.2918&NA\\
                    hyp-s&1&0.000126&0.001184&0.011303&0&-1963.430&984.7149&0.001392\\
                    hyp&1&0.000003&0.001174&0.011305&0.000057&-1961.439&984.7197&0.000288\\
                    N &0&0&0.001237&0.000163&0&-1846.459&924.2295&0
                \end{tabular}
                \caption{Marginal models for the Power2007 log-return series}
            %\end{center}
        \end{table}
}


\subsubsection{Copulas}
\frame{\frametitle{Copula Modelling}

Let $F_i(t)$ be a marginal return distribution
of asset $i$.

One can use of a copula function $C$ to obtain the multivariate
 distribution
$$
F(t_1, \ldots t_n)=C(F_1(t_1),\ldots,F_n(t_n)).
$$

Flexibility in the choice of copulae allows to consider several
features of the dependence
structure.




\frame{
\frametitle{Facts on Copula Functions}


A copula $C$ is a multivariate distribution with standard uniform
marginal distributions. That is $C$ is a mapping $[0,1]^m
\rightarrow [0,1]$ with
\begin{itemize}
\item<1-> $C(u_1,\ldots,u_m)$ is increasing in each component $u_i$
\item<2-> $C(1,\ldots,1,u_i,1,\ldots,1)=u_i$ for all
$i\in\{1,\ldots,m\}$, $u_i\in[0,1]$
\item<3-> For all $(a_1,\ldots,a_m), (b_1,\ldots,b_m)\in[0,1]^m$ with
$a_i\leq b_i$ we have:
$$
\sum_{i_1=1}^2\cdots\sum_{i_d=1}^2(-1)^{i_1+\ldots+i_d}C(u_{1,i_1},\ldots,u_{m,u_m})
\geq 0,
$$
where $u_{j,1}=a_j$ and $u_{j,2}=b_j$ for all $j \in
\{1,\ldots,m\}$.
\end{itemize}



\frame{
\frametitle{Facts on Copula Functions}

Thus a copula function is a multivariate distribution function
such that its marginal distributions are standard uniform. We use
the notation
$$
C:[0,1]^m \rightarrow [0,1], \A (u_1, \ldots, u_m)\rightarrow
C(u_1, \ldots, u_m).
$$

If $\bv{X}=(X_1,\ldots,X_m)'$ has joint distribution $F$ with
continuous marginals $F_1,\ldots,F_m$, then the distribution
function of the transformed vector
$$
(F_1(X_1),\ldots,F_m(X_m))
$$
is a copula $C$, and
$$
F(x_1,\ldots,x_m)=C(F_1(x_1),\ldots,F_m(x_m)).
$$




\frame{
\frametitle{Examples of Copula Functions}


The Gaussian copula is given by:
$$
C^m_{\Phi,\Gamma}(\bv{u})=\bv{\Phi}_{\Gamma}(\Phi^{-1}(u_1),\ldots,\Phi^{-1}(u_m)),
$$
where $\bv{\Phi}_{\Gamma}$ denotes the multivariate Gaussian
distribution function with linear correlation matrix $\Gamma$.


The $t$-copula can be expressed by:
$$
C^m_{t,\nu,\Gamma}(\bv{u})=t^m_{\nu,\Gamma}(t_\nu^{-1}(u_1),\ldots,t_{\nu}^{-1}(u_m)),
$$
where $t^m_{\nu,\Gamma}$ denotes the distribution function of an
m-variate $t$-distributed random vector with parameter $\nu>2$ and
linear correlation matrix $\Gamma$. Furthermore, $t^{-1}_\nu$ is
the inverse of  the univariate $t$-distribution function with
parameter $\nu$.




\frame{\frametitle{Examples of Copulas}

\begin{center}
\begin{figure}
\includegraphics[width=0.6\textwidth]{../../../pics/copula-density.pdf}
\end{figure}
\end{center}







\frame{\frametitle{Archimedean copula}

An Archimedean copula function $C:[0,1]^{I}\rightarrow[0,1]$ is a
copula function which can be represented in the following form
$$
C(\bv{x})=\varphi^{-1}\left(\sum_{i=1}^{I}\varphi(x_i)\right)
$$
with a suitable function $\varphi:[0,1]\rightarrow\mathbb{R}_t$
with $\varphi(0)=\infty, \varphi(1)=0$. The function $\varphi$ is
called the generator of the copula.

Examples of this class are
\begin{itemize}
\item Clayton: $\varphi(t)=(t^{-\theta}-1);
\varphi^{-1}(s)=(1+s)^{-1/\theta},\; \theta\geq 0$.\\
Lower tail dependency, no upper tail dependency.
\item
Gumbel: $\varphi(t)=(-\log t)^{\theta},
\varphi^{-1}(s)=e^{-s1/\theta},\; \theta\geq 1$.\\
Upper tail dependency, no lower tail dependency.
\end{itemize}


\frame{ \frametitle{Copula Examples}


\includegraphics<1>[height=6cm]{../../../pics/copula_examples.pdf}%




\frame{\frametitle{Copulas}

\begin{itemize}
\item We fit 5 copula models (Normal, t, Frank, Gumbel, Clayton)
to the data. \item Fitting is done via the IFM (Inference for
margins) method, which means we fit the marginal models
separately, employ the quantile transformation on each margin and
then fit the copula on this transformed dataset. \item For each
margin we choose the best (ranked by its AIC) generalized
hyperbolic model as seen in the tables before. \item We rank the
copulas by their AIC values.

\end{itemize}



\frame{\frametitle{Copula fit}

\begin{table}[ht]
            \vspace{0.5cm}
            \begin{center}
                \begin{tabular}{c|c|c|c}
                    name&\parbox{3cm}{\centering number of\\ fitted parameters}&AIC&log-likelihood\\\hline
                    t&7&-229.780909&121.890455\\
                    normal&6&-225.226240&118.613120\\
                    clayton&1&-92.833737&47.416868\\
                    frank&1&-67.382373&34.691187\\
                    gumbel&1&-62.799001&32.399500\\
                \end{tabular}
                \caption{Copula models for the year 2007 log-return series}
            \end{center}
        \end{table}




\frame{\frametitle{Copula fit -- Results}

t and normal copula provide the best fit,  because
\begin{itemize}
\item they have way more parameters to capture the dependence
structure \item they have more flexibility to fit the dependence
between the margins of our models.
\end{itemize}





\subsubsection{Risk Management Application}
\frame{\frametitle{Coal-fired Power Plant}

We again compute the Value at Risk (VaR) and the one-day Expected
Shortfall (ES).

Now we need to simulate the loss distribution of the portfolio

\begin{itemize}
\item Generate $N:=100000$ realizations of our multivariate copula
based model; i.e. $N$ realizations of the estimated copula. \item
Employ the quantile transform for each margin to get $S_t^{(i)}$.
\item Transform these samples into realizations of our portfolio
loss distribution.

\end{itemize}

 \frame{\frametitle{Coal-fired Power Plant}

We consider a coal-fired power plant with an output of $1000000$
MWh of electricity. To produce this amount of electricity one
needs $330000$ tons of coal and as a byproduct the firing process
generates $900000$ tons of CO$_2$. So our coal-fired power plant
is nothing else than the following portfolio:
        \begin{itemize}
            \item a long position of $1000000$ MWh of electricity,
            \item a short position of $330000$ tons of coal,
            \item and a short position of $900000$ tons of CO$_2$.
        \end{itemize}

 \frame{\frametitle{Coal-fired Power Plant -- Margins} \tiny
\begin{table}[ht]
                \begin{center}
                    \begin{tabular}{c|c|c|c|c|c|c|c|c}
                        model&$\lambda$&$\overline{\alpha}$&$\mu$&$\sigma$&$\gamma$&AIC&log-likelihood&p-value\\\hline
                        &\multicolumn{8}{c}{Power2007}\\\cline{2-9}
                        t&-1.203832&0&0.001321&0.014477&0&-2561.138&1283.569&0.843597\\\hline
                        &\multicolumn{8}{c}{Coal2007}\\\cline{2-9}
                        hyp&1&3.139811&0.003589&0.009880&-0.003154&-2510.830&1259.415&0.715009\\\hline
                        &\multicolumn{8}{c}{Carbon2007}\\\cline{2-9}
                        t&-1.007697&0&0.003084&0.193450&0&-1655.236&830.6180&0.719724\\\hline
                        &\multicolumn{8}{c}{Power2008}\\\cline{2-9}
                        t&-1.535741&0&0.000637&0.009909&0&-3937.290&1971.645&0.782617\\\hline
                        &\multicolumn{8}{c}{Coal2008}\\\cline{2-9}
                        NIG&-0.5&4.107861&0.004122&0.009177&-0.003702&-3804.225&1906.112&1\\\hline
                        &\multicolumn{8}{c}{Carbon2008}\\\cline{2-9}
                        sk t&-1.481606&0&0.004414&0.032939&-0.004633&-2548.610&1278.305&0.352221\\\hline
                    \end{tabular}
                    \caption{Marginal models for the 3-variate coal-fired power plant data}
                \end{center}
            \end{table}
} \frame{\frametitle{Coal-fired Power Plant -- Copula} \small
\begin{table}[ht]
                \begin{center}
                    \begin{tabular}{c|c|c|c}
                        name&\parbox{3cm}{\centering number of\\ fitted parameters}&AIC&log-likelihood\\\hline
                        &\multicolumn{3}{c}{2007}\\\cline{2-4}
                        t&4&-238.455625&123.227812\\
                        normal&3&-236.101537&121.050768\\
                        gumbel&1&-115.395942&58.697971\\
                        frank&1&-114.767269&58.383634\\
                        clayton&1&-114.432476&58.216238\\\hline
                        &\multicolumn{3}{c}{2008}\\\cline{2-4}
                        normal&3&-308.304908&157.152454\\
                        t&4&-306.578106&157.289053\\
                        frank&1&-130.467471&66.233736\\
                        clayton&1&-118.717110&60.358555\\
                        gumbel&1&-99.941605&50.970803\\
                    \end{tabular}
                    \caption{Copula models for the 3-variate coal-fired power plant data}
                \end{center}
            \end{table}
}} \frame{\frametitle{One Day Risk Measures (Mio EURO)} {\tiny

            \begin{table}[!ht]
                \begin{center}
                    \begin{tabular}{c|c|c|c|c|c|c|c|c|c|c}
                        &t&normal&gumbel&clayton&frank&t&normal&gumbel&clayton&frank\\\hline
                        &\multicolumn{5}{c|}{2007, 0,9M t CO$_2$}&\multicolumn{5}{|c}{2007, 0,1M t CO$_2$}\\\cline{2-11}
                        VaR$_{0.9}$&0.53&0.54&0.70&0.77&0.72&0.50&0.50&0.50&0.29&0.50\\
                        VaR$_{0.95}$&0.77&0.77&1.01&1.13&1.07&0.75&0.74&0.76&0.44&0.76\\
                        VaR$_{0.99}$&1.65&1.74&2.16&2.79&2.38&1.63&1.64&1.68&0.86&1.68\\
                        ES$_{0.9}$&1.21&1.20&9.23&2.39&1.59&1.01&1.00&1.89&0.64&1.02\\
                        ES$_{0.95}$&1.79&1.77&17.63&3.86&2.32&1.41&1.40&3.17&0.91&1.44\\
                        ES$_{0.99}$&4.81&4.60&82.61&12.85&5.68&2.96&2.87&11.62&2.21&2.95\\\cline{2-11}
                        &\multicolumn{5}{c|}{2008, 0,9M t CO$_2$}&\multicolumn{5}{|c}{2008, 0,1M t CO$_2$}\\\cline{2-11}
                        VaR$_{0.9}$&0.53&0.53&0.71&0.79&0.72&0.49&0.49&0.49&0.28&0.48\\
                        VaR$_{0.95}$&0.71&0.71&0.97&1.04&0.98&0.69&0.69&0.71&0.40&0.69\\
                        VaR$_{0.99}$&1.18&1.20&1.65&1.73&1.73&1.28&1.29&1.35&0.67&1.35\\
                        ES$_{0.9}$&0.83&0.84&1.13&1.20&1.16&0.84&0.84&0.87&0.46&0.86\\
                        ES$_{0.95}$&1.04&1.06&1.43&1.50&1.50&1.11&1.10&1.15&0.59&1.15\\
                        ES$_{0.99}$&1.72&1.74&2.35&2.38&2.52&1.99&1.95&2.06&0.92&2.11
                    \end{tabular}
                    \caption{Risk measures for the coal-fired power plant (with CO$_2$)}
                \end{center}
            \end{table}
} \frame\frametitle{Coal-fired Power Plant}
\begin{figure}
\begin{center}
%\includegraphics[width=0.49\textwidth]{pics/EScoal09ret.pdf}
\includegraphics[heigth=5cm, width=7cm]{../../../pics/VaRPowerCoalCarbon2007.pdf}
\end{center}
\end{figure}

}\frame{\frametitle{Coal-fired
Power Plant} %\vspace{-2cm}
\begin{figure}
\begin{center}
\includegraphics[heigth= 5cm,width=7cm]{../../../pics/ESPowerCoalCarbon2007.pdf}
%\includegraphics[width=0.49\textwidth]{pics/VARcoal09ret.pdf}
\end{center}
\end{figure}

}


\frame{\frametitle{Comparison of Models}
\begin{itemize}
\item<1-> Copula-based models produce more accurate Risk Numbers
\item<2-> Marginal tails seem to be more important than joint tail behavior (see copulas)
\item<3-> Parametric approach allows dynamic modelling
\end{itemize}

