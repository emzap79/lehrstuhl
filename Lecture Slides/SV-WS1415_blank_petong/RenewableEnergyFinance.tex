% !TEX root = StructuringValuation_ws1314UDE.tex
\section{Renewable Energy Finance}
\subsection{Emission Certificates}
\begin{frame}
  \frametitle{Basic idea of cap-and-trade systems}
  \begin{itemize}
  \item<1-> \textbf{At the beginning} of the compliance period, the regulator \textbf{allocates} permits to the companies
  \item<2-> \textbf{During} the compliance period, the companies can \textbf{trade} permits among each other
  \item<3-> \textbf{At the end} of the compliance period, a regulated company has to \textbf{hand in} one permit or \textbf{pay a penalty fee} per unit of emission
  \end{itemize}
\end{frame}

\begin{frame}
  \frametitle{Permit price in the EU ETS during the first phase}
\begin{center}
\begin{figure}[h!]
\centering
\rotatebox{0}{
\scalebox{0.6}{
\includegraphics[width=1.45\textwidth, height=\textheight]{../../../pics/car-00-1-data.pdf}}}
\caption{EUA-Dec07 futures price (22 April 2005 - 17 December 2007).}
\label{fig:plotCar00-Data}
\end{figure}
\end{center}
\end{frame}


%\subsection{Calculating Permit Prices}
\begin{frame}
  \frametitle{Cumulative Emissions}
    To calculate permit prices, we specify the process for the cumulative emissions e.g. as     $$
      q_{[0,t]} = \int_0^t Q_s ds
    $$
    where the emission rate $Q_t$ follows a Geometric Brownian motion.


  There is no closed-form density for $q_{[0,t]}$ available.

\end{frame}

%8. Folie
\begin{frame}
  \frametitle{Approximation Approaches}
\begin{itemize}
\item<1-> Linear approximation approach
   $$
     q_{[t_1,t_2]} \approx \tilde{q}^{Lin}_{[t_1,t_2]} = Q_{t_2} (t_2 - t_1)
   $$
   %has the shortcoming that the moments of the cumulative emissions do not match the true ones.
\item<2-> Log-normal moment matching
$$
q_{[t_1,t_2]} \approx \tilde{q}^{Log}_{[t_1,t_2]} = logN \left(\mu_L(t_1,t_2), \sigma^2_L(t_1,t_2) \right) \label{ECumApprox2}
$$
where the parameters $\mu_L(t_1,t_2)$, $\sigma_L(t_1,t_2)$ are chosen such that the first two moments of $\tilde{q}^{Log}_{[t_1,t_2]}$ and $\tilde{q}^{IG}_{[t_1,t_2]}$, respectively, match those of $q_{[t_1,t_2]}$.
\end{itemize}
\end{frame}

%9. Folie
\begin{frame}
  \frametitle{Moment matching requires two steps}
    \begin{itemize}
   \item<1-> Compute the first two moments $m_k$ of a log-normal random variable and solve for the parameters. \\
     In the log-normal case we have that $m_k = e^{k\mu + k^2 \frac{\sigma^2}{2}}$ and
     $$
     \sigma^2 = \ \ln\left( \frac{m_2}{m^2_1}\right)  \qquad \mu = \ \ln(m_1) - \frac{1}{2}  \sigma^2
     $$
    \item<2-> Compute the first two moments of the integral over a geometric Brownian motion
    $$\begin{array}{lll}
\EW \left[ q_{[t_1,t_2]} \right] &=& \ Q_{t_1} \alpha_{t_2-t_1} \\
\EW \left[ \left( q_{[t_1,t_2]} \right)^2 \right] &=& \ 2 Q_{t_1}^2 \beta_{t_2-t_1}
\end{array}
$$
    and plug those into the above equation.
    \end{itemize}
\end{frame}

\begin{frame}
  \frametitle{Auxiliary functions for moments of integral over GBM}
\begin{align}
\alpha_{t_2-t_1}    =& \ \left\{
         \begin{array}{ll}
            \frac{1}{\mu} \left( e^{\mu \left( t_2 - t_1 \right)} - 1 \right)
            & \quad \mbox{if $\mu \ne 0$} \\
            t_2 - t_1
            & \quad \mbox{if $\mu = 0$} \\
         \end{array} \right. \label{MomIntAlpha} \\
%\beta_{t_2-t_1} =& \  \frac{ \mu e^{(2 \mu + \sigma^2) \left( t_2 - t_1 \right)} + \mu + \sigma^2 - \left( 2\mu + \sigma^2\right) e^{\mu \left( t_2 - t_1 \right)}}{\mu(\mu + \sigma^2)(2\mu + \sigma^2)} \label{MomIntBeta}
\beta_{t_2-t_1}    =& \ \left\{
         \begin{array}{ll}
            \frac{ \mu e^{(2 \mu + \sigma^2) \left( t_2 - t_1 \right)} + \mu + \sigma^2 - \left( 2\mu + \sigma^2\right) e^{\mu \left( t_2 - t_1 \right)}}{\mu(\mu + \sigma^2)(2\mu + \sigma^2)}
            & \quad \mbox{if $\mu \ne 0$} \\
            \frac{1}{\sigma^4} \left( e^{\sigma^2\left(t_2-t_1\right)} - 1\right)
            & \quad \mbox{if $\mu = 0$} \\
         \end{array} \right. \label{MomIntBeta}
\end{align}
\end{frame}

%11. Folie
%\subsection[Permit Prices]{Permit prices for different approximation approaches}
\begin{frame}
  \frametitle{Permit price - linear approximation}

    The permit price at time t is given by
  $$
  S_t^{Lin} = \ \left\{
         \begin{array}{ll}
            P e^{-r\tau}
            & \quad \mbox{if $q_{[0,t]} \ge N$} \\
            P e^{-r\tau} \cdot \Phi \left(\frac{-\ln\left( \frac{1}{\tau} \left[ \frac{N - q_{[0,t]}}{Q_t} \right] \right) + \left( \mu - \frac{\sigma^2}{2}\right)\tau}{\sigma \sqrt{\tau}} \right) & \quad \mbox{if $q_{[0,t]} < N$} \\
         \end{array} \right.
$$
where \\
$\tau = T - t$ is the time to compliance. \\
$\Phi(\cdot)$ denotes the c.d.f. of a standard normal random variable.

\end{frame}

%12. Folie
\begin{frame}
  \frametitle{Permit price - log-normal moment matching}

    The permit price at time t is given by
$$
S_t^{Log} = \ \left\{
         \begin{array}{ll}
            P e^{-r\tau}
            &\mbox{if $q_{[0,t]} \ge N$} \\
            P e^{-r\tau} \cdot \Phi \left(\frac{-\ln\left( \frac{N - q_{[0,t]}}{Q_t} \right) + 2\ln(\alpha_{\tau}) - \frac{1}{2} \ln(2\beta_{\tau})}{\sqrt{\ln(2\beta_{\tau}) - 2\ln(\alpha_{\tau})}} \right) & \mbox{if $q_{[0,t]} < N$} \\
         \end{array} \right.
$$
where \\
$\tau = T - t$ is the time to compliance and \\
$\alpha_{\tau}, \beta_{\tau}$ are obtained by calculating the first and the second moment of the integral over a geometric Brownian motion. \\
$\Phi(\cdot)$ denotes the c.d.f. of a standard normal random variable.

\end{frame}


%14. Folie
\begin{frame}
  \frametitle{Relating theoretical permit prices to allocation}
    We introduce the following two random variables that are very easy to interpret\\
    {\bf Time needed to exhaust the remaining permits}
$$
x_t := \frac{N - q_{[0,t]}}{Q_t}
$$

and
{\bf Over-/Underallocation in years}
$$
x_t - (T-t)
$$
\end{frame}



\begin{frame}
  \frametitle{Implied over-/underallocation during the first phase of the EU ETS}
\begin{center}
\begin{figure}[h!]
\centering
\rotatebox{0}{
\scalebox{0.6}{
\includegraphics[width=1.35\textwidth, height=1.05\textheight]{../../../pics/fig-implied1.pdf}}}
{\tiny \caption{Implied $x_t - (T-t)$ for first phase for fixed  $\mu = 0.02$ and $\sigma = 0.05$. Linear approximation approach (straight line), log-normal moment matching (dashed line). Positive values correspond to overallocation.}}
\label{fig:plot10}
\end{figure}
\end{center}
\end{frame}


%\subsection[2006 Price Slump]{Theoretical discussion of permit price slump in 2006}




% frametitle
{Price Floor Using a Subsidy}
\begin{itemize}
\item<1-> The severe permit price drop, followed by a price hovering above zero for more than five months during the first phase of the EU ETS, persuaded several policy makers that new cap-and-trade schemes would need additional safety-valve features.
\item<2-> In particular, policy makers have been concerned about permit prices that are either too low or too high.
\item<3-> Thus setting a price floor and/or ceiling has been proposed.
\end{itemize}



% frametitle
{Price Floor Using a Subsidy -- Regulation}
\begin{itemize}
\item<1-> A company with a permit shortage at compliance date faces a penalty $P$.
\item<2-> If a company ends up with an excess of permits, it receives a subsidy $S$ per unit of permit.
\item<3-> Let $0<S\leq P$ and let $N$ be the initial amount of permits allocated to relevant companies.
\end{itemize}


% frametitle
{Permit Price in hybrid system}
Denote the futures permit price by $\tilde{F}(t,T)$:
\begin{align*}
\tilde{F}(t,T) =& \ P \cdot \PM \left( q_{[0,T]} > N \mid \Ft \right) + S \cdot \PM \left( q_{[0,T]} \le N \mid \Ft \right) \\
  =& \ P \cdot \PM \left( q_{[0,T]} > N \mid \Ft \right) + S \cdot \left( 1 - \PM \left( q_{[0,T]} > N \mid \Ft \right)\right) \\
 % =& \ S + (P-S) \cdot \PM \left( q_{[0,T]} > N \mid \Ft \right)
  =& \ S + \frac{P-S}{P} \cdot P \cdot \PM \left( q_{[0,T]} > N \mid \Ft \right) \\
  =& \ S + \frac{P-S}{P} \cdot F(t,T) = \ F(t,T) + S\left(1 - \frac{F(t,T)}{P}\right),
\end{align*}
where $F(t,T) = P \cdot \PM \left( q_{[0,T]} > N \mid \Ft \right)$ is the futures permit price in an ordinary system.



% frametitle
{Decomposition of permit price in hybrid system}
Computing the value of a put with strike $S$ shows that the price in the hybrid scheme is the price in the ordinary scheme plus the value of a put option on the price in the ordinary scheme with strike $S$ and maturity $T$:
$$
\begin{array}{ll}
&\EW\left[ (S-F(T,T))^+ \mid \Ft \right]\\*[12pt]
=& \ \EW\left[ \left( S- P \textbf{1}_{\left\{ q_{[0,T]} > N \right\}} \right)^+ \mid \Ft \right] \\*[12pt]
=& \ \left( S- P \right)^+ \PM \left( q_{[0,T]} > N \mid \Ft \right) + \left( S- 0 \right)^+ \PM \left( q_{[0,T]} \le N \mid \Ft \right) \\*[12pt]
 \stackrel{S<P}{=}& \ S \cdot \PM \left( q_{[0,T]} \le N \mid \Ft \right).
\end{array}
$$




% frametitle
{Enforcement costs for regulator}
\begin{itemize}
\item<1-> A price floor ensured by the presence of a subsidy is relatively easy to implement and has the further advantage of lowering the expected enforcement costs for regulated companies.
\item<2-> The presence of the subsidy could induce a higher stimulus in technology and abatement investments, favoring the achievement of emission reduction targets.
\item<3-> However, the implementation of such a hybrid system might result in a significant financial burden for the environmental policy regulator. The current magnitude of this burden can be obtained by calculating the price of the put option.
\end{itemize}




%\subsection{Dynamic Reduced Form Models}
\begin{frame}
\frametitle{Permit Prices}
\begin{itemize}
\item<1-> Recall the emission rate
$$
dQ_t = Q_t(\mu dt + \sigma dW_t)
$$
\item<2->
The cumulative emissions are
$$
q_{[0,t]} = \int_0^t Q_s ds
$$
\item<3->
The futures permit price is given as
$$
F(t,T) = P \prb\left[q_{[0,T]}>N |\F_t\right]
$$
\end{itemize}
\end{frame}

\begin{frame}
\frametitle{Approximative Pricing}
\begin{itemize}
\item<1-> Linear approximation approach
$$
\begin{array}{lll}
q_{[t_1,t_2]} &\approx& \tilde{q}^{Lin}_{[t_1,t_2]} = Q_{t_2} (t_2 - t_1) \\*[12pt]
&=&\displaystyle   Q_{t_1} e^{\left\{\log (t_2 - t_1) + \left(\mu-\frac{\sigma^2}{2}\right)(t_2-t_1)+\int_{t_1}^{t_2}\sigma dW_t\right\}}
\end{array}
$$
\item<2-> Moment matching
$$
\begin{array}{lll}
q_{[t_1,t_2]} &\approx& \tilde{q}^{Log}_{[t_1,t_2]}\\*[12pt]
&=& Q_{t_1} \exp\left\{ \int_{t_1}^{t_2}\mu_t dt + \int_{t_1}^{t_2} \sigma_t dW_t\right\}
\end{array}
$$
where the functions $\mu_t$ and $\sigma_t$ are defined by the functions $\alpha_t, \beta_t$ from the moment matching.
\end{itemize}
\end{frame}
%\subsection[Reduced Form Dynamics]{Dynamics of the permit process in the reduced form model}

\begin{frame}
\frametitle{Carmona-Hinz Approach}
\begin{itemize}
\item<1-> Use a lognormal process
$$
\Gamma_{T}= \Gamma_0  \exp{\left\{\int_{0}^{T}\sigma_t dW_t -\frac{1}{2}\int_0^T \sigma^2_t dt\right\}}
$$
with $\Gamma_0 >0$ and $\sigma(.)$ a deterministic square-integrable function.
\item<2-> Define the futures price under a risk-neutral measure $\Q$ as
$$
F(t,T) = P \Q\left[\Gamma_T>1 |\F_t\right]
$$
\end{itemize}
\end{frame}


% frametitle
{Reduced-Form Dynamics}
The martingale
$$
a_t = \EX^\Q\left[\IF_{\{\Gamma_T>1\}} |\F_t\right]
$$
is given by
$$
a_t= \Phi \left[\frac{\Phi^{-1}(a_0) \sqrt{\int_0^T \sigma^2_s ds}+\int_0^t \sigma_s dW_s}{\sqrt{\int_t^T \sigma^2_s ds}}\right]
$$
and solves the stochastic differential equation
$$
da_t = \Phi'\left(\Phi^{-1}(a_t)\right)\sqrt{z_t}dW_t
$$
with
$$
z_t=\frac{\sigma_t^2}{\int_t^T \sigma^2_u du}
$$


% frametitle
{Reduced-Form Dynamics -- Proof}
\begin{itemize}
\item<1-> $a_t$ formula is straightforward calculation
\item<2-> For dynamics use that
$$
a_t = \Phi(\xi_t)
$$
with
$$
\xi_t = \frac{\xi_{0,T}+\int_0^t\sigma_s dW_s}{\sqrt{\int_t^T\sigma_s^2ds}}\; \mbox{and}\;  \xi_{0,T}=\log \Gamma_0 - \frac{1}{2} \int_0^T\sigma_s^2ds.
$$
Starting with the dynamics of $\xi_t$ an application of It{\^o}'s formula gives the result.
\end{itemize}


%\subsection{Historical Calibration}

% frametitle
{Model Parametrization}
\begin{itemize}
\item<1-> For constant $\sigma$ we find $z_t=(T-t)^{-1}$, so a richer specification is needed.
\item<2-> A standard model is
$$
da_t = \Phi'\left(\Phi^{-1}(a_t)\right)\sqrt{\beta(T-t)^{-\alpha}}dW_t
$$
which specifies a family $\sigma_s(\alpha,\beta)$.
\item<3->
So $z_t(\alpha, \beta) = \beta(T-t)^{-\alpha}$ and
$$
\begin{array}{lll}
\sigma_t^2(\alpha,\beta)&=& \displaystyle z_t(\alpha, \beta) \exp\left\{-\int_0^t z_s(\alpha, \beta) ds \right\}\\*[12pt]
&=&\displaystyle
\left\{
\begin{array}{ll}
\beta(T-t)^{-\alpha} e^{-\frac{\beta}{1-\alpha}[T^{1-\alpha}-(T-1)^{1-\alpha}]} &\alpha \not=1\\
\beta(T-t)^{\beta-1}T^{-\beta} &\alpha=1.
\end{array}
\right.
\end{array}
$$
\end{itemize}


% frametitle
{Objective Measure}
\begin{itemize}
\item<1-> We do a historical calibration and change measure to the objective measure.
\item<2-> The standard change of measure gives
$$
\frac{d\prb}{d\Q} = \exp\left\{\int_0^T H_s dW_s -\frac{1}{2} \int_0^T H_s^2ds \right\}).
$$
\item<3->
Under constant market price of risk $H_t \equiv h$ and by Girsanov's theorem
$$
\tilde{W}_t = W_t - ht
$$
is a $\prb$ Brownian motion.
\end{itemize}


% frametitle
{Objective Measure}
\begin{itemize}
\item<1->
Under $\prb$
$$
d\xi_t = \left(\frac{1}{2} z_t \xi_t + h \sqrt{z_t} \right)dt + \sqrt{z_t} d\tilde{W}_t,
$$
so $\xi_{\tau}$ given $\xi_t$ is Gaussian.
\item<2-> So we can invert permit prices to obtain $\xi$ values and calculate the log-likelihood to obtain
estimates for $\alpha$ and $\beta$.
\end{itemize}



%\subsection{Option Pricing}

% frametitle
{Pricing Formula}
For a European call with strike $K$ and maturity $\tau$ the option price is
$$
C_t = e^{-\int_t^\tau r_s ds} \int_{-\infty}^\infty (P\Phi(x)-K)^+ \Phi_{\mu_{t,\tau}, \sigma_{t,\tau}}(dx)
$$
with
$$
\mu_{t,\tau}=
\left\{
\begin{array}{ll}
\xi_t \left(\frac{T-t}{T-\tau}\right)^{\frac{\beta}{2}} & \alpha =1\\
\xi_t \exp\left\{\frac{\beta}{2(1-\alpha)}[(T-t)^{1-\alpha}-(T-\tau)^{1-\alpha}]\right\} & \alpha \not= 1.
\end{array}
\right.
$$
and
$$
\sigma^2_{t,\tau}=
\left\{
\begin{array}{ll}
\left(\frac{T-t}{T-\tau}\right)^\beta-1 & \alpha =1\\
\exp\left\{\frac{\beta}{1-\alpha}[(T-t)^{1-\alpha}-(T-\tau)^{1-\alpha}]\right\}-1 & \alpha \not= 1.
\end{array}
\right.
$$





%\subsection{Reduced Form Option Pricing in Multi-period Models}

% frametitle
{Two-period Model}
\begin{itemize}
\item<1-> We consider a two-period model, $[0,T]$ and $[T,T']$, with banking and withdrawal
\item<2-> Let $\Q$ be a martingale measure and $(A_t)_{t\in[0,T]}$ and $(A'_t)_{t\in[0,T']}$
be the futures contracts which are $\Q$ martingales.
\item<3-> Let $N\in \F_T$ resp. $N' \in \F_{T'}$ be non-compliance in the first resp. second period.
\end{itemize}


% frametitle
{(Non-) Compliance at $T$}
\begin{itemize}
\item<1-> In case of compliance, i.e. event $\Omega-N$, since $A_T$ is the spot price at $T$ and the permit can be banked,  we have
$$
A_T\IF_{\{\Omega-N\}}= \kappa A'_T \IF_{\{\Omega-N\}}
$$
with $\kappa= \exp\{-\int_T^{T'}r_s ds\}$ a discount factor.
\item<2-> In case of non-compliance
$$
A_T\IF_{N}= \kappa A'_T \IF_{N} + P\IF_N
$$
\item<3-> Thus
$$
A_t-\kappa A'_t = \EX_t^\Q[A_T-\kappa A'_T]= \EX^\Q_t[P \IF_N]
$$
\end{itemize}



% frametitle
{Reduced-Form Model}
\begin{itemize}
\item<1-> As in the one period case,  we set
$$
A_t-\kappa A'_t = P \Phi(\xi_t^1)
$$
with $\xi^1_t$ a Gaussian process driven by a Brownian motion $W^1_t$.
\item<2-> Assume that the ETS ends after the second period, then
$$
A'_t = P \Phi(\xi^2_t)
$$
with $\xi^2_t$ a Gaussian process driven by a Brownian motion $W^2_t$.
\end{itemize}


% frametitle
{Pricing Formula}
For a European call written on a futures in the first period with strike $K$ and maturity $\tau$ we decompose the payoff
$$
(A_\tau-K)^+= (A_\tau - \kappa A'_\tau + \kappa A'_\tau -K )^+= (P\Phi(\xi^1_t) + \kappa P \Phi(\xi_t^2) -K)^+
$$
We obtain for the option price
$$
C_t = e^{-\int_t^\tau r_s ds} \int_{\setR^2} (P\Phi(x_1)+\kappa P \Phi(x_2) -K)^+ \Phi_{\mu_\tau, \sigma_\tau}(dx_1dx_2)
$$
where the parameters of the two-dimensional Gaussian distribution depend on the individual drift and volatility terms and the correlation
of $\xi^1$ and $\xi^2$.


\subsection{Renewable Energy Projects on Capital Markets}


% frametitle
{Carbon Revenue Bonds}
\begin{itemize}
\item<1-> To finance high initial cost for Renewable Energy (RE) projects future returns of the project are securitized
\item<2->
\begin{itemize}
\item Sell future electricity from RE project
\item Sell environmental credits from RE project
\end{itemize}
\item<3-> Only revenue from environmental credits is used for bond
\item<4-> Rigorous forecast analysis of revenues, sensitivity tests and risk analysis is required.
\end{itemize}




% frametitle
{Structure of Bond}
\begin{itemize}
\item<1-> pass-through: all revenues are directly passed trough to the owner of the bond
\item<2->
\begin{itemize}
\item maturity: T
\item revenues in year 1: $c_i$
\item rate of return: $r$
\item initial price: $x$
\end{itemize}
\item<3-> Fair price
$$
x= \sum_{i=1}^T \frac{c_i}{(1+r)^i}
$$
\end{itemize}



% frametitle
{EUA Time Series}
\begin{figure}[h!]
\centering
\includegraphics[width=0.9\textwidth, height=0.7\textheight]{../../../pics/EUA-TimeSeries.pdf}
%\caption{EUA Time Series}
\label{fig:EUA-TS}
\end{figure}


% frametitle
{QQ-Plots for EUA fits}
\begin{figure}[h!]
\centering
\includegraphics[width=0.9\textwidth, height=0.7\textheight]{../../../pics/QQ-CarbonFit.pdf}
%\caption{EUA fits}
\label{fig:EUA-fits}
\end{figure}


% frametitle
{Carbon Bond Histogram}
\begin{figure}[h!]
\centering
\includegraphics[width=0.9\textwidth, height=0.7\textheight]{../../../pics/carbon-bond-histogram.pdf}
%\caption{Carbon Bond Histogram}
\label{fig:Carbon-Bond-Histogram}
\end{figure}



%\subsection{Impact of Carbon Markets on Investment Markets}


% frametitle
{Financial Implications of Carbon Policies}
\begin{itemize}
\item<1-> Modern risk management has to include the consequences of climate change.
\item<2-> Regulatory risk (reduction targets for carbon emissions) transforms into financial risk  for several asset classes.
\item<3-> Carbon inefficient firms tend to have a higher credit spread and higher refinancing costs.
\end{itemize}



% frametitle
{Factors Affecting Carbon Risk}
\begin{itemize}
\item<1-> Energy intensity and fuel mix -- firms that are dependent on fossil fuels face higher costs.
\item<2-> Direct, indirect and embedded emissions of a firm's product affect market position.
\item<3-> Marginal abatement costs.
\item<4-> Technology trajectory -- progress in adapting low-carbon emission technologies.
\end{itemize}



% frametitle
{Motivation for Investors to Invest in Carbon-related Assets}
\begin{itemize}
\item<1-> Financial Motivation
\begin{itemize}
\item Portfolio diversification
\item Potential fundamental price appreciation of carbon
\item Hedging financial risk due to carbon price increases
\end{itemize}
\item<2-> Green Motivation
\begin{itemize}
\item Compliance with UN principles of responsible investment (UN PRI)
\item Public opinion, behaviour as corporate citizen
\item Incentivizing the corporate sector by taking carbon credits from the market
\end{itemize}
\end{itemize}



% frametitle
{Risk In Renewable Energy Companies}
\begin{itemize}
\item<1-> Costs: as the costs of producing RE come down while the costs of producing fossil fuels rise, a
substitution will occur.
\item<2-> Capital: Government and private capita
\item<3-> Competition: between governments as they try to build greener economies
\item<4-> China: huges efforts to establish a green economy
\item<5-> Consumers: demand products with less impact on the economy
\item<6-> Climate Change: will be tackled by investment in greener technologies.
\end{itemize}



% frametitle
{CAPM Analysis of RE Companies}
\begin{itemize}
\item<1-> Empirical evidence shows that RE companies have a $\beta$ close to two.
\item<2-> Model: $i$ firm, $t$ time
$$
R_{it}= \alpha_i + \beta_{it} R_{mt}+\epsilon_{it}
$$
$R$ returns, $\alpha$ component that is independent of the market.
\item<3-> Higher beta values indicate a higher equity cost of capital.Investors must then be compensated
through higher expected returns in order to take on the higher risk. A higher equity cost of capital can affect borrowing costs and
it can also affect the discount rate used in net present value calculations.
\end{itemize}



% frametitle
{CAPM Analysis of RE Companies II}
\begin{itemize}
\item<1-> Which factors affect systematic risk (the $\beta$)?
\item<2-> Empirical analysis shows:
\begin{itemize}
\item Increases in sales growth reduce market risk
\item Increases in oil price returns increases systematic risk
\end{itemize}
\item<3-> In order to foster RE companies governments can implement policies to increase sales of such companies (e.g. PV-industry and feed-in tariffs).
\end{itemize}




