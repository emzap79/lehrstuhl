\section{Derivatives Risk Management}

\subsection{Introduction}
\begin{frame}[fragile]
\frametitle{Motivation}
A company that has a position in options, as a financial derivative or implicit
in a contract, has to think about managing the risk of the position. The risk
manager should ask the following questions:
\begin{itemize}
  \item What are the risk drivers influencing the value of the position?
  \item How does the value of the position change if risk drivers change?
  \item How can you protect (i.e., hedge) yourself against those risk?
\end{itemize}
We focus on answering the last two questions.
\end{frame}

\begin{frame}[fragile]
\frametitle{Some Terms}
A \emph{hedge} is a trade that reduces the risk of a position.
\begin{itemize}
  \item A \emph{static} hedge is set up once and not touched until maturity of
  the derivative/position that is hedged.
  \item A \emph{dynamic} hedge is set up and adjusted continuously (in practice:
  regularly).
  \item A \emph{semi-static} hedge is set up and adjusted at infrequent
  intervals.
\end{itemize}
A position for which no hedge is in place is called \emph{naked}, a position
that is fully hedged is called \emph{covered} (e.g., sell a forward and buy the
underlying stock).
\end{frame}

\begin{frame}[fragile]
\frametitle{A simple Hedging Strategy: Stop-Loss}
We consider a short position in one European call option with strike $K$.
\begin{itemize}
  \item A stop-loss hedge involves buying one share of the underlying stock when
  the price moves above $K$ and selling it again when the price falls below $K$.
  \item The strategy usually does not work well in practice because the trader
  cannot know whether the stock price moves above $K$ or falls below $K$ until
  it is too late. Therefore, he buys the stock at $K+\epsilon$ and sells at
  $K-\epsilon$, resulting in a loss.
  \item Also, discounting has to be taken into account.
\end{itemize}
\end{frame}

\subsection{The Greeks}
\begin{frame}[fragile]
\frametitle{The Greeks}
\begin{itemize}
\item In order to quantify the risk associated with an instrument, one looks at
\emph{ how much the price of the instrument changes if one of the underlying
risk drivers changes} its value. Those risk measures are often called
Greeks.
\item Mathematically speaking, the Greeks are just the derivative of the price
of the instrument with respect to the value of the risk driver.
\item Once those Greeks are known for a portfolio, one can easily calculate how
much the value of an option or a portfolio changes \emph{marginally}, if one of
the variables changes \emph{marginally}, all others remain fixed.
\item The most important Greeks are Delta, Gamma, Vega, Rho, and Theta.
\end{itemize}
\end{frame}

\begin{frame}[fragile]
\frametitle{The Delta}
\begin{itemize}
\item Delta is the derivative of the instrument price with respect to the price
of the underlying.
\item The delta of the underlying security is one.
\item If the payoff is not linear (for example if the instrument is an option),
Delta is not constant.
\item As Delta is the derivative with respect to the underlying, it tells how
much the value of the instrument changes if the value of the underlying changes
marginally.
\item Thus in a continuous time model, Delta is the amount of the underlying
needed to be sold in order to offset the price change of the instrument. The
ability to neutralize the trading book with respect to price changes in the
underlying makes Delta the most important Greek.
\end{itemize}
\end{frame}

\begin{frame}[fragile]
\frametitle{Delta Hedging I}
\begin{itemize}
  \item Assume that you have an options position with delta $\Delta$. This means
  that if the price of the underlying moves by a (very) small amount $\epsilon$, the
  price of the option position moves approximately by $\Delta \epsilon$.
  \item A \emph{delta hedge} consists of selling $\Delta$ units of the
  underlying (buying if $\Delta <0$) and gives a portfolio that does not
  change its value if the price of the underlying changes by a marginal amount.
  The portfolio has a delta of zero and is called \emph{delta neutral}.
\end{itemize}
\end{frame}

\begin{frame}[fragile]
\frametitle{Delta Hedging II}
\begin{itemize}
  \item BUT: Delta is not constant, so the portfolio has to be rebalanced.
  \item In the Black-Scholes model, a delta hedge with continuous rebalancing is
  a perfect hedge.
  \item In practice, continuous rebalancing is not possible so that the
  portfolio is not protected against larger market movements. Transaction
  costs and bid-ask spread also result in losses.
\end{itemize}
\end{frame}



\begin{frame}[fragile]
\frametitle{Delta in the Black-Scholes Model}
In the Black-Scholes model, the Greeks can be computed explicitly for
European call and put options. We have:
\begin{align*}
  \Delta = \frac{\partial}{\partial S}Call_{BS}(S,K,\sigma,r,t,T) = N(d_1)
\end{align*}
where, as usual, $N$ denotes the c.d.f. of the standard normal distribution and
\begin{align*}
  d_1 = \frac{\log \left( \frac{S}{K} \right) + \left( r + \frac{\sigma^2}{2}
  \right)(T-t)}{\sigma \sqrt{T-t}}.
\end{align*}
\end{frame}

\begin{frame}[fragile]
\frametitle{Delta of a Call Option in the Black-Scholes Model}
%\usepackage{graphics} is needed for \includegraphics
\begin{figure}[htp]
\begin{center}
  \includegraphics[width=0.8\textwidth]{../pics/delta}
  \caption{Delta for a European call in the BS model, $T=1$, $r=1\%$,
  $\sigma=20\%$, $K=100$.}
  \label{fig:deltaBS}
\end{center}
\end{figure}
\end{frame}

\begin{frame}[fragile]
\frametitle{Example: Delta Hedge}
You are long USD $1,000$ in the $104$ call. Interest rate is $5\%$,
stock price today is $99$, time to maturity $1$ month, and implied volatility is
$15.7\%$.
\begin{itemize}
  \item How can you make your portfolio delta neutral by investing in the stock?
  \item You set up the delta neutral portfolio and the stock price jumps to
  USD$100$ immediately. What is your P/L for the portfolio?
\end{itemize}

\end{frame}

\begin{frame}[fragile]
\frametitle{Example: Delta Hedge}
\begin{itemize}
  \item Compute the price of the call with the BS formula:
  \begin{align*}
    Call_{BS} = 0.3858 \text{USD}.
  \end{align*}
  \item The position consists of $N=1,000/Call_{BS}=2592$ call options.
  \item The delta of each option is
  \begin{align*}
    \Delta &= N(d_1) =N(\frac{\log \left( S/K \right) + (r+\sigma^2/2)(T-t)
    }{\sigma\sqrt{T-t}}) \\
    	&= N(\frac{\log \left( 99/104 \right) + (0.05+0.157^2/2)(1/12)
    }{0.157\sqrt{1/12}})\\
     	&= 0.1654.
  \end{align*}
\end{itemize}
\end{frame}


\begin{frame}[fragile]
\frametitle{Example: Delta Hedge}
\begin{itemize}
  \item The delta of the position is long $\Delta_P=N\cdot \Delta=428.70$.
  \item The stock has $\Delta=1$.
  \item To make the position delta neutral, you have to enter a short position
  of $428.70$ shares.
\end{itemize}
\end{frame}

\begin{frame}[fragile]
\frametitle{Example: Delta Hedge}
\begin{itemize}
  \item The loss from the short position in the stock is $428.70 \cdot
  1=428.70$.
  \item To compute the gain from the long options position, we have to compute
  the option price for $S=100$. Using the BS formula, we obtain
  \begin{align*}
    Call_{BS}(S=100) = 0.5808.
  \end{align*}
  \item The gain from the options position is $2592\cdot(0.5808-0.3858)=505.37$.
  \item Our profit is $505.37-428.70=76.67$.
\end{itemize}
\end{frame}

\begin{frame}[fragile]
\frametitle{The Gamma}
\begin{itemize}
  \item Gamma is the second derivative of the instrument price with respect to
  the price of the underlying.
  \item If the instrument has a linear payoff, Gamma is zero.
  \item As delta is the first derivative of the option price with respect to the
  underlying, gamma is the derivative of delta with respect to the underlying
  and thus measures, how much Delta changes if the underlying changes.
  \item This is an important information in risk management as it tells the
  trader how much of the underlying he has to buy or sell if the underlying
  itself changes price.
  \item Geometrically, gamma might be seen as the slope of Delta.
\end{itemize}
\end{frame}

\begin{frame}[fragile]
\frametitle{Gamma Neutral Portfolios}
As the delta of a portfolio changes with the price of the underlying, a delta
hedge has to be rebalanced frequently.
\begin{itemize}
  \item A delta hedge for a portfolio with a high (absolute) gamma has to be
  monitored more carefully than a delta hedge of a portfolio with gamma close to
  zero.
  \item To decrease the hedging error, a trader might want to make a delta
  neutral portfolio gamma neutral.
  \item The gamma cannot be changed by investing in the underlying because its
  gamma is zero.
  \item Strategy: Make the portfolio gamma neutral by investing in an option,
  then make it delta neutral by investing in the underlying.
\end{itemize}
\end{frame}


\begin{frame}[fragile]
\frametitle{Gamma in the Black-Scholes Model}
The gamma of a European call option in the Black-Scholes model is given by
\begin{align*}
  \Gamma = \frac{\partial^2}{\partial S^2}Call_{BS}(S,K,\sigma,r,t,T) =
  \frac{\varphi(d_1)}{S\sigma \sqrt{T-t}}.
\end{align*}
Here, $\varphi(x)$ denotes the p.d.f. of the standard normal distribution.
\end{frame}

\begin{frame}[fragile]
\frametitle{Gamma of a Call Option in the Black-Scholes Model}
\begin{figure}[htp]
\begin{center}
  \includegraphics[width=0.8\textwidth]{../pics/gamma}
  \caption{Gamma for a European call in the BS model, $T=1$, $r=1\%$,
  $\sigma=20\%$, $K=100$.}
  \label{fig:gammaBS}
\end{center}
\end{figure}
\end{frame}

\begin{frame}[fragile]
\frametitle{Gamma of a Call Option in the Black-Scholes Model}
\begin{figure}[htp]
\begin{center}
  \includegraphics[width=0.8\textwidth]{../pics/gamma2}
  \caption{Gamma for a European call in the BS model, $T=10$, $r=1\%$,
  $\sigma=20\%$, $K=100$.}
  \label{fig:gamma2BS}
\end{center}
\end{figure}
\end{frame}

\begin{frame}[fragile]
\frametitle{Example: Gamma Neutral Portfolio}
You are in the same position as in the previous example. Additionally, you can
trade in the $97$ call.
\begin{itemize}
  \item Put together a portfolio that is delta and gamma neutral.
  \item What is your P/L if the stock price jumps to $100$?
\end{itemize}
\end{frame}

\begin{frame}[fragile]
\frametitle{Example: Gamma Neutral Portfolio}
\begin{itemize}
  \item The gamma of the $104$ call is
  \begin{align*}
  \Gamma_{104C} = \frac{\varphi(d_1)}{S\sigma \sqrt{T-t}} = 0.0554.
  \end{align*}
  \item The gamma of the $97$ call is
  \begin{align*}
  \Gamma_{97C} = \frac{\varphi(d_1)}{S\sigma \sqrt{T-t}} = 0.0758.
  \end{align*}
  \item To make the portfolio gamma neutral, we have to build up a position of
  $n$ $97$ calls so that $2592\cdot 0.0554 +n\cdot 0.0758 = 0$.
  \item We find that $n=-1894.74$ makes the portfolio gamma neutral, i.e., we
  sell $1894.74$ units of the $97$ call.
\end{itemize}
\end{frame}

\begin{frame}[fragile]
\frametitle{Example: Gamma Neutral Portfolio}
\begin{itemize}
  \item We compute the delta of the portfolio consisting of the two positions in
  the calls.
  \item The $97$ call has a delta of $\Delta_{97C}=0.7139$.
  \item The delta of the position is
  \begin{align*}
    \Delta_{P'} = 2592 \cdot 0.1654 - 1894.74 \cdot 0.7139 = -924.02
  \end{align*}
  \item We have to buy $924.02$ units of the underlying to make the portfolio
  delta neutral. By buying the underlying, we do not change the gamma of the
  portfolio, it remains zero.
\end{itemize}
\end{frame}

\begin{frame}[fragile]
\frametitle{Example: Gamma Neutral Portfolio}
\begin{itemize}
  \item The price of the $97$ call for $S=99$ is $Call_{BS,97}(S=99)=3.2235$.
  \item The price of the $97$ call for $S=100$ is $Call_{BS,97}(S=100)=3.9735$.
  \item The P/L from the position in the $97$ call is
  $-1894.97*(3.9735-3.2235)=-1421.11$.
  \item The P/L from the position in the stock is $924.02$.
  \item The portfolio P/L is $924.02-1421.11+505.37=-8.28.$
\end{itemize}
\end{frame}

\begin{frame}[fragile]
\frametitle{The Vega}
\begin{itemize}
  \item Vega is the derivative of the instrument price with respect to
  implied volatility.
  \item Thus, vega indicates how much the option price changes if the implied
  volatility changes.
  \item Vega is not a Greek letter.
\end{itemize}
\end{frame}

\begin{frame}[fragile]
\frametitle{Vega in the Black-Scholes Model}
The vega of a European call option in the Black-Scholes model is given by
\begin{align*}
  \nu = \frac{\partial}{\partial \sigma}Call_{BS}(S,K,\sigma,r,t,T) =
  S\sqrt{T-t} \varphi(d_1).
\end{align*}
Therefore, we have
\begin{align*}
  \nu = S^2 \sigma (T-t) \Gamma.
\end{align*}
\end{frame}

\begin{frame}[fragile]
\frametitle{Vega of a Call Option in the Black-Scholes Model II}
\begin{figure}[htp]
\begin{center}
  \includegraphics[width=0.8\textwidth]{../pics/vega}
  \caption{Vega for a European call in the BS model, $T=1$, $r=1\%$,
  $\sigma=20\%$, $K=100$.}
  \label{fig:vegaBS}
\end{center}
\end{figure}
\end{frame}

\begin{frame}[fragile]
\frametitle{Vega of a Call Option in the Black-Scholes Model III}
\begin{figure}[htp]
\begin{center}
  \includegraphics[width=0.8\textwidth]{../pics/vega_impliedvol}
  \caption{Vega for a European call in the BS model, $T=1$, $r=1\%$, $K=100$.}
  \label{fig:vega2BS}
\end{center}
\end{figure}
\end{frame}

\begin{frame}[fragile]
\frametitle{The Theta}
\begin{itemize}
  \item Theta is the derivative of the instrument price with respect to
  time to maturity.
  \item Thus, theta indicates how much the option price changes as time moves
  closer to maturity.
  \item Theta is usually negative and therefore is also called \emph{time decay}
  or \emph{rent}.
  \item Note: The passage of time is deterministic. It does not make sense to
  hedge against these losses.
\end{itemize}
\end{frame}

\begin{frame}[fragile]
\frametitle{Theta of a Call Option in the Black-Scholes Model}
\begin{figure}[htp]
\begin{center}
  \includegraphics[width=0.8\textwidth]{../pics/theta}
  \caption{Theta for a European call in the BS model, $\sigma=20\%$, $r=1\%$,
  $K=100$.}
  \label{fig:thetaBS}
\end{center}
\end{figure}
\end{frame}

\begin{frame}[fragile]
\frametitle{The Rho}
\begin{itemize}
  \item Rho is the derivative of the instrument price with respect to the risk
  free interest rate.
  \item It measures, how the price of the instrument changes if the interest
  rate changes.
  \item Rho is particular important for fixed-income portfolios. If the hedging
  portfolio of the instrument consists of a large portion of debt and only a
  small amount of initial capital, the influence of changes in the risk free
  rate might become quite big.
\end{itemize}
\end{frame}

\begin{frame}[fragile]
\frametitle{Rho of a Call Option in the Black-Scholes Model}
\begin{figure}[htp]
\begin{center}
  \includegraphics[width=0.8\textwidth]{../pics/rho}
  \caption{Rho for a European call in the BS model, $\sigma=20\%$, $T=1$,
  $K=100$.}
  \label{fig:rhoBS}
\end{center}
\end{figure}
\end{frame}

\subsection{Applications}

\begin{frame}[fragile]
\frametitle{Portfolio Insurance}
\begin{itemize}
  \item Portfolio managers are often interested in acquiring put options on
  their portfolio. This protects them from losses while retaining the upside
  potential. This is especially important if fixed targets have to be met as,
  e.g., for guaranteed pensions.
  \item Such options are usually not traded because each portfolio manager has
  her individual portfolio.
  \item To achieve his goal, the portfolio manager can use options on market
  indices to approximate the option on the portfolio, or he can create the
  option synthetically by replication. The latter technique is called portfolio
  insurance in this context.
\end{itemize}
\end{frame}

\begin{frame}[fragile]
\frametitle{Creating a Put Option}
A put option on a portfolio can be created synthetically by
\begin{itemize}
  \item calculating the delta $\Delta$ of a put on the portfolio in the BS model;
  \item selling a proportion of $\vert \Delta \vert$ of the original portfolio;
  \item repurchasing the original portfolio so that it again matches the
  portfolio delta after an increase of the portfolio value;
  \item selling additional parts of the original portfolio after a decrease in
  portfolio value to match the new delta.
\end{itemize}
\end{frame}

\begin{frame}[fragile]
\frametitle{Portfolio Insurance: Problems}
\begin{itemize}
  \item The portfolio manager buys after prices have increased and sells after
  prices have declined.
  \item Frequent monitoring is necessary to obtain a good replication.
  \item Transaction costs and bis-ask spread generate losses.
\end{itemize}
\end{frame}

\begin{frame}[fragile]
\frametitle{Portfolio Insurance and the Black Monday 1987}
\begin{itemize}
  \item On Friday, Oct. 16, 1987, the DJIA fell by almost $5\%$, having declined
  significantly in the days before.
  \item Consequently, portfolio insurers had to sell large portions of their
  portfolios to maintain their hedge.
  \item They could not complete the necessary transactions until the market
  closed on Friday afternoon.
  \item The additional sales were conducted on Monday, Oct. 19, and contributed
  to the largest one-day percentage decline in the history of the DJIA, nearly
  $23\%$.
  \item Other traders added to this by anticipating the actions of portfolio
  insurers.
\end{itemize}
\end{frame}



\subsection{Manipulating VaR}
\frame{\frametitle{Changing VaR using Options}
\begin{figure}
	\centering
		\includegraphics[width=1\textwidth]{../pics/VaR-options}
\end{figure}
}

\frame{\frametitle{Changing VaR using Options}
\begin{itemize}
\item<1-> The red line depicts the distribution function of the profit/loss of a portfolio of common stock. The $95\%$ VaR
of this portfolio is about $3500$.
\item<2-> Write put options with a strike price so that the real world probability of the options being in the money at maturity is less than $5\%$.
The blue line depicts the distribution function of the profit/loss after buying and selling the options (VaR is now just $2400$).
\item<3-> The VaR is decreased, i.e. we are  able to decrease VaR to an arbitrary level by selling put options of this security.
However, the trading strategy increases the potential for large loss.
\end{itemize}
}
\subsection{Calculating Portfolio VaR}
\frame{\frametitle{Delta-Gamma Approach}
\begin{itemize}
\item<1->
The portfolio value can be written as a polynomial expansion in the market prices.
\item<2-> The \textcolor{red}{Delta-Gamma} approach considers---additional to
the linear terms---the quadratic terms to calculate the changes in the value of
the portfolio.
\item<3-> We write the change in the portfolio value as:
\begin{align*}
	\Delta V(F_1,F_2,\ldots,F_n)& \approx \sum_{i=1}^n \frac{\partial V}{\partial F_i} \Delta F_i \\
	&+ \frac 1 2 \sum_{i=1}^n \sum_{j=1}^n \frac{\partial^2 V}{\partial F_i \partial F_j} \Delta F_i \Delta F_j.
\end{align*}
\end{itemize}
}

\frame{\frametitle{Delta-Gamma Approach}
\begin{itemize}
\item<1-> This approach is adequate for non-linear positions, such as options, because their Gamma is also considered.
\item<2-> Qualitatively a portfolio with a positive Gamma (a convex payoff structure) reduces its market price sensitivity (i.e. its Delta position) in the case of decreasing market prices.
\item<3-> If market prices are increasing there is a positive effect from an increasing Delta.
\item<4-> A VaR calculated with the Delta-Gamma method is therefore too high. Vice versa, the calculated VaR will be too low if the portfolio has a negative Gamma.
\end{itemize}
}

\section{Monte Carlo Simulation}
\subsection{Basics}

\begin{frame}[fragile]
\frametitle{Simulation}
``\emph{simulation},  in industry, science, and education, a research or
teaching technique that reproduces actual events and processes under test conditions.
Developing a simulation is often a highly complex mathematical process.
Initially a set of rules, relationships, and operating procedures are
specified, along with other variables. The interaction of these phenomena
create new situations, even new rules, which further evolve as the simulation
proceeds. Simulation implements range from paper-and-pencil and board-game
reproductions of situations to complex computer-aided interactive systems.''

from Encyclop{\ae}dia Brittanica
\end{frame}

\begin{frame}[fragile]
\frametitle{Monte Carlo Simulation}
Monte Carlo simulation methods work as follows:
\begin{enumerate}
  \item\label{l:enum1} Randomly draw from a set of possible outcomes (the
  simulation).
  \item Repeat step \ref{l:enum1} multiple times and record the results.
  \item This provides us with a (empirical) probability distribution.
  \item We take this empirical distribution as an approximation of the
  underlying distribution and use it to compute quantities such as
  expectation, variance, and quantiles.
\end{enumerate}
Mathematically, this works because
\begin{itemize}
  \item The law of large numbers ensures that our results get better as we
  	increase the number of random draws (i.e., the estimate converges).
  \item The central limit theorem gives us error estimates.
\end{itemize}
\end{frame}


\begin{frame}[fragile]
\frametitle{Example: Rolling Dice}
We have a random variable $X$ representing a fair 6-sided die. Of course,
\begin{align*}
  \E[X] = \sum_{i=1}^6 \Pr(X=i)i=\frac{1}{6}\left( 1 + 2 + 3 + 4 + 5 + 6
  \right)= 3.5
\end{align*}
How can we get this result with MC-simulation?
\begin{itemize}
  \item Roll the dice many times.
  \item Record the numbers rolled.
  \item Compute the mean of those numbers.
\end{itemize}
\end{frame}

\begin{frame}[fragile]
\frametitle{Example: Rolling Dice II}
\begin{center}
\includemedia[
label=roll_dice, % defines JavaScript object annotRM
width=0.6\textwidth,height=0.6\textwidth,
activate=pageopen,
addresource=video/rollDice.mp4, %two video files
flashvars={
source=video/rollDice.mp4
&loop=false % loop video
&scaleMode=letterbox % preserve aspect ratio while scaling the video
}
]{}{VPlayer.swf}\\
\PushButton[
onclick={
annotRM['roll_dice'].activated=true;
annotRM['roll_dice'].callAS('playPause');
}]{\fbox{Play/Pause}}

\end{center}
\end{frame}

\begin{frame}[fragile]
\frametitle{Monte Carlo Methods in Finance: Motivation}
Why would we use Monte Carlo methods for option pricing if we have the
Black-Scholes formula? There usually are no formulas for the prices of
\begin{itemize}
  \item many exotic options or path dependent options.
  \item vanilla options in complex models that include, e.g., stochastic
  volatility and jumps.
  \item options that depend on multiple correlated assets.
\end{itemize}
Monte Carlo methods work in all of these settings.
\end{frame}

\begin{frame}[fragile]
\frametitle{Option Pricing via Monte Carlo Methods I}
Recall: In the binomial option pricing model, we can find so
called risk neutral probabilities for up- and down-movements of the underlying
such that the  expectation of the discounted option payoffs gives us the price
of the option. This generalizes to other, time-continuous models.

For the time-$0$ price $V_0$ of a European option with payoff $f(S_T)$ at time
$T$, we can write
\begin{align*}
  V_0 = \E^\mathbb{\Q}[D_T f(S_T)]
\end{align*}
where $D_T$ is a discount factor and $E^\mathbb{\Q}$ denotes the expectation
under the risk-neutral measure.
\end{frame}

\begin{frame}[fragile]
\frametitle{Option Pricing via Monte Carlo Methods II}
We can obtain a Monte Carlo estimate of the price as follows.
\begin{itemize}
  \item Develop a model for the price process $S_t$ of the underlying security
  using risk-neutral probabilities.
  \item Generate realizations of $S_T$ (and the discount factor $D_T$ if it is
  stochastic).
  \item Compute the option-payoff for each realization.
  \item Estimate the expectation of the discounted cash flows as above. This is
  the price.
\end{itemize}
The technique can be generalized easily to include path-dependent options.
\end{frame}


\begin{frame}[fragile]
\frametitle{Example: European Call in the Black-Scholes Model}
In the Black-Scholes model, the log-returns follow a normal distribution. Under
the risk-neutral measure, we have
\begin{align*}
  \log\left(\frac{S(t+\delta)}{S(t)} \right) \sim
  N\left(\left( r-\frac{\sigma^2}{2}\right)\delta,\sigma \sqrt{\delta}\right).
\end{align*}
\begin{itemize}
  \item Generate $n$ realizations of $\log(S_{T}/S_0)$ and calculate
  $p_i=(S_{T,i}-K)^+, i=1,\ldots,n$.
  \item Compute the mean payoff $\hat{p}_n=\sum_{i=1}^n p_i/n$.
  \item Then, $\hat{V}_0 = e^{-rT}\hat{p}_n$.
\end{itemize}
\end{frame}

\begin{frame}[fragile]
\frametitle{Example: European Call in the Black-Scholes Model}
\begin{center}
\includemedia[
label=sim_BS, % defines JavaScript object annotRM
width=0.6\textwidth,height=0.6\textwidth,
activate=pageopen,
addresource=video/simBSCall.mp4, %two video files
flashvars={
source=video/simBSCall.mp4
&loop=false % loop video
&scaleMode=letterbox % preserve aspect ratio while scaling the video
}
]{}{VPlayer.swf}\\
\PushButton[
onclick={
annotRM['sim_BS'].activated=true;
annotRM['sim_BS'].callAS('playPause');
}]{Play/Pause}
% \PushButton[
% onclick={
% annotRM['sim_BS'].activated=true;
% annotRM['sim_BS'].callAS('pause');
% }]{Pause}

\end{center}
\end{frame}

\begin{frame}[fragile]
\frametitle{Example: European Call in the Black-Scholes Model}
\begin{figure}[htp]
\begin{center}
  \includegraphics[width=0.9\textwidth]{../pics/BS_Call15K}
  \caption{MC simulation for a European Option in the Black-Scholes model.}
\end{center}
\end{figure}
\end{frame}


\begin{frame}[fragile]
\frametitle{Strengths and Weaknesses of Monte Carlo Methods}
\begin{itemize}
  \item[+] Very flexible method for pricing, works for path dependent options
  and with sophisticated models;
  \item[+] Does not suffer from ``curse of dimensions'';
  \item[+] Provides error estimates;
  \item[o\hspace{1px}] Implementation
  \item[\textendash\hspace{1px}] Computationally expensive, slow convergence;
  \item[\textendash\hspace{1px}] No ``exact'' prices;
  \item[\textendash\hspace{1px}] Numerical problems (discretization,\ldots).
\end{itemize}
\end{frame}







\begin{frame}[fragile]
\frametitle{Example: A Portfolio Based on Two Risk Factors}
From the property of normal log-returns, we can construct a discretized process
of the stock price. If we want to evaluate the stock price at times $0 \leq t_1
\leq \ldots \leq t_n=T$ with $t_{i+1}-t_i=\delta$ for $i=1,\ldots,n-1$:
\begin{align*}
  S_{t+1} &= S_t \exp\left( \mu \delta + \sigma \sqrt{\delta} \epsilon_t
  \right)\\
  &\approx S_{t} + S_{t} \left( \mu \delta + \sigma
  \sqrt{\delta} \epsilon_{t} \right)
\end{align*}
where $\epsilon_t\sim N(0,1)$ and the $\epsilon_t$ are independent over time.
\end{frame}

\begin{frame}[fragile]
\frametitle{Example: A Portfolio Based on Two Risk Factors II}
If we want to model at the same time, we can generalize the model simply by
setting $S_{t}=(S_{t,1},S_{t,2})$ and
\begin{align*}
  S_{t+1,j} = S_{t,j} + S_{t,j} \left( \mu_j \delta + \sigma_j
  \sqrt{\delta} \epsilon_{t,j} \right), j=1,2.
\end{align*}
Stock returns are rarely independent. How can we select $\epsilon_{t,j}$ such
that the correlation is $\rho$, that is
\begin{align*}
  C= \Cor(S_{t}) = \left[ \begin{array}{cc}1 & \rho \\ \rho &1
  \end{array}\right] ?
\end{align*}
\end{frame}

\begin{frame}[fragile]
\frametitle{Example: A Portfolio Based on Two Risk Factors III}
\begin{itemize}
  \item Computer programs can generate realizations of random vectors
	$Z_t=(Z_{t,1},Z_{t,2})$ with independent standard normal components and
	independent over time.
  \item We want to use $Z_t$ to obtain correlated random numbers by using a
	transformation: find a matrix $A$ such that $\epsilon_t=AZ^T_t$ has two
	standard normal components with correlation $\rho$.
  \item We can do this with the Cholesky decomposition.
\end{itemize}




\end{frame}

\begin{frame}[fragile]
\frametitle{Example: A Portfolio Based on Two Risk Factors IV}
If $C$ is a positive semi-definite correlation matrix, we can write
		\begin{align*}
		  C = AA^T
		\end{align*}
where $A$ is a lower triangular matrix. In two dimensions we have:
\begin{align*}
  C = \left[\begin{array}{cc} 1& \rho \\
  	\rho &1 \end{array}\right] &= \left[\begin{array}{cc} a_{11}& 0 \\
  	a_{12} &a_{22} \end{array}\right] \cdot \left[\begin{array}{cc} a_{11}&
  	a_{12} \\ 0 &a_{22} \end{array}\right]
   &= \left[\begin{array}{cc} a^2_{11}&
   a_{11}a_{12} \\ a_{11}a_{12} &a^2_{12}+a^2_{22} \end{array}\right].
\end{align*}
We find that
\begin{align*}
  a_{11} &= a^2_{11} = 1 \\
  a_{11} a_{12} &= \rho \\
  a^2_{12}+a^2_{22} &= 1
\end{align*}
so that $a_{12}=\rho$ and $a_{22}=\sqrt{1-\rho^2}$.
\end{frame}

\begin{frame}[fragile]
\frametitle{Example: A Portfolio Based on Two Risk Factors V}
We set
\begin{align*}
  \left[\begin{array}{c} \epsilon_{t,1}\\
  	\epsilon_{t,2} \end{array}\right] = A Z_t^T =
  	\left[\begin{array}{cc} 1& 0 \\
  	\rho &\sqrt{1-\rho^2} \end{array}\right]\left[\begin{array}{c} Z_{t,1} \\
  	Z_{t,2} \end{array}\right] = \left[\begin{array}{c} Z_{t,1}\\
  	\rho Z_{t,1} +\sqrt{1-\rho^2}Z_{t,2} \end{array}\right].
\end{align*}
Let us check that we did the right thing:
\begin{itemize}
  \item $\epsilon_{t,1}=Z_{t,1}$ is a standard normal random variable.
  \item The sum of two independent normally distributed random variables has a
  normal distribution, so $\epsilon_{t,2}=\rho Z_{t,1} +\sqrt{1-\rho^2}Z_{t,2}$
  also has a normal distribution.
  \item We have
  \abovedisplayskip=2pt
  	\begin{align*}
  		\E[\rho Z_{t,1} +\sqrt{1-\rho^2}Z_{t,2}] &= \rho\E[Z_{t,1}] +
  		\sqrt{1-\rho^2}\E[Z_{t,2}] = 0\\
  	\intertext{and}
  	\Var(\rho Z_{t,1} +\sqrt{1-\rho^2}Z_{t,2}) &= \rho^2\Var(z_{t,1}) + (\sqrt{1-\rho^2})^2
  \Var(z_{t,2})\\
   &= \rho^2 + 1 -\rho^2 =1
	\end{align*}
\end{itemize}
\end{frame}

\begin{frame}[fragile]
\frametitle{Example: A Portfolio Based on Two Risk Factors VI}
\begin{itemize}
  \item Therefore $\epsilon_{t,2}$ has a normal distribution with mean $zero$ and
	variance $1$, i.e., a standard normal distribution.
  \item It remains to check the correlation:
	\begin{align*}
	  \Cor(\epsilon_{t,1},\epsilon_{t,2}) &= \Cor\left(z_{t,1},\rho z_{t,1} +
	  \sqrt{1-\rho^2}z_{t,2}\right)\\
	   &= \rho \Cor(z_{t,1},z_{t,1}) =\rho.
	\end{align*}
\end{itemize}
This principle generalizes to dimensions $n>2$ although estimation of the
covariance matrix becomes more difficult.
\end{frame}

\begin{frame}[fragile]
\frametitle{Example: A Portfolio Based on Two Risk Factors VII}
\begin{center}
\includemedia[
label=sim_Assets, % defines JavaScript object annotRM
width=0.6\textwidth,height=0.6\textwidth,
activate=pageopen,
addresource=video/sim2assets.mp4, %two video files
flashvars={
source=video/sim2assets.mp4
&loop=false % loop video
&scaleMode=letterbox % preserve aspect ratio while scaling the video
}
]{}{VPlayer.swf}\\
\PushButton[
onclick={
annotRM['sim_Assets'].activated=true;
annotRM['sim_Assets'].callAS('playPause');
}]{\fbox{Play/Pause}}

\end{center}
\end{frame}

\subsection{Monte Carlo Methods in Risk Management}
\begin{frame}[fragile]
\frametitle{Computing VaR and ES based on Simulations}
VaR and ES can be computed based on MC simulations as follows:
\begin{itemize}
  \item Develop a model for the development of the $m$ risk factors
  $S=(S_1,\ldots,S_m)$ that influence the portfolio value using real-world
  probabilities.
  \item Generate paths $(S_t)_{0\leq t\leq T}\in \R^m$ of the risk factors
  from $t=0$ to up to the target horizon $T$.
  \item For each path, price the portfolio at $T$.
  \item Compute the risk measure from the empirical distribution.
\end{itemize}
\end{frame}

\begin{frame}[fragile]
\frametitle{Risk Management vs. Pricing}
\begin{block}{Probability Measures}
To price options and derivatives, you have to model the underlying securities
(and interest rates, exchange rates,\ldots) under a risk-neutral measure!

To compute risk measures, you have to model the risk factors of your portfolio
under the real-world measure!
\end{block}
\end{frame}



\begin{frame}[fragile]
\frametitle{Calculating the Greeks with Monte Carlo Methods}
The conceptually easiest way to calculate the Greeks in a Monte Carlo setting
is the following finite-difference approximation:
\begin{itemize}
  \item Compute a MC-estimate $\hat{V}$.
  \item Change a single model parameter $\theta$ by ``a little bit'' to
  $\theta_\epsilon=\theta+\epsilon$.
  \item Compute the MC-estimate $\hat{V}_\epsilon$ using this new parameter
  $\theta_\epsilon$, keeping all other parameters fixed.
  \item Approximate the derivative with the difference quotient
   	\begin{align*}
  		\frac{\partial V}{\partial \theta} \approx \frac{\hat{V}_\epsilon
  		- \hat{V}}{\epsilon}.
	\end{align*}
\end{itemize}
\end{frame}

\begin{frame}[fragile]
\frametitle{Finite-Difference Monte Carlo Greeks}
The finite difference method introduced in the previous slide is easy to
understand but has some serious drawbacks:
\begin{itemize}
  \item It requires (at least) a full MC simulation and pricing run for each
  Greek.
  \item It does not work well for options with discontinuous
  payoffs, e.g., digital options (=binary options = cash-or-nothing options).
\end{itemize}
Alternative methods such as pathwise derivatives, likelihood ratio method and
Malliavin Greeks are much more complex, but can yield vastly improved results.
\end{frame}

\begin{frame}[fragile]
\frametitle{Example: Finite Difference vs. Mallivin Greeks}
%\usepackage{graphics} is needed for \includegraphics
\begin{figure}[htp]
\begin{center}
  \includegraphics[width=0.8\textwidth]{../pics/Gamma_Digital_BS}
  \caption{Gamma for a Digital Option in the Black-Scholes model.}
\end{center}
\end{figure}
\end{frame}

