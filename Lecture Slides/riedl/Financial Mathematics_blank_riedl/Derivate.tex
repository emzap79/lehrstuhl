% !TEX root = FinancialMathematics_ws1314UDE.tex
\part{Derivate}
\section{Basic Derivatives}

Derivative Background

A derivative security, or contingent claim, is a financial
contract whose value at expiration date $T$ (more briefly, expiry)
is determined exactly \index{contingent claim} by the price (or
prices within a prespecified time-interval) of the underlying
financial assets (or instruments) at time $T$ (within the time
interval $[0,T]$).


Derivative securities can be grouped under three general headings:
{\it Options, Forwards and Futures} and {\it Swaps}. During this
lectures we will encounter all this structures and further variants.



Modelling Assumptions

We impose the following set of assumptions on the financial
markets:
	{\it No market frictions: } No transaction costs, no bid/ask
	spread, no taxes, no margin requirements, no restrictions on short sales.
	
	{\it No default risk:} Implying same interest for borrowing
	and lending  
	
	{\it Competitive markets:}  Market participants act as price takers  
	
	{\it Rational agents} Market participants prefer more to less


Arbitrage
The concept of arbitrage lies at the centre of the relative pricing theory. 
All we need to assume additionally is
that economic agents prefer
more to less, or more precisely, an increase in consumption
without any costs will always be accepted.

The essence of the technical sense of arbitrage is that it should
not be possible to guarantee a profit without exposure to risk.
Were it possible to do so, arbitrageurs (we use the French
spelling, as is customary) would do so, in unlimited quantity,
using the market as a \lq {money-pump}' to extract arbitrarily
large quantities of riskless profit.

{\it We assume that arbitrage opportunities do not exist!} }

\subsection{Options}


Options

An option is a financial instrument giving one the {\it right but
not the obligation} to make a specified transaction at (or by) a
specified date at a specified price. {\it Call} options give one
the right to buy. {\it Put} options give one the right to sell.
{\it European} options give one the right to buy/sell on the
specified date, the expiry date, on which the option expires or
matures.

{\it American} options give one the right to buy/sell at any time
prior to or at expiry.

The simplest call and put options are now so standard they are
called {\it vanilla} options.

Many kinds of options now exist, including so-called {\it exotic}
options.  Types include: {\it Asian} options, which depend on the
{\it average} price over a period, {\it lookback} options, which
depend on the  {\it maximum} or {\it minimum} price over a period
and {\it barrier} options, which depend on some price level being
attained or not. }


Terminology}

The asset to which the option refers is called the {\it underlying
asset} or the {\it underlying}. The price at which the transaction
to buy/sell the underlying, on/by the expiry date (if exercised),
is made, is called the {\it exercise price} or {\it strike price}.
We shall usually use $K$ for the strike price, time $t = 0$ for
the initial time (when the contract between the buyer and the
seller of the option is struck), time $t = T$ for the expiry or
final time.

Consider, say, a European call option, with strike price $K$;
write $S(t)$ for the value (or price) of the underlying at time
$t$.  If $S(t) > K$, the option is {\it in the money}, if $S(t) =
K$, the option is said to be {\it at the money} and if $S(t) < K$,
the option is {\it out of the money}.


Payoff

The payoff from the option is $$ S(T) - K \mbox{ if } S(T)
> K\A \mbox{ and }\A 0 \;\; \mbox{otherwise} $$ (more briefly
written as  $(S(T) - K)^+$).

Taking into account the initial payment of an investor one obtains
the profit diagram below.

Payoff
%
\begin{figure}\label{payoffeurocall}
\unitlength1cm \thicklines
\begin{picture}(10,7)
\put(1,2){\vector(1,0){7}} \put(8,1.5){$S(T)$} \put(4,2){$K$}
\put(2,1){\vector(0,1){5}} \put(1.4,6.5){profit}
\put(4,1.5){\line(1,1){4}} \put(2,1.5){\line(1,0){2}}
\end{picture}
\caption{Profit diagram for a European call}
\end{figure}


Underlying Securities
	We will mainly use Commodities or Commodity Futures;
	Fixed income instruments: T-Bonds, Interest Rates (LIBOR, EURIBOR);
	Other classes are possible: (one or several) stocks; Currencies (FX);
	Also Derivatives may be used as underlying for compound derivatives (call on call).


Arbitrage Relationship- Example

We now use the principle of no-arbitrage to obtain bounds for
option prices. We focus on
European options (puts and calls) with identical underlying (say a
stock $S$), strike $K$ and expiry date $T$. Furthermore we assume
the existence of a risk-free bank account (bond) with constant
interest rate $r$ (continuously compounded) during the time
interval $[0,T]$. We start with a fundamental relationship:


We have the following  put-call parity between the prices of the
underlying asset $S$ and European call and put options on stocks
that pay no dividends:
\begin{equation}\label{Europutcall}
S_t + P_t - C_t = K e^{-r(T-t)}.
\end{equation}



Arbitrage Relationship - Example

Consider a portfolio consisting of one stock, one put
and a short position in one call (the holder of the portfolio has
written the call); write $V(t)$ for the value of this portfolio.
Then
$$
V(t) = S(t) + P(t) - C(t)
$$
for all $t \in [0,T]$. At expiry we have
$$\begin{array}{lll}
V(T)&=&S(T)+(S(T)-K)^--(S(T)-K)^+\\*[12pt]
&=&S(T)+K-S(T)=K.
\end{array}
$$
This portfolio thus guarantees a payoff $K$ at time $T$. Using the
principle of no-arbitrage, the value of the portfolio must at any
time $t$ correspond to the value of a sure payoff $K$ at $T$, that
is $V(t)=K e^{-r(T-t)}$. \hfill \eb



\subsection{Forwards and Futures}

Basic Structure
	A {\it forward contract}
	is an agreement to buy or sell an asset $S$ at a certain future
	date $T$ for a certain price $K$.

	The agent who agrees to
	buy the underlying asset is said to have a {\it long} position,
	the other agent assumes a {\it short} position.
 
	The settlement
	date is called {\it delivery date} and the specified price is
	referred to as {\it delivery price}.


Forwards
	The {\it forward
	price} $F(t,T)$ is the delivery price which would make the
	contract have zero value at time $t$.
	
	At the time the contract is set up, $t=0$,
	the forward price therefore equals the delivery price, hence
	$F(0,T) = K$.

	The forward prices $F(t,T)$ need not (and will not)
	necessarily be equal to the delivery price $K$ during the
	life-time of the contract.

	The payoff from a long position in a forward contract on one unit
	of an asset with price $S(T)$ at the maturity of the contract is
	$$ S(T)-K.$$
 
	Compared with a call option with the same maturity
	and strike price $K$ we see that the investor now faces a downside
	risk, too. He has the obligation to buy the asset for price $K$.


Futures
	Futures can be defined as standardized forward contracts traded at exchanges where a clearing house acts as a central counterparty for all transactions.
	Usually an initial margin is paid as a guarantee.
	Each trading day a settlement price is determined and gains or losses are immediately realized at a margin account.
	Thus credit risk is eliminated, but there is exposure to interest rate risk.

\subsection{Swaps}

Swaps

A {\it swap} is an agreement whereby two parties
undertake to exchange, at known dates in the future, various
financial assets (or cash flows) according to a prearranged
formula that depends on the value of one or more underlying
assets. Examples are currency swaps (exchange currencies) and
interest-rate swaps (exchange of fixed for floating set of
interest payments).


Example: Fixed for Floating Interest Rate Swap

Consider the case of a {\it forward swap settled in arrears} characterized by:
	a fixed time $t$, the  contract time, 
	
	dates $T_0 < T_1, \ldots < T_n$, equally distanced $T_{i+1}-T_i = \delta$,
	
	$R$,  a prespecified fixed rate of interest,
	
	$K$,  a nominal amount.



Example: Fixed for Floating Interest Rate Swap

A swap contract $S$ with $K$ and $R$ fixed for the period $T_0,
\ldots T_n$ is a sequence of payments, where the amount of money
paid out at $T_{i+1}, \; i=0, \ldots, n-1$ is defined by
$$
X_{i+1} = K \delta (L(T_i,T_i)-R).
$$

The floating rate over $[T_i, T_{i+1}]$ observed at $T_i$ is a
simple rate defined as
$$
p(T_i, T_{i+1}) = \frac{1}{1 + \delta L(T_i,T_i)}.
$$



\section{Option Pricing}
\subsection{Binomial Tree Models}
 
A fundamental example
	We consider a one-period model, i.e. we allow trading only at
	$t=0$ and $t=T=1$(say).
		
		Our aim is to value at $t=0$  a European contingent claim on a stock $S$
		with maturity $T$.
		
		The payoff $H$ of the instrument is a function $f$ of the stock price at
		$T$. For a European call option with strike $K$, we would have
		$H=f(S_T)=(S_T-K)^+$.
		
		We assume that stocks do not pay dividends.
		
		Investors can borrow and lend at the risk-free rate $r$.


Replication: The Black-Scholes-Merton Approach
If we can find a \emph{replicating portfolio} of instruments with known prices
that has the same payoff as the contingent claim $H$ in every state of the world, the value of the
two positions has to be equal by the no arbitrage assumption.


An Example I
We calculate the price of a European call option on a stock $S$ with strike
$K=50$ and maturity in $T=1$ in a one-period model. The stock price today is
$50$ and can move up or down by $10\%$.

\center
\includegraphics[height=3cm]{../../../pics/COpt1}


An Example II
In this example, we use $r=1\%$. We try to replicate the option with
investments of $x$ in the stock and $y$ in the risk-free bank account.
\begin{align*}
  5 &= x \cdot 55 + y \cdot 1.01 \\
  0 &= x \cdot 45 + y \cdot 1.01 \\
  \Rightarrow  x &= 0.5, y= -22.277
\end{align*}
The value of our investment today, and, by no arbitrage, the value of the option
today, is
\begin{align*}
  V_0 = 0.5 \cdot 50 -22.277 \cdot 1 = 2.72
\end{align*}


Risk-Neutral Valuation
	In the example above, we did not use any assumption about the risk
  preferences of investors and we did not need the probability with which the
  stock moves up or down.
  
	Idea: Pretend that investors are indifferent about risk. Then, every
  asset has to earn an expected return equal to the return of the risk-free
  investment (if there is no arbitrage). Find the probabilities for up- and
  down-movements of the stock in such a world and use them to find the price of
  the option.


An Example III}
In the setup from our example, the risk-free rate of return is $1\%$. Therefore,
\begin{align*}
  1.01 &\stackrel{!}{=} \text{prob}_{up} \cdot 1.1 + \text{prob}_{down} \cdot 0.9 \\
  \Leftrightarrow 1.01 &\stackrel{!}{=} \text{prob}_{up} \cdot 1.1 + (1-\text{prob}_{up})
  \cdot 0.9 \\
  \Leftrightarrow \text{prob}_{up} &= 55\%
\end{align*}
We get the price of the option
\begin{align*}
  V_0 = (0.55 \cdot 5 + (1-0.55)\cdot 0)/1.01 = 2.72
\end{align*}



Risk-Neutral Valuation for Options
The probabilities found in this way are called ``risk-neutral'' probabilities.
They define a probability measure, the ``risk-neutral'' or ``pricing''
measure.\\
\vspace{0.5cm}

The price $V(t)$ at time $t$ of an option with payoff $P(S_T)$ at time $T$ is

   the expectation
   under the risk-neutral measure $\Q$
   of the discounted
   payoff.

\begin{block}{}
\begin{align*}
  V(t) = \uncover<2->{\E\uncover<3->{^\Q}\left[ \uncover<4->{e^{-r(T-t)}}
  \uncover<5->{P(S_T)}\right]}
\end{align*}
\end{block}


The Cox-Ross-Rubinstein (CRR) Model}
The Cox-Ross-Rubinstein model is a binomial tree model. The figure below shows a
two-period tree, but it can easily be extended to multiple periods.
%\usepackage{graphics} is needed for \includegraphics
\begin{figure}[htp]
\begin{center}
\includegraphics[height=3.5cm]{../../../pics/CRR_tree}
\end{center}
\end{figure}


CRR Model: Choice of Parameters
The parameters $u$ and $d$ that control the movements of the stock in a binomial
model can be chosen to match the volatility $\sigma$ of the stock. In the CRR
model, we have:
		$u=e^{\sigma\sqrt{\Delta t}}$ and $d=e^{-\sigma\sqrt{\Delta t}}$
		
		To match the expected return $\mu$ in the real world, the probability of
		an up movement has to be $p^*=\frac{e^{\mu \Delta t} -d}{u-d}$.
		
		For pricing, we need the risk neutral probabilities that can be derived
		as in the example. For an up movement, we get: $p=\frac{e^{r \Delta t}-d}{u-d}$.


Convergence
In a one-period binomial model, we only consider two points in time, $t=0$ and
$t=T$. This approximation is rather rough as it allows only two states of the
world in $t=T$. We can add realism by dividing the interval $[0,T]$ into more
steps. If we add more and more steps, the results get closer and closer to the
results from the Black-Scholes model that we consider next. Mathematically
speaking, the CRR model converges to the Black-Scholes model as the number of
time steps in the interval increases to infinity.


\subsection{The Black-Scholes-Merton Model}


The Black-Scholes-Merton Model}
	In 1973, Black and Scholes (1973) and Merton (1973)
	developed the Black-Scholes (or Black-Scholes-Merton) model which was a major breakthrough in
	option valuation. It can be derived in different ways:

		 As the limit of the CRR model;
		 Via the Black-Scholes partial differential equation (PDE);
		 Via risk-neutral pricing.

	For the second and third method, we need the additional assumption that stock
	prices follow a geometric Brownian motion with constant drift and volatility.
	This is not needed for the first method because there is an implicit distributional assumption in the CRR model.


The Black-Scholes Formula
The time-$t$ price $c(t)$ of a European call option with strike $K$ and
maturity $T$ on a non-dividend-paying stock $S$ with volatility $\sigma$ in the
Black-Scholes model with risk-free interest rate $r$ is
\begin{align*}
  c(t) &= S_t \Phi(d_1) - e^{-r(T-t)}K\Phi(d_2)\\
  d_1 &= \frac{\ln \left(\frac{S_t}{K} \right) +
  \left(r+\frac{\sigma^2}{2}\right)(T-t)}{\sigma\sqrt{T-t}}\\
  d_2 &= d_1 - \sigma \sqrt{T-t}
\end{align*}
where $\Phi(.)$ denotes the cumulative distribution function (cdf) of the standard
normal distribution.


Example: Option Prices in the Black Scholes Model}
Find the price of a European call option in the BS model with strike $K=100$,
time to maturity $1$ year, on a stock with time-$0$ price $100$. The risk free
rate is $5\%$ p.a., the volatility of the stock is $20\%$ p.a.
\begin{align*}
  d_1&=\frac{\ln \left(\frac{100}{100} \right) +
  \left(0.05+\frac{0.2^2}{2}\right)(1-0)}{0.2 \sqrt{1-0}}\\&=0.35\\
  d_2&=0.35-0.2*\sqrt{1-0} \\&= 0.15\\
  c(0)&=100 N(0.35) - e^{-0.05\cdot 1}100N(0.15)\\&=10.45.
\end{align*}