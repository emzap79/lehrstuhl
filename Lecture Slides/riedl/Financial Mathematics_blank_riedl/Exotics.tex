% !TEX root = FinancialMathematics_ws1314UDE.tex
\part{Exotic Options}
\section{Univariate Exotic Options}
 
Change of num{\'e}raire technique
	Consider the option to exchange $K$ assets $Z_2$ against asset
	$Z_1$ at time $T$, which  gives one the right to the cash-flow
	$(Z_1(T) - KZ_2(T))^+$ (with $K=1$). Using $Z_2$ as a
	num\'{e}raire, its price $\pi_{ex}(0)$ is such that
		$$
		\frac{\pi_{ex}(0)}{Z_2(0)} =
		 \EX_{\Q_{Z_2}} \left[ \left(\frac{Z_1(T)}{Z_2(T)}-K\right)^+\right]\!,
		$$
	or
		$$
		\pi_{ex}(0) = Z_1(0) \Q_{Z_1}(A) - K Z_2(0) \Q_{Z_2}(A),
		$$
	where $\Q_{Z_i}$ is the equivalent martingale measure with $Z_i$
	as num\'{e}raire and $A= \{\om: Z_1(T,\om) >K Z_2(T,\om) \}$.



\subsection{Barrier Options}
Barrier Options
	One-barrier options specify a stock-price level, $H$ say, such
	that the option pays (`knocks in') or not (`knocks out') according
	to whether or not level $H$ is attained, from below (`up') or
	above (`down').  There are thus four possibilities: `up and in',
	`up and out', `down and in' and `down and out'.  Since barrier
	options are path-dependent (they involve the behaviour of the
	path, rather than just the current price or price at expiry), they
	may be classified as exotic; alternatively, the four basic
	one-barrier types above may be regarded as `vanilla barrier'
	options, with their more complicated variants as `exotic barrier'
	options.


Down-and-out Call
	Consider a down-and-out call option with strike $K$ and barrier $H$. The payoff is
		$$
		\begin{array}{ll}
		&\DSE (S(T) - K)^+ \IF_{\{\min S(.) \geq H\}}\\*[18pt] =&\DSE
		(S(T) - K) \IF_{\{S(T) \geq K, \min S(.) \geq H\}},
		\end{array}
		$$
	so by risk-neutral pricing the value of the option ${DOC}_{K,H}$ is
		$$
		\EX^*\left[e^{-rT} (S(T) - K) \IF_{\{S(T) \geq K, \min S(.) \geq
		H\}}\right]\!,
		$$
	where $S$ is geometric Brownian motion.


max and min of BM}
	Write $c := \frac{\left(b - \frac{1}{2}{\sigma}^2\right)}{\sigma}$; then
		$$
		\min S(.) \geq H
		$$
	iff
		$$
		\min (ct + W(t)) \geq {\sigma}^{-1} \log (H/p_0).
		$$
	Writing $X$ for $X(t) := ct + W(t)$ -- drifting Brownian motion
	with drift $c$, $m^X$, $M^X$ for its minimum and maximum processes
		$$
		m^X(t) := \min \{X(s): s \in [0,t] \},
		$$
		$$
		M^X(t) := \max \{X(s): s \in [0,t]\},
		$$
	the payoff function involves the bivariate process $(X,m)$, and
	the option price involves the joint law of this process.


Reflection Principle
	Consider $c = 0$. We require the joint law of standard BM and its
	maximum $M$ (or minimum), $(W,M)$.

	We choose a level $b > 0$, and run the process until the {\it
	first-passage time} $ {\tau}(b) := \inf \{t \geq 0: W(t) \geq b
	\}. $ This is a stopping time and by the strong Markov property
	the process begins afresh at level $b$, and by symmetry the
	probabilistic properties of its further evolution are invariant
	under {\it reflection} in the level $b$. This {\it reflection
	principle} leads to L\'evy's joint density formula ($x<y$)
	$$
	\begin{array}{ll}
	&\DSE \prb_0\left( W(t) \in dx, M(t) \in dy\right)\\*[12pt] =&\DSE
	\frac{2(2y - x)}{\sqrt{2 \pi t^3}} \exp\left\{- \frac{1}{2} (2y -
	x)^2 /t \right\}.
	\end{array}
	$$


Density of $(X(t),M^X(t))$
	L\'evy's formula for the joint density of $(W(t), M(t))$ may be
	extended to the case of general drift $c$ by the usual method for
	changing drift, Girsanov's theorem.  The general result is
		$$
		\begin{array}{lll}
		&\DSE \prb_0 \left(X(t) \in dx, M^X(t) \in dy\right)\\*[12pt] =&\DSE
		\frac{2(2y - x)}{\sqrt{2 \pi t^3}} \exp\left\{- \frac{(2y -
		x)^2}{2t} + cx - \frac{1}{2}c^2 t \right\}.
		\end{array}
		$$
	Here as before $0 \leq x \leq y$. 


Valuation Formula
	It is convenient to decompose the price $DOC_{K,H}$ of the
	down-and-out call into the (Black-Scholes) price of the
	corresponding vanilla call, $C_K$ say, and the {\it knockout
	discount}, $KOD_{K,H}$ say, by which the knockout barrier at $H$
	lowers the price: \index{knockout discount}
		$$
		DOC_{K,H} = C_K - KOD_{K,H}.
		$$
	The option formula is, writing $\lambda := r - \frac{1}{2}{\sigma}^2$,
		$$
		\begin{array}{lll}
		KOD_{K,H} &=&\DSE p_0 (H/p_0)^{2 + 2 \lambda/{{\sigma}^2}}
		\Phi(c_1(p_0,T)))\\*[12pt] &&\DSE - K e^{-rT} (H/p_0)^{2 \lambda
		/{{\sigma}^2}} \Phi(c_2(p_0,T)),
		\end{array}
		$$
	where $c_1$, $c_2$ are given by
		$$
		c_{1,2}(p,t) = \frac{\log (H^2 /pK) + (r \pm
		\frac{1}{2}{\sigma}^2) t} {\sigma \sqrt{t}}
		$$


\subsection{Binary Options}
Binary Options
	We assume that the underlying model is a  Black-Scholes model. Recall the payoffs for a European
	call resp. put
		$$
		B^{call} = (S(T)-K)\IF_{\{S(T) >K\}} \A \mbox{resp.} \A B^{put} = (K-S(T))\IF_{\{S(T)
		<K\}}.
		$$

		A binary (or digital) option is a contract whose
		payoff depends in a discontinuous way on the
		terminal price of the underlying asset.

		{\it Simple Binary Option.}
		The terminal payment at $t=T$ of the call resp. put are
			$$
			B_d^{call} = \IF_{\{S(T) >K\}} \A \mbox{resp.} \A B_d^{put} = \IF_{\{S(T)
			<K\}}.
			$$
		The valuation formula follows from the risk-neutral valuation principle
		and is
			$$
			\begin{array}{llll}
			\pi_d^{call}(t) &=&\DSE e^{-r(T-t)} \Phi(d_2(S, T)), \A &\mbox{"Digital call"}\\*[12pt]
			\pi_d^{put}(t) &=&\DSE e^{-r(T-t)} \Phi(-d_2(S, T)), \A &\mbox{"Digital put"}\\*[12pt]
			\end{array}
			$$
		with
			$$
			d_2(s,t) =\DSE
			\frac{\log(s/K) + (r -
			\frac{\sigma^2}{2})t}{\sigma \sqrt{t}}.
			$$

		{\it Cash-or-nothing options.}  Here the payoffs at
		expiry of the European call resp. put are given by
			$$
			BC_d^{call} = C \IF_{\{S(T) >K\}} \A \mbox{resp.} \A BC_d^{put} = C \IF_{\{S(T)
			<K\}}.
			$$
		{\it Asset-or-nothing options.}  Here the corresponding payoffs are
			$$
			BA_d^{call} = S(T) \IF_{\{S(T) >K\}} \A \mbox{resp.} \A BA_d^{put} = S(T) \IF_{\{S(T)
			<K\}}.
			$$
			Observe that we have the decomposition
			$$
			BA_d^{call} = B^{call}+K B_d^{call}
			$$


Gap Options
{\it Gap options,} with payoffs
	$$
	B_{Gap}^{call} = (S(T)-C)\IF_{\{S(T) >K\}} \A \mbox{resp.} \A B_{Gap}^{put} = (C-S(T))\IF_{\{S(T)
	<K\}}.
	$$
Here the relevant decomposition for the call is
	$$
	B_{Gap}^{call}= B^{call}-(C-K)B_d^{call}.
	$$

{\it Super-share options,} with payoffs
	$$
	B_{SS}^{call} = \frac{S(T)}{K_1} \IF_{\{K_1 < S(T) <K_2\}}\!.
	$$


Paylater Options
	{\it Paylater options} have final payoff
		$$
		B_{PL}^{call} = (S(T)-(K+D^{call}))\IF_{\{S(T) >K\}} \A \mbox{resp.} \A B_{PL}^{put} = ((K-D^{put})-S(T))\IF_{\{S(T)
		<K\}}
		$$
	where $D^{call}, D^{put}$ have to be determined in such a way that the prices of the paylater
	options equal zero at $t=0$.

	The relevant decompositions are
		$$
		B_{PL}^{call}= B^{call}-D^{call}B_d^{call} \A \mbox{resp.} \A B_{PL}^{put}= B^{put}-D^{put}B_d^{put},
		$$
	so
		$$
		D^{call}= \frac{\pi^{call}(0)}{\pi_d^{call}(0)} \A \mbox{resp.} \A
		D^{put}= \frac{\pi^{put}(0)}{\pi_d^{put}(0)}.
		$$


\subsection{Compound Options}

Compound Options
	{\it Compound Options} give the right to buy (sell) at $t=T$ another option with maturity $T_1\geq T$.
	$$
	\begin{array}{llll}
	B_{com}^{CC} &=&\DSE (\pi^{call}(T)-K)^+ \A &\mbox{"Call on a call"}\\*[12pt]
	B_{com}^{CP} &=&\DSE (\pi^{put}(T)-K)^+ \A &\mbox{"Call on a put"}\\*[12pt]
	B_{com}^{PC} &=&\DSE (K-\pi^{call}(T))^+ \A &\mbox{"Put on a call"}\\*[12pt]
	B_{com}^{PP} &=&\DSE (K-\pi^{put}(T))^+ \A &\mbox{"Put on a put"}.
	\end{array}
	$$

Valuation formula
There exists a uniquely determined $p^*>0$ for $T\leq T_1$ such that for $S(T)=p^*$ we have
	$$
	\pi^{call}(T,p^*)=K
	$$
Define
	$$
	\begin{array}{lll}
	g_1(t) &=&\DSE \frac{\log(S(t)/p^*) + (r +
	\frac{\sigma^2}{2})(T-t)}{\sigma \sqrt{(T-t)}},\\*[12pt] g_2(t) &=&\DSE
	g_1(t) -\sigma \sqrt{T-t}
	\end{array}
	$$
and
	$$
	\begin{array}{lll}
	h_1(t) &=&\DSE \frac{\log(S(t)/K)_1 + (r +
	\frac{\sigma^2}{2})(T_1-t)}{\sigma \sqrt{T_1-t}},\\*[12pt] h_2(t) &=&\DSE
	h_1(t) -\sigma \sqrt{T_1-t}.
	\end{array}
	$$
Then the price of a call on a call is
	$$
	\begin{array}{lll}
	\pi^{CC}_{com}(t) &=&\DSE S(t) N^{(\rho_1)}(g_1(t),h_1(t)) \\*[12pt]
	&&\DSE  - K_1 e^{-r(T_1-t)} N^{(\rho_1)}(g_2(t),h_2(t))\\*[12pt]
	&&\DSE-Ke^{-r(T-t)} \Phi(g_2(t)),
	\end{array}
	$$
for $t\in[0,T]$, where $N^{(\rho)}(x,y)$ is the distribution of a
bivariate standard normal distribution with correlation
coefficient $\rho$ and where $\rho_1=\sqrt{\frac{T-t}{T_1-t}}$. }


\section{Multi-look Exotic Options}
Asian Options
	An Asian option is an option on a time average of the underlying
	asset.

	Asian calls and puts have payoffs
		$$(\bar{S}-K)^+$$
		and
		$$(\bar{S}-K)^-,$$
	where the strike $K$ is a constant and
		$$
		\bar{S}=\frac{1}{n}\sum_{i=1}^n S(t_i)
		$$ 
	is the average price of the underlying asset over the discrete set of monitoring dates

	Thus: Asian option under discrete monitoring.

	There are no exact formulas for the price of such options, because
	the distribution of $\bar{S}$ (average of lognormals) is
	intractable.\\*[12pt]

	Variants are $(\bar{S}-S(t))^+$ and $(\bar{S}-S(t))^-$.

	For Asian options under continuous monitoring
		$$
		\bar{S}=\frac{1}{T-u}\int_u^T S(t)dt
		$$
	over an interval $[u,T]$.

	There are pricing approaches using transform analysis (Laplace transforms!).


Geometric average option
	There the average $\bar{S}$ is replaced by
		$$
		\left(\prod_{i=1}^n S(t_i)\right)^{1/n}
		$$
	i.e. the geometric average of the underlying.

	In the Black Scholes setting we have
	$$
	\prod_{i=1}^nS(t_i)^{1/n}=
	S(0)\exp\left\{\left(r-\frac{1}{2}\sigma^2\right)\cdot\frac{1}{n}\sum_{i=1}^n t_i+
	\frac{\sigma}{n}\sum_{i=1}^n W(t_i)\right\}.
	$$

	We know $\cov(W(t_i)W(t_j))=\min(t_i,t_j)=\sigma_{ij}$.
	Also the linear transformation stability of the normal
	distribution implies that for $X\sim{\cal N}(\mu,\Sigma)$, we have
	$a'X\sim{\cal N}(a'\mu,a'\Sigma a)$.

	So
		$$
		\sum_{i=1}^nW(t_i)\sim{\cal N}\left(0,(1,\ldots,1)(\sigma_{ij})_{n,n}
		\begin{pmatrix} 1 \\ \vdots \\ 1 \end{pmatrix}\right)
		={\cal N}(0,\sum_{i=1}^n(2i-1)t_{n+1-i}).
		$$

	So the geometric average of $S(t_1),\ldots,S(t_n)$ has the
	same distribution as the value of a time $T$
	$GBM(r-\delta,\bar{\sigma}^{2})$ with
		$$
		T=\frac{1}{n}\sum_{i=1}^nt_i,\qquad\bar{\sigma}^2=\frac{\sigma^2}{n^2T}\sum_{i=1}^2(2i-1)t_{n+1+j},
		\qquad\delta=\frac{1}{2}\sigma^2-\frac{1}{2}\bar{\sigma}^2.
		$$
	Thus an option on the geometric average can be priced using the
	standard Black Scholes formula.


Pricing by simulation
	Valuation of a European Call with maturity $T$ strike $K$ on $S$ under constant interest rate $r$.
	By risk-neutral pricing
		$$
		C(0)=\EX(e^{-rT}(S(T)-K)^+).
		$$
	For the Black-Scholes model
		$$
		dS_t=S_t(rdt+\sigma dW_t)
		$$
	so
		$$
		S_T=S_0\cdot \exp\left\{\left( r-\frac{1}{2}\sigma^2\right)T+\sigma W_T\right\}
		$$
	with $W_T=\sqrt{T}\cdot Z\; Z\sim{\cal N}(0,1)$.

	Also
		$$
		\begin{array}{lll}
		C(0) &=&\DSE S_0\Phi\left(\frac{log(S_0/K)+(r+\frac{1}{2}\sigma^2)T}{\sigma\sqrt{T}}\right)\\*[18pt]
		&&\DSE   -e^{-rT}K\cdot\Phi\left(\frac{log(S/K)+(r-\frac{1}{2}\sigma^2)T)}{\sigma\sqrt{T}}\right).
		\end{array}
		$$
	We can now compare the Monte-Carlo approach with the exact valuation formula.

	{\it {\bf Algorithm}:\\*[6pt]
  for $i=1,\ldots n$ generate $Z_i\sim{\cal N}(0,1)$ independent\\
  set $S_i(T)=\exp((r-\frac{1}{2}\sigma^2)T+\sigma\sqrt{T}Z_i)$\\
  set $C_i=e^{-rT}(S_i(T)-K)^+$\\
  set $\hat{C}_n=\frac{1}{n}(C_1+\ldots +C_n)$
	}

	We note that $\hat{C}_n$ is unbiased, i.e. $\EX(\hat{C}_n)=C$
	and strongly consistent: $\hat{C}_n-C\rightarrow 0$ a.s. $(n\rightarrow
	\infty)$. Also we can provide a confidence interval. Let
	$$
	S_c=\sqrt{\frac{1}{n-1}\sum_{i=1}^n(C_i-\hat{C}_n)^2}
	$$
	denote the sample standard deviation and let $z_\delta$ denote the
	$1-\delta$ quantile of the standard normal
	$(\Phi(z_\delta)=1-\delta)$.
	
	Then
		$$
		\left[\hat{C}_n-z_{\delta/2}\frac{S_c}{\sqrt{n}},
			\hat{C}_n+z_{\delta/2}\frac{S_c}{\sqrt{n}}\right]
		$$
	is an asymptotically valid $1-\delta$ confidence interval for C.


	Recall that $\hat{C}_n-\C\approx{\cal N}(0,S_c/\sqrt{n})$, then
	$\Phi(-z_\delta)=1-\Phi(z_\delta)=\delta$
		$$
		\begin{array}{cll}
		\prb(-z_{\delta/2}\leq(C-\hat{C}_n)\frac{\sqrt{n}}{S_c}\leq z_{\delta/2}) & = & 1-\delta\\*[12pt]
		 \Leftrightarrow \prb(\hat{C}_n-z_{\delta/2}\frac{S_c}{\sqrt{n}}\leq C\leq\hat{C}_n+z_{\delta/2}\frac{S_c}{\sqrt{n}})
		 & =& 1-\delta
		\end{array}
		$$
	For $\delta=0.05$ we have $z_{\delta/2}\approx 1.96$ for example.


Pricing by simulation -- Asian Options
	Consider the payoff $(\bar{S}-K)^+$
	with $\bar{S}=\frac{1}{m}\sum_{j=1}^m S(t_j)$ for a set of fixed
	dates $0=t_0<t_1<\ldots<t_m\leq T$ with $T$ the date at which
	the payoff is received.

	Now we need the values of $S(t)$ at a
	number of intermediate points. However the basic problem is the
	same, simulate $S(t_{j+1})$ given $S(t_j)$, so
	$$S(t_{j+1})=S(t_j)\cdot
	\exp((r-\frac{1}{2}\sigma^2)(t_{j+1}-t_j)+\sigma\sqrt{t_{j+1}-t_j}Z_{j+1})$$
	where $Z_1,\ldots ,Z_m$ are independent $N(0,1)$.
	

Pricing by simulation -- Asian Options}
	{\it {\bf Algorithm}\\*[6pt]
	for $i=1,\ldots,n$\\
	for $j=1,\ldots,m$ generate $Z_{ij}\sim {\cal N}(0,1)$ independent\\
	set $S_i(t_j)=S_i(t_{j-1})\exp(\ldots Z_{ij})$\\
	set $\bar{S}_i=\frac{1}{m}(S_i(t_1)+\ldots +S_i(t_m))$\\
	set $C_i=e^{-rT}(\bar{S}_i-K)^+$\\
	set $\hat{C}_n=(C_1+\ldots+C_n)/n$
	}



\section{Exotic Options with more than one underlying}
Indexed Options
	Consider the following financial market model
		$$
		\begin{array}{llll}
		\mbox{bank account}\, & dB(t) &=& r B(t) dt\\*[6pt]
		\mbox{stock } 1\, & dS_1(t) &=& S_1(t) \left( b_1 dt + \s_{11} dW_1(t) +
		\s_{12} dW_2(t) \right); \A S_1(0)=s_1\\*[12pt]
		\mbox{stock } 2\, & dS_2(t) &=& S_2(t) \left( b_2 dt + \s_{21} dW_1(t) +
		\s_{22} dW_2(t) \right)\A S_2(0)=s_2,
		\end{array}
		$$
	with all coefficients constant, and the volatility matrix 
	$\s =
		\left[\begin{array}{ll}
		\s_{11} & \s_{12}\\
		\s_{21} & \s_{22}
		\end{array}
		\right]
	$
	non-singular.
	
	An indexed option with parameters $a_1,a_2$ is given by the final payout
		$$
		B_{ind}=(a_1S_1(T)-a_2S_2(T))^+
		$$

	Valuation can be done by using the change-of-num\'{e}raire formula
	outlined above. With the notation
		$$
		\begin{array}{lll}
		\tilde{\sigma}_1^2 &=&\DSE \s_{11}^2 + \s_{12}^2; \;\;
		\tilde{\sigma}_2^2 =\DSE \s_{21}^2 + \s_{22}^2\\*[12pt]
		\rho &=&\DSE \frac{\s_{11}\s_{21} + \s_{12}\s_{22}}{\tilde{\s}_1\tilde{\s}_2}
		\end{array}
		$$
	we obtain
		$$
		\begin{array}{lll}
		\pi_{ind}(0) &=&\DSE a_1s_1 \Phi\left(
		\frac{\log(s_1a_1/(s_2a_2) +(\tilde{\s}_1^2+\tilde{\s}_2^2-2\rho \tilde{\s}_1\tilde{\s}_2)T/2)}
		{\sqrt{(\tilde{\s}_1^2+\tilde{\s}_2^2-2\rho \tilde{\s}_1\tilde{\s}_2)T}}   \right)\\*[12pt]
		&&\DSE - a_2s_2 \Phi\left(
		\frac{\log(s_1a_1/(s_2a_2)-(\tilde{\s}_1^2+\tilde{\s}_2^2-2\rho \tilde{\s}_1\tilde{\s}_2)T/2)}
		{\sqrt{(\tilde{\s}_1^2+\tilde{\s}_2^2-2\rho \tilde{\s}_1\tilde{\s}_2)T}}   \right)\\*[12pt]
		\end{array}
		$$

Options on the minimum/maximum of two stocks
	The payoffs are given by
		$$
		\begin{array}{llll}
		B_{min}^{Call} &=&\DSE (\min\{S_1(T),S_2(T)\}-K)^+ \A &\mbox{"Call on minimum"}\\*[12pt]
		B_{max}^{Call} &=&\DSE (\max\{S_1(T),S_2(T)\}-K)^+ \A &\mbox{"Call on maximum"}\\*[12pt]
		B_{min}^{Put} &=&\DSE (K-\min\{S_1(T),S_2(T)\})^+ \A &\mbox{"Put on minimum"}\\*[12pt]
		B_{min}^{Call} &=&\DSE (K-\max\{S_1(T),S_2(T)\})^+ \A &\mbox{"Call on maximum"}
		\end{array}
		$$
	
	We use the notation
		$$
		\begin{array}{lll}
		\s^2 &=&\DSE \tilde{\s}_1^2+\tilde{\s}_2^2-2\rho \tilde{\s}_1\tilde{\s}_2\\*[12pt]
		d_1 &=& \DSE \frac{log(s_1/K) +\left(r+\frac{1}{2}\tilde{\s}_1^2\right)T}{\tilde{\s}_1\sqrt{T}}\\*[12pt]
		d_2 &=& \DSE \frac{log(s_2/K) +\left(r+\frac{1}{2}\tilde{\s}_2^2\right)T}{\tilde{\s}_2\sqrt{T}}\\*[12pt]
		d_3 &=& \DSE \frac{log(s_1/s_2) -\left(r+\frac{1}{2}\s_1^2\right)T}{\s_1\sqrt{T}}\\*[12pt]
		d_4 &=& \DSE \frac{log(s_1/s_2) -\left(r+\frac{1}{2}\s_1^2\right)T}{\s_1\sqrt{T}}\\*[12pt]
		\end{array}
		$$

	The prices of the minimum/maximum options are given
		$$
		\begin{array}{lll}
		X_{min}^{Call}(0) &=& \DSE s_1 \Phi^{(\tilde{\rho})}(d_1,d_3)+ s_2 \Phi^{(\tilde{\rho})}(d_2,d_4)\\*[12pt]
		&&\DSE - K e^{-rT} \Phi^{(\tilde{\rho})}(d_1-\tilde{\s}_1\sqrt{T},d_2-\tilde{\s}_2\sqrt{T})\\*[12pt]
		X_{min}^{Put}(0) &=& X_{min}^{Call}(0)+Ke^{-rT}-s_1\Phi(d_3)-s_2\Phi(d_4)\\*[12pt]
		X_{max}^{Call}(0) &=& \DSE X_{(1)}^{Call}(0)+X_{(2)}^{Call}(0)- X_{min}^{Call}(0)\\*[12pt]
		X_{max}^{Put}(0) &=& \DSE X_{(1)}^{Put}(0)+X_{(2)}^{Put}(0)- X_{min}^{Put}(0).
		\end{array}
		$$
