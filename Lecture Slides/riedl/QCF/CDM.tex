% !TEX root = QCF_ss13UDE.tex
\section{Flexible Mechanisms under the Kyoto Protocol}
\subsection{Structure}
\frame{\frametitle{Project-based Mechanisms}
\begin{itemize}
\item<1-> 2005:  Emissions Trading (EU ETS) was launched in the European Union as a measure to meet the Kyoto commitment
\item<2-> Flexible Mechanisms as Clean Development Mechanism (CDM) and Joint Implementation (JI) can be used by countries to meet their Kyoto commitment and also by companies within the framework of EU emissions trading
\end{itemize}
Objectives of CDM/JI
\begin{itemize}
\item<1-> To save the same or a higher amount of CO2 for the same financial effort and by using fewer financial resources (abatement costs are lower)
\item<2-> Sustainability, emissions reduction, technology transfer, contribution towards economic development
 \end{itemize}
}



							% Folie 4
\frame{\frametitle{Cost Advantage}
\begin{figure}[h!]
\centering
\includegraphics[width=0.9\textwidth, height=0.7\textheight]{../../../pics/abatementcosts.pdf}
\caption{Cost advantage due to CDM/JI}
\label{fig:CDM/JI}
\end{figure}
}

							% Folie 5

\frame{\frametitle{Clean Development Mechanism (CDM)}
\begin{itemize}
\item <1-> Emission reduction projects in a developing or an emerging country (e.g. China, India, Brazil, Malaysia)
\item <2-> Accepted technologies (photovoltaic, wind power, biomass, energy efficiency; no nuclear projects!)
\item <3-> Projects generate Certified Emission Reductions (CER) in the amount of tonnes of emissions that have been avoided (compared to the 'business-as-usual'-scenario, that means claim for additionality)
\end{itemize}
\textbf{Secondary CERs:} issued, tradable credits with guaranteed delivery \\
\textbf{Primary CERs:} purchased forward directly from project (or fonds), subject to individual project and delivery risks
}

							% Folie 6


\frame{\frametitle{Joint Implementation (JI)}
\begin{itemize}
\item <1-> Emission reduction projects in an industrialized or a transition country (e.g. Russia, Ukraine), which has committed to a cap, solely in sectors not covered in the ETS
\item <2-> Analogous to CDM, but reduces the emission budget of the country where the project takes place
\item <3-> Projects generate ERU (Emissions Reduction Unit) in the amount of the avoided emissions (compared to the 'business-as-usual'-scenario)
\item <4-> from 2008 onwards accepted in the EU ETS
\end{itemize}
CERs and ERUs can be used by \\
1) Countries, for compliance under the Kyoto Protocol \\
2) Companies within the EU ETS (up to 22\%), for compliance
}


							% Folie 7

\frame{\frametitle{Example of the Project-based Mechanisms}
\begin{figure}[t]
\begin{minipage}[t]{0.475\textwidth}
Construction of a wind farm \\
abroad:
\vspace*{-0.2c
\begin{figure}[h!]
\centering
\includegraphics[width=0.6\textwidth, height=0.3\textheight]{../../../pics/windfarm.jpg}
\end{figure}
\vspace*{-0.4cm}
\begin{itemize}
\item<1-> annual electricity production: 300 GWh
\item<1-> no emissions occur during production
\end{itemize}
\end{minipage}
\begin{minipage}[t]{0.475\textwidth}
Electricity production at a coal \\
power plant:
\vspace*{-0.2cm}
\begin{figure}[h!]
\centering
\includegraphics[width=0.6\textwidth, height=0.3\textheight]{../../../pics/coalpowerplant.jpg}
\end{figure}
\vspace*{-0.4cm}
\begin{itemize}
\item<1-> 300,000 tonnes of CO2 emissions annually
\end{itemize}
\end{minipage}
\end{figure}
}
						% Folie 8

\frame{\frametitle{Example of the Project-based Mechanisms}
\begin{itemize}
\item<1-> By the construction of the wind farm, emissions amounting to 300,000 tonnes per year have been avoided
\item<2-> Therefore, emission certificates amounting to 300,000 tonnes per year will be generated
\end{itemize}
}

						% Folie 9

\frame{\frametitle{Distribution of CDM Projects and generated CERs by host country }
\begin{figure}[t]
\begin{minipage}[t]{0.475\textwidth}
Expected average annual CERs \\
from registered projects \\
(total: 605,014,135):
\vspace*{-0.7cm}
\begin{figure}
\centering
\includegraphics[width=1.25\textwidth, height=0.4\textheight]{../pics/ExpectedAverAnnCERs2.png}
%Expected average annual CERs from registered projects by host party. Total: 605,014,135. Source: http:\cdm.unfccc.int (c) 20.06.2012 15:56
\end{figure}
\vspace*{-0.8cm}
\begin{itemize}
\item<1-> 68\% renewables projects %Source: http://www.cdmpipeline.org/cdm-projects-type.htm
\end{itemize}
\vspace*{-0.9cm}

\end{minipage}
\begin{minipage}[t]{0.475\textwidth}
Registered project activities\\
(total: 4,248):
\vspace*{-0.7cm}
\begin{figure}
\centering
\includegraphics[width=1.2\textwidth, height=0.4\textheight]{../../../pics/RegisteredProjActivity2.png}
% registered project activities by host party. Total: 4,248. Source: http:\cdm.unfccc.int (c) 20.06.2012 15:56
\end{figure}
\vspace*{-0.8cm}
\begin{itemize}
\item<1-> 5600  projects worldwide
\item<2-> 2,7 bn CERs expected until end of 2012
\end{itemize}
\end{minipage}
\end{figure}
}
\subsection{Risks}

\frame{\frametitle{Types of Risk Involved}
\begin{itemize}
\item<1-> Market risk
\item<2-> Volume Risk
\item<3-> Credit Risk
\item<4-> Delivery Risk
\end{itemize}
}

\frame{\frametitle{CDM/JI Risk Management Example}
\begin{itemize}
\item<1-> We combine a CDM project with an ERPA
\item<2-> Recall an Emission Reduction Purchase Argreement (ERPA) is a  transaction that transfers carbon credits between two parties under the Kyoto Protocol. The buyer pays the seller cash in exchange for carbon credits, thereby allowing the purchaser to emit more carbon dioxide into the atmosphere.
\item<3-> Assume an ERPA with $10$ \euro/ t CO2
\end{itemize}
}

\frame{\frametitle{CDM/JI Risk Management Example II}
\begin{itemize}
\item<1-> Sell the expected volume $V_{exp}$ of 50 000 t CO2 forward at $15$ \euro /t CO2
\item<2-> Now the volume delivered is a random variable $V$ and the certificate spot price $S$ is random.
\item<3-> So the portfolio value $P$ at delivery is
\begin{equation}\nonumber
P = 15 \times V_{exp} - 10 \times V + (S-15) \times (V_{exp}-V)^+ + S \times (V-V_{exp})^+
\end{equation}
\item<4-> So we face two risky scenarios
\begin{itemize}
\item Higher volume with low spot
\item lower volume with higher spot
\end{itemize}
\end{itemize}

}

\frame{\frametitle{Allowance Price Versus CERs}
\begin{figure}[h!]
\centering
\includegraphics[width=0.9\textwidth, height=0.7\textheight]{../../../pics/EUAvsSCER.pdf}
\caption{Price Difference EUA vs SCER}
\label{fig:EUAvsSCER}
\end{figure}

}

\frame{\frametitle{Allowance Price Versus CERs}
\begin{itemize}
\item<1-> Reason for price difference: Limited number of SCER for use in EU
\item<2-> Intensive discussion of that in Barrieu and Fehr (2010).
\end{itemize}

}
\frame{\frametitle{Challenges in CDM Projects}
\begin{itemize}
\item<1-> Regulatory  Risk
\begin{itemize}
\item Possible changes of CDM frame after 2012 with an impact on private investments
\item Acceptance of project-types and countries (China, Brazil)
\end{itemize}
\item<2-> CDM Acceptance
\begin{itemize}
\item Long and bureaucratic
\item not transparent, concrete methodic unknown
\end{itemize}
\item<3-> Country Risk
\begin{itemize}
\item local (CDM-) infrastructure: people, infrastructure
\item local energy infrastructure
\item local political risk
\end{itemize}

\end{itemize}
}

\section{Capital Market and Renewable Energy Projects}
\subsection{Carbon Bonds}
\frame{\frametitle{Carbon Revenue Bonds}
\begin{itemize}
\item<1-> To finance high initial cost for Renewable Energy (RE) projects future returns of the project are securitized
\item<2->
\begin{itemize}
\item Sell future electricity from RE project
\item Sell environmental credits from RE project
\end{itemize}
\item<3-> Only revenue from environmental credits is used for bond
\item<4-> Rigorous forecast analysis of revenues, sensitivity tests and risk analysis is required.
\end{itemize}
}


\frame{\frametitle{Structure of Bond}
\begin{itemize}
\item<1-> pass-through: all revenues are directly passed trough to the owner of the bond
\item<2->
\begin{itemize}
\item maturity: T
\item revenues in year 1: $c_i$
\item rate of return: $r$
\item initial price: $x$
\end{itemize}
\item<3-> Fair price
$$
x= \sum_{i=1}^T \frac{c_i}{(1+r)^i}
$$
\end{itemize}
}

\frame{\frametitle{EUA Time Series}
\begin{figure}[h!]
\centering
\includegraphics[width=0.9\textwidth, height=0.7\textheight]{../../../pics/EUA-TimeSeries.pdf}
%\caption{EUA Time Series}
\label{fig:EUA-TS}
\end{figure}
}
\frame{\frametitle{QQ-Plots for EUA fits}
\begin{figure}[h!]
\centering
\includegraphics[width=0.9\textwidth, height=0.7\textheight]{../../../pics/QQ-CarbonFit.pdf}
%\caption{EUA fits}
\label{fig:EUA-fits}
\end{figure}
}
\frame{\frametitle{Carbon Bond Histogram}
\begin{figure}[h!]
\centering
\includegraphics[width=0.9\textwidth, height=0.7\textheight]{../../../pics/carbon-bond-histogram.pdf}
%\caption{Carbon Bond Histogram}
\label{fig:Carbon-Bond-Histogram}
\end{figure}

}

\subsection{Impact of Carbon Markets on Investment Markets}

\frame{\frametitle{Financial Implications of Carbon Policies}
\begin{itemize}
\item<1-> Modern risk management has to include the consequences of climate change.
\item<2-> Regulatory risk (reduction targets for carbon emissions) transforms into financial risk  for several asset classes.
\item<3-> Carbon inefficient firms tend to have a higher credit spread and higher refinancing costs.
\end{itemize}
}

\frame{\frametitle{Factors Affecting Carbon Risk}
\begin{itemize}
\item<1-> Energy intensity and fuel mix -- firms that are dependent on fossil fuels face higher costs.
\item<2-> Direct, indirect and embedded emissions of a firm's product affect market position.
\item<3-> Marginal abatement costs.
\item<4-> Technology trajectory -- progress in adapting low-carbon emission technologies.
\end{itemize}
}

\frame{\frametitle{Motivation for Investors to Invest in Carbon-related Assets}
\begin{itemize}
\item<1-> Financial Motivation
\begin{itemize}
\item Portfolio diversification
\item Potential fundamental price appreciation of carbon
\item Hedging financial risk due to carbon price increases
\end{itemize}
\item<2-> Green Motivation
\begin{itemize}
\item Compliance with UN principles of responsible investment (UN PRI)
\item Public opinion, behaviour as corporate citizen
\item Incentivizing the corporate sector by taking carbon credits from the market
\end{itemize}
\end{itemize}
}

\frame{\frametitle{Risk In Renewable Energy Companies}
\begin{itemize}
\item<1-> Costs: as the costs of producing RE come down while the costs of producing fossil fuels rise, a
substitution will occur.
\item<2-> Capital: Government and private capita
\item<3-> Competition: between governments as they try to build greener economies
\item<4-> China: huges efforts to establish a green economy
\item<5-> Consumers: demand products with less impact on the economy
\item<6-> Climate Change: will be tackled by investment in greener technologies.
\end{itemize}
}

\frame{\frametitle{CAPM Analysis of RE Companies}
\begin{itemize}
\item<1-> Empirical evidence shows that RE companies have a $\beta$ close to two.
\item<2-> Model: $i$ firm, $t$ time
$$
R_{it}= \alpha_i + \beta_{it} R_{mt}+\epsilon_{it}
$$
$R$ returns, $\alpha$ component that is independent of the market.
\item<3-> Higher beta values indicate a higher equity cost of capital.Investors must then be compensated
through higher expected returns in order to take on the higher risk. A higher equity cost of capital can affect borrowing costs and
it can also affect the discount rate used in net present value calculations.
\end{itemize}
}

\frame{\frametitle{CAPM Analysis of RE Companies II}
\begin{itemize}
\item<1-> Which factors affect systematic risk (the $\beta$)?
\item<2-> Empirical analysis shows:
\begin{itemize}
\item Increases in sales growth reduce market risk
\item Increases in oil price returns increases systematic risk
\end{itemize}
\item<3-> In order to foster RE companies governments can implement policies to increase sales of such companies (e.g. PV-industry and feed-in tariffs).
\end{itemize}
}


	





