\pdfminorversion=4
\makeatletter\let\ifGm@compatii\relax\makeatother
\documentclass[%
    %presentation,
    11pt,
   handout, % Erstellen eines Handouts
]{beamer}
\mode<presentation>
{
  \usetheme{myulm}
  \setbeamercovered{transparent}
}
%\usepackage[ngerman]{babel}
%\usepackage[utf8]{inputenc}
\usepackage[ansinew]{inputenc}
\usepackage{amsmath,amssymb,amsfonts}
\usepackage{times}
\usepackage{graphicx}
\usepackage{fancyvrb}
\usepackage{array}
\usepackage{colortbl}
\usepackage[OT1]{fontenc}
\usepackage{amsthm}
\usepackage{pdfpages}
\usepackage{bbm}
\usepackage{sidecap}
\usepackage[]{subfigure}
\usepackage{graphicx}
\usepackage{enumerate}
\usepackage{rotating}
\usepackage{totpages}
\usepackage[left]{eurosym}
\usepackage{hyperref}
\usepackage{booktabs}
%\usepackage[breaklinks]{hyperref}

%\usepackage[ansinew]{inputenc}  % ANSI
\usepackage[T1]{fontenc}        % Zeichen-Codierung
\usepackage{textcomp}



%\usepackage{harvard}
\usepackage{natbib}
%\usepackage[longnamesfirst]{natbib}
%\usepackage{har2nat}


%\usepackage{biblatex}
%\bibliography{QCF-literature}


\definecolor{purelightblue}{rgb}{0.93,0.93,1}
\definecolor{pureblue}{rgb}{0.1882,0.2510,0.5647}
\definecolor{UDE}{rgb}{0.1882,0.2510,0.5647}
\setbeamertemplate{blocks}[shadow=true]%[rounded]
\setbeamercolor*{block title}{fg=purelightblue,bg=pureblue}
\setbeamercolor*{block body}{bg=purelightblue}

\definecolor{pcolor}{rgb}{0.0,0.6,0.0}
\definecolor{qcolor}{rgb}{0.75,0.0,0.0}
\definecolor{obscolor}{rgb}{0.58,0.70,0.84}


\newcounter{realpage}

%%%%%%%%%%%%%%%%%%%%%%%%%%%%%%%%%%%%%%%%%%%%%%%%%%%%%%%%%%%%%%%%%%%%%%%%%%%%%%%%%%%%%%%%%


%Anfang der Kopfzeile
\newcommand{\zwischentitel}{}
\newcommand{\leitthema}{Quantitative Climate Finance}
\newcommand{\ort}{Essen}
%Datum kann weiter unten geändert werden (\presdatum{})
%Ende der Kopfzeile


%Anfang zur Titelseite
\title[QCF]{Quantitative Climate Finance}
\author{Professor Dr. R{\"u}diger Kiesel}
\newcommand{\presdatum}{SS 2013}
\institute{Chair for Energy Trading and Finance}
\subtitle{Lecture - Summer 2013}
%Ende der Titelseite
\AtBeginPart{
\frame[plain]{\titlepage}
\frame{\partpage}
\frame{\tableofcontents[part=\insertpartnumber]}
}



\AtBeginSection[] % Do nothing for \section*
{
\begin{frame}{Agenda}{\quad}
\only{\tableofcontents[currentsection,hideothersubsections]}
\end{frame}
}

\input{../../def}
\begin{document}

\frame[plain]{\titlepage}

	

% !TEX root = EnergyTrading_ss13UDE.tex


\section{Introduction}
\subsection{Energy Markets}

Liberalisation
	The German Electricity market went into Liberalization in April 1998.\\
	The Pre - Liberalisation system was based on calculatory costs: the price was according to the 'cost-plus' rule
		Integrated value-chain: production, grid, distribution
		Electricity production to secure supply within a regional monopole
		Long-term supply contracts
		No liquid market on the whole sale market
		Regulated consumer prices, regulated investments


Liberalisation
	Post - Liberalisation system based on forces of market: higher volatility of prices, flexibility has value.
		Unbundling of value-chain
		Power plants are used optimally -- no obligation to secure supply
		New players and products
		Trading in Long- and Short-positions on a liquid whole sale market
		Investments based on market expectations


Markets

Since the deregulation of electricity markets in the end of the
1990s, power can be traded at exchanges like the Nordpool, http://www.nordpoolspot.com/  or the
European Energy Exchange (EEX), http://www.eex.com/en. All exchanges have established
spot and futures markets.


Spot prices
\begin{center}
\includegraphics[heigth=0.9 \textheight, width=0.9 \textwidth]{../../../pics/phelixBase2002_12.pdf}
\end{center}

\begin{center}
\includegraphics[heigth=0.9 \textheight, width=0.9 \textwidth]{../../../pics/phelixBase2002_08.pdf}
\end{center}

\begin{center}
\includegraphics[heigth=0.9 \textheight, width=0.9 \textwidth]{../../../pics/phelixBase2008_12.pdf}
\end{center}


\subsection{Options, Forwards and Swaps}

Derivative Background

A derivative security, or contingent claim, is a financial
contract whose value at expiration date $T$ (more briefly, expiry)
is determined exactly \index{contingent claim} by the price (or
prices within a prespecified time-interval) of the underlying
financial assets (or instruments) at time $T$ (within the time
interval $[0,T]$).


Derivative securities can be grouped under three general headings:
{\it Options, Forwards and Futures} and {\it Swaps}. During this
lectures we will encounter all this structures and further variants.


Underlying Securities
	We will mainly use Commodities or Commodity Futures;
	Fixed income instruments: T-Bonds, Interest Rates (LIBOR, EURIBOR);
	Other classes are possible: (one or several) Stocks; Currencies (FX);
	Also Derivatives may be used as underlying for compound derivatives (call on call).


Modelling Assumptions (Financial Markets)
	We impose the following set of assumptions on the financial
	markets:
		{\it No market frictions: } No transaction costs, no bid/ask spread, no taxes, no margin requirements, no restrictions on short sales.
		
		{\it No default risk:} Implying same interest for borrowing and lending.  
		{\it Competitive markets:}  Market participants act as price takers.  
		{\it Rational agents:} Market participants prefer more to less.


Arbitrage
	The concept of arbitrage lies at
	the centre of the relative pricing theory. All we need to assume additionally is
	that economic agents prefer
	more to less, or more precisely, an increase in consumption
	without any costs will always be accepted.

	The essence of the technical sense of arbitrage is that it should
	not be possible to guarantee a profit without exposure to risk.
	Were it is possible to do so, arbitrageurs (we use the French
	spelling, as is customary) would do so, in unlimited quantity,
	using the market as a \lq {money-pump}' to extract arbitrarily
	large quantities of riskless profit.

	{\it We assume that arbitrage opportunities do not exist!} }


Options
	An option is a financial instrument giving one the {\it right but
	not the obligation} to make a specified transaction at (or by) a
	specified date at a specified price. {\it Call} options give one
	the right to buy. {\it Put} options give one the right to sell.
	{\it European} options give one the right to buy/sell on the
	specified date, the expiry date, on which the option expires or
	matures.

	{\it American} options give one the right to buy/sell at any time
	prior to or at expiry.


Options

The simplest call and put options are now so standard that they are
called {\it vanilla} options.

Many kinds of options now exist, including so-called {\it exotic}
options.  Types include: {\it Asian} options, which depend on the
{\it average} price over a period, {\it lookback} options, which
depend on the  {\it maximum} or {\it minimum} price over a period
and {\it barrier} options, which depend on some price level being
attained or not. }

 
Terminology

The asset to which the option refers is called the {\it underlying
asset} or the {\it underlying}. The price at which the transaction
to buy/sell the underlying, on/by the expiry date (if exercised),
is made, is called the {\it exercise price} or {\it strike price}.
We shall usually use $K$ for the strike price, time $t = 0$ for
the initial time (when the contract between the buyer and the
seller of the option is struck), time $t = T$ for the expiry or
final time.

Consider, say, a European call option, with strike price $K$;
write $S(t)$ for the value (or price) of the underlying at time
$t$.  If $S(t) > K$, the option is {\it in the money}, if $S(t) =
K$, the option is said to be {\it at the money} and if $S(t) < K$,
the option is {\it out of the money}.




Payoff
	The payoff from the option is $$ S(T) - K \mbox{ if } S(T)
	> K\A \mbox{ and }\A 0 \;\; \mbox{otherwise} $$ (more briefly
	written as  $(S(T) - K)^+$).

	Taking into account the initial payment of an investor one obtains
	the profit diagram below.

	Profit diagram for a European call
	\begin{figure}\label{payoffeurocall}
	\unitlength1cm \thicklines
	\begin{picture}(10,7)
	\put(1,2){\vector(1,0){7}} \put(8,1.5){$S(T)$} \put(4,2){$K$}
	\put(2,1){\vector(0,1){5}} \put(1.4,6.5){profit}
	\put(4,1.5){\line(1,1){4}} \put(2,1.5){\line(1,0){2}}
	\end{picture}
	\caption{Profit diagram for a European call}
	\end{figure}


Example: Options
	A trader purchases a European call option maturing in 6 month for 100 Barrels crude oil
	with strike 82 USD/Barrel. He pays a premium of 2 USD/Barrel.\\
	If the oil price rises to 87 USD/Barrel at maturity of the option, then the trader can 
	exercise the call and buy 100 Barrels of crude oil for 82 USD/Barrel and sell them 
	at 87 USD/Barrel in the market, making a profit of 300 USD.\\
	If, however, the price of crude oil at maturity drops to 81 USD/Barrel below the strike
	price, then the trader would not exercise the option, making a loss (limited to the cost 
	of the call premium) of 200 USD.


Arbitrage Relationship- Example

We now use the principle of no-arbitrage to obtain bounds for
option prices. We focus on
European options (puts and calls) with identical underlying (say a
stock $S$), strike $K$ and expiry date $T$. Furthermore we assume
the existence of a risk-free bank account (bond) with constant
interest rate $r$ (continuously compounded) during the time
interval $[0,T]$. We start with a fundamental relationship:

We have the following  put-call parity between the prices of the
underlying asset $S$ and European call and put options on stocks
that pay no dividends:
\begin{equation}\label{Europutcall}
S_t + P_t - C_t = K e^{-r(T-t)}.
\end{equation}

Consider a portfolio consisting of one stock, one put
and a short position in one call (the holder of the portfolio has
written the call); write $V(t)$ for the value of this portfolio.
Then
	$$
	V(t) = S(t) + P(t) - C(t)
	$$
for all $t \in [0,T]$. At expiry we have
	$$
	\begin{array}{lll}
	V(T)&=&S(T)+(S(T)-K)^--(S(T)-K)^+\\*[12pt]
	&=&S(T)+K-S(T)=K.
	\end{array}
	$$
This portfolio thus guarantees a payoff $K$ at time $T$. Using the
principle of no-arbitrage, the value of the portfolio must at any
time $t$ correspond to the value of a sure payoff $K$ at $T$, that
is $V(t)=K e^{-r(T-t)}$. \hfill \eb


European Call Price
	For a European call $X = (S(T)-K)^+$ and  we can evaluate the
	above expected value

	The Black-Scholes price
	pro\-cess of a European call is given by
	$$
	\begin{array}{lll}
	C(t) &=&\DSE S(t) \Phi(d_1(S(t), T-t))\\*[12pt]
	&&- K e^{-r(T-t)} \Phi(d_2(S(t), T-t)).
	\end{array}
	$$
	The functions $d_1(s,t)$ and $d_2(s,t)$ are given by
	$$
	\begin{array}{lll}
	d_1(s,t) &=&\DSE \frac{\log(s/K) + (r +
	\frac{\sigma^2}{2})t}{\sigma \sqrt{t}},\\*[12pt] d_2(s,t) &=&\DSE
	 \frac{\log(s/K) + (r -
	\frac{\sigma^2}{2})t}{\sigma \sqrt{t}}
	\end{array}
	$$




Forwards and Futures
	A {\it forward contract} is an agreement to buy or sell an asset $S$ at a certain future
	date $T$ for a certain price $K$.

	The agent who agrees to buy the underlying asset is said to have a {\it long} position,
	the other agent assumes a {\it short} position.
 
	The settlement date is called {\it delivery date} and the specified price is
	referred to as {\it delivery price}.


Forwards
	The {\it forward price} $F(t,T)$ is the delivery price which would make the
	contract have zero value at time $t$.
	
	At the time the contract is set up, $t=0$, the forward price therefore equals the
	delivery price, hence $F(0,T) = K$.

	The forward prices $F(t,T)$ need not (and will not)
	necessarily be equal to the delivery price $K$ during the
	life-time of the contract.


Forwards
	The payoff from a long position in a forward contract on one unit
	of an asset with price $S(T)$ at the maturity of the contract is
	$$ S(T)-K.$$
 
	Compared with a call option with the same maturity
	and strike price $K$ we see that the investor now faces a downside
	risk, too. He has the obligation to buy the asset for price $K$.


Futures
	Futures can be defined as standardized forward contracts traded at exchanges where a clearing house acts as a central counterparty for all transactions.
	
	Usually an initial margin is paid as a guarantee.
	
	Each trading day a settlement price is determined and gains or losses are immediately realized at a margin account.
	
	Thus credit risk is eliminated, but there is exposure to interest rate risk.


Black's Formula
	We use the same notation -
	strike $K$, expiry $T$ as in the spot case, and write $\Phi$ for the
	standard normal distribution function.
	{\it
	The arbitrage price $C$ of a European futures call option is
	$$
	C(t)= c(f(t), T-t),
	$$
	where $c(f,t)$ is given by Black's futures options formula:
	$$
	c(f,t) := e^{-rt} (f \Phi(\tilde{d}_1 (f,t)) - K \Phi(\tilde{d}_2 (f,t))),
	$$
	where
	$$
	\tilde{d}_{1,2} (f,t) := \frac{\log (f/K) \pm \frac{1}{2} {\sigma}^2 t}{
	\sigma \sqrt{t}}.
	$$
	}


Swaps
	A {\it swap} is an agreement whereby two parties
	undertake to exchange, at known dates in the future, various
	financial assets (or cash flows) according to a prearranged
	formula that depends on the value of one or more underlying
	assets. Examples are currency swaps (exchange currencies) and
	interest-rate swaps (exchange of fixed for floating set of
	interest payments).


Spread Options
	Spread options can be used by owners of corresponding plants to
	manage market risk.

	The pay off of a typical spread is
	$$C_{\mbox{spread}}^{(T)}=\max(S_1(T)-S_2(T)-K,0)$$ with $S_i$ the
	underlyings, $K$ the strike.

	For $K=0$ (exchange option) there is an analytic formula due to
	Margrabe (1978).
	$$\begin{array}{lll}
	 C_{\mbox{spread}}(t) & = & e^{-r(T-t)}(S_1(t)\Phi(d_1)-S_2(t)\Phi(d_2))
	 \\*[12pt]
	 %P_{\mbox{spread}}(t) & = & e^{-r(T-t)}(S_2(t)\Phi(-d_2)-S_1(t)\Phi(-d_1))
	 %\\*[12pt]
	 \mbox{where}\quad d_1 & = & \frac{\log(S_1(t)/S_2(t))+\sigma^{2}(T-t)/2}{\sqrt{\sigma^{2}(T-t)}}\quad ,d_2=d_1-\sqrt{\sigma^{2}(T-t)}
	 \\*[12pt]
	 \mbox{and}\quad \sigma & = & \sqrt{\sigma_1^2-2\rho\sigma_1\sigma_2+\sigma_2^2}
	\end{array}$$
	where $\rho$ is the correlation between the two underlyings.

	For $K\neq 0$ no easy analytic formula is available.


%\setcounter{part}{2}
% !TEX root = QCF_ss14UDE.tex
\section{Emission Trading Schemes}

% frametitle
{Example for Emission Trading}

% begin itemize

	Consider two companies A and B each emitting 100 000 metric tons of $\textnormal{CO}_2$ per year

	Each has been allocated 95 000 metric tons under its national allocation plan

	Credits are trading at 10\euro\ per metric ton

	Company A can cut 10 000 metric tons of emission at 5\euro\ per ton (marginal abatement costs, MAC)

	Company B has MAC of 15\euro\ per ton

	Company A receives 50 000\euro\ for its surplus and covers the costs of its own reduction

	Company B meets the cap at cost 50 000\euro\ instead of 75 000\euro

% end itemize

\subsection[Overview]{Overview of different ETS}

% frametitle
{Overview of different emission trading systems  -- USA}
  % begin itemize

  Regional Greenhouse Gas Initiative (launched in 2009)
        The Regional Greenhouse Gas Initiative (RGGI) was the first mandatory cap-and-trade program in the United States to limit carbon dioxide (CO2) from the power sector. It consists of Connecticut, Delaware, Maine, Maryland, Massachusetts, New Hampshire, New York, Rhode Island, and Vermont.

   California started an ETS in 2012.

   \url{http://www.arb.ca.gov/cc/capandtrade/capandtrade.htm}

  and

  \url{http://www.c2es.org/us-states-regions/key-legislation/california-cap-trade}

% end itemize

% frametitle
{Overview of different emission trading systems}
  % begin itemize

 Hubei Province became the sixth jurisdiction in China to launch a pilot carbon emissions trading program, joining Shenzhen, Shanghai, Beijing, Tianjin, and Guangdong Province. In the coming months, two additional programs will be introduced in Chongqing and Qingdao.

	\textbf{Other GHG trading systems}
        % begin itemize

	Australian ETS was supposed to start in 2014. There was a  tax-based first phase. The new government abolished the tax and shelved the plan of an ETS.
        \url{http://www.climatechange.gov.au/}

	New Zealand ETS (New Zealand, launched in 2008)
        \url{https://www.climatechange.govt.nz/emissions-trading-scheme}
        % end itemize
   % end itemize

%\subsection{The EU ETS}
%8. Folie

% frametitle
{Characteristics of EU ETS ($\textnormal{CO}_2$)}
  % begin itemize

	EU ETS is split up into three phases
  % begin itemize

	\textbf{Phase I (2005-07)}

	\textbf{Phase II (2008-12)} coinciding with commitment period of Kyoto protocol

	\textbf{Phase III (2013-20)} inducing significant changes compared to the two previous periods, according to Directive 2009/29/EC
  % end itemize

	Scheme covers approximately 12,000 large emitters in the EU that are responsible for 50\% of total $\textnormal{CO}_2$ emissions. Regulated sectors include energy industry, combustion, cement, etc.

	Emission allowances are traded mostly OTC (approx 60\%), bilateral (approx 10\%) and on eight different exchanges (approx 30\%):
  ECX in London, Nord Pool in Oslo, Powernext in Paris, EEX in Leipzig, The Green Exchange (NYMEX), Sende $\textnormal{CO}_2$, EXAA, New Values Climex.
  % end itemize
%Sources: DB Research (2011), ETS (Directive 2009.29.EC), http://eur-lex.europa.eu/Notice.do?mode=dbl&lang=en&ihmlang=en&lng1=en,de&lng2=bg,cs,da,de,el,en,es,et,fi,fr,hu,it,lt,nl,pl,pt,ro,sk,sl,&val=463502:cs&page=

% frametitle
{Characteristics of EU ETS ($\textnormal{CO}_2$) - Phases I and II}
  % begin itemize

	Process steps concerning the distribution of the allowances according to Phases I and II:
  % begin itemize

	Each country submits a NAP (National Allocation Plan) to the European Commission (EC)

	EC adjusts NAPs if necessary and countries distribute EUAs among regulated firms according to the final NAP as approved by the EC
  % end itemize
   %are distributed in each phase 200x EU defines overall target and targets for each country
  %200x Each country allocates EUAs to different sectors and companies via NAP (National Allocation Plan)
  % March Allocation of allowances for the current year (in advance)
  %January Compliance time, i.e. pollutors have to deliver allowances for the last year to the regulator

	At the end of the current phase regulated firms have to pay a fine of 100 Euro for each emitted ton of $\textnormal{CO}_2$ that is not covered by an allowance (excess emissions penalty).
  % end itemize

% frametitle
{Characteristics of EU ETS ($\textnormal{CO}_2$) - Phase III}
  % begin itemize

	Process steps concerning the distribution of the allowances according to Phase III:
  % begin itemize

	NAPs were  abolished

	Since 2013  \textbf{Auctioning} has been  introduced as default method of initial allowance allocation

	The initial auctioning share  in the power sector was 100\%

	For all other sectors the initial auction share was  20\% and is to be increased to 70\% by 2020 (and to 100\% respectively by 2027)

	Non-auctioned allowances will be distributed on the basis of benchmarks
  % end itemize
   %are distributed in each phase 200x EU defines overall target and targets for each country
  %200x Each country allocates EUAs to different sectors and companies via NAP (National Allocation Plan)
  % March Allocation of allowances for the current year (in advance)
  %January Compliance time, i.e. pollutors have to deliver allowances for the last year to the regulator

	The amount of the excess emissions penalty in the third phase is indexed to the annual inflation rate of the Eurozone.  %Source: DB Research 2011, CMA 2011
  % end itemize

% frametitle
{Structural Reform of the EU ETS}
 % begin itemize

	At the start of phase III in 2013 the surplus of allowances stood at almost two billion, double its level in early 2012.

	As a short-term measure, the Commission has posted the auctioning of 900 million allowances until 2019-2020 to allow demand to pick up.

   As back-loading is only a temporary measure, the Commission proposes to establish a market stability reserve at the beginning of the next trading period in 2021.
 \url{http://ec.europa.eu/clima/policies/ets/reform/index_en.htm}
   % end itemize

\subsection{ETS and Flexibility}

% frametitle
{{\it Where} flexibility}

% begin itemize

	The ambition should be a global market. However, there are various constraints such as policy differences, differences in the traded good, etc.

	Linking different national and regional trading systems can approximate a global market.

	Linking markets increases liquidity and thus reduces the cost of trading.

	However, different designs of schemes have to be taken into account.

% end itemize

% frametitle
{{\it Where} flexibility: Gains from Trade}
\begin{center}
\begin{figure}[h!]
\centering
\includegraphics[width=0.9\textwidth, height=0.8\textheight]{../../../pics/carbon-gainsfromtrade.pdf}
%\caption{}
\end{figure}
\end{center}
 %Source: VividEconomics (2008) S.15

% frametitle
{{\it When} flexibility  -- Banking}

% begin itemize

	effectively increases the depth and liquidity of the market, reducing price volatility by
making current prices a function of a longer time span of activity, rather than being
entirely determined by events today;

	creates an incentive for firms to take early action;

	firms with banked allowances have a
vested interest in higher prices and the continuation (and success) of the system, to maximise the value of
their allowance assets;

	banking can also prevent a price collapse between commitment
periods;

% end itemize

% frametitle
{{\it When} flexibility -- Borrowing}

% begin itemize

	the regulator may not be well-equipped to assess the credit worthiness and
solvency of firms who borrow allowances, who thereby become debtors;

	borrowing enables firms to delay action if they assume that targets will prove too
onerous and will subsequently be softened;

	firms with borrowed allowances have an active interest to lobby for weaker targets,
or even for scrapping emissions trading altogether, so that their debts are cancelled;

	the political desire to (be seen to) act early, and potential benefits of early action,
also imply that politicians may prefer to place constraints on borrowing;

% end itemize

% frametitle
{{\it When} flexibility}

% begin itemize

	banking is usually allowed between periods (Exemption EU ETS Phase I);

	there is typically no borrowing (or only very limited);

	when there are limits on borrowing between periods, the length of the commitment period is relevant to 'when' flexibility
and to market efficiency

% begin itemize

	investments to reduce
emissions may require many years for investors to recover their costs

	in case of short periods, investors have to guess the emissions caps set by
future governments, and attempt to anticipate changes in the underlying structure of the
carbon trading framework

% end itemize

% end itemize

\subsection{ETS and Tax}

% frametitle
{ETS vs Tax: Generalities}

% begin itemize

	Since compliance costs are uncertain the choice of instrument depends on the relative curvatures
of the marginal benefit curve and the marginal abatement costs curve.

	In case of CO2, where damage does not depend on the flow of emissions but on their accumulation in
the atmosphere, scientific results suggest that a carbon tax is more economically efficient under
uncertainty than emissions trading.

	In practice, however, the analysis of efficiency
under uncertainty has had little influence on the choice of policy instruments. The preference
for carbon trading over carbon taxes is driven largely by powerful political economy concerns.
Trading systems are easier to implement politically.
  % end itemize

% frametitle
{ETS vs Tax}

% begin itemize

	The market for emission
reductions has a demand schedule, which is determined by the marginal abatement costs of
regulated agents, and a supply schedule, which is determined by policy.

	Under a pure tax
system, the supply of allowances is infinitely elastic. The market is effectively supplied with as
many allowances as agents wish to buy at a fixed price (the tax rate).

	Under a pure allowance
system, supply is completely inelastic as the amount of allowances is exogenously fixed.

	Hybrid systems create a supply curve that is neither fully flat (a pure tax) nor fully vertical
(pure cap-and-trade) but (stepwise) upward sloping.

% end itemize

% frametitle
{Price ceilings and price floors}

% begin itemize

	a price ceiling and floor provide significantly
greater clarity to investors to deliver dynamic efficiency (in the form of optimal investment
over longer time frames).

	the price floor would guarantee a certain minimum return on
investment in low-carbon technologies, reducing the risk faced by innovating firms.

	the price ceiling may enhance policy credibility. Because it caps the costs of
compliance, a ceiling reduces the risk of a policy reversal if abatement costs turn out to be
injuriously high.

% end itemize

% frametitle
{Price ceilings and price floors}

% begin itemize

	a price ceiling can be established through an unlimited commitment from the regulator to sell allowances onto the market at the price ceiling

	drawback: compliance with the emissions cap is sacrificed

	a price floor can be established
through an unlimited commitment from the regulator to buy back
allowances from the market at the price floor

	drawback: the floor would be achieved at the risk of
imposing a liability on the public balance sheet.

% end itemize

% frametitle
{Hybrid System with Safety Valve}

\begin{center}
\begin{figure}[h!]
\centering
\includegraphics[width=0.9\textwidth, height=0.8\textheight]{../../../pics/hybridscheme.pdf}
%\caption{}
\end{figure}
\end{center}
%Source: new and Old policies S.12

% frametitle
{Ecological Effectiveness}

% begin itemize

	Emission trading systems sets a cap, which in theory establishes precisely the level of emissions which is desired (100\% effectiveness)

	Tax system cannot guarantee an exact amount of emissions as an  outcome. The tax level is set under uncertainty about the marginal abatement costs.

	Hybrid systems reach the target value as long as the system is in its trade area. As soon as the trigger level is reached the ecological target becomes diluted due to the additional certificates.

% end itemize

% frametitle
{Political Feasibility}
Political feasibility is the level of acceptance a policy has in the public. A major factor is the number of people affected.

% begin itemize

	Emission trading systems have a good political enforceability since costumers feel no direct effect by government action.

	Price rises are blamed on companies, especially in case allocation is free. In case of auctioning extra government revenues can be used for redistribution.

	Tax systems are generally met with scepticism (especially in the US). Additional costs on emissions will effect consumers more directly via cost increases.

	Hybrid systems also generate an extra revenue after the trigger is met, which may be viewed positively.

% end itemize

\subsection{Multiple Policy Instruments}

% frametitle
{Multiple Instruments I}

% begin itemize

	Emission regulation is directed at internalizing externalities and economic theory indicates that only one instrument is needed to
internalize one externality.

	Policy often involves multiple instruments such as command-and-control regulation, subsidies, taxes, trading schemes, etc.

	This process reflects an ad-hoc policy-accretion process driven by the multiplicity of national institutions or ...

	the temptation of politician to fix everything.

% end itemize

% frametitle
{Multiple Policy Instruments II}

% begin itemize

	Combinations of permit trading schemes, carbon taxes, technology-specific subsidies and regulatory standards

	Taxes introduced by Sweden, Norway, Denmark, Ireland

	Subsidies for renewable energy by Germany, Spain

	Academic literature gives justification for multiple instruments used in a complimentary way, i.e. hybrid systems. Justification in terms of presence of multiple market failures, asymmetric information, principle-agent relation
 % end itemize

\subsubsection{Symmetric Policy Combinations}

% frametitle
{Tax and Trade}

% begin itemize

	 Carbon tax $t$ (Euro per tonne)

	 Cap-and-trade scheme with price $p$

	 Firm must pay tax and procure certificates for emissions
\begin{tabular}{ll}
$e_0$ & baseline emission \\
$e$ & emissions after abatement \\
$a$ & $=e_0 - e$; \\
$c(a)$ & abatement costs, $c' > 0$, $c'' > 0$ \\
\end{tabular} \\

	 Optimization problem
\begin{align}
\min_{e} & \left\{c(e_0-e)+te+pe\right\} = \min_{e}\left\{f(e)\right\}
\end{align}

	 First-order-condition (*)
\[
c'(e_0-e^*)=t+p
\]

% end itemize

							% Folie 5

% frametitle
{Tax and Trade: optimization problem}

% begin itemize

	 Since $c'>0$ we can invert $e^*=e_0-c'^{-1}(t+p)=e^*(t,p)$ \\

	 Since $f''(e)=c''(e_0-e)>0$ it is a minimum \\

	 Differentiation of (*) w.r.t. the variable t:
\[
\frac{\partial}{\partial{t}}c'(e_0-e^*(t,p))=\frac{\partial}{\partial{t}}(t+p)
\]
\[
-c''(e-e^*(t,p)) \frac{\partial}{\partial{t}} (e^*(t,p))=1
\]
\[
e^*_t=-\frac{1}{c''} < 0
\]

	 Thus increases in tax reduce emissions; By symmetry $e^*_p = -\frac{1}{c''} < 0$ increases in price reduce emissions and $e^*_t=e^*_p$

% end itemize

							% Folie 6

% frametitle
{Tax and Trade: optimization problem}
Assume a cap $E$, $n$ identical firms, such that the aggregate emissions are $E=ne^*$. \\
For constant tax $t$ formally
\[
dE = ne^*_pdp \textnormal{, so } \frac{dp}{dE} = (ne^*_p)^{-1} < 0
\]
that is an increase in the cap reduces the price. \\
For fixed cap we have
\[
ne^*_tdt+ne^*_pdp=0
\]
\[
\frac{dp}{dt}=-\frac{e^*_t}{e^*_p}=-1.
\]
That is ''a small increase in tax results one-for-one in an equivalent reduction of the permit price". \\

							% Folie 7

% frametitle
{Tax and Trade: optimization problem}
\begin{center}
\begin{figure}[h!]
\centering
\rotatebox{0}{
\scalebox{0.6}{
\includegraphics[width=1.35\textwidth, height=\textheight]{../../../pics/cost-of-switching.pdf}}}
\caption{Relation of tax and permit price.}
\end{figure}
\end{center}
%MAC = cost of switching,  % marginal abatement cost;
%$a = e_0 - e$

							% Folie 8

% frametitle
{Tax and Trade}

% begin itemize

	 MAC to use (for example) wind generation is how much it costs to generate electricity by wind compared with the cheapest alternative. \\

	 So tax increase reduces MAC, because it reduces the opportunity cost of wind generation. Thus opportunity cost of abatement have decreased since the firm will pay a higher penalty not to abate.

% end itemize

% frametitle
{Tax and Trade}

% begin itemize

	Model shows that an increase in tax reduces the permit price

	No additional abatement will be achieved

	The average carbon price will be reduced and the risk that the price system will collapse is increased

% end itemize

% frametitle
{Subsidies and Trade}

% begin itemize

	Model shows that an increase in subsidies reduces the permit price

	A higher level of subsidy for an abatement technology reduce the abatement cost at any given level of production

	For a given emission cap, demand for permits will be lower and so will be the price.
  % end itemize

							% Folie 9

% frametitle
{Subsidies and Trade: optimization problem}
Subsidy $s$ applies equally to all firms and technologies provided accordingly to the level of abatement $a = e_0 - e$ \\

% begin itemize

	 Optimization problem
\begin{align}
\min_{e} & \left\{c(e_0-e)-s(e_0-e)+pe\right\} = \min_{e}\left\{f(e)\right\}
\end{align}

	 First-order-condition
\[
f'(e)=-c'(e_0-e)+ s+p = 0
\]
so
\[
c'(e_0-e)=s+p
\]
and
\[
e^*=e_0-c'^{-1}(s+p)=e^*(s,p)
\]

% end itemize

							% Folie 10

% frametitle
{Subsidies and Trade: optimization problem}
The same calculation as above shows
\[
\frac{dp}{ds} = -\frac{e^*_s}{e^*_p}=-1.
\]
The higher the subsidy, the lower the permit price.

% frametitle
{Trade and Trade}

% begin itemize

	Two separate trading programs apply upstream to firms that produce electricity and downstream to firms that consume it.

	Example: UK with EU ETS for electricity producers and Carbon Reduction Commitment for firms and organizations that are primarily electricity consumers.

% end itemize

% frametitle
{Trade and Trade}
Compliance upstream

% begin itemize

	 higher electricity price

	 equivalent to tax on energy consumption (linked to carbon price)

	 downstream implicit price of carbon

% end itemize

Increase in upstream permit prices

% begin itemize

	 increase tax downstream

	 decrease downstream carbon price (see part 1)

% end itemize

\subsubsection{Asymmetric Policy Combinations}

% frametitle
{Uniliteral Tax and Trade}

% begin itemize

	Firms are identical

	Tax only affects a fraction $f$ of firms

	Uniliteral tax leads to diverging marginal costs and so increased mitigation costs.

% end itemize

% frametitle
{Unilateral tax and trade}

% begin itemize

	 Firms are identical with optimal emissions $e^*(t,p)$; $e^*_p = e^*_t = -\frac{1}{c''}$

	 Tax affects only a fraction of firms $f$ (a fraction of the system-wide emissions) $0 < f < 1$ \\
 So
\begin{align}
E = fne^*(t,p)+(1-f)ne^*(p)
\end{align}
Assuming
\[
e^{*f}_p = e^{*(1-f)}_p
\]
We find
\[
dE = fne^*_tdt+ne^*_pdp
\]

% end itemize

							% Folie 5

% frametitle
{Unilateral tax and trade}
Fix $E$, then the impact of the tax on the permit price is
\[
0 =  fe^*_tdt+e^*_pdp \Rightarrow \frac{dp}{dt} = -f \frac{e^*_t}{e^*_p} = -f
\]
The impact of the tax change on permit prices is diluted and different for the two categories
\[
\frac{de^{*f}}{dt} = e^{*f}_t + e^{*f}_p \frac{dp}{dt} = -\frac{1}{c''} + (-\frac{1}{c''})(-f) = -(1-f)\frac{1}{c''} < 0
\]
\[
\frac{de^{*1-f}}{dt} = e^{*(1-f)}_p\frac{dp}{dt} = \frac{f}{c''} > 0
\]
So the emissions for firms subject to tax fall, but the emissions for firms with no tax increase.

							% Folie 6

% frametitle
{Unilateral tax and trade}
The impact on marginal costs for taxed firms is \\
(use $c'(e_0-e^*) = t+p)$)
\[
\frac{d(c^f)'}{dt} = \frac{dt}{dt} + \frac{dp}{dt} = 1-f > 0
\]
and for untaxed firms
\[
\frac{d(c^{1-f})'}{dt} = \frac{dp}{dt} = -f < 0
\]
Diverging marginal costs mean that the gains from trade are, at least in part, reversed. (Remember firms where identical, now mitigation costs increase.)

							% Folie 7

% frametitle
{Technology Policies and Trade}

% begin itemize

	A technology-specific measure affects only part of the MAC curve and will lead to a compositional reorientation of the curve

	In EU ETS fuel switching from coal to gas is targeted

	Trading price may fall, but mitigation cost will rise in general.

% end itemize

%\setcounter{part}{3}
% !TEX root = QCF_ss13UDE.tex
\section{Equilibrium Models}
\subsection{Deterministic Equilibrium Model}

Rubin 1996: Firm i's optimization problem
	Firm i minimizes its cost by buying/selling an optimal quantity of emission permits and by emitting an optimal quantity of emissions, i.e.
		\begin{align}
		\min_{\theta_i, e_i} & \left\{ \int_0^T e^{-rt} [C_i(e_i(t)) + P(t)\theta_i(t)]dt \right\} \\
		\textnormal{subject to }
		&\dot{B}_i = S_i(t) - e_i(t) + \theta_i(t) \\
		&            B_i(0) = 0 \textnormal{ and } B_i(t) \ge 0 \\
		&            e_i(t) \ge 0
		\end{align}

	{Explanation of variables}
	\begin{tiny}
	\begin{tabular}{cl}
	$e_i(t)$ & quantity of emissions \\
	$\theta_i(t)$ & quantity of emission permits bought or sold \\
	$S_i(t)$ & endowment of emissions \\
	$B_i(t)$ & level of emissions in the bank\\
	$C_i(e_i(t))$ & abatement cost function where $C'_i(e_i) < 0$ and $C''_i(e_i) > 0$ \\
	$r$ & interest rate \\
	\end{tabular}
	\end{tiny}


%16. Folie

Rubin 1996: Market equilibrium
	An intertemporal market equilibrium in emission permits over a T-period horizon consists of \\
		$\qquad P^*(t) \ge 0$ (permit price) \\
		$\qquad \theta^*(t) = (\theta^*_1(t), \ldots, \theta^*_N(t))$ (vector of optimal trading volumes) \\
		$\qquad E^*(t) = (e^*_1(t), \ldots, e^*_N(t))$ (vector of optimal emission levels) \\
	such that for a given $P^*(t)$,
		$\theta^*(t)$ and $E^*(t)$ minimize each firm's costs subject to each firm's constraints as given in (2) - (4) and \\
	the following two conditions hold

	Market clearing condition on permits \\
		$
		\qquad \sum_{i=1}^N \theta_i^*(t) = 0
		$
 Terminal stock condition \\
		$
		\qquad P^*(T)\sum_{i=1}^N B^*_i(T) = 0
		$



%17. Folie

Rubin 1996: Joint optimization problem
	A fictitious central planner minimizes total costs by choosing optimal quantities of emissions, i.e.
		\begin{align}
		\min_{e_1, \ldots, e_N} &\left\{ \int_0^T e^{-rt} \sum_{i=1}^N C_i(e_i(t)) dt \right\} \\
		\textnormal{subject to }
		&\dot{B}(t) = \sum_{i=1}^N \left(S_i(t) - e_i(t)\right) \\
		&            B(0) = 0 \textnormal{ and } B(t) \ge 0 \\
		&            e_i(t) \ge 0 \quad \textnormal{ for all } i = 1, \ldots, N
		\end{align}
	{Explanation of variables}
		\begin{tiny}
		\begin{tabular}{cl}
		$S_i(t)$ & firm i's endowment of emissions \\
		$B(t)$ & sum of emissions banked by the firms at time t\\
		$C_i(e_i(t))$ & firm i's abatement cost when emitting $e_i(t)$ where $C'_i(e_i) < 0$ and $C''_i(e_i) > 0$ \\
		$r$ & interest rate \\
		\end{tabular}
		\end{tiny}


%18. Folie
Rubin 1996: Theorem (Market equilibrium and joint optimization problem)
	There exists an intertemporal market equilibrium in emission permits over a T-period horizon
	The market equilibrium solution is at least as inexpensive as the result of the joint cost optimization


%19. Folie
Rubin 1996: Theorem (Permit price)
	The permit price equals the marginal abatement costs
		\[
		P(t) = - C'_i(e_i)
		\]
	The permit price 
		follows Hotelling's rule if banking/borrowing are allowed
		grows at a rate less than the interest rate r if there are restrictions on borrowing
		\[
		\frac{\dot{P}}{P} =
				\left\{ \begin{array}{ll}
					r  &
						\mbox{if $\Phi_i = 0$ } \\
					r - \frac{e^{rt}\Phi_i}{P} &
						\mbox{if $\Phi_i > 0$}
				\end{array}
				\right.
		\]
		where $\Phi_i$ is the adjoint variable of the borrowing constraint


		Proof: \\
		Follows from evaluating the necessary conditions (hereby use the Hamiltonian)



%20. Folie

Rubin 1996: Necessary (and sufficient) conditions for firm i's minimization problem
\begin{tiny}
	The \textbf{Hamiltonian} is given by
		\[
		H_i = e^{-rt} [C_i(e_i(t)) + P(t)\theta_i(t)] + \lambda_i[S_i - e_i + \theta_i]
		\]
	and the \textbf{generalized Hamiltonian} is given by
		$
		L_i = H_i  - \Phi_i B_i - \tau_i e_i
		$ \\
	Hence
		\begin{align}
		\dot{B}_i &= \frac{\partial L_i}{\partial \lambda_i} = S_i - e_i + \theta_i \\
		\dot{\lambda}_i &= -\frac{\partial L_i}{\partial B_i} = \Phi_i \\
		B_i &\ge 0, \quad \Phi_i \ge 0, \quad \Phi_i B_i = 0 \\
		\frac{\partial L_i}{\partial e_i} &= e^{-rt}C'_i(e_i) - \lambda_i \ge 0 \\
		e_i &\ge 0, \quad \tau_i \ge 0, \quad e_i \frac{\partial H_i}{\partial e_i} = 0 \\
		B_i(T) &\ge 0, \quad \lambda_i(T) \le 0, \quad B_i(T) \lambda_i(T) = 0
		\end{align}
\end{tiny}


\subsection{Stochastic Equilibrium Model}
%\subsubsection{Full Economy Model}

Carmona et al. 2008: Firm i's optimization problem
	For given forward permit price $A$ and prices of the produced goods $S$ the firm i maximizes its expected 
	terminal wealth by  buying/selling an optimal number of permits and producing an optimal quantity of goods, i.e.
	\begin{align}
		\sup_{\theta^i, \xi^i} \mathbb{E} \left[ \underbrace{S^i(\xi^i) - C^i(\xi^i)}_{production} + 
		\underbrace{T^i(\theta^i)}_{trading} - \underbrace{\Pi \left(\varepsilon^i + e^i(\xi^i) - \Delta^i - \theta^i_T \right)^+}_{penalty} \right]
	\end{align}

Variables
\begin{tiny}
	\begin{tabular}{ll}
	$S^i(\xi^i) =    \sum_{t=0}^{T-1} \sum_{j, k} S^k_t \xi^{i,j,k}_t$ & revenues from selling the produced goods \\
	$C^i(\xi^i) =    \sum_{t=0}^{T-1} \sum_{j, k} C^k_t \xi^{i,j,k}_t$ & costs from producing the goods \\
	$T^i(\theta^i) = \sum_{t=0}^{T-1} \theta^i_t (A_{t+1} - A_t) - \theta_T^i A_T$ & profit/loss from trading emission permits \\
	$e^i(\xi^i) =    \sum_{t=0}^{T-1} \sum_{j, k} S^k_t \xi^{i,j,k}_t$ & firm i's emissions in [0,T] from the production\\
	$\Delta^i = \sum_{t=0}^{T-1} \Delta^i_t$ & number of emission permits allocated to firm i in [0,T] \\*[12pt]
	\end{tabular}

	\begin{tabular}{ll}
	$\varepsilon^i$ & quantity of firm i's emissions in [0,T] that cannot be controlled\\
	$\theta^i_t$ & number of forward contracts on emission permits held by firm i at time t\\
	$\Pi$ & penalty per emission unit \\
	$S_t^k$ & price of product k\\
	$C_t^{i,j,k}$ & firm i's marginal production costs of product k using production technology j\\
	$e_t^{i,j,k}$ & emission factor of firm i, production technology j and product k \\
	\end{tabular}
\end{tiny}


%22. Folie
Market equilibrium
	A market equilibrium in emission permits consists of
		$\quad A^*$ (one-dimensional stochastic process for forward price on permits)
		
		$\quad S^*$ (multi-dim. stochastic process for the prices of the products)
		
		$\quad \theta^*$ (multi-dim. stochastic process of optimal trading strategies)
		
		$\quad \xi^*$ (multi-dim. stochastic process of optimal production strategies)

	such that for given $A^*$ and $S^*$,
		$\theta^*$ and $\xi^*$ lead to a situation where all the firms are satisfied by their strategy.

	Formally
		$
		\qquad \mathbb{E}\left[ L^{A^*, S^*, i}\left(\theta^{*i},\xi^{*i}\right) \right] \ge \mathbb{E}\left[ L^{A^*, S^*, i}\left(\theta^{i},\xi^{i}\right) \right] 
		\textnormal{ for all } \left(\theta^i, \xi^i \right)
		$ \\
	and the following two conditions hold

	Market clearing condition on permits
		$$
		\sum_{i} \theta^{*i}_t = 0
		$$
	Supply meets demand for each good
		$$
		\sum_{i, j} \xi^{*i,j,k}_t = D_t^k
		$$

%23. Folie

Global optimization problem
	A fictitious central planner minimizes expected total costs by producing an optimal quantity of goods $\xi^*$, i.e. it faces the optimization problem
		\begin{align}
		\inf_{\xi} \mathbb{E} \left[ \underbrace{C(\xi)}_{production} - \underbrace{\Pi \left(\varepsilon + e(\xi) - \Delta \right)^+}_{penalty} \right]
		\end{align}
	where \\
		\begin{tabular}{ll}
		$C(\xi) =  \sum_{i} C^i(\xi^i)$ & total production costs\\
		$e(\xi) = \sum_{i} e^i(\xi^i)$ & total emissions from production in [0,T] \\
		$\varepsilon = \sum_{i} \varepsilon^i$ & total emissions in [0,T] that are not controllable \\
		$\Delta = \sum_{i} \Delta^i$ & total emission certificates handed out by the regulator \\
		$\Pi$ & penalty per emission unit \\
		\end{tabular}


%24. Folie
Theorem: Market equilibrium and joint optimization problem
	If $(A^*,S^*)$ is a market equilibrium with associated strategies
	$(\theta^*,\xi^*)$ then $\xi^*$ is a solution of the global optimization problem
	
	There exists a solution $\bar{\xi}$ of the global optimization problem
	
	If $\bar{\xi}$ is a a solution of the global optimization problem then $(\bar{A}, \bar{S})$ 
	is a market equilibrium and the equilibrium allowance price process is almost surely unique


%25. Folie
Theorem: Equilibrium prices
	Let $(A^*,S^*)$ be a market equilibrium with associated strategies $(\theta^*,\xi^*)$ then
	%\vspace{-0.5cm}

	Forward prices on permits are almost surely given by 
		\[
		A^*_t = \Pi \cdot \mathbb{E} \left[ \chi_{\left\{ \varepsilon + e(\xi) - \Delta \ge 0 \right\}} | \mathcal{F}_t \right]
		\]

	Spot prices $S^{*k}$ of the goods and the optimal production strategy $\xi^{*i}$ correspond to a merit-order-type
	equilibrium with adjusted costs $C_t^{i,j,k} + e^{i,j,k} A_t^*$, i.e. at time t and for each good k
		all the production means of the economy are ranked by increasing adjusted production costs
	
		demand is met by producing from the cheapest production means
		
		k's equilibrium spot price is the marginal cost of production of the most expensive production means used to meet demand $D_t^k$

		\[
		S_t^{*k} = \max_{i,j} \left\{ \left(C_t^{i,j,k} + e^{i,j,k} A_t^* \right) \chi_{\left\{\xi_t^{i,j,k}>0\right\}}\right\}
		\]


\subsection{Dynamics of CO2 permit prices}
\subsubsection{Reduced Equilibrium Model}

Basic Model
	Risk-neutral companies with total initial endowment $e_0$
	
	Total emissions dynamics are
		\begin{equation}
		dy_t= \mu(t, y_t)dt + \sigma(t, y_t)dW_t
		\end{equation}
	with deterministic drift and volatility.
	
	Central planner who minimizes total expected cost over a trading period $[0,T]$ by deciding at any time instant
	whether to costly abate some of the CO2 emissions or not.
	
	At the end of the period actual accumulated emissions and penalty costs are determined.


Basic Model II
	$x_t$ are the total expected emissions over the trading period
	
	Then
		\begin{equation}
		x_t=-\int_0^tu_s ds + \EX_t\left[\int_0^T y_s ds \right]
		\end{equation}
	
	$u_t$ is the optimal rate of abatement which is  actively chosen by the central planner.
	
	So $x_t$ is a controlled stochastic process.


Total Emissions
	$x_T$ are the realized emissions that relate to a potential penalty function
	
	Without abatement total expected emissions  are
		$$
		x_0=\EX\left[\int_0^T y_s ds \right]
		$$
	
	The dynamics of the total expected emissions are
		\begin{equation}
		dx_t=-u_t dt + G(t) dW_t
		\end{equation}
	
	$G(t)$ is the volatility of the uncontrolled part of $x_t$ and depends both on the drift $\mu(t, y_t)$
	and the volatility $\sigma(t,y_t)$ of the emission rate.


Optimisation problem of the central planner I
	\begin{equation}
	\max_{u_t} \EX_0\left[\int_0^Te^{-rt}C(t,u_t)dt + e^{-rT}P(x_T)\right]
	\end{equation}
	with
	$$
	\begin{array}{lll}
	C(t,u_t) &=& - \frac{1}{2}c u_t^2 \\*[12pt]
	P(x_T) &=& \min[0,p(e_0-x_T)]
	\end{array}
	$$


Optimisation problem of the central planner II
	$C(t,u_t)$ are the abatement costs per unit of time. $c$ constant implies no change in technology occurs.
	The quadratic form implies linearly increasing marginal abatement costs.
	
	$P(x_T)$ is the penalty function, with $p$ the penalty including all costs.
	
	$r$ is the constant interest rate.


Solution of the control problem}
	Let $V(t,x_t)$ be the expected value of the optimal policy given $x_t$. By a standard
	Hamilton-Jacobi-Bellman argument we arrive at
		\begin{equation}
		V_t=-\frac{1}{2}(G(t))^2 V_{xx}-\frac{1}{2c}e^{rt}(V_x)^2
		\end{equation}
	with boundary condition
		$$
		V(T, x_T)=e^{-rT}P(x_T)
		$$
	and optimal control
		$$
		u_t=-\frac{1}{c} e^{rt}V_x
		$$


Permit Prices
	\begin{center}
	\begin{figure}[h!]
	\centering
	\rotatebox{0}{
	\scalebox{0.6}{
	\includegraphics[width=1.45\textwidth, height= 1.2\textheight]{../pics/pic1-SUHW.pdf}}}
	\end{figure}
	\end{center}


Permit Price Dynamics
	Recall that the permit price must equal the marginal abatement costs, so
		\begin{equation}
		S(t,x_t) = c u_t = -e^{rt} V_x(t,x_t)
		\end{equation}
	
	Using It{\^o}'s formula and the HJB-PDE we find that the discounted permit price is a martingale.
 
	Its dynamics are
		\begin{equation}
		dS(t,x_t)= G(t)S_x(t,x_t) dW_t
		\end{equation}


Implied Permit Price Volatility
	\begin{center}
	\begin{figure}[h!]
	\centering
	\rotatebox{0}{
	\scalebox{0.6}{
	\includegraphics[width=1.45\textwidth, height= 1.2\textheight]{../pics/pic2-SUHW.pdf}}}
	\end{figure}
	\end{center}


\subsubsection{Central Planner and Equilibrium}
Individual Company Models
	Each individual company has an endowment $e_{i0}$
	
	Individual emissions dynamics are
	\begin{equation}
	dy_{it}= \mu(t, y_{it})dt + \sigma(t, y_{it})dW_{it}
	\end{equation}
	with deterministic drift and volatility.


Individual Emissions
	$x_{it}$ are the total expected emissions of company $i$ over the trading period
 
	Then
	\begin{equation}
	x_{it}=-\int_0^tu_{is} ds -\int_0^tz_{is}ds + \EX_t\left[\int_0^T y_{is} ds \right]
	\end{equation}
 
	$u_{it}$ is the individual rate of abatement
	
	and $z_{it}$ is the instantaneous amount of permits bought or sold.


Individual Emissions Dynamics
	The dynamics of the total expected emissions are
		\begin{equation}
		dx_{it}=-[u_{it}+z_{it}] dt + G_i(t) dW_{it}
		\end{equation}
	
	$G_i(t)$ is the volatility of the uncontrolled part of $x_{it}$ and depends both on the drift $\mu_i(t, y_{it})$
	and the volatility $\sigma_i(t,y_{it})$ of the emission rate.


Optimisation Problem for the individual Company
		\begin{equation}
		\max_{u_{it}, z_{it}} \EX\left[\int_0^Te^{-rt}C_i(t,u_{it})dt - \int_0^T e^{-rt}S(t)z_{it}dt+ e^{-rT}P_i(x_{iT})\right]
		\end{equation}
	with $S(t)$ the permit price and
		$$
		\begin{array}{lll}
		C_i(t,u_{it}) &=& - \frac{1}{2}c_i u_{it}^2 \\*[12pt]
		P_i(x_{iT}) &=& \min[0,p(e_{i0}-x_{iT})]
		\end{array}
		$$


Solution of the control problem
	Let $V^i(t,x_{it})$ be the expected value of the optimal policy for company $i$. By a standard
	Hamilton-Jacobi-Bellman argument we arrive at
		$$
		\begin{array}{lll}
		0&=\max_{u_{it},z_{it}}&\left[e^{-rt}(C_i(t,u_{it}) - S(t) z_{it})\right.\\*[12pt]
		&&+\left.V^i_t -V_x^i(u_{it}+z_{it}) + \frac{1}{2}(G_i(t))^2 V^i_{xx}\right]
		\end{array}
		$$
	with boundary condition
		$$	
		V^i(T, x_{iT})=e^{-rT}P_i(x_{iT}).
		$$


Equilibrium Solution
	We solve the HJB for $N$ companies and use the market clearing condition
		$$
		\sum_{i=1}^N z_{it}^*=0
		$$
	The first-order conditions give
		$$
		\begin{array}{llll}
		u_{it}^* &=& -\frac{1}{c_i} e^{rt} V^i_x & i=1, \ldots N \\*[12pt]
		S(t) &=& - e^{rt} V^i_{x} & i=1, \ldots N
		\end{array}
		$$
	So again
		$$
		S(t) = c_i u_{it}^*, \; i=1, \ldots N.
		$$


Joint Cost Problem I
	Again we image a central planner who has to solve
		\begin{equation}
		\max_{u_{it}} \EX\left[\int_0^Te^{-rt}\sum_{i=1}^N C_i(t,u_{it})dt + e^{-rT} \sum_{i=1}^N P_i(x_{iT})\right]
		\end{equation}
	with $C_i$ and $P_i$ as before.

	We assume only one source of randomness, i.e. $W_{it}= W_t$, then we have the  joint value function as
		$$
		V(t, x_{1t}, \ldots, x_{Nt}) = \sum_{i=1}^N V_i(t,x_{it}).
		$$

	%Here, $z_{it}$ are irrelevant due to the market clearing condition.


Joint Cost Problem II
	The joint cost problem now is
		$$
		\begin{array}{lll}
		0&=\displaystyle \max_{\{u_{it},i=1, \ldots, N\}}&\displaystyle \left[e^{-rt}\sum_{i=1}^N C_i(t,u_{it})\right.\\*[12pt]
		&&+\displaystyle \left.\sum_{i=1}^N (V^i_t -V_x^iu_{it}) + \frac{1}{2}\sum_{i=1}^N(G_i(t))^2 V^i_{xx}\right]
		\end{array}
		$$
	with boundary condition
		$$
		\sum_{i=1}^N V^i(T, x_{iT})=e^{-rT}\sum_{i=1}^N P_i(x_{iT}).
		$$
	

Joint Problem Solution
	The first-order conditions give
		$$
		u_{it}^{**}= -\frac{1}{c_i} e^{rt} V^i_x, \; i=1, \ldots N
		$$
	Due to linearity we also have
		$$
		u_{it}^{**}= \argmax \left\{ \max_{u_{it}} \left[e^{-rt} C_i(t,u_{it}) + V^i_t -V_x^iu_{it} + \frac{1}{2}(G_i(t))^2 V^i_{xx}\right]
		\right\}
		$$
	Again
		$$
		S(t) = c_i u_{it}^{**}= -e^{rt}V^i_x, \; i=1, \ldots N.
		$$


Equivalence to Equilibrium Solution
	$$
	\begin{array}{lll}
	&u_{it}^{**}&\\*[12pt]
	=&  \argmax \left\{ \displaystyle \max_{u_{it},z_{it}} \right. &\left[e^{-rt} C_i(t,u_{it}) -e^{-rt}S(t)z_{it} +e^{-rt}S(t)z_{it} \right.\\*[12pt]
	 &&\left.\left. + V^i_t -V_x^iu_{it} + \frac{1}{2}(G_i(t))^2 V^i_{xx}\right]
	\right\}\\*[12pt]
	=&  \argmax \left\{ \displaystyle \max_{u_{it},z_{it}} \right. &\left[e^{-rt} (C_i(t,u_{it}) -S(t)z_{it}) \right. \\*[12pt]
	&&\left.\left. + V^i_t +V_x^i(-u_{it}-z_{it}) + \frac{1}{2}(G_i(t))^2 V^i_{xx}\right]
	\right\}\\*[12pt]
	=&u_{it}^*&
	\end{array}
	$$


\subsubsection{Multi-Period Models}
% \subsection{General Equilibrium model}

Basic Model
	Total emissions dynamics are
		\begin{equation}
		dy_t= \mu(y_t)dt + \sigma(y_t)dW_t
		\end{equation}
	with deterministic drift and volatility.
	
	buy or sell $z_t$  permits in the market
	
	abate $u_t$ with cost function $C(u_t)$
	
	pay penalty costs


Basic Model II
	$x_{t,T_k}$ are the total expected emissions in $[0,T_k]$
	
	Then
		\begin{equation}
		x_{t,T_k}=-\int_0^tu_s ds  -\int_0^t z_u du + \EX_t\left[\int_0^{T_k} y_s ds \right]
		\end{equation}


Multi-period Framework
	Consider $n$ consecutive trading periods $[0,T_1], [T_1, T_2], \ldots [T_{n-1}, T_n]$ with inter-period banking (no borrowing).
	
	Initial endowment of $e_{T_{k-1}}$ at the beginning of each period $[T_{k-1}, T_k]$.
	
	Have to pay penalty if emissions $x_{T_k}$ from $0$ to $T_k$ exceed the total allocated permits
		$$
		e_{T_k}= \sum_{T_j < T_k}e_{T_j}.
		$$
	With $R(x_{T_k})=e_{T_k}-x_{T_k}$ penalty cost are
		$$
		P(x_{T_k}) = P \min\{0, R(x_{T_k})\}
		$$


Multi-period Optimisation problem
	$$
	\begin{array}{lll}
	\max_{u_t, z_t} & \EX_0& \left[ \displaystyle\int_0^{T_n}e^{-rt}C(u_t)dt - \int_0^{T_n}e^{-rt}S(t) z_tdt \right. \\*[12pt]
	& & \displaystyle \left. + \sum_{j=1}^n e^{-rT_j}P(x_{T_j}) + R(X_{T_n}) S_{end}\right]
	\end{array}
	$$

Equilibrium Solution
	N companies with
		$u_{it}$ is the individual rate of abatement
	
		and $z_{it}$ is the instantaneous amount of permits bought or sold
		
		solve their individual cost problem
		
		with market clearing condition
			$$
			\sum_{i=1}^N z_{it}^*=0 \;\;  \forall t \in [0, T_n]
			$$


Permit Price Process
	The general structure is still
		$$
		S(t) = \sum_{T_j >t} e^{-r(T_t-t)} P \EX_t\left[\IF_{\{R(x_{T_j})<0\}}\right] + e^{rt} S_{end}
		$$
	The first-order conditions give
		$$
		S(t) = c_i u_{it}^*, \; i=1, \ldots N.
		$$


Solution Strategy

 Start with the last period
		find the characteristic PDE with boundary conditions from optimality principle of stochastic control
		
		solve for strategy value function $V_n$

	step back one period
		find the characteristic PDE
 
		solve for strategy value with boundary condition from next periods value

	derive abatement strategy from HJB


%\setcounter{part}{4}
% !TEX root = QCF_ss14UDE.tex
\section{Reduced Form Models}
\subsection{EU ETS First Phase Price Collapse}

% frametitle
{Permit price in the EU ETS during the first phase}
\begin{center}
\begin{figure}[h!]
\centering
\rotatebox{0}{
\scalebox{0.6}{
\includegraphics[width=1.45\textwidth, height=\textheight]{../../../pics/car-00-1-data.pdf}}}
\caption{EUA-Dec07 futures price (22 April 2005 - 17 December 2007).}
\label{fig:plotCar00-Data}
\end{figure}
\end{center}
\end{frame}

%\subsection{Calculating Permit Prices}

% frametitle
{Cumulative Emissions}
    To calculate permit prices, we specify the process for the cumulative emissions in the framework of Carmona et al. by
    $$
      q_{[0,t]} = \int_0^t Q_s ds
    $$
    where the emission rate $Q_t$ follows a Geometric Brownian motion.

  There is no closed-form density for $q_{[0,t]}$ available.

\end{frame}

%8. Folie

% frametitle
{Approximation Approaches}


% begin itemize




	Linear approximation approach of Chesney and Taschini (2008)
   $$
     q_{[t_1,t_2]} \approx \tilde{q}^{Lin}_{[t_1,t_2]} = Q_{t_2} (t_2 - t_1)
   $$
   %has the shortcoming that the moments of the cumulative emissions do not match the true ones.


	Moment matching of Gr{\"u}ll and Kiesel (2009): Log-normal (moment matching)
$$
q_{[t_1,t_2]} \approx \tilde{q}^{Log}_{[t_1,t_2]} = logN \left(\mu_L(t_1,t_2), \sigma^2_L(t_1,t_2) \right) \label{ECumApprox2}
$$
where the parameters $\mu_L(t_1,t_2)$, $\sigma_L(t_1,t_2)$ are chosen such that the first two moments of $\tilde{q}^{Log}_{[t_1,t_2]}$ and $\tilde{q}^{IG}_{[t_1,t_2]}$, respectively, match those of $q_{[t_1,t_2]}$.


% end itemize


\end{frame}

%9. Folie

% frametitle
{Moment matching requires two steps}
    % begin itemize


	Compute the first two moments $m_k$ of a log-normal random variable and solve for the parameters. \\
     In the log-normal case we have that $m_k = e^{k\mu + k^2 \frac{\sigma^2}{2}}$ and
     $$
     \sigma^2 = \ \ln\left( \frac{m_2}{m^2_1}\right)  \qquad \mu = \ \ln(m_1) - \frac{1}{2}  \sigma^2
     $$


	Compute the first two moments of the integral over a geometric Brownian motion
    $$\begin{array}{lll}
\EW \left[ q_{[t_1,t_2]} \right] &=& \ Q_{t_1} \alpha_{t_2-t_1} \\
\EW \left[ \left( q_{[t_1,t_2]} \right)^2 \right] &=& \ 2 Q_{t_1}^2 \beta_{t_2-t_1}
\end{array}
$$
    and plug those into the above equation.
    % end itemize
\end{frame}

% frametitle
{Auxiliary functions for moments of integral over GBM}
\begin{align}
\alpha_{t_2-t_1}    =& \ \left\{
         \begin{array}{ll}
            \frac{1}{\mu} \left( e^{\mu \left( t_2 - t_1 \right)} - 1 \right)
            & \quad \mbox{if $\mu \ne 0$} \\
            t_2 - t_1
            & \quad \mbox{if $\mu = 0$} \\
         \end{array} \right. \label{MomIntAlpha} \\
%\beta_{t_2-t_1} =& \  \frac{ \mu e^{(2 \mu + \sigma^2) \left( t_2 - t_1 \right)} + \mu + \sigma^2 - \left( 2\mu + \sigma^2\right) e^{\mu \left( t_2 - t_1 \right)}}{\mu(\mu + \sigma^2)(2\mu + \sigma^2)} \label{MomIntBeta}
\beta_{t_2-t_1}    =& \ \left\{
         \begin{array}{ll}
            \frac{ \mu e^{(2 \mu + \sigma^2) \left( t_2 - t_1 \right)} + \mu + \sigma^2 - \left( 2\mu + \sigma^2\right) e^{\mu \left( t_2 - t_1 \right)}}{\mu(\mu + \sigma^2)(2\mu + \sigma^2)}
            & \quad \mbox{if $\mu \ne 0$} \\
            \frac{1}{\sigma^4} \left( e^{\sigma^2\left(t_2-t_1\right)} - 1\right)
            & \quad \mbox{if $\mu = 0$} \\
         \end{array} \right. \label{MomIntBeta}
\end{align}
\end{frame}

%11. Folie
%\subsection[Permit Prices]{Permit prices for different approximation approaches}

% frametitle
{Permit price - linear approximation}
    \begin{block}{}
    The permit price at time t is given by
  $$
  S_t^{Lin} = \ \left\{
         \begin{array}{ll}
            P e^{-r\tau}
            & \quad \mbox{if $q_{[0,t]} \ge N$} \\
            P e^{-r\tau} \cdot \Phi \left(\frac{-\ln\left( \frac{1}{\tau} \left[ \frac{N - q_{[0,t]}}{Q_t} \right] \right) + \left( \mu - \frac{\sigma^2}{2}\right)\tau}{\sigma \sqrt{\tau}} \right) & \quad \mbox{if $q_{[0,t]} < N$} \\
         \end{array} \right.
$$
where \\
$\tau = T - t$ is the time to compliance. \\
$\Phi(\cdot)$ denotes the c.d.f. of a standard normal random variable.
    \end{block}
\end{frame}

%12. Folie

% frametitle
{Permit price - log-normal moment matching}
    \begin{block}{}
    The permit price at time t is given by
$$
S_t^{Log} = \ \left\{
         \begin{array}{ll}
            P e^{-r\tau}
            &\mbox{if $q_{[0,t]} \ge N$} \\
            P e^{-r\tau} \cdot \Phi \left(\frac{-\ln\left( \frac{N - q_{[0,t]}}{Q_t} \right) + 2\ln(\alpha_{\tau}) - \frac{1}{2} \ln(2\beta_{\tau})}{\sqrt{\ln(2\beta_{\tau}) - 2\ln(\alpha_{\tau})}} \right) & \mbox{if $q_{[0,t]} < N$} \\
         \end{array} \right.
$$
where \\
$\tau = T - t$ is the time to compliance and \\
$\alpha_{\tau}, \beta_{\tau}$ are obtained by calculating the first and the second moment of the integral over a geometric Brownian motion. \\
$\Phi(\cdot)$ denotes the c.d.f. of a standard normal random variable.
    \end{block}
\end{frame}

% frametitle
{Permit price - reciprocal gamma moment matching}
    \begin{block}{}
    The permit price at time t is given by
$$
S_t^{IG}  = \ \left\{
         \begin{array}{ll}
            P e^{-r\tau}
            & \quad \mbox{if $q_{[0,t]} \ge N$} \\
            P e^{-r\tau} \cdot G \left(\frac{Q_t}{N - q_{[0,t]}} | \frac{4\beta_{\tau} - \alpha^2_{\tau}}{2\beta_{\tau} - \alpha^2_{\tau}} , \frac{2\beta_{\tau} - \alpha^2_{\tau}}{2 \alpha_{\tau} \beta_{\tau}}  \right)
            & \quad \mbox{if $q_{[0,t]} < N$} \\
         \end{array} \right.
$$
where \\
$\tau = T - t$ is the time to compliance and \\
$\alpha_{\tau}, \beta_{\tau}$ are obtained by calculating the first and the second moment of the integral over a geometric Brownian motion. \\
$G(x|a,b)$ denotes the c.d.f. of a gamma random variable with shape parameter $a$ and scale parameter $b$.
    \end{block}
\end{frame}

%14. Folie

% frametitle
{Relating theoretical permit prices to allocation}
    We introduce the following two random variables that are very easy to interpret
    \begin{block}{Time needed to exhaust the remaining permits}
$$
x_t := \frac{N - q_{[0,t]}}{Q_t}
$$
    \end{block}
and
\begin{block}{Over-/Underallocation in years}
$$
x_t - (T-t)
$$
    \end{block}
\end{frame}

%15. Folie

% frametitle
{Numerical illustrations }
\begin{center}
\begin{figure}[h!]
\centering
\rotatebox{0}{
\scalebox{0.6}{
\includegraphics[width=1.45\textwidth, height=\textheight]{../../../pics/fig05a.pdf}}}
\caption{Trajectory of $x_t$ for $t \in [0,1]$,  $N = Q_0 = 100$, $\mu = 0.02$ and $\sigma = 0.05$.}
\label{fig:plot4}
\end{figure}
\end{center}
\end{frame}

% frametitle
{Numerical illustrations}
\begin{center}
\begin{figure}[h!]
\centering
\rotatebox{0}{
\scalebox{0.6}{
\includegraphics[width=1.45\textwidth, height=\textheight]{../../../pics/fig08.pdf}}}
\caption{Trajectory of $S_t^{Lin}(x_t)$ (left), $S_t^{Log}(x_t)$ (middle) and $S_t^{IG}(x_t)$ (right) for $t \in [0,1]$,  $N = Q_0 = 100$, $\mu = 0.02$ and $\sigma = 0.05$.}
\label{fig:plot8}
\end{figure}
\end{center}
\end{frame}

% frametitle
{Implied over-/underallocation during the first phase of the EU ETS}
\begin{center}
\begin{figure}[h!]
\centering
\rotatebox{0}{
\scalebox{0.6}{
\includegraphics[width=1.35\textwidth, height=1.05\textheight]{../../../pics/fig-implied1.pdf}}}
\caption{\tiny Implied $x_t - (T-t)$ for first phase for fixed  $\mu = 0.02$ and $\sigma = 0.05$. Linear approximation approach (straight line), log-normal moment matching (dashed line). Positive values correspond to overallocation.}
\label{fig:plot10}
\end{figure}
\end{center}
\end{frame}

%\subsection[2006 Price Slump]{Theoretical discussion of permit price slump in 2006}

%18. Folie

% frametitle
{Permit price Delta}
For $t \in [0,T)$ and $q_{[0,t]} < N$


% begin itemize




	
$$
\frac{dS_t^{Lin}}{dx_t}(x_t) := \ - \frac{P e^{-r \tau}}{\sigma\sqrt{\tau}} \cdot \frac{1}{x_t} \phi \left( \frac{-\ln\left( \frac{1}{\tau} x_t \right) + \left( \mu - \frac{\sigma^2}{2} \right) \tau}{\sigma \sqrt{\tau}}\right) < 0
$$


	
$$
\frac{S_t^{Lin}((1+h)x_t)-S_t^{Lin}(x_t)}{S_t^{Lin}(x_t)}=
  -\frac{\phi \left( \frac{-\ln\left( \frac{1}{\tau} x_t \right) + \left( \mu - \frac{\sigma^2}{2} \right) \tau}{\sigma \sqrt{\tau}}\right)}{\Phi \left( \frac{-\ln\left( \frac{1}{\tau} x_t \right) + \left( \mu - \frac{\sigma^2}{2} \right) \tau}{\sigma \sqrt{\tau}}\right)}\cdot \frac{h}{\sigma \sqrt{\tau}}
$$


% end itemize



% frametitle
{Price slumps and allocation}
We show that a price slump of more than 50\% can be related to an implicit change in $x_t$ of less than $5\%$.\\

We introduce the following notation


% begin itemize




	$t - \Delta$ is the date before the publication of verified emissions that affected the permit price (28 April 2006)


	$t$ is the date of the announcement of cumulative emissions (15 May 2006)


% end itemize


\end{frame}

% frametitle
{Price slumps and allocation}
Using


% begin itemize




	the cumulative emissions until $t$ denoted by $q_{[0,t]}$


	the futures permit price at and before publication of emission data denoted by $F(t,T)$ and $F(t - \Delta,T)$, respectively


% end itemize


the implicit time needed to exhaust the remaining permits before the announcement was $h(\sigma)$ per cent larger than the previous estimate $\bar{x}_t$ where
$$
h(\sigma) = \frac{F(t,T) - F(t-\Delta,T)}{P \phi\left( \Phi^{-1}\left(\frac{F(t,T)}{P}\right) \right)} \cdot f^{approx}(\sigma,t,\bar{x}_t)
$$
\end{frame}

%19. Folie

% frametitle
{Price slumps and allocation}
\begin{center}
\begin{figure}[h!]
\centering
\rotatebox{0}{
\scalebox{0.6}{
\includegraphics[width=1.45\textwidth, height=\textheight]{../../../pics/fig-implied2.pdf}}}
\caption{Linear approximation ("1"), log-normal moment matching ("2").}
\label{fig:plot11}
\end{figure}
\end{center}
\end{frame}

% frametitle
{Price Floor Using a Subsidy}


% begin itemize




	The severe permit price drop, followed by a price hovering above zero for more than five months during the first phase of the EU ETS, persuaded several policy makers that new cap-and-trade schemes would need additional safety-valve features.


	In particular, policy makers have been concerned about permit prices that are either too low or too high.


	Thus setting a price floor and/or ceiling has been proposed.


% end itemize



% frametitle
{Price Floor Using a Subsidy -- Regulation}


% begin itemize




	A company with a permit shortage at compliance date faces a penalty $P$.


	If a company ends up with an excess of permits, it receives a subsidy $S$ per unit of permit.


	Let $0<S\leq P$ and let $N$ be the initial amount of permits allocated to relevant companies.


% end itemize



% frametitle
{Permit Price in hybrid system}
Denote the futures permit price by $\tilde{F}(t,T)$:
\begin{align*}
\tilde{F}(t,T) =& \ P \cdot \PM \left( q_{[0,T]} > N \mid \Ft \right) + S \cdot \PM \left( q_{[0,T]} \le N \mid \Ft \right) \\
  =& \ P \cdot \PM \left( q_{[0,T]} > N \mid \Ft \right) + S \cdot \left( 1 - \PM \left( q_{[0,T]} > N \mid \Ft \right)\right) \\
 % =& \ S + (P-S) \cdot \PM \left( q_{[0,T]} > N \mid \Ft \right)
  =& \ S + \frac{P-S}{P} \cdot P \cdot \PM \left( q_{[0,T]} > N \mid \Ft \right) \\
  =& \ S + \frac{P-S}{P} \cdot F(t,T) = \ F(t,T) + S\left(1 - \frac{F(t,T)}{P}\right),
\end{align*}
where $F(t,T) = P \cdot \PM \left( q_{[0,T]} > N \mid \Ft \right)$ is the futures permit price in an ordinary system.

% frametitle
{Decomposition of permit price in hybrid system}
Computing the value of a put with strike $S$ shows that the price in the hybrid scheme is the price in the ordinary scheme plus the value of a put option on the price in the ordinary scheme with strike $S$ and maturity $T$:
$$
\begin{array}{ll}
&\EW\left[ (S-F(T,T))^+ \mid \Ft \right]\\*[12pt]
=& \ \EW\left[ \left( S- P \textbf{1}_{\left\{ q_{[0,T]} > N \right\}} \right)^+ \mid \Ft \right] \\*[12pt]
=& \ \left( S- P \right)^+ \PM \left( q_{[0,T]} > N \mid \Ft \right) + \left( S- 0 \right)^+ \PM \left( q_{[0,T]} \le N \mid \Ft \right) \\*[12pt]
 \stackrel{S<P}{=}& \ S \cdot \PM \left( q_{[0,T]} \le N \mid \Ft \right).
\end{array}
$$

% frametitle
{Expected enforcement costs for regulated companies}
 Let $f_q$ be the probability density function of the cumulative emissions $q_{[0,T]}$ in the entire regulated period. The expected enforcement costs for relevant companies in an ordinary system are
$$
EEC = \ P \int_N^{\infty} (x-N) f_q(x) dx \ge 0.
$$
Similarly, the expected enforcement costs for regulated companies in this hybrid system are
$$
EEC^{PF} = \ P \int_N^{\infty} (x-N) f_q(x) dx - S \int_0^N (N-x) f_q(x) dx.
$$
So, the total expected enforcement costs for regulated companies under this hybrid system are lower than under an ordinary system.
$$
EEC - EEC^{PF} = \ S \int_0^N (N-x) f_q(x) dx \ \ge 0.
$$

% frametitle
{Enforcement costs for regulator}


% begin itemize




	A price floor ensured by the presence of a subsidy is relatively easy to implement and has the further advantage of lowering the expected enforcement costs for regulated companies.


	The presence of the subsidy could induce a higher stimulus in technology and abatement investments, favoring the achievement of emission reduction targets.


	However, the implementation of such a hybrid system might result in a significant financial burden for the environmental policy regulator. The current magnitude of this burden can be obtained by calculating the price of the put option.


% end itemize



% frametitle
{Hybrid systems }
{\tiny
\begin{table}
\centering
\begin{tabular}{|l|l|l|l|l|}
\hline
Scheme & Price bound & Prices can & Link with & Description of the mechanism \\
 &  & exceed bounds & offsets market &  \\
\hline
\hline
\multicolumn{5}{|l|}{\textbf{Existing cap-and-trade scheme}} \\
\hline
%EU ETS & - & - & Yes & Fixed limit on the use of CERs \\
% &  &  & & Possibility to trade options quoted on exchanges \\
%\hline
Offset safety-valve & Upper & Yes & Yes & Flexible limit on the use of offsets \\
\hline
\hline
\multicolumn{5}{|l|}{\textbf{Proposed safety-valve mechanisms for cap-and-trade schemes}} \\
\hline
Subsidy price floor & Lower & No & No & Subsidy \\
\hline
Price collar & Upper \&  & No & No & Regulator sells unlimited amount of \\
& Lower & & &                                   permits at the price ceiling and\\
  &   &  &  &                               buys unlimited amount of permits    \\
 &   &  &  &                              at the price floor \\
\hline
Allowance reserve & Upper \&  & Yes & No & Regulator sells limited amount of  \\
 & Lower  & &  &                                     permits at the price ceiling and buys  \\
 &   & &  &                                     limited amount permits at price floor \\
\hline
Regulator offers  & Upper \&  & No (for owner & No & Regulator sells options  \\
options                        &       Lower         & of options)    &    &  at a market price\\
\hline
\end{tabular}
%\caption{Overview on the main results of the mechanisms under investigation in this paper and description of how they work in practice.}
\end{table}
}

\subsection{Dynamic Reduced Form Models}

% frametitle
{Permit Prices}


% begin itemize




	Recall the emission rate
$$
dQ_t = Q_t(\mu dt + \sigma dW_t)
$$


	
The cumulative emissions are
$$
q_{[0,t]} = \int_0^t Q_s ds
$$


	
The futures permit price is given as
$$
F(t,T) = P \prb\left[q_{[0,T]}>N |\F_t\right]
$$


% end itemize


\end{frame}

% frametitle
{Approximative Pricing}


% begin itemize




	Linear approximation approach
$$
\begin{array}{lll}
q_{[t_1,t_2]} &\approx& \tilde{q}^{Lin}_{[t_1,t_2]} = Q_{t_2} (t_2 - t_1) \\*[12pt]
&=&\displaystyle   Q_{t_1} e^{\left\{\log (t_2 - t_1) + \left(\mu-\frac{\sigma^2}{2}\right)(t_2-t_1)+\int_{t_1}^{t_2}\sigma dW_t\right\}}
\end{array}
$$


	Moment matching
$$
\begin{array}{lll}
q_{[t_1,t_2]} &\approx& \tilde{q}^{Log}_{[t_1,t_2]}\\*[12pt]
&=& Q_{t_1} \exp\left\{ \int_{t_1}^{t_2}\mu_t dt + \int_{t_1}^{t_2} \sigma_t dW_t\right\}
\end{array}
$$
where the functions $\mu_t$ and $\sigma_t$ are defined by the functions $\alpha_t, \beta_t$ from the moment matching.


% end itemize


\end{frame}
%\subsection[Reduced Form Dynamics]{Dynamics of the permit process in the reduced form model}

% frametitle
{Carmona-Hinz Approach}


% begin itemize




	Use a lognormal process
$$
\Gamma_{T}= \Gamma_0  \exp{\left\{\int_{0}^{T}\sigma_t dW_t -\frac{1}{2}\int_0^T \sigma^2_t dt\right\}}
$$
with $\Gamma_0 >0$ and $\sigma(.)$ a deterministic square-integrable function.


	Define the futures price under a risk-neutral measure $\Q$ as
$$
F(t,T) = P \Q\left[\Gamma_T>1 |\F_t\right]
$$


% end itemize


\end{frame}

% frametitle
{Reduced-Form Dynamics}
The martingale
$$
a_t = \EX^\Q\left[\IF_{\{\Gamma_T>1\}} |\F_t\right]
$$
is given by
$$
a_t= \Phi \left[\frac{\Phi^{-1}(a_0) \sqrt{\int_0^T \sigma^2_s ds}+\int_0^t \sigma_s dW_s}{\sqrt{\int_t^T \sigma^2_s ds}}\right]
$$
and solves the stochastic differential equation
$$
da_t = \Phi'\left(\Phi^{-1}(a_t)\right)\sqrt{z_t}dW_t
$$
with
$$
z_t=\frac{\sigma_t^2}{\int_t^T \sigma^2_u du}
$$

% frametitle
{Reduced-Form Dynamics -- Proof}


% begin itemize




	$a_t$ formula is straightforward calculation


	For dynamics use that
$$
a_t = \Phi(\xi_t)
$$
with
$$
\xi_t = \frac{\xi_{0,T}+\int_0^t\sigma_s dW_s}{\sqrt{\int_t^T\sigma_s^2ds}}\; \mbox{and}\;  \xi_{0,T}=\log \Gamma_0 - \frac{1}{2} \int_0^T\sigma_s^2ds.
$$
Starting with the dynamics of $\xi_t$ an application of It{\^o}'s formula gives the result.


% end itemize



%\subsection{Historical Calibration}

% frametitle
{Model Parametrization}


% begin itemize




	For constant $\sigma$ we find $z_t=(T-t)^{-1}$, so a richer specification is needed.


	A standard model is
$$
da_t = \Phi'\left(\Phi^{-1}(a_t)\right)\sqrt{\beta(T-t)^{-\alpha}}dW_t
$$
which specifies a family $\sigma_s(\alpha,\beta)$.


	
So $z_t(\alpha, \beta) = \beta(T-t)^{-\alpha}$ and
$$
\begin{array}{lll}
\sigma_t^2(\alpha,\beta)&=& \displaystyle z_t(\alpha, \beta) \exp\left\{-\int_0^t z_s(\alpha, \beta) ds \right\}\\*[12pt]
&=&\displaystyle
\left\{
\begin{array}{ll}
\beta(T-t)^{-\alpha} e^{-\frac{\beta}{1-\alpha}[T^{1-\alpha}-(T-1)^{1-\alpha}]} &\alpha \not=1\\
\beta(T-t)^{\beta-1}T^{-\beta} &\alpha=1.
\end{array}
\right.
\end{array}
$$


% end itemize



% frametitle
{Objective Measure}


% begin itemize




	We do a historical calibration and change measure to the objective measure.


	The standard change of measure gives
$$
\frac{d\prb}{d\Q} = \exp\left\{\int_0^T H_s dW_s -\frac{1}{2} \int_0^T H_s^2ds \right\}).
$$


	
Under constant market price of risk $H_t \equiv h$ and by Girsanov's theorem
$$
\tilde{W}_t = W_t - ht
$$
is a $\prb$ Brownian motion.


% end itemize



% frametitle
{Objective Measure}


% begin itemize




	
Under $\prb$
$$
d\xi_t = \left(\frac{1}{2} z_t \xi_t + h \sqrt{z_t} \right)dt + \sqrt{z_t} d\tilde{W}_t,
$$
so $\xi_{\tau}$ given $\xi_t$ is Gaussian.


	So we can invert permit prices to obtain $\xi$ values and calculate the log-likelihood to obtain
estimates for $\alpha$ and $\beta$.


% end itemize



%\subsection{Option Pricing}

% frametitle
{Pricing Formula}
For a European call with strike $K$ and maturity $\tau$ the option price is
$$
C_t = e^{-\int_t^\tau r_s ds} \int_{-\infty}^\infty (P\Phi(x)-K)^+ \Phi_{\mu_{t,\tau}, \sigma_{t,\tau}}(dx)
$$
with
$$
\mu_{t,\tau}=
\left\{
\begin{array}{ll}
\xi_t \left(\frac{T-t}{T-\tau}\right)^{\frac{\beta}{2}} & \alpha =1\\
\xi_t \exp\left\{\frac{\beta}{2(1-\alpha)}[(T-t)^{1-\alpha}-(T-\tau)^{1-\alpha}]\right\} & \alpha \not= 1.
\end{array}
\right.
$$
and
$$
\sigma^2_{t,\tau}=
\left\{
\begin{array}{ll}
\left(\frac{T-t}{T-\tau}\right)^\beta-1 & \alpha =1\\
\exp\left\{\frac{\beta}{1-\alpha}[(T-t)^{1-\alpha}-(T-\tau)^{1-\alpha}]\right\}-1 & \alpha \not= 1.
\end{array}
\right.
$$

%\subsection{Reduced Form Option Pricing in Multi-period Models}

% frametitle
{Two-period Model}


% begin itemize




	We consider a two-period model, $[0,T]$ and $[T,T']$, with banking and withdrawal


	Let $\Q$ be a martingale measure and $(A_t)_{t\in[0,T]}$ and $(A'_t)_{t\in[0,T']}$
be the futures contracts which are $\Q$ martingales.


	Let $N\in \F_T$ resp. $N' \in \F_{T'}$ be non-compliance in the first resp. second period.


% end itemize



% frametitle
{(Non-) Compliance at $T$}


% begin itemize




	In case of compliance, i.e. event $\Omega-N$, since $A_T$ is the spot price at $T$ and the permit can be banked,  we have
$$
A_T\IF_{\{\Omega-N\}}= \kappa A'_T \IF_{\{\Omega-N\}}
$$
with $\kappa= \exp\{-\int_T^{T'}r_s ds\}$ a discount factor.


	In case of non-compliance
$$
A_T\IF_{N}= \kappa A'_T \IF_{N} + P\IF_N
$$


	Thus
$$
A_t-\kappa A'_t = \EX_t^\Q[A_T-\kappa A'_T]= \EX^\Q_t[P \IF_N]
$$


% end itemize



% frametitle
{Reduced-Form Model}


% begin itemize




	As in the one period case,  we set
$$
A_t-\kappa A'_t = P \Phi(\xi_t^1)
$$
with $\xi^1_t$ a Gaussian process driven by a Brownian motion $W^1_t$.


	Assume that the ETS ends after the second period, then
$$
A'_t = P \Phi(\xi^2_t)
$$
with $\xi^2_t$ a Gaussian process driven by a Brownian motion $W^2_t$.


% end itemize



% frametitle
{Pricing Formula}
For a European call written on a futures in the first period with strike $K$ and maturity $\tau$ we decompose the payoff
$$
(A_\tau-K)^+= (A_\tau - \kappa A'_\tau + \kappa A'_\tau -K )^+= (P\Phi(\xi^1_t) + \kappa P \Phi(\xi_t^2) -K)^+
$$
We obtain for the option price
$$
C_t = e^{-\int_t^\tau r_s ds} \int_{\setR^2} (P\Phi(x_1)+\kappa P \Phi(x_2) -K)^+ \Phi_{\mu_\tau, \sigma_\tau}(dx_1dx_2)
$$
where the parameters of the two-dimensional Gaussian distribution depend on the individual drift and volatility terms and the correlation
of $\xi^1$ and $\xi^2$.

% frametitle
{Parameters of the Pricing Formula}
The means are
\begin{eqnarray}\nonumber
\mu^1_{t,\tau} &=& \Phi^{-1}\left(\frac{A_t - \kappa A'_t}{P}\right) \sqrt{\left(\frac{T-t}{T-\tau}\right)\beta_1}\\\nonumber
\mu^2_{t,\tau} &=& \Phi^{-1}\left(\frac{\kappa A'_t}{P}\right) \sqrt{\left(\frac{T'-t}{T'-\tau}\right)\beta_2}\\\nonumber
\end{eqnarray}
and the covariance matrix is
\begin{eqnarray}\nonumber
\nu^{1,1}_{t,\tau} &=& \var(\xi_\tau^1) =  \left(\frac{T-t}{T-\tau}\right)^{\beta_1}-1\\\nonumber
\nu^{2,2}_{t,\tau} &=& \var(\xi_\tau^2) =  \left(\frac{T'-t}{T'-\tau}\right)^{\beta_2}-1\\\nonumber
\nu^{1,2}_{t,\tau}= \nu^{2,1}_{t,\tau} &=& \beta_1^\frac{1}{2}\beta_2^\frac{1}{2}
\frac{\int_t^\tau (T-u)^\frac{\beta_1-1}{2} (T'-u)^\frac{\beta_2-1}{2} \rho du}{(T-\tau)^\frac{\beta_1}{2} (T'-\tau)^\frac{\beta_2}{2}} \\\nonumber
\end{eqnarray}


% !TEX root = QCF_ss14UDE.tex
\section{Capital Market and Renewable Energy Projects}
\subsection{Carbon Bonds}

% frametitle
{Carbon Revenue Bonds}


% begin itemize


\item<1-> To finance high initial cost for Renewable Energy (RE) projects future returns of the project are securitized
\item<2->


% begin itemize


\item Sell future electricity from RE project
\item Sell environmental credits from RE project


% end itemize


\item<3-> Only revenue from environmental credits is used for bond
\item<4-> Rigorous forecast analysis of revenues, sensitivity tests and risk analysis is required.


% end itemize



% frametitle
{Structure of Bond}


% begin itemize


\item<1-> pass-through: all revenues are directly passed trough to the owner of the bond
\item<2->


% begin itemize


\item maturity: T
\item revenues in year 1: $c_i$
\item rate of return: $r$
\item initial price: $x$


% end itemize


\item<3-> Fair price
$$
x= \sum_{i=1}^T \frac{c_i}{(1+r)^i}
$$


% end itemize



% frametitle
{EUA Time Series}
\begin{figure}[h!]
\centering
\includegraphics[width=0.9\textwidth, height=0.7\textheight]{../../../pics/EUA-TimeSeries.pdf}
%\caption{EUA Time Series}
\label{fig:EUA-TS}
\end{figure}

% frametitle
{QQ-Plots for EUA fits}
\begin{figure}[h!]
\centering
\includegraphics[width=0.9\textwidth, height=0.7\textheight]{../../../pics/QQ-CarbonFit.pdf}
%\caption{EUA fits}
\label{fig:EUA-fits}
\end{figure}

% frametitle
{Carbon Bond Histogram}
\begin{figure}[h!]
\centering
\includegraphics[width=0.9\textwidth, height=0.7\textheight]{../../../pics/carbon-bond-histogram.pdf}
%\caption{Carbon Bond Histogram}
\label{fig:Carbon-Bond-Histogram}
\end{figure}

\subsection{Impact of Carbon Markets on Investment Markets}

% frametitle
{Financial Implications of Carbon Policies}


% begin itemize


\item<1-> Modern risk management has to include the consequences of climate change.
\item<2-> Regulatory risk (reduction targets for carbon emissions) transforms into financial risk  for several asset classes.
\item<3-> Carbon inefficient firms tend to have a higher credit spread and higher refinancing costs.


% end itemize



% frametitle
{Factors Affecting Carbon Risk}


% begin itemize


\item<1-> Energy intensity and fuel mix -- firms that are dependent on fossil fuels face higher costs.
\item<2-> Direct, indirect and embedded emissions of a firm's product affect market position.
\item<3-> Marginal abatement costs.
\item<4-> Technology trajectory -- progress in adapting low-carbon emission technologies.


% end itemize



% frametitle
{Motivation for Investors to Invest in Carbon-related Assets}


% begin itemize


\item<1-> Financial Motivation


% begin itemize


\item Portfolio diversification
\item Potential fundamental price appreciation of carbon
\item Hedging financial risk due to carbon price increases


% end itemize


\item<2-> Green Motivation


% begin itemize


\item Compliance with UN principles of responsible investment (UN PRI)
\item Public opinion, behaviour as corporate citizen
\item Incentivizing the corporate sector by taking carbon credits from the market


% end itemize




% end itemize



% frametitle
{Risk In Renewable Energy Companies}


% begin itemize


\item<1-> Costs: as the costs of producing RE come down while the costs of producing fossil fuels rise, a
substitution will occur.
\item<2-> Capital: Government and private capita
\item<3-> Competition: between governments as they try to build greener economies
\item<4-> China: huges efforts to establish a green economy
\item<5-> Consumers: demand products with less impact on the economy
\item<6-> Climate Change: will be tackled by investment in greener technologies.


% end itemize



% frametitle
{CAPM Analysis of RE Companies}


% begin itemize


\item<1-> Empirical evidence shows that RE companies have a $\beta$ close to two.
\item<2-> Model: $i$ firm, $t$ time
$$
R_{it}= \alpha_i + \beta_{it} R_{mt}+\epsilon_{it}
$$
$R$ returns, $\alpha$ component that is independent of the market.
\item<3-> Higher beta values indicate a higher equity cost of capital.Investors must then be compensated
through higher expected returns in order to take on the higher risk. A higher equity cost of capital can affect borrowing costs and
it can also affect the discount rate used in net present value calculations.


% end itemize



% frametitle
{CAPM Analysis of RE Companies II}


% begin itemize


\item<1-> Which factors affect systematic risk (the $\beta$)?
\item<2-> Empirical analysis shows:


% begin itemize


\item Increases in sales growth reduce market risk
\item Increases in oil price returns increases systematic risk


% end itemize


\item<3-> In order to foster RE companies governments can implement policies to increase sales of such companies (e.g. PV-industry and feed-in tariffs).


% end itemize




%\setcounter{part}{5}
%\input{ClimateFinance}





%\begin{frame}
%\tiny
%\bibliographystyle{apalike}
%\bibliography{QCF-literature}
%\nocite{*}
%\end{frame}

\end{document} 