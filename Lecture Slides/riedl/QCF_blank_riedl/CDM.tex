% !TEX root = QCF_ss13UDE.tex
\section{Flexible Mechanisms under the Kyoto Protocol}
\subsection{Structure}

Project-based Mechanisms
	2005:  Emissions Trading (EU ETS) was launched in the European Union as a measure to meet the Kyoto commitment
	
	Flexible Mechanisms as Clean Development Mechanism (CDM) and Joint Implementation (JI) can be used by countries
	to meet their Kyoto commitment and also by companies within the framework of EU emissions 	trading

Objectives of CDM/JI
	To save the same or a higher amount of CO2 for the same financial effort and by using fewer financial resources (abatement costs are lower)
	
	Sustainability, emissions reduction, technology transfer, contribution towards economic development
 

							% Folie 4
Cost Advantage
	\begin{figure}[h!]
	\centering
	\includegraphics[width=0.9\textwidth, height=0.7\textheight]{../../../pics/abatementcosts.pdf}
	\caption{Cost advantage due to CDM/JI}
	\label{fig:CDM/JI}
	\end{figure}


							% Folie 5
Clean Development Mechanism (CDM)
	Emission reduction projects in a developing or an emerging country (e.g. China, India, Brazil, Malaysia)
	
	Accepted technologies (photovoltaic, wind power, biomass, energy efficiency; no nuclear projects!)
	
	Projects generate Certified Emission Reductions (CER) in the amount of tonnes of emissions that have been 
	avoided (compared to the 'business-as-usual'-scenario, that means claim for additionality)

	\textbf{Secondary CERs:} issued, tradable credits with guaranteed delivery \\
	
	\textbf{Primary CERs:} purchased forward directly from project (or fonds), subject to individual project and delivery risks


							% Folie 6
Joint Implementation (JI)
	Emission reduction projects in an industrialized or a transition country (e.g. Russia, Ukraine), which has committed to a cap, solely in sectors not covered in the ETS
 
	Analogous to CDM, but reduces the emission budget of the country where the project takes place
	
	Projects generate ERU (Emissions Reduction Unit) in the amount of the avoided emissions (compared to the 'business-as-usual'-scenario)
 
	from 2008 onwards accepted in the EU ETS

CERs and ERUs can be used by \\
	1) Countries, for compliance under the Kyoto Protocol \\
	2) Companies within the EU ETS (up to 22\%), for compliance


							% Folie 7
Example of the Project-based Mechanisms}
\begin{figure}[t]
	\begin{minipage}[t]{0.475\textwidth}
		Construction of a wind farm \\
		abroad:
		\vspace*{-0.2cm}
			\begin{figure}[h!]
			\centering
			\includegraphics[width=0.6\textwidth, height=0.3\textheight]{../../../pics/windfarm.jpg}
			\end{figure}
		\vspace*{-0.4cm}
			annual electricity production: 300 GWh
			no emissions occur during production
	\end{minipage}

	\begin{minipage}[t]{0.475\textwidth}
		Electricity production at a coal \\
		power plant:
			\vspace*{-0.2cm}
			\begin{figure}[h!]
			\centering
			\includegraphics[width=0.6\textwidth, height=0.3\textheight]{../../../pics/coalpowerplant.jpg}
			\end{figure}
			\vspace*{-0.4cm}
			300,000 tonnes of CO2 emissions annually
	\end{minipage}
\end{figure}


						% Folie 8
Example of the Project-based Mechanisms
	By the construction of the wind farm, emissions amounting to 300,000 tonnes per year have been avoided
	
	Therefore, emission certificates amounting to 300,000 tonnes per year will be generated


						% Folie 9
Distribution of CDM Projects and generated CERs by host country 
\begin{figure}[t]
	\begin{minipage}[t]{0.475\textwidth}
		Expected average annual CERs \\
		from registered projects \\
		(total: 605,014,135):
		\vspace*{-0.7cm}
		\begin{figure}
		\centering
		\includegraphics[width=1.25\textwidth, height=0.4\textheight]{../pics/ExpectedAverAnnCERs2.png}
		%Expected average annual CERs from registered projects by host party. Total: 605,014,135. Source: http:\cdm.unfccc.int (c) 20.06.2012 15:56
		\end{figure}
		\vspace*{-0.8cm}
		68\% renewables projects %Source: http://www.cdmpipeline.org/cdm-projects-type.htm
	\vspace*{-0.9cm}
	\end{minipage}
	
	\begin{minipage}[t]{0.475\textwidth}
		Registered project activities\\
		(total: 4,248):
		\vspace*{-0.7cm}
		\begin{figure}
		\centering
		\includegraphics[width=1.2\textwidth, height=0.4\textheight]{../../../pics/RegisteredProjActivity2.png}
		% registered project activities by host party. Total: 4,248. Source: http:\cdm.unfccc.int (c) 20.06.2012 15:56
		\end{figure}
		\vspace*{-0.8cm}
		5600  projects worldwide
		2,7 bn CERs expected until end of 2012
	\end{minipage}
\end{figure}


\subsection{Risks}

Types of Risk Involved
	Market risk
	Volume Risk
	Credit Risk
	Delivery Risk


CDM/JI Risk Management Example
	We combine a CDM project with an ERPA
	
	Recall an Emission Reduction Purchase Argreement (ERPA) is a  transaction that transfers carbon credits between
	two parties under the Kyoto Protocol. The buyer pays the seller cash in exchange for carbon credits, thereby allowing the purchaser to emit more carbon dioxide into the atmosphere.
	
	Assume an ERPA with $10$ \euro/ t CO2


CDM/JI Risk Management Example II
	Sell the expected volume $V_{exp}$ of 50 000 t CO2 forward at $15$ \euro /t CO2
	
	Now the volume delivered is a random variable $V$ and the certificate spot price $S$ is random.
	
	So the portfolio value $P$ at delivery is
		\begin{equation}\nonumber
		P = 15 \times V_{exp} - 10 \times V + (S-15) \times (V_{exp}-V)^+ + S \times (V-V_{exp})^+
		\end{equation}
 
	So we face two risky scenarios
		Higher volume with low spot
		lower volume with higher spot


Allowance Price Versus CERs
	\begin{figure}[h!]
	\centering
	\includegraphics[width=0.9\textwidth, height=0.7\textheight]{../../../pics/EUAvsSCER.pdf}
	\caption{Price Difference EUA vs SCER}
	\label{fig:EUAvsSCER}
	\end{figure}


Allowance Price Versus CERs
	Reason for price difference: Limited number of SCER for use in EU
	Intensive discussion of that in Barrieu and Fehr (2010).	



Challenges in CDM Projects
	Regulatory  Risk
		Possible changes of CDM frame after 2012 with an impact on private investments
		Acceptance of project-types and countries (China, Brazil)

	CDM Acceptance
		Long and bureaucratic
		not transparent, concrete methodic unknown

	Country Risk
		local (CDM-) infrastructure: people, infrastructure
		local energy infrastructure
		local political risk


\section{Capital Market and Renewable Energy Projects}
\subsection{Carbon Bonds}

Carbon Revenue Bonds
	To finance high initial cost for Renewable Energy (RE) projects future returns of the project are securitized
		Sell future electricity from RE project
		Sell environmental credits from RE project

	Only revenue from environmental credits is used for bond
		Rigorous forecast analysis of revenues, sensitivity tests and risk analysis is required.


Structure of Bond
	pass-through: all revenues are directly passed trough to the owner of the bond
		maturity: T
		revenues in year 1: $c_i$
		rate of return: $r$
		initial price: $x$

	Fair price
		$$
		x= \sum_{i=1}^T \frac{c_i}{(1+r)^i}
		$$


EUA Time Series
	\begin{figure}[h!]
	\centering
	\includegraphics[width=0.9\textwidth, height=0.7\textheight]{../../../pics/EUA-TimeSeries.pdf}
	%\caption{EUA Time Series}
	\label{fig:EUA-TS}
	\end{figure}

QQ-Plots for EUA fits
	\begin{figure}[h!]
	\centering
	\includegraphics[width=0.9\textwidth, height=0.7\textheight]{../../../pics/QQ-CarbonFit.pdf}
	%\caption{EUA fits}
	\label{fig:EUA-fits}
	\end{figure}

Carbon Bond Histogram
	\begin{figure}[h!]
	\centering
	\includegraphics[width=0.9\textwidth, height=0.7\textheight]{../../../pics/carbon-bond-histogram.pdf}
	%\caption{Carbon Bond Histogram}
	\label{fig:Carbon-Bond-Histogram}
	\end{figure}


\subsection{Impact of Carbon Markets on Investment Markets}

Financial Implications of Carbon Policies
	Modern risk management has to include the consequences of climate change.
	
	Regulatory risk (reduction targets for carbon emissions) transforms into financial risk  for several asset classes.
	
	Carbon inefficient firms tend to have a higher credit spread and higher refinancing costs.


Factors Affecting Carbon Risk
	Energy intensity and fuel mix -- firms that are dependent on fossil fuels face higher costs.
	Direct, indirect and embedded emissions of a firm's product affect market position.
	Marginal abatement costs.
	Technology trajectory -- progress in adapting low-carbon emission technologies.


Motivation for Investors to Invest in Carbon-related Assets}
	Financial Motivation
		Portfolio diversification
		Potential fundamental price appreciation of carbon
		Hedging financial risk due to carbon price increases

	Green Motivation
		Compliance with UN principles of responsible investment (UN PRI)
		Public opinion, behaviour as corporate citizen
		Incentivizing the corporate sector by taking carbon credits from the market


Risk In Renewable Energy Companies}
	Costs: as the costs of producing RE come down while the costs of producing fossil fuels rise, a
	substitution will occur.
	
	Capital: Government and private capita
	
	Competition: between governments as they try to build greener economies
	
	China: huges efforts to establish a green economy
	
	Consumers: demand products with less impact on the economy
	
	Climate Change: will be tackled by investment in greener technologies.



CAPM Analysis of RE Companies
	Empirical evidence shows that RE companies have a $\beta$ close to two.
	
	Model: $i$ firm, $t$ time
		$$
		R_{it}= \alpha_i + \beta_{it} R_{mt}+\epsilon_{it}
		$$
	$R$ returns, $\alpha$ component that is independent of the market.
 
	Higher beta values indicate a higher equity cost of capital.Investors must then be compensated
	through higher expected returns in order to take on the higher risk. A higher equity cost of capital can affect borrowing costs and
	it can also affect the discount rate used in net present value calculations.


CAPM Analysis of RE Companies II
	Which factors affect systematic risk (the $\beta$)?
	
	Empirical analysis shows:
		Increases in sales growth reduce market risk
		Increases in oil price returns increases systematic risk

	In order to foster RE companies governments can implement policies to increase sales of such companies (e.g. PV-industry and feed-in tariffs).

