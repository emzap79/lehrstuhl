% !TEX root = riskmanagement_ws13UDE.tex
\part{Risk Measures}
\section{Value at Risk (VaR)}


\subsection{Basic Properties}

\frame{\frametitle{Risk Measures}
\begin{itemize}
\item<1->
For risk management purposes one is primarily concerned with
containing the downside risk, i.e. to place bounds on the
potential losses, or the amount by which the final value of a
trading position falls short of the expected return.
\item<2-> This leads to
the definition of measures of risk that focus on the lower tail of
the distribution of the random variable representing the final
position.
\item<3-> Here, the classical measure - variance - is being replaced using a
full (empirical) distribution.
\end{itemize}
}

\frame{\frametitle{Value at Risk (VaR)}
\begin{itemize}
\item<1->
For a given portfolio, value-at-risk measures the potential future loss (in
terms of market value) that, under normal market conditions, will not be exceeded with a defined confidence
level $\alpha$ in a defined period.
\item<2->More formally, let $X$ be a random variable $X$ that represents the change in value of some position during some trading period. Then we are looking for the largest number  $x_{\alpha}$, such that the probability 
$$
\prb(X \geq q_{\alpha})= \alpha.
$$
\item<3-> Then $q_{\alpha}$ is the loss level that will be exceeded on $1-\alpha \%$ occassions. 
$$
-q_\alpha= VaR_\alpha
$$ 
is called the Value-at-Risk at level $\alpha$.
\item<4-> An equivalent definition uses $Y=-X$ the loss of the
position. 
\end{itemize}
}



\frame{\frametitle{Value at Risk}
The worst $1-\alpha$ scenarios are above the $-VaR_\alpha$, all others are below.
\begin{figure}
	\centering
		\includegraphics[width=0.8\textwidth]{../../../pics/VaR}
	\label{fig:VaR}
\end{figure}
}


\frame{\frametitle{VaR: Deutsche Bank Risk Report -- Usage of VaR}
Quote p7.: We use the value-at-risk approach to derive quantitative measures for our trading book
market risks under normal market conditions. Our value-at-risk figures play a role in both internal and
external (regulatory) reporting. For a given portfolio, value-at-risk measures affect the potential future loss (in
terms of market value) that, under normal market conditions, will not be exceeded with a defined confidence
level in a defined period. The value-at-risk for a total portfolio represents a measure of our diversified
market risk (aggregated, using pre-determined correlations) in that portfolio.
}

\frame{\frametitle{VaR: Deutsche Bank Risk Report -- Usage of VaR}
Quote p37: Our primary instrument to manage trading market risk is the limit setting process. Our Management Board,
supported by Market Risk Management, which is part of our independent legal, risk \& capital function, sets a
Group-wide value-at-risk and economic capital limits for the market risk in the trading book. Market Risk Management
sub-allocates this overall limit to our group divisions (e.g., Global Markets and Corporate Finance)
and individual business areas (e.g., Global Rates, Global Markets Equity, etc.) based on anticipated business
plans and risk appetite. Within the individual business areas, the business heads or Chief Operating Officers
may establish business limits by sub-allocating the Market Risk Management limit down to individual portfolios
or geographical regions.
}


\frame{\frametitle{VaR: Deutsche Bank Risk Report -- Assessment of VaR:}
Value-at-risk is a quantitative measure of the potential loss (in value) of trading positions due to market
movements that will not be exceeded in a defined period of time and with a defined confidence level.\\
Our value-at-risk for the trading businesses is based on our own internal value-at-risk model. In October 1998,
the German Banking Supervisory Authority (now the BaFin) approved our internal value-at-risk model for
calculating the regulatory market risk capital for our general and specific market risks. Since then the model
has been periodically refined and approval has been maintained.\\
We calculate value-at-risk using a 99 \% confidence level and a holding period of one day. This means we
estimate there is a 1 in 100 chance that a mark-to-market loss from our trading positions will be at least as
large as the reported value-at-risk. For regulatory reporting, the holding period is ten days.\\
}

\frame{\frametitle{VaR: Deutsche Bank Risk Report -- Assessment of VaR}
We use historical market data to estimate value-at-risk, with an equally-weighted 261 trading day history. The
calculation employs a Monte Carlo simulation technique, and we assume that changes in risk factors follow a
certain distribution, e.g., normal or logarithmic normal distribution. To determine our aggregated value-at-risk,
we use observed correlations between the risk factors during this 261 trading day period.\\
Our value-at-risk model is designed to take into account the following risk factors: interest rates, credit
spreads, equity prices, foreign exchange rates and commodity prices, as well as their implied volatilities and
common basis risk. The model incorporates both linear and, especially for derivatives, nonlinear effects of the
risk factors on the portfolio value.
}

\frame{\frametitle{VaR: Deutsche Bank Risk Report -- Assessment of VaR}
The value-at-risk measure enables us to apply a constant and uniform measure across all of our trading businesses
and products. It allows a comparison of risk in different businesses, and also provides a means of
aggregating and netting positions within a portfolio to reflect correlations and offsets between different asset
classes. Furthermore, it facilitates comparisons of our market risk both over time and against our daily trading
results
}

\frame{\frametitle{VaR: Deutsche Bank Risk Report -- Limitations of VaR}

The use of historical data may not be a good indicator of potential future events, particularly those that are
extreme in nature. This backward-looking limitation can cause value-at-risk to understate risk (as in
2008), but can also cause it to be overstated. In 2009 we observed fewer outliers than would be predicted
by the model. In a strict statistical sense, the value-at-risk in 2009 was over-conservative, and had overestimated
the risk in the trading books. As discussed, our value-at-risk model bases estimates of future
volatility on market data observed over the previous year. For much of 2009, this estimate incorporated
the extreme market volatility observed in the fourth quarter of 2008 following the bankruptcy of Lehman
Brothers. As markets normalized in 2009, estimated volatility exceeded actual volatility, and fewer outliers
occurred than expected.
}

\frame{\frametitle{VaR: Deutsche Bank Risk Report -- Limitations of VaR}
\begin{itemize}
\item<1->
Assumptions concerning the distribution of changes in risk factors, and the correlation between different
risk factors, may not hold true, particularly during market events that are extreme in nature. While we
believe our assumptions are reasonable, there is no standard value-at-risk methodology to follow. Different
assumptions would produce different results.
\item<2->The one day holding period does not fully capture the market risk arising during periods of illiquidity, when
positions cannot be closed out or hedged within one day.
\item<3->Value-at-risk does not indicate the potential loss beyond the 99th quantile.
\item<4-> Intra-day risk is not captured.
\item<5-> Although we consider the material risks to be covered by our value-at-risk model and we further enhance
it, there still may be risks in the trading book that are not covered by the value-at-risk model.
\end{itemize}
}

\frame{\frametitle{Value at Risk - Shortcomings}
\begin{itemize}
\item<1->
The Value at Risk is a highly aggregated form of a risk measure. It projects the whole distribution of the risk to just one number.
\item<2-> Thus, the VaR is not able to capture all properties of the risk. Particularly, the VaR gives no hint about what happens beyond the $\alpha$-level.
%\item<3-> Deutsche Bank Risk Report p. 40 (also Discussion pp 41)
\end{itemize}
}


\frame{\frametitle{Value at Risk}
The figure illustrates a portfolio with a low VaR compared to its risk.
\begin{figure}
	\centering
		\includegraphics[width=0.7\textwidth]{../../../pics/Shortcoming_VaR}
	\label{fig:Shortcoming_VaR}
\end{figure}
}


\subsection{Calculation Approaches}

\frame{\frametitle{Value at Risk - Parameters}
\begin{itemize}
\item<1->
The parameters of the VaR are time horizon and confidence level.
\item<2-> If the changes in the portfolio have independent identical normal distributions, then we get for an $N$-period VaR
$$
	VaR(\alpha,N) = \sqrt{N}VaR(\alpha,1).
$$
\item<3-> This square-root rule
is often used as an approximation.
\item<4-> The proper time horizon for VaR calculations has to be chosen
according to the liquidity of the market. For the trading book of banks
typically $N=10$ days. In Solvency II, the horizon is 1 year.
\item<5-> Confidence levels between $95\%$ and $99.5\%$
 are often used. For normally distributed risks, the VaR can be computed as a
 multiple of the standard deviation.
\end{itemize}
}



%\subsection{VaR Computation}

\frame{\frametitle{Absolute and Relative VaR}
\begin{itemize}
\item<1->
We use the definitions \textcolor{red}{Absolute VaR is the maximum expected loss over a given horizon period
at a given level of confidence} and \textcolor{blue}{Relative VaR is the maximum expected loss over a given horizon period
at a given level of confidence measured relative to the expected revenue over that period}.
\item<2-> The return $R$ on the portfolio is the revenue $REV$ divided by the initial value of the portfolio $W$. If we denote the
quantile cut-off points in the distributions by $R^*$ and $REV^*$, then the
absolute VaR is $$
VaR_{abs} = -REV^* = -R^* W
$$
and the relative VaR (distance from the mean) is
$$
VaR_{rel}= -REV^*+\bar{REV}= -R^*W+\mu W
$$
where $\bar{REV}$ is the mean revenue and $\mu$ is the mean return.
\end{itemize}

}
\frame{\frametitle{Parametric VaR}
\begin{itemize}
\item<1-> Now we assume that the return $R$ over the holding period has some density function $f$, then for a level of
confidence $\alpha$
$$
\prob(R <R^*)=\int_{-\infty}^{R^*}f(x)dx = 1-\alpha
$$
\item<2-> A typical assumption is that $R \sim N(\mu, \sigma)$. Then we can transform $R$ into a standard normal random variable $Z$
$$
\prob(R<R^*)=\prob(Z< (R^*-\mu)/\sigma)
$$
\item<3-> We can now use the quantiles $q_\alpha$ of a standard normal to compute VaR numbers (for example $q_{0.05}=-1.65$).
So $R^*= \mu + q_\alpha\sigma$ and
$$
VaR_{abs}=-\mu W-q_\alpha\sigma W \;\; \mbox{ and } \;\; VaR_{rel}= -q_{\alpha}\sigma W.
$$
\end{itemize}
}

\frame{\frametitle{Portfolio VaR I}
\begin{itemize}
\item<1-> Consider a portfolio with value $W$ consisting of two assets, 1 and 2,
with a relative amount $w_1$ held in asset 1 and a relative amount $w_2$ held in asset 2 (so $w_1+w_2=1$). 
\item<2->Asset returns have a joint normal distribution with mean $\mu_i$,
variance $\sigma^2_i$ for $i=1,2$, and correlation $\rho_{1,2}$.
\item<3->Asset $i$ has a value-at-risk of $VaR_\alpha^{(i)}=-q_{1-\alpha}
w_i\sigma_i W$ for $i=1,2$.
\end{itemize}
}

\begin{frame}[fragile]
\frametitle{Portfolio VaR II}
\begin{itemize}
  \item The variance of the portfolio is
  	\begin{align*}
  		\sigma_p^2= \left[w_1^2\sigma_1^2+ w_2^2\sigma_2^2+2
  		w_1w_2\rho_{1,2}\sigma_1\sigma_2\right]
	\end{align*}
  \item The VaR of the portfolio is
	\begin{align*}
  	\begin{array}{lll}
		VaR_p &=& - q_{1-\alpha}\sigma_p W \\
		  &=& -q_{1-\alpha} \sqrt{\left(w_1^2\sigma_1^2+w_2^2\sigma_2^2 +2
		    w_1w_2\rho_{1,2}\sigma_1\sigma_2\right)}W\\
		  &=& \sqrt{-q_{1-\alpha}^2\left( w_1^2\sigma_1^2+w_2^2\sigma_2^2 +2
		    w_1w_2\rho_{1,2}\sigma_1\sigma_2\right) W^2 )}\\
		  &=& \sqrt{VaR_1^2+ VaR_2^2+2 \rho_{1,2}VaR_1VaR_2}
	\end{array}
	\end{align*}
\end{itemize}
\end{frame}



\section{Expected Shortfall (ES) }
\frame{\frametitle{Expected Shortfall}
\begin{itemize}
\item<1-> The Expected Shortfall specifies the expectation of the loss in case the loss is above the VaR.
\item<2-> ES is also called Tail Value-at-Risk (TVaR) or Conditional
Value-at-Risk (CVaR).
\end{itemize}
}

\frame{\frametitle{Illustration of ES}
\begin{figure}
	\centering
		\includegraphics[width=1\textwidth]{../../../pics/Expected_Shortfall}
	\label{fig:Expected_Shortfall}
\end{figure}
}

\frame{\frametitle{Expected Shortfall -- Definition}
\begin{itemize}
\item<1->
For a loss $X$ with  cdf $F_X$ the expected
shortfall at confidence level $\alpha\in(0,1)$ is defined as
$$
\EX S_{\alpha}=\frac{1}{\alpha}\int_0^{\alpha} q_u(F_X)du
=\frac{1}{\alpha}\int_0^{\alpha} VaR_u(X)du
$$
where
$q_u(F_X)=F_X^{\leftarrow}(u)$ is the quantile function of $F_X$.
\item<2-> For an integrable loss $X$ with continuous cdf $F_X$ and any
$\alpha\in(0,1)$ we have
$$
\EX S_{\alpha}=\frac{\EX(X; X\leq
q_{\alpha}(X))}{\alpha}=\EX(X|X\leq VaR_{\alpha})
$$
(where we have used the notation $\EX(X; A)=\EX(X \IF_A)$ for $X$ a
rv, $A\in\cal F $)
\end{itemize}
}



\section{Calculating Risk Measures}
\subsection{Classification of Computational Approaches}
\frame{\frametitle{VaR and ES Computation}
% There are many approaches of how to calculate the VaR/ES. For all approaches,
% the risk factors have to be identified first. The most common approaches of
% calculating the VaR/ES are:
\begin{itemize}
\item<1-> Historical simulation: Estimate the VaR/ES from historical data.
 Take the return of the last $k$ trading days and estimate the VaR as the $\alpha \cdot k$-th highest loss. The ES can be approximated by the average of the losses larger or equal to the $\alpha \cdot k$-th highest loss.
\item<2-> Analytical method: Assume a distribution for market prices and derive an analytical formula for the VaR and the ES by expansion of the portfolio value in terms of the driving risk factors.
\item<3-> Structured Monte Carlo method: Assume a stochastic model for the risk factors, simulate $M$ sample paths and determine the portfolio value on those paths. The  $M \cdot \alpha$-th lowest portfolio value gives an estimate of the VaR the ES can be obtained by averaging.
\end{itemize}
}

\subsection{Example: DAX}
\frame{\frametitle{VaR and CVaR - historical method}
\begin{itemize}
\item<1-> We calculate  VaR and CVaR for DAX returns in 2011 and 2012.
\item<2-> So we have  512 trading days and  511 daily log-returns.
\end{itemize}
}

\frame{\frametitle{VaR and CVaR - historical method}
We get the following histogram for the log-returns:
\begin{figure}
		\centering
		\includegraphics[width=1.00\textwidth]{../../../pics/Histogram}
\end{figure}
}


\frame{\frametitle{VaR and CVaR - historical method}
The worst log-returns are:

\begin{tabular}{r|rr|rr|r}

         0 &   -6.00\% &           9 &   -3,46\% &	  18 &   -3,02\% \\  

         1 &   -5.99\% &          10 &   -3,46\% &   	  19 &   -2,90\% \\

         2 &   -5,42\% &          11 &   -3,42\% &        20 &   -2,88\% \\

         3 &   -5,27\% &          12 &   -3,42\% &        21 &   -2,87\% \\

         4 &   -5,15\% &          13 &   -3,41\% &        22 &   -2,82\% \\

         5 &   -5,13\% &          14 &   -3,41\% &        23 &   -2,76\% \\

         6 &   -5,09\% &          15 &   -3,28\% &        24 &   -2,52\% \\

         7 &   -4,12\% &          16 &   -3,24\% &        25 &   -2,52\% \\
         
         8 &   -3,48\% & 	  17 &   -3,23\% &	  26 &   -2,51\% \\

\end{tabular}\\
Thus $VaR_{99\%} = 5.12\%$ (by Interpolation between  5 and 6) and $CVaR_{99\%} = 5.34\%$. Observe, that  $CVaR_{99\%}$ is the mean of  $1\%$ (0-6)  of the worst results. }



\frame{\frametitle {VaR and CVaR - parametric method }
We estimate the parameters  $\mu$ and 
$\sigma$ of the DAX log-returns:
$$
\mu = 0,0002, \; \sigma = 0,0153
$$
So from 
$$
q_\alpha(X) = \EX(X) + q_{\alpha}\sigma(X)
$$
we find that
$$
VaR_{99\%} = 3,54\%.
$$
}
\frame{\frametitle{VaR and CVaR - parametric method}
From
$$
CVaR_\alpha(X) = -\EX[X|X \le q_\alpha (X)]
$$
we get the  parametric CVaR using the above parameters and the addition of a multiple of the standard deviation:\\
$$
CVaR_\alpha(X)=-\EX(X) + \frac{\varphi(q_{\alpha})}{1-\alpha}\sigma(X),
$$
So
$$
CVaR_{99\%} = 4,02\%.
$$
($\varphi(.)$ is the density of the standard normal distribution .) 
}


\frame{\frametitle{Backtesting VaR}
\begin{itemize}
\item<1-> To backtest VaR we compare the VaR prediction with the actual performance during a (past) time period. 
\item<2-> We simply count the number of VaR violations and compare them with the prediction. 
\item<3-> For a good VaR model the number of violations should be close to number which is implied by the quantile. In addition, the violations should be randomly distributed over the observation time period. 
\end{itemize}
}
\frame{\frametitle{Tests of the VaR model }
\begin{figure}
		\centering
		\includegraphics[width=1.00\textwidth]{../../../pics/VaR-exceedance-sjur.jpg}
\end{figure}
There exist standard tests for both backtesting approaches (Kupiec 1995, Christoffel 1998).

}


\subsection{General Principles for Risk Measures}
\frame{\frametitle{Coherent Risk Measures}

A coherent risk measure is a function $\rho$ on abstract risk positions such that
\begin{itemize}
\item<1-> \emph{Monotonicity}: if $X\geq 0$, then $\rho(X)\leq 0$;
\item<2-> \emph{Translation anti-variance}: if $k \in \R$, then
$\rho(X+k)=\rho(X)-k$;
\item<3-> \emph{Homogeneity}: if $\lambda \geq 0$ in $\R$, then $\rho(\lambda X)
= \lambda \rho (X)$;
\item<4-> \emph{Subadditivity}: $\rho(X+Y)\leq \rho(X)+\rho(Y)$.
\end{itemize}
What risk measures are coherent?
\begin{itemize}
  \item VaR is not a coherent risk measure as it is not subadditive in general.
  \item ES is a coherent risk measure.
\end{itemize}

}


