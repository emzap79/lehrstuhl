% !TEX root = riskmanagement_ws13UDE.tex
\part{Market Risk and Derivatives}

\section{Basic Derivatives}
 
Derivative Background
	A derivative security, or contingent claim, is a financial
	contract whose value at expiration date $T$ (more briefly, expiry)
	is determined exactly \index{contingent claim} by the price (or
	prices within a pre-specified time-interval) of the underlying
	financial assets (or instruments) at time $T$ (within the time
	interval $[0,T]$).

	Derivative securities can be grouped under three general headings:
	{\it Options, Forwards and Futures}, and {\it Swaps}.


Modelling Assumptions
We impose the following set of assumptions on the financial markets:
	{\it No market frictions: } No transaction costs, no bid/ask spread, no taxes, no margin requirements, no restrictions on short sales.
 
	{\it No default risk:} Implying same interest for borrowing and lending
	
	{\it Competitive markets:}  Market participants act as price takers
	
	{\it Rational agents:} Market participants prefer more to less.
	
	\emph{We assume that arbitrage opportunities do not exist!}


\subsection{Options}
Options
	An option is a financial instrument giving the holder the {\it right but
	not the obligation} to make a specified transaction at (or by)
	specified date(s) at a specified price ({\it the strike}).

	{\it Call} options give one the right to buy an asset (\emph{the
	underlying}).

	{\it Put} options give one the right to sell an asset.
	
	{\it European} options give the holder the right
	to buy/sell on the specified date, the expiry date, on which the option expires
	or matures.
	
	{\it American} options give the holder the right to buy/sell at any time
	prior to or at expiry.
	
	\emph{Bermudan} options give the holder the right to buy/sell at one of a
	number of specified days.


European Call Options
	Consider, say, a European call option, with strike price $K$;
	write $S(t)$ for the value (or price) of the underlying at time
	$t$.  If $S(t) > K$, the option is {\it in the money}, if $S(t) =
	K$, the option is said to be {\it at the money} and if $S(t) < K$,
	the option is {\it out of the money}.
 
	The payoff from the option is $$ S(T) - K \mbox{ if } S(T)
	> K\A \mbox{ and }\A 0 \;\; \mbox{otherwise} $$ (more briefly
	written as  $(S(T) - K)^+$).
 
	We obtain the payoff diagram on the next slide. 


\subsection{Forwards, Futures and Swaps}

Basic Structure
	A {\it forward contract}
	is an agreement to buy or sell an asset $S$ at a certain future
	date $T$ for a certain price $K$.

	The agent who agrees to buy the underlying asset is said to have a {\it long}
	position, the other agent assumes a {\it short} position.
 
	The settlement date is called {\it delivery date} and the specified
	price is referred to as {\it delivery price}.


Forwards
	The {\it forward price} $F(t,T)$ is the delivery price which would make the
	contract have zero value at time $t$.
	
	At the time the contract is set up, $t=0$, the forward price therefore 
	equals the delivery price, hence $F(0,T) = K$.

	The forward prices $F(t,T)$ need not (and will not)
	necessarily be equal to the delivery price $K$ during the
	life-time of the contract.


Forwards
	The payoff from a long position in a forward contract on one unit
	of an asset with price $S(T)$ at the maturity of the contract is
	$$ S(T)-K.$$
	
	Compared with a call option with the same maturity
	and strike price $K$ we see that the investor now faces a downside
	risk, too. He has the obligation to buy the asset for price $K$.


Value of a Forward Contract
	The forward price is in general \emph{not} equal to the value of a forward
	contract.
	
	The value of a forward contract with delivery price $K$ and expiration time $T$
	at $0\leq t \leq T$ is
		\begin{align*}
			f(t,S(t)) = S(t) - e^{-r(T-t)}K
		\end{align*}
	where we have assumed that $r$ is the constant risk-free rate of return.


Futures
	Futures can be defined as standardized forward contracts traded at
	exchanges where a clearing house acts as a central counterparty for all transactions.
	
	Usually an initial margin is paid as a guarantee.
	
	Each trading day a settlement price is determined and gains or losses are immediately realized at a margin account.
	
	Thus credit risk is (virtually) eliminated, but there is exposure to
	interest rate risk.


%\subsection{Swaps}

Swaps
A {\it swap} is an agreement whereby two parties
undertake to exchange, at known dates in the future, various
financial assets (or cash flows) according to a prearranged
formula that depends on the value of one or more underlying
assets. Examples are currency swaps (exchange currencies) and
interest-rate swaps (exchange of fixed for floating set of
interest payments).


\section{Option Pricing}
\subsection{Binomial Tree Models}

A fundamental example
	We consider a one-period model, i.e. we allow trading only at $t=0$ and $t=T=1$(say).
		Our aim is to value at $t=0$  a European contingent claim on a stock $S$
		with maturity $T$.
		
		The payoff $H$ of the instrument is a function $f$ of the stock price at
		$T$. For a European call option with strike $K$, we would have
		$H=f(S_T)=(S_T-K)^+$.
		
		We assume that stocks do not pay dividends.
		
		Investors can borrow and lend at the risk-free rate $r$.


Replication: The Black-Scholes-Merton Approach
	If we can find a \emph{replicating portfolio} of instruments with known prices
	that has the same payoff as the contingent claim $H$ in every state of the world, the value of the
	two positions has to be equal by the no arbitrage assumption.


An Example I
	We calculate the price of a European call option on a stock $S$ with strike
	$K=50$ and maturity in $T=1$ in a one-period model. The stock price today is
	$50$ and can move up or down by $10\%$.
	
	\center
	\includegraphics[height=3cm]{../../../pics/COpt1}


An Example II
	In this example, we use $r=1\%$. We try to replicate the option with
	investments of $x$ in the stock and $y$ in the risk-free bank account.
		\begin{align*}
			5 &= x \cdot 55 + y \cdot 1.01 \\
			0 &= x \cdot 45 + y \cdot 1.01 \\
			\Rightarrow  x &= 0.5, y= -22.277
		\end{align*}
	The value of our investment today, and, by no arbitrage, the value of the option
	today, is
		\begin{align*}
			V_0 = 0.5 \cdot 50 -22.277 \cdot 1 = 2.72
		\end{align*}


Risk-Neutral Valuation
	In the example above, we did not use any assumption about the risk
  preferences of investors and we did not need the probability with which the
  stock moves up or down.
  
	Idea: Pretend that investors are indifferent about risk. Then, every
  asset has to earn an expected return equal to the return of the risk-free
  investment (if there is no arbitrage). Find the probabilities for up- and
  down-movements of the stock in such a world and use them to find the price of
  the option.


An Example III
	In the setup from our example, the risk-free rate of return is $1\%$. Therefore,
		\begin{align*}
			1.01 &\stackrel{!}{=} \text{prob}_{up} \cdot 1.1 + \text{prob}_{down} \cdot 0.9 \\
			\Leftrightarrow 1.01 &\stackrel{!}{=} \text{prob}_{up} \cdot 1.1 + (1-\text{prob}_{up})
			\cdot 0.9 \\
			\Leftrightarrow \text{prob}_{up} &= 55\%
		\end{align*}
		
	We get the price of the option
		\begin{align*}
			V_0 = (0.55 \cdot 5 + (1-0.55)\cdot 0)/1.01 = 2.72
		\end{align*}


Risk-Neutral Valuation for Options
	The probabilities found in this way are called ``risk-neutral'' probabilities.
	They define a probability measure, the ``risk-neutral'' or ``pricing''
	measure.\\
	\vspace{0.5cm}

	The price $V(t)$ at time $t$ of an option with payoff $P(S_T)$ at time $T$ is
		the expectation 
		under the risk-neutral measure $\Q$
		of the discounted
		payoff.

	\begin{align*}
		V(t) = \uncover<2->{\E\uncover<3->{^\Q}\left[ \uncover<4->{e^{-r(T-t)}}
		\uncover<5->{P(S_T)}\right]}
	\end{align*}


The Cox-Ross-Rubinstein (CRR) Model}
	The Cox-Ross-Rubinstein model is a binomial tree model. The figure below shows a
	two-period tree, but it can easily be extended to multiple periods.
	%\usepackage{graphics} is needed for \includegraphics
	\begin{figure}[htp]
	\begin{center}
	\includegraphics[height=3.5cm]{../../../pics/CRR_tree}
	\end{center}
	\end{figure}


\subsection{The Black-Scholes-Merton Model}

The Black-Scholes-Merton Model
	In 1973, Black and Scholes (1973) and Merton (1973)
	developed the Black-Scholes (or Black-Scholes-Merton) model which was a major breakthrough in
	option valuation. It can be derived in different ways:
		As the limit of the CRR model;
		Via the Black-Scholes partial differential equation (PDE);
		Via risk-neutral pricing.

	For the second and third method, we need the additional assumption that stock
	prices follow a geometric Brownian motion with constant drift and volatility.
	This is not needed for the first method because there is an implicit
	distributional assumption in the CRR model.


The Black-Scholes Formula
	The time-$t$ price $c(t)$ of a European call option with strike $K$ and
	maturity $T$ on a non-dividend-paying stock $S$ with volatility $\sigma$ in the
	Black-Scholes model with risk-free interest rate $r$ is
		\begin{align*}
			c(t) &= S_t \Phi(d_1) - e^{-r(T-t)}K\Phi(d_2)\\
			d_1 &= \frac{\ln \left(\frac{S_t}{K} \right) +
			\left(r+\frac{\sigma^2}{2}\right)(T-t)}{\sigma\sqrt{T-t}}\\
			d_2 &= d_1 - \sigma \sqrt{T-t}
		\end{align*}
	where $\Phi(.)$ denotes the cumulative distribution function (cdf) of the standard
	normal distribution.


Example: Option Prices in the Black Scholes Model
	Find the price of a European call option in the BS model with strike $K=100$,
	time to maturity $1$ year, on a stock with time-$0$ price $100$. The risk free
	rate is $5\%$ p.a., the volatility of the stock is $20\%$ p.a.
	\begin{align*}
		d_1&=\frac{\ln \left(\frac{100}{100} \right) +
		\left(0.05+\frac{0.2^2}{2}\right)(1-0)}{0.2 \sqrt{1-0}}\\&=0.35\\
		d_2&=0.35-0.2*\sqrt{1-0} \\&= 0.15\\
		c(0)&=100 \Phi(0.35) - e^{-0.05\cdot 1}100\Phi(0.15)\\&=10.45.
	\end{align*}


Implied Volatility}
	To obtain a price from the Black-Scholes formula, we have to input a volatility.
	If we observe option prices on the market, we can compute the volatility that
	would give us the observed price, the so-called \emph{implied volatility}. The
	assumption in the BS model is that this implied volatility is constant for
	options on the same underlying. But we often observe the following:
		The implied volatility is decreasing with the strike in equity markets
		(volatility skew), U-shaped, e.g., in FX markets (volatility smile).
		
		The implied volatility changes with time to maturity (term structure of
		volatility).

	We can visualize both effects using a volatility surface.


Volatility Surface
\begin{center}
\includegraphics[height=6.5cm]{../../../pics/BBVol_SF_DAX}
\end{center}


Volatility Surface
\begin{center}
\includegraphics[height=6.5cm]{../../../pics/BBImpl_Vol_DAX}
\end{center}


%\subsection{Put-Call Parity}

The Put-Call Parity I
	The put-call parity is a relationship between European call (with price $c(t)$)
	and put (with price $p(t)$) options with the same strike, the same maturity, and
	with the same underlying security. Consider a portfolio consisting of a long
	call and a short put. The payoff at expiry is $S_T-K$.
	\begin{figure}\label{payofflcsp}
	\unitlength1cm \thicklines
	\begin{picture}(10,4.5)
	\put(1,2){\vector(1,0){5.5}} \put(6,1.5){$S(T)$} \put(3.5,1.5){$K$}
	\put(2,0.5){\vector(0,1){4}} \put(0.5,4){payoff}\color{red}
	\put(2,0.5){\line(1,1){3.5}}
	\end{picture}
	\end{figure}


Put-Call Parity II}
	This is the same payoff function as for a forward contract. Therefore the value
	of the portfolio consisting of a short put and a long call must be equal to the
	value of a forward contract for any $0\leq t \leq T$:
		\begin{align*}
			c(t) - p(t) = S(t) - e^{-r(T-t)}K
		\end{align*}
	where $r$ is the risk-free interest rate.


\section{Derivatives Risk Management}

%\subsection{Introduction}

Motivation
	A company that has a position in options, as a financial derivative or implicit
	in a contract, has to think about managing the risk of the position. The risk
	manager should ask the following questions:
			What are the risk drivers influencing the value of the position?
			How does the value of the position change if risk drivers change?
			How can you protect (i.e., hedge) yourself against those risk?

	We focus on answering the last two questions.


Some Terms
	A \emph{hedge} is a trade that reduces the risk of a position.
		A \emph{static} hedge is set up once and not touched until maturity of
		the derivative/position that is hedged.
		
		A \emph{dynamic} hedge is set up and adjusted continuously (in practice:
		regularly).
		
		A \emph{semi-static} hedge is set up and adjusted at infrequent
		intervals.

	A position for which no hedge is in place is called \emph{naked}, a position
	that is fully hedged is called \emph{covered} (e.g., sell a forward and buy the
	underlying stock).


A simple Hedging Strategy: Stop-Loss
	We consider a short position in one European call option with strike $K$. 
	A stop-loss hedge involves buying one share of the underlying stock when
  the price moves above $K$ and selling it again when the price falls below $K$.
  
	The strategy usually does not work well in practice because the trader
  cannot know whether the stock price moves above $K$ or falls below $K$ until
  it is too late. Therefore, he buys the stock at $K+\epsilon$ and sells at
  $K-\epsilon$, resulting in a loss.
   
	Also, discounting has to be taken into account.


\subsection{The Greeks}

The Greeks
	In order to quantify the risk associated with an instrument, one looks at
	\emph{ how much the price of the instrument changes if one of the underlying
	risk drivers changes} its value. Those risk measures are often called
	Greeks.
 
	Mathematically speaking, the Greeks are just the derivative of the price
	of the instrument with respect to the value of the risk driver.
	
	Once those Greeks are known for a portfolio, one can easily calculate how
	much the value of an option or a portfolio changes \emph{marginally}, if one of
	the variables changes \emph{marginally}, all others remain fixed.
	
	The most important Greeks are Delta, Gamma, Vega, Rho, and Theta.


The Delta
	Delta is the derivative of the instrument price with respect to the price
	of the underlying.
	
	The delta of the underlying security is one.
	
	If the payoff is not linear (for example if the instrument is an option),
	Delta is not constant.
	
	As Delta is the derivative with respect to the underlying, it tells how
	much the value of the instrument changes if the value of the underlying changes
	marginally.
	
	Thus in a continuous time model, Delta is the amount of the underlying
	needed to be sold in order to offset the price change of the instrument. The
	ability to neutralize the trading book with respect to price changes in the
	underlying makes Delta the most important Greek.


Delta Hedging I
	Assume that you have an options position with delta $\Delta$. This means
  that if the price of the underlying moves by a (very) small amount $\epsilon$, the
  price of the option position moves approximately by $\Delta \epsilon$.
  
	A \emph{delta hedge} consists of selling $\Delta$ units of the
  underlying (buying if $\Delta <0$) and gives a portfolio that does not
  change its value if the price of the underlying changes by a marginal amount.
  The portfolio has a delta of zero and is called \emph{delta neutral}.


Delta Hedging II
	BUT: Delta is not constant, so the portfolio has to be rebalanced.
	
	In the Black-Scholes model, a delta hedge with continuous rebalancing is
  a perfect hedge.
  
	In practice, continuous rebalancing is not possible so that the
  portfolio is not protected against larger market movements. Transaction
  costs and bid-ask spread also result in losses.

 
Delta in the Black-Scholes Model
	In the Black-Scholes model, the Greeks can be computed explicitly for
	European call and put options. We have:
		\begin{align*}
			\Delta = \frac{\partial}{\partial S}Call_{BS}(S,K,\sigma,r,t,T) = \Phi(d_1)
		\end{align*}
	where, as usual, $\Phi$ denotes the c.d.f. of the standard normal distribution and 
		\begin{align*}
			d_1 = \frac{\log \left( \frac{S}{K} \right) + \left( r + \frac{\sigma^2}{2}
			\right)(T-t)}{\sigma \sqrt{T-t}}.
		\end{align*}


Delta of a Call Option in the Black-Scholes Model
	%\usepackage{graphics} is needed for \includegraphics
	\begin{figure}[htp]
	\begin{center}
		\includegraphics[width=0.8\textwidth]{../../../pics/delta}
		\caption{Delta for a European call in the BS model, $T=1$, $r=1\%$,
		$\sigma=20\%$, $K=100$.}
		\label{fig:deltaBS}
	\end{center}
	\end{figure}


Example: Delta Hedge
	You are long USD $1,000$ in the $104$ call. Interest rate is $5\%$,
	stock price today is $99$, time to maturity $1$ month, and implied volatility is
	$15.7\%$. 
		How can you make your portfolio delta neutral by investing in the stock?
		
		You set up the delta neutral portfolio and the stock price jumps to
		USD$100$ immediately. What is your P/L for the portfolio?


Example: Delta Hedge
	Compute the price of the call with the BS formula:
		\begin{align*}
			Call_{BS} = 0.3858 \text{USD}.
		\end{align*}
	
	The position consists of $N=1,000/Call_{BS}=2592$ call options.
  
	The delta of each option is 
		\begin{align*}
			\Delta &= \Phi(d_1) =\Phi\left(\frac{\log \left( S/K \right) + (r+\sigma^2/2)(T-t)
			}{\sigma\sqrt{T-t}}\right) \\
				&= \Phi\left(\frac{\log \left( 99/104 \right) + (0.05+0.157^2/2)(1/12)
			}{0.157\sqrt{1/12}}\right)\\
				&= 0.1654.
		\end{align*}


Example: Delta Hedge
	The delta of the position is long $\Delta_P=N\cdot \Delta=428.70$.
   
	The stock has $\Delta=1$.
  
	To make the position delta neutral, you have to enter a short position
  of $428.70$ shares.

	The loss from the short position in the stock is $428.70 \cdot
  1=428.70$.
  
	To compute the gain from the long options position, we have to compute
  the option price for $S=100$. Using the BS formula, we obtain
		\begin{align*}
			Call_{BS}(S=100) = 0.5808.
		\end{align*}
	
	The gain from the options position is $2592\cdot(0.5808-0.3858)=505.37$.
  
	Our profit is $505.37-428.70=76.67$.


The Gamma
	Gamma is the second derivative of the instrument price with respect to
	the price of the underlying.
   
	If the instrument has a linear payoff, Gamma is zero.
   
	As delta is the first derivative of the option price with respect to the
  underlying, gamma is the derivative of delta with respect to the underlying
  and thus measures, how much Delta changes if the underlying changes.
   
	This is an important information in risk management as it tells the
  trader how much of the underlying he has to buy or sell if the underlying
  itself changes price.
   
	Geometrically, gamma might be seen as the slope of Delta.


Gamma Neutral Portfolios
	As the delta of a portfolio changes with the price of the underlying, a delta
	hedge has to be rebalanced frequently. 
		A delta hedge for a portfolio with a high (absolute) gamma has to be
		monitored more carefully than a delta hedge of a portfolio with gamma close to
		zero.
		
		To decrease the hedging error, a trader might want to make a delta
		neutral portfolio gamma neutral.
		
		The gamma cannot be changed by investing in the underlying because its
		gamma is zero.
		
		Strategy: Make the portfolio gamma neutral by investing in an option,
		then make it delta neutral by investing in the underlying.


Gamma in the Black-Scholes Model
	The gamma of a European call option in the Black-Scholes model is given by
		\begin{align*}
			\Gamma = \frac{\partial^2}{\partial S^2}Call_{BS}(S,K,\sigma,r,t,T) =
			\frac{\varphi(d_1)}{S\sigma \sqrt{T-t}}.
		\end{align*}
	Here, $\varphi(x)$ denotes the p.d.f. of the standard normal distribution.


Gamma of a Call Option in the Black-Scholes Model
	\begin{figure}[htp]
	\begin{center}
		\includegraphics[width=0.8\textwidth]{../../../pics/gamma}
		\caption{Gamma for a European call in the BS model, $T=1$, $r=1\%$,
		$\sigma=20\%$, $K=100$.}
		\label{fig:gammaBS}
	\end{center}
	\end{figure}


Gamma of a Call Option in the Black-Scholes Model
	\begin{figure}[htp]
	\begin{center}
		\includegraphics[width=0.8\textwidth]{../../../pics/gamma2}
		\caption{Gamma for a European call in the BS model, $T=10$, $r=1\%$,
		$\sigma=20\%$, $K=100$.}
		\label{fig:gamma2BS}
	\end{center}
	\end{figure}


Example: Gamma Neutral Portfolio
	You are in the same position as in the previous example. Additionally, you can
	trade in the $97$ call. 
		Put together a portfolio that is delta and gamma neutral.
		
		What is your P/L if the stock price jumps to $100$?}

		The gamma of the $104$ call is 
		\begin{align*}
		\Gamma_{104C} = \frac{\varphi(d_1)}{S\sigma \sqrt{T-t}} = 0.0554.
		\end{align*}
		
		The gamma of the $97$ call is 
		\begin{align*}
		\Gamma_{97C} = \frac{\varphi(d_1)}{S\sigma \sqrt{T-t}} = 0.0758.
		\end{align*}
		
		To make the portfolio gamma neutral, we have to build up a position of
		$n$ $97$ calls so that $2592\cdot 0.0554 +n\cdot 0.0758 = 0$.
		
		We find that $n=-1894.74$ makes the portfolio gamma neutral, i.e., we
		sell $1894.74$ units of the $97$ call.

		We compute the delta of the portfolio consisting of the two positions in
		the calls.
		
		The $97$ call has a delta of $\Delta_{97C}=0.7139$.
		
		The delta of the position is
			\begin{align*}
				\Delta_{P'} = 2592 \cdot 0.1654 - 1894.74 \cdot 0.7139 = -924.02
			\end{align*}
		We have to buy $924.02$ units of the underlying to make the portfolio
		delta neutral. By buying the underlying, we do not change the gamma of the
		portfolio, it remains zero.

		The price of the $97$ call for $S=99$ is $Call_{BS,97}(S=99)=3.2235$.
		
		The price of the $97$ call for $S=100$ is $Call_{BS,97}(S=100)=3.9735$.
		
		The P/L from the position in the $97$ call is
		$-1894.97*(3.9735-3.2235)=-1421.11$.
		
		The P/L from the position in the stock is $924.02$.
		
		The portfolio P/L is $924.02-1421.11+505.37=-8.28.$


The Vega
	Vega is the derivative of the instrument price with respect to
  implied volatility.
	
	Thus, vega indicates how much the option price changes if the implied
  volatility changes.
	
	Vega is not a Greek letter.



Vega in the Black-Scholes Model
	The vega of a European call option in the Black-Scholes model is given by
		\begin{align*}
			\nu = \frac{\partial}{\partial \sigma}Call_{BS}(S,K,\sigma,r,t,T) =
			S\sqrt{T-t} \varphi(d_1).
		\end{align*}

	Therefore, we have
		\begin{align*}
			\nu = S^2 \sigma (T-t) \Gamma.
		\end{align*}


Vega of a Call Option in the Black-Scholes Model II
	\begin{figure}[htp]
	\begin{center}
		\includegraphics[width=0.8\textwidth]{../../../pics/vega}
		\caption{Vega for a European call in the BS model, $T=1$, $r=1\%$,
		$\sigma=20\%$, $K=100$.}
		\label{fig:vegaBS}
	\end{center}
	\end{figure}


Vega of a Call Option in the Black-Scholes Model III}
	\begin{figure}[htp]
	\begin{center}
		\includegraphics[width=0.8\textwidth]{../../../pics/vega_impliedvol}
		\caption{Vega for a European call in the BS model, $T=1$, $r=1\%$, $K=100$.}
		\label{fig:vega2BS}
	\end{center}
	\end{figure}


The Theta
	Theta is the derivative of the instrument price with respect to
  time to maturity.
  
	Thus, theta indicates how much the option price changes as time moves
  closer to maturity.
  
	Theta is usually negative and therefore is also called \emph{time decay}
  or \emph{rent}.
   
	Note: The passage of time is deterministic. It does not make sense to
  hedge against these losses. 


Theta of a Call Option in the Black-Scholes Model}
	\begin{figure}[htp]
	\begin{center}
		\includegraphics[width=0.8\textwidth]{../../../pics/theta}
		\caption{Theta for a European call in the BS model, $\sigma=20\%$, $r=1\%$,
		$K=100$.}
		\label{fig:thetaBS}
	\end{center}
	\end{figure}


\subsection{Applications}
\subsubsection{Portfolio Insurance}

Portfolio Insurance
	Portfolio managers are often interested in acquiring put options on
	their portfolio. This protects them from losses while retaining the upside
	potential. This is especially important if fixed targets have to be met as,
	e.g., for guaranteed pensions.
   
	Such options are usually not traded because each portfolio manager has
  her individual portfolio. 
   
	To achieve his goal, the portfolio manager can use options on market
  indices to approximate the option on the portfolio, or he can create the
  option synthetically by replication. The latter technique is called portfolio
  insurance in this context.


Creating a Put Option
	A put option on a portfolio can be created synthetically by
		calculating the delta $\Delta$ of a put on the portfolio in the BS model;
		
		selling a proportion of $\vert \Delta \vert$ of the original portfolio;
		
		repurchasing the original portfolio so that it again matches the
		portfolio delta after an increase of the portfolio value;
		
		selling additional parts of the original portfolio after a decrease in
		portfolio value to match the new delta.


Portfolio Insurance: Problems
	The portfolio manager buys after prices have increased and sells after
  prices have declined.
   
	Frequent monitoring is necessary to obtain a good replication.
   
	Transaction costs and bis-ask spread generate losses.


Portfolio Insurance and the Black Monday 1987
	On Friday, Oct. 16, 1987, the DJIA fell by almost $5\%$, having declined
  significantly in the days before.
   
	Consequently, portfolio insurers had to sell large portions of their
  portfolios to maintain their hedge. 
   
	They could not complete the necessary transactions until the market
  closed on Friday afternoon.
   
	The additional sales were conducted on Monday, Oct. 19, and contributed
  to the largest one-day percentage decline in the history of the DJIA, nearly
  $23\%$.
   
	Other traders added to this by anticipating the actions of portfolio
  insurers.


\subsubsection{Volatility Hedge}

VDAX-NEW: Definition
	The volatility index VDAX-NEW was developed by Deutsche B{\"o}rse and Goldman Sachs. 
	It tracks the degree of fluctuation expected by the derivatives market, i.e. the implied 
	volatility, for the DAX index. The index expresses in percentage terms what degree of volatility 
	is to be expected for the following 30 days.
   
	VDAX-NEW started on 20 April 2005 and will replace VDAX in the medium-term.


VDAX-NEW: Examples
	Examples:\\
		A DAX of 4,000 and a VDAX-NEW of 10 indicate that the DAX stock index is expected to fluctuate 
		between 3,885 and 4,115 over the next thirty days:
		$$4000\pm4000\times0.1\times\sqrt{\frac{30}{365}}\approx4000\pm115.$$
		
		A DAX of 4,000 and a VDAX-NEW of 20 indicate that the DAX stock index is expected to fluctuate 
		between 3,770 and 4,230 over the next thirty days:
    $$4000\pm4000\times0.2\times\sqrt{\frac{30}{365}}\approx4000\pm230.$$


VDAX-NEW
	Volatility Index VDAX-NEW (in percentage points)
		$$\includegraphics[scale=0.6]{../../../pics/VDaxNew.png}$$


VDAX-NEW: Calculation
	Index is based on 8 sub-indices (option series) which include DAX Options from 2-24 months expiration.
  The main rolling index is calculated 30 days to expiration on linear interpolation of the two sub-indices closest to the 30 days expiration.
  In addition to at-the-money options (VDAX), out-of-the-money options are also considered (VDAX-NEW).
  Calculation frequency: Once a minute on every trading day at Eurex between 8.50am and 5.30pm CET.


VDAX-NEW: Features
	Expresses market expectation of the amplitude of fluctuation in DAX.
  Index is able to react only to changes in volatility.
	Allows better replication for derivatives and structured products.
  Due to ATM and OTM options VDAX-NEW captures more of the volatility skew than VDAX.
  Index establishes volatiliy as a tradable and separate asset class for investors.


VDAX-NEW: Use
	Investment/Trading: Speculation on future levels of volatility.
  Hedging: Cover short volatility positions.
  Diversification of German equity portfolios (correlation to DAX is -0.5689, correlation to MDAX is -0.4668).
  Benchmark for German equity volatility.


VDAX-NEW: Diversification
	Portfolio of 80\% DAX and 20\% VDAX-NEW between
	1992 and 2004 would have generated a 3.8 percentage
	points on average higher yield than investments in
	DAX only. And this with lower risk!

	$$\includegraphics[scale=0.45]{../../../pics/VDax_div.png}$$


\subsubsection{Calculating Portfolio VaR}

Delta-Gamma Approach
	The portfolio value can be written as a polynomial expansion in the market prices.
	
	The \textcolor{red}{Delta-Gamma} approach considers---additional to	
	the linear terms---the quadratic terms to calculate the changes in the value of
	the portfolio.
	
	We write the change in the portfolio value as:
		\begin{align*}
			\Delta V(F_1,F_2,\ldots,F_n)& \approx \sum_{i=1}^n \frac{\partial V}{\partial F_i} \Delta F_i \\
			&+ \frac 1 2 \sum_{i=1}^n \sum_{j=1}^n \frac{\partial^2 V}{\partial F_i \partial F_j} \Delta F_i \Delta F_j.
		\end{align*}

	This approach is adequate for non-linear positions, such as options, because their Gamma is also considered.
	
	Qualitatively a portfolio with a positive Gamma (a convex payoff structure) reduces its 
	market price sensitivity (i.e. its Delta position) in the case of decreasing market prices.
	
	If market prices are increasing there is a positive effect from an increasing Delta.
	
	A VaR calculated with the Delta-Gamma method is therefore too high. Vice versa, the 
	calculated VaR will be too low if the portfolio has a negative Gamma.


\section{Monte Carlo Simulation}
\subsection{Basic Concept}

Simulation
	``\emph{simulation},  in industry, science, and education, a research or
	teaching technique that reproduces actual events and processes under test conditions.
	Developing a simulation is often a highly complex mathematical process.
	Initially a set of rules, relationships, and operating procedures are
	specified, along with other variables. The interaction of these phenomena
	create new situations, even new rules, which further evolve as the simulation
	proceeds. Simulation implements range from paper-and-pencil and board-game
	reproductions of situations to complex computer-aided interactive systems.''

	from Encyclop{\ae}dia Brittanica



Monte Carlo Simulation
	Monte Carlo simulation methods work as follows:
		\label{l:enum1} Randomly draw from a set of possible outcomes (the
		simulation).
		
		Repeat step \ref{l:enum1} multiple times and record the results.
		
		This provides us with a (empirical) probability distribution.
		
		We take this empirical distribution as an approximation of the
		underlying distribution and use it to compute quantities such as
		expectation, variance, and quantiles.
		
	Mathematically, this works because
		The law of large numbers ensures that our results get better as we
  	increase the number of random draws (i.e., the estimate converges).
		
		The central limit theorem gives us error estimates.



Example: Rolling Dice
	We have a random variable $X$ representing a fair 6-sided die. Of course,
		\begin{align*}
			\E[X] = \sum_{i=1}^6 \Pr(X=i)i=\frac{1}{6}\left( 1 + 2 + 3 + 4 + 5 + 6
			\right)= 3.5
		\end{align*}
	How can we get this result with MC-simulation?
		Roll the dice many times.
		Record the numbers rolled.
		Compute the mean of those numbers.


\subsection{Monte Carlo Methods in Risk Management}
\subsubsection{Calculating Option Prices}

Monte Carlo Methods in Finance: Motivation
	Why would we use Monte Carlo methods for option pricing if we have the
	Black-Scholes formula? There usually are no formulas for the prices of
		many exotic options or path dependent options.
		
		vanilla options in complex models that include, e.g., stochastic
		volatility and jumps.
		
		options that depend on multiple correlated assets.

	Monte Carlo methods work in all of these settings.


Option Pricing via Monte Carlo Methods I
	Recall: In the binomial option pricing model, we can find so
	called risk neutral probabilities for up- and down-movements of the underlying
	such that the  expectation of the discounted option payoffs gives us the price
	of the option. This generalizes to other, time-continuous models.

	For the time-$0$ price $V_0$ of a European option with payoff $f(S_T)$ at time
	$T$, we can write
		\begin{align*}
			V_0 = \E^\mathbb{\Q}[D_T f(S_T)]
		\end{align*}
	where $D_T$ is a discount factor and $E^\mathbb{\Q}$ denotes the expectation
	under the risk-neutral measure.



Option Pricing via Monte Carlo Methods II
	We can obtain a Monte Carlo estimate of the price as follows.
		Develop a model for the price process $S_t$ of the underlying security
		using risk-neutral probabilities.
		
		Generate realizations of $S_T$ (and the discount factor $D_T$ if it is
		stochastic).
		
		Compute the option-payoff for each realization.
		
		Estimate the expectation of the discounted cash flows as above. This is
		the price.

	The technique can be generalized easily to include path-dependent options.


Example: European Call in the Black-Scholes Model
	In the Black-Scholes model, the log-returns follow a normal distribution. Under
	the risk-neutral measure, we have
		\begin{align*}
		\log\left(\frac{S(t+\delta)}{S(t)} \right) \sim
			N\left(\left( r-\frac{\sigma^2}{2}\right)\delta,\sigma \sqrt{\delta}\right).
		\end{align*}

  Generate $n$ realizations of $\log(S_{T}/S_0)$ and calculate
  $p_i=(S_{T,i}-K)^+, i=1,\ldots,n$.
   
	Compute the mean payoff $\hat{p}_n=\sum_{i=1}^n p_i/n$.
   
	Then, $\hat{V}_0 = e^{-rT}\hat{p}_n$.

	\begin{figure}[htp]
	\begin{center}
		\includegraphics[width=0.9\textwidth]{../../../pics/BS_Call15K}
		\caption{MC simulation for a European Option in the Black-Scholes model.}
	\end{center}
	\end{figure}


Strengths and Weaknesses of Monte Carlo Methods
	[+] Very flexible method for pricing, works for path dependent options
  and with sophisticated models;
  
	[+] Does not suffer from ``curse of dimensions'';
  
	[+] Provides error estimates;
  
	[o\hspace{1px}] Implementation
  
	[\textendash\hspace{1px}] Computationally expensive, slow convergence;
  
	[\textendash\hspace{1px}] No ``exact'' prices;
  
	[\textendash\hspace{1px}] Numerical problems (discretization,\ldots).


\subsubsection{Simulating Portfolio Values}

Example: A Portfolio Based on Two Risk Factors
	From the property of normal log-returns, we can construct a discretized process
	of the stock price. If we want to evaluate the stock price at times $0 \leq t_1
	\leq \ldots \leq t_n=T$ with $t_{i+1}-t_i=\delta$ for $i=1,\ldots,n-1$:
		\begin{align*}
			S_{t+1} &= S_t \exp\left( \mu \delta + \sigma \sqrt{\delta} \epsilon_t
			\right)\\
			&\approx S_{t} + S_{t} \left( \mu \delta + \sigma
			\sqrt{\delta} \epsilon_{t} \right)
		\end{align*}
	where $\epsilon_t\sim N(0,1)$ and the $\epsilon_t$ are independent over time.


Example: A Portfolio Based on Two Risk Factors II
	If we want to model at the same time, we can generalize the model simply by
	setting $S_{t}=(S_{t,1},S_{t,2})$ and
		\begin{align*}
			S_{t+1,j} = S_{t,j} + S_{t,j} \left( \mu_j \delta + \sigma_j
			\sqrt{\delta} \epsilon_{t,j} \right), j=1,2.
		\end{align*}

	Stock returns are rarely independent. How can we select $\epsilon_{t,j}$ such
	that the correlation is $\rho$, that is
		\begin{align*}
		C= \Cor(S_{t}) = \left[ \begin{array}{cc}1 & \rho \\ \rho &1 
		\end{array}\right] ?
		\end{align*}


Example: A Portfolio Based on Two Risk Factors III
	Computer programs can generate realizations of random vectors
	$Z_t=(Z_{t,1},Z_{t,2})$ with independent standard normal components and
	independent over time.
  
	We want to use $Z_t$ to obtain correlated random numbers by using a
	transformation: find a matrix $A$ such that $\epsilon_t=AZ^T_t$ has two
	standard normal components with correlation $\rho$.
  
	We can do this with the Cholesky decomposition. 


Example: A Portfolio Based on Two Risk Factors IV}
	If $C$ is a positive semi-definite correlation matrix, we can write
		\begin{align*}
		  C = AA^T
		\end{align*} 
	
	where $A$ is a lower triangular matrix. In two dimensions we have:
		{\small \begin{align*}
			C = \left[\begin{array}{cc} 1& \rho \\
				\rho &1 \end{array}\right] &= \left[\begin{array}{cc} a_{11}& 0 \\
				a_{12} &a_{22} \end{array}\right] \cdot \left[\begin{array}{cc} a_{11}&
				a_{12} \\ 0 &a_{22} \end{array}\right]
			 &= \left[\begin{array}{cc} a^2_{11}&
			 a_{11}a_{12} \\ a_{11}a_{12} &a^2_{12}+a^2_{22} \end{array}\right].
		\end{align*}
		}
	
	We find that
		\begin{align*}
			a_{11} &= a^2_{11} = 1 \\
			a_{11} a_{12} &= \rho \\
			a^2_{12}+a^2_{22} &= 1
		\end{align*}
	
	so that $a_{12}=\rho$ and $a_{22}=\sqrt{1-\rho^2}$.


Example: A Portfolio Based on Two Risk Factors V
	We set
		\begin{align*}
			\left[\begin{array}{c} \epsilon_{t,1}\\
				\epsilon_{t,2} \end{array}\right] = A Z_t^T =
				\left[\begin{array}{cc} 1& 0 \\
				\rho &\sqrt{1-\rho^2} \end{array}\right]\left[\begin{array}{c} Z_{t,1} \\
				Z_{t,2} \end{array}\right] = \left[\begin{array}{c} Z_{t,1}\\
				\rho Z_{t,1} +\sqrt{1-\rho^2}Z_{t,2} \end{array}\right].
		\end{align*}
	
	Let us check that we did the right thing:
		$\epsilon_{t,1}=Z_{t,1}$ is a standard normal random variable.
		
		The sum of two independent normally distributed random variables has a
		normal distribution, so $\epsilon_{t,2}=\rho Z_{t,1} +\sqrt{1-\rho^2}Z_{t,2}$
		also has a normal distribution.
		
		We have
			\abovedisplayskip=2pt
				\begin{align*}
					\E[\rho Z_{t,1} +\sqrt{1-\rho^2}Z_{t,2}] &= \rho\E[Z_{t,1}] +
					\sqrt{1-\rho^2}\E[Z_{t,2}] = 0\\
				\intertext{and}
				\Var(\rho Z_{t,1} +\sqrt{1-\rho^2}Z_{t,2}) &= \rho^2\Var(z_{t,1}) + (\sqrt{1-\rho^2})^2
			\Var(z_{t,2})\\
			 &= \rho^2 + 1 -\rho^2 =1
			\end{align*}


Example: A Portfolio Based on Two Risk Factors VI
		Therefore $\epsilon_{t,2}$ has a normal distribution with mean $zero$ and
		variance $1$, i.e., a standard normal distribution.
		
		It remains to check the correlation:
			\begin{align*}
				\Cor(\epsilon_{t,1},\epsilon_{t,2}) &= \Cor\left(z_{t,1},\rho z_{t,1} +
				\sqrt{1-\rho^2}z_{t,2}\right)\\ 
				 &= \rho \Cor(z_{t,1},z_{t,1}) =\rho.
			\end{align*}

	This principle generalizes to dimensions $n>2$ although estimation of the
	covariance matrix becomes more difficult.


\subsubsection{Calculating Risk Measures}

Computing VaR and ES based on Simulations
	VaR and ES can be computed based on MC simulations as follows:
		Develop a model for the development of the $m$ risk factors
		$S=(S_1,\ldots,S_m)$ that influence the portfolio value using real-world
		probabilities.
		
		Generate paths $(S_t)_{0\leq t\leq T}\in \R^m$ of the risk factors
		from $t=0$ to up to the target horizon $T$.
		
		For each path, price the portfolio at $T$.
		
		Compute the risk measure from the empirical distribution.


Risk Management vs. Pricing
	Probability Measures
		To price options and derivatives, you have to model the underlying securities
		(and interest rates, exchange rates,\ldots) under a risk-neutral measure!

		To compute risk measures, you have to model the risk factors of your portfolio
		under the real-world measure!