% !TEX root = StructuringValuation_ws1314UDE.tex
\section{Power Plants}
\subsection{What is the value of a fossile power plant?}


%% Hier Strom alleine
Phelix Base 2002-12 
\begin{figure}[htp]
\centering
\includegraphics[width=0.9\textwidth, heigth=0.9\textheigth]{../../../pics/PhelixBase2002_12.pdf}
\end{figure}


%% Strom 2001-- 2008
Phelix Base 2002-2008 
\begin{figure}[htp]
\centering
\includegraphics[width=0.9\textwidth, heigth=0.9\textheigth]{../../../pics/PhelixBase2002_08.pdf}
%\captionPower, gas and carbon prices.
\end{figure}


%% Strom ab 2008
Phelix Base 2008-2012 
\begin{figure}[htp]
\centering
\includegraphics[width=0.9\textwidth, heigth=0.9\textheigth]{../../../pics/PhelixBase2008_12.pdf}
\end{figure}


%% Installed Photovoltaik
Photovoltaik 
\begin{figure}[htp]
\centering
\includegraphics[width=\textwidth]{../../../pics/PV-2012.pdf}
%\captionPower, gas and carbon prices.
\label{prices} 
\end{figure}


Wind 
\begin{figure}[htp]
\centering
\includegraphics[width=\textwidth]{../../../pics/Wind-2012.pdf}
%\captionPower, gas and carbon prices.
\label{prices} 
\end{figure}


Gas power plant 
\begin{figure}[htp]
\includegraphics[width=\textwidth]{../../../pics/GuD-Lingen}
\label{prices} 
\end{figure}


A day in august 
\begin{figure}[htp]
\includegraphics[width=\textwidth]{../../../pics/day-profile-august}
%\captionWind, Sonne und Strompreise
\end{figure}


Wind, sun and electricity 
\begin{figure}[htp]
\centering \includegraphics[width=\textwidth]{../../../pics/week1-14Nov.png}
%\captionWind, Sonne und Strompreise
\end{figure}


Value lost -- back on the envelope calculations 
	Installed capacity 876 MW 
	Variable costs ca. 60 Euro/MWh 
	
	Profitable hours per year 
		2010 (993), 
		2011 (2309), 
		2012 (737), 
	with an average profit 6.9 Euro per MWh. 

	Typical investment assumption 
		3500 profitable hours 
		10 Euro profit per MWh 
	loss per year 

	2010: (3500-993){*}876{*}10=21961320 Euro, 

	2011: (3500-2309){*}876{*}10 = 10433160 Euro, 

	2012: (3500-737){*}876{*}10= 24203880 Euro. 


%%%%%%%%%% Clean Spark Spread

Does it get better? 
\begin{figure}[htp]
\centering \includegraphics[width=\textwidth]{../../../pics/Spark-Spread-2012.pdf}
\end{figure}


RWE Response 14.August 2013 
\begin{figure}[htp]
\centering
\includegraphics[width=0.9\textwidth, heigth= 0.9 \textheigth]{../../../pics/RWE-Decommission}
%\caption{Wind, Sonne und Strompreise}
\end{figure}


\subsection{Clean Spread Options}
Clean Spark Spreads
	A gas power plant is long electricity, short gas and short carbon. 
	Financially, we can express that as the Clean Spark Spread 
		\begin{equation}
		V_{t}=\max\{P_{t}-h\, G_{t}-c_{E}\; E_{t}-C,0\},\label{spark_spread_value}
		\end{equation}
	here $P_{t}$ is the price of electricity, $G_{t}$ the price of gas,
	$E_{t}$ the price of emission certificates and $C$ fixed costs ($h$
	and $c_{E}$ are conversion factors). 


%\subsectionValuation of Spread Options
Valuation of Simple Spread Options
	For $K=0$ (exchange option) there is an analytic formula due to Margrabe (1978). 
	$$\begin{array}{lll}
	 C_{\mbox{spread}}(t) & = & (S_1(t)\Phi(d_1)-S_2(t)\Phi(d_2))
	 \\*[12pt]
	 P_{\mbox{spread}}(t) & = & (S_2(t)\Phi(-d_2)-S_1(t)\Phi(-d_1))
	 \\*[12pt]
	 \mbox{where}\quad d_1 & = & \frac{\log(S_1(t)/S_2(t))+\sigma^{2}(T-t)/2}{\sqrt{\sigma^{2}(T-t)}},\quad d_2=d_1-\sqrt{\sigma^{2}(T-t)}
	 \\*[12pt]
	 \mbox{and}\quad \sigma & = & \sqrt{\sigma_1^2-2\rho\sigma_1\sigma_2+\sigma_2^2}
	\end{array}$$
	where $\rho$ is the correlation between the two underlyings.

	For $K\neq 0$ no easy analytic formula is available.


Spread Option Value and Correlation 
	The value of a spread option depends strongly on the correlation between the two underlyings. 
	$$\tiny\text{$S_1=S_2=100$, $T=3$, $r=0.02$, $\sigma_1=0.6$, $\sigma_2=0.4$.}$$

	\includegraphics[scale=0.3]{../../../pics/corr}

	The higher the correlation between the two underlyings the lower is
	the volatility of the spread and hence the value of the spread option. 


Approximation by Kirk's Formula (2 Assets)
	Again for $r=0$ we have the formula $$\begin{array}{lll}
	 C_{\mbox{K2}}(t) & = & (S_1(t)\Phi(d_{1,K})-(S_2(t)+K)\Phi(d_{2,K}))
	 \\*[12pt]
	 \quad d_{1,K} & = & \frac{\log(S_1(t)/(S_2(t)+K))+\sigma_K^{2}(T-t)/2}{\sqrt{\sigma_K^{2}(T-t)}},\\*[12pt] 
		\quad d_{2,K} &=&d_{1,K}-\sqrt{\sigma_K^{2}(T-t)}
	 \\*[12pt]
	 \quad \sigma_K & = & \sqrt{\sigma_1^2-2b_1\rho\sigma_1\sigma_2+b_1^2\sigma_2^2}\\*[12pt]
	 \quad b_1 &=& \frac{S_2(t)}{S_2(t)+K}
	\end{array}$$
	and $\rho$ is the correlation between the two underlyings.


Approximation by Kirk's Formula (2 Assets)
	One can show that with $\tau=T-t$ 
	$$
	 C_{\mbox{K2}}(S_1(t), S_2(t), K, \tau) \approx
	 C_{\mbox{BS}}(S_1(t), S_2(t)+K, \sigma_K, \tau)
	 $$
	for  $\tau$ small. 


Approximation for three asset case 
	We consider the payoff 
	\begin{equation}
	\max\{S_{1}(T)-S_{2}(T)-S_{3}(T)-K,0\},\label{Three_asset_value}
	\end{equation}

	We value this as a classical vanilla option with random strike $S_{2}+S_{3}+K$
	and appropriate volatility. 


Approximation by Kirk's Formula (3 Assets)
	Again for $r=0$ we have for  $\tau$ small the formula 
	{\small
	\begin{equation}
	 C_{\mbox{K3}}(S_1(t), S_2(t), S_3(t), K, \tau) \approx
	 C_{\mbox{BS}}(S_1(t), S_2(t)+S_3(t)+K, \sigma_S, \tau)
	\label{kirk3}
	\end{equation}
	with
	 $$
	 \begin{array}{lll}
	 \sigma_S & = & \sqrt{\sigma_1^2+b_2^2\sigma_2^2 +b_3^2\sigma_3^2
	 - 2\rho_{12}\sigma_1\sigma_2b_2 - 2\rho_{13}\sigma_1\sigma_3b_1 + 2\rho_{23}\sigma_2\sigma_3b_2b_3}\\*[12pt]
	 b_2 &=& \frac{S_2(t)}{S_2(t)+S_3(t) K}
	 \;\;\mbox{and}  \;\;\
		b_3 = \frac{S_3(t)}{S_2(t)+S_3(t) K}
	\end{array}$$
	}
	and $\rho_{ij}$ is the correlation between the underlying $i,j$.


\subsection{Power Plant Valuation}
\subsubsection{Prices  Emission}
Certificates We model the emission price as a geometric Brownian motion

\begin{equation}
d{E}_{t}=\alpha^{E}\, E_{t}\, d{t}+\sigma^{E}\, E_{t}\, d{W}_{t}^{E},\label{co2}
\end{equation}


Gas Price 
	We model the gas price as a mean-reverting process 
		\begin{eqnarray}
		G_{t} & = & e^{g(t)+Z_{t}},\nonumber \\
		d{Z}_{t} & = & -\alpha^{G}\, Z_{t}\, d{t}+\sigma^{G}\, d{W}_{t}^{G},\label{gas}
		\end{eqnarray}

	$\alpha^{G}$ is the speed of mean-reversion for gas prices. 


Power Price 
	We model the power price as a sum of two mean-reverting processes
		\begin{eqnarray}
		P_{t} & = & e^{f(t)+X_{t}+Y_{t}},\nonumber \\
		d{X}_{t} & = & -\alpha^{P}\, X_{t}\, d{t}+\sigma^{P}\, d{W}_{t}^{P},\nonumber \\
		d{Y}_{t} & = & -\beta\, Y_{t}\, d{t}+J_{t}\, d{N}_{t},\label{power}
		\end{eqnarray}

	$\alpha^{P}$ and $\beta$ are speeds of mean-reversion for the smooth and the jump component of power prices. 

	$N$ is a Poisson process with intensity $\lambda$. 

	$J_{t}$ are independent identically distributed (i.i.d) random variables representing the jump size. 


Seasonal components 
	$g(t)$ and $f(t)$ are seasonal trend components for gas and power, respectively, defined as

	\begin{eqnarray}
	f(t) & = & a_{1}+a_{2}\, t+a_{3}\cos(a_{5}+2\pi t)+a_{4}\cos(a_{6}+4\pi t),\nonumber \\
	g(t) & = & b_{1}+b_{2}\, t+b_{3}\cos(b_{5}+2\pi t)+b_{4}\cos(b_{6}+4\pi t),\nonumber \\
	\label{grseasonality}
	\end{eqnarray}

	where $a_{1}$ and $b_{1}$ may be viewed as production expenses,
	$a_{2}$ and $b_{2}$ are the slopes of increase in these costs. The
	rest of the parameters are responsible for two seasonal changes in
	summer and winter respectively. 

	Dependence Structure In the current setting we also assume that
	$W^{E}$, $W^{G}$ and $N$ are mutually independent processes, but
	there is some correlation between $W^{P}$ and $W^{G}$

	\begin{equation}
	d{W}_{t}^{P}\, d{W}_{t}^{G}=\rho\, d{t}.\label{corr}
	\end{equation}


Data sources 
%\item All the data sets are taken from the European Energy Exchange, \textttwww.eex.com.
	Phelix Day Base: It is the average price of the hours 1 to 24 for
	electricity traded on the spot market. It is calculated for all calendar
	days of the year as the simple average of the auction prices for the
	hours 1 to 24 in the market area Germany/Austria. (EUR/MWh), 

	NCG: Delivery is possible at the virtual trading hub in the market
	areas of NetConnect Germany GmbH \& Co KG. daily price (EUR/MWh), 

	Emission certificate daily price: One EU emission allowance confers
	the right to emit one tonne of carbon dioxide or one tonne of carbon
	dioxide equivalent. (EUR/EUA). 

	We cover the last three years: 25.09.2009 - 08.06.2012. 


Price Paths, 25.09.2009 - 08.06.2012. 
	\begin{figure}[htp]
	\centering
	\includegraphics[width=\textwidth]{../../../pics/prices.pdf}
	%\caption{Power, gas and carbon prices.}
	\label{prices}
	\end{figure}}


Clean Spark Spread, 25.09.2009 - 08.06.2012. 
	\begin{figure}[htp]
	\centering
	\includegraphics[width=\textwidth]{../../../pics/spread.pdf}
	%\caption{Spark spread path.}
	\label{spread}


\subsubsection{Simple Valuation}
Real Option Approach 
	Here we value the flexibility of the power plant as well. In particular,
	the impact of the volatility of the underlying price process on the
	plant value becomes clear. 

	The approach allows to formulate optimisation problems which allow
	to maximise the value of the flexibility. 

	Flexibility adds value compared to the discounted cash flow approach. 


Real Option Approach -- Forward Based 
	A first step is to consider the flexibility within a period exactly
	as in traded forwards (e.g. flexibility only for a whole month), which
	need to be liquidly traded. 

	Then the power plant is just a sequence of call-options on the clean
	(spark) spread (as we assume that there are no further restrictions) 

	We can then use Kirk's approximation with the appropriate forward
	specification. 


Heath-Jarrow-Morton (HJM) Approach
	The Heath-Jarrow-Morton model uses the entire forward rate curve as
	(infinite-dimensional) state variable. The dynamics of the forward
	rates $F(t,T)$ are \foreignlanguage{english}{\textit{exogenously}
	given by 
	$$
	dF(t,T)=\alpha(t,T)dt+\s(t,T)dW(t).
	$$
	For any fixed maturity $T$, the initial condition of the stochastic
	differential equation is determined by the current value of the empirical
	(observed) forward rate for the future date $T$ which prevails at
	time $0$.}


Factor GBM Specification 
	\textbf{One-Factor GBM:} Here the volatility is 
	$$
	\sigma_1(t,T)=e^{-\kappa (T-t)}\sigma
	$$
	and
	$$
	dF(t,T)=F(t,T)\sigma_1(t,T)dW(t)
	$$

	\textbf{Two-Factor GBM Specification:} 
	ere the volatilities are 
	$$
	\sigma_1(t,T)=e^{-\kappa (T-t)}\sigma_1 \; \mbox{ and } \; \sigma_2>0
	$$
	and
	$$
	\frac{dF(t,T)}{F(t,T)}=\sigma_1(t,T)dW_1(t)+\sigma_2dW_2(t)
	$$


Modelling Approach 
	We use the HJM-framework to model the forward dynamics directly. 

	We distinguish between forward contracts with a fixed time delivery
	and forward contracts with a delivery period, called \emph{swaps}. 

	A typical lognormal dynamics of the swap price is, 
	\begin{equation}
	dF(t,T_{1},T_{2})=\Sigma(t,T_{1},T_{2})F(t,T_{1},T_{2})\, dW(t).\label{eqn: lognormal dynamics}
	\end{equation}
	
	The only parameter in this model is the volatility function $\Sigma$
	which has to capture all movements of the swap price and especially
	the time to maturity effect.

Volatility Functions 
	We assume that the swap price dynamics for
	all swaps is given by (\ref{eqn: lognormal dynamics}) where $\Sigma(t,T_{1},T_{2})$
	is a continuously differentiable and positive function.

	Starting out with a given volatility function for a fixed time forward
	contract the volatility function $\Sigma$ for the swap contract is
	given by 
	\begin{equation}
	\Sigma(t,T_{1},T_{2})=\int_{T_{1}}^{T_{2}}\hat{w}(u,T_{1},T_{2})\sigma(t,u)\, du.\label{eqn: swap volatility creation}
	\end{equation}


Forward Schwartz Volatility 
	For the related volatility function of the forward we obtain 
	\begin{equation}
	\sigma(t,u)=ae^{-b(u-t)}\label{vol-schwartz}
	\end{equation}
	where $a,b>0$ are constant. 

	The time to maturity effect is modeled by a negative exponential
	function. 

	When the time to maturity tends to infinity the volatility function
	converges to zero. 

	The exponential function causes that the volatility increases as the
	time to maturity decreases which leads to an increased volatility
	when the contract approaches the maturity. 


Swap Schwartz Volatility
	Applying this forward volatility to (\ref{eqn: swap volatility creation}) the swap volatility is: 
		\begin{align}
		\Sigma(t,T_{1},T_{2}) & =a\,\varphi(T_{1},T_{2})
		\end{align}
		where 
		\begin{align}
		\varphi(T_{1},T_{2})=\frac{e^{-b(T_{1}-t)}-e^{-b(T_{2}-t)}}{b(T_{2}-T_{1})}\label{volatility function varphi}
		\end{align}
	The Black-76 specification of the forward volatility can be obtained
	if $\varphi(T_{1},T_{2})=1$, that is $b=0$ in (\ref{vol-schwartz}).

	The associated swap price volatility is then given by $\Sigma(t,T_{1},T_{2})=a$.

  
A Two-Factor Model 
	For a fixed delivery start $T$ and delivery period 1 month, let
	the dynamics of a Forward $F_{t,T}$ be given by the two factor model:
		\begin{eqnarray*}
		F(t,T) & = & F(0,T)\exp\left\{ \mu(t,T)+\int_{0}^{t}\hat{\sigma_{1}}(s,T)dW_{s}^{(1)}+\sigma_{2}W_{t}^{(2)}\right\} 
		\end{eqnarray*}

	$W^{(1)}$ and $W^{(2)}$ independent Brownian motions 

	$\hat{\sigma_{1}}(s,T)=\sigma_{1}e^{-\kappa(T-s)}$ 

	$\sigma_{1}$, $\sigma_{2}$, $\kappa>0$ constants 

	$\mu(t,T)$ being the risk-neutral martingale drift 


Model Parameters
$\sigma_{1}$ affects the level at the short end of the volatility curve
	\begin{center}
	\includegraphics[height=6cm, width=10cm]{../../../pics/sigma1}
	\par\end{center}

 $\kappa$ affects the slope of the volatility curve at the short end
		\begin{center}
		\includegraphics[height=6cm, width=10cm]{../../../pics/kappa}
		\end{center}

 $\sigma_{2}$ affects the level at the long end of the volatility curve
	\begin{center}
	\includegraphics[height=6cm, width=10cm]{../../../pics/sigma2}
	\par\end{center}


\subsubsection{Simulation-based Valuation}

Present Value of a Power Plant 
	The operator acts on the spot market. The specific daily configuration
	of the power plant is not traded, so we use historical probabilities. 

	We don't consider any further restrictions. 

	The plant runs for another few years, so future values will be discounted. 


Spark Spread Analysis I
	In our investigation we will focus on the clean spark spread to model
	the value of virtual gas power plant. We will now use spot price processes
	in order to assess the day-by-day risk position of such a plant. Thus,
	we will model the daily profit (or loss) of a power plant as
		\begin{equation}
		V_{t}=\max\{P_{t}-h\, G_{t}-c_{E}\; E_{t},0\},\label{spark_spread_value}
		\end{equation}
	where $P_{t}$ is the power price, $G_{t}$ is the gas price, $E_{t}$
	is the carbon certificate price, $h$ is the heat rate, $c_{E}$ emission
	conversion rate. 


Spark Spread Analysis II 
	We compute the spark spread value $V_{t}$ given in (\ref{spark_spread_value})
	for every day $t$ for a time period of three years. 

	The formula for the total value then read 
		$$VPP(t,T) = \int_{t}^{T}e^{-r(s-t)}\,V(s)\,ds.$$

	Then, by fixing all the parameters except of one (e.g. correlation)
	and setting the shift value (e.g. 1\%), we compute shifted up and
	down spark spread values, i.e. $V_{t}^{up}$ and $V_{t}^{down}$,
	which we may use for sensitivity analysis. 


Power Plant Analysis I 
	We compute the value of the power plant
	(VPP) by means of Monte Carlo simulations. For a fixed large number
	$N$ and a fixed period $T=3$ years we have 
	$$VPP(t,T) = \frac{1}{N}\sum_{i=1}^{N}VPP_i(t,T),$$ where $$VPP_i(t,T) = \sum_{s=t}^{T}e^{-r(T-s)}\,V_i(s).$$


Power Plant Analysis II 
	We also compute shifted both up and down power plant values, i.e.
	$VPP^{up}(t,T)$ and $VPP^{down}(t,T)$ (i.e. w.r.t. shifted spark
	spread values) and calculate the sensitivity 
		$$sVPP(\theta_0) = \frac{VPP^{up}(t,T) - VPP^{down}(t,T)}{2 \cdot shift}.$$

	Finally, we compute the bid and ask prices, i.e. we use the closed
	formula for AVaR to get the risk-captured prices by subtracting and
	adding risk-adjustment value to $VPP(t,T)$ respectively. 

	For a specified significance level $\alpha\in(0,1)$ this risk-adjustment
	value is computed as 
		$$\frac{\varphi(\Phi^{-1}(\alpha))}{\alpha}\sqrt{\frac{sVPP(\theta_0)' \cdot \Sigma \cdot sVPP(\theta_0) }{N}}.$$


Parameter-risk implied bid-ask spread w.r.t. correlation parameter, Gaussian jumps. 
	\begin{columns}[t]
	\begin{column}[l]{0.6\textwidth}
	\includegraphics[width=\textwidth]{../../../pics/ba_prices_alpha_corr_normal_5000.pdf}
	\end{column}
	\begin{column}[r]{0.6\textwidth}
	\includegraphics[width=\textwidth]{../../../pics/ba_width_alpha_corr_normal_5000.pdf}
	\end{column}
	\end{columns}

Parameter-risk implied bid-ask spread w.r.t. the gas
	\begin{columns}[t]
	\begin{column}[l]{0.6\textwidth}
	\includegraphics[width=\textwidth]{../../../pics/ba_prices_alpha_gas_normal_5000.pdf}
	\end{column}
	\begin{column}[r]{0.6\textwidth}
	\includegraphics[width=\textwidth]{../../../pics/ba_width_alpha_gas_normal_5000.pdf}
	\end{column}
	\end{columns}


Estimation Procedures: Emissions and Gas 
	Apply a standard procedure to de-seasonalize gas (don't change notation). 

	$\log E_{t}$ and $\log G_{t}$ are normally distributed. 

	Thus, we can use standard Maximum Likelihood Methods. 


Estimation Procedures: Power I 
	The estimation procedure for the power price includes several steps: 

	Estimation of the seasonal trend and deseasonalisation. 

	With an iterative procedure we filter out returns with absolute values
	greater than three times the standard deviation of the returns of
	the series at the current iteration. The process is repeated until
	no further outliers can be found. 

	As a result we obtain a standard deviation of the jumps, $\sigma_{j}$,
	and a cumulative frequency of jumps, $l$. The latter is defined as
	the total number of filtered jumps divided by the annualised number
	of observations. 


Estimation Procedures: Power II 
	Once we have filtered the $X_{t}$ process, we can identify it as
	a first order autoregressive model in continuous time, i.e. so-called
	AR(1) process. Discretizing the process and estimating it by maximum
	likelihood method (MLE) yields the estimates. 

%\item To estimate the mean-reversion rate for the jump process one can take an advantage of the approach based on the autocorrelation function (ACF) suggested by Barndorff-Nielsen, %Shephard (2001) (implemented Benth, Nazarova, Kiesel 2011).




%%%%%%%%%%%%%%%%%%%%%%%%%% Energy Empirics %%%%%%%%%%%%%%%%%%%%%%%%%%%%%%%%%
\sectionEmpirical Analysis of Commodity Returns \subsectionMarginal Distributions
Marginal Distributions %\vspace-1cm
The analysis consists of following commodities:\\
 Power (Baseload), Brent Crude Oil, Coal, Natural Gas, CO$_{2}$ allowances

We highlight important features using some of them.

The commodities have a wide variety of different statistical properties,
e. g. 

\begin{center}
\begin{figure}
\includegraphics[width=0.45\textwidth,bb = 0 0 200 100, draft, type=eps]{../../../pics/qqoil.pdf}
\includegraphics[width=0.45\textwidth,bb = 0 0 200 100, draft, type=eps]{../../../pics/qqpower.pdf} 
\end{figure}

\par\end{center}

 Marginal Distributions -- Basic Statistics {\small{}}
\begin{table}[ht]
{\small{\vspace{0.5cm}
 }}{\small \par}

\centering{}{\small{}}%
\begin{tabular}{c|c|c|c|c}
 & {\small{Power2007}} & {\small{Brent2007}} & {\small{Coal2007}} & {\small{Carbon2007}}\tabularnewline
\hline 
{\small{Length}} & \multicolumn{4}{c}{{\small{- 314 -}}}\tabularnewline
{\small{Minimum}} & {\small{-0.070490}} & {\small{-0.037960}} & {\small{-0.034940}} & {\small{-0.336500}}\tabularnewline
{\small{Maximum}} & {\small{0.088390}} & {\small{0.059270}} & {\small{0.025550}} & {\small{0.569600}}\tabularnewline
{\small{Mean}} & {\small{0.001237}} & {\small{0.001497}} & {\small{0.000180}} & {\small{0.000094}}\tabularnewline
{\small{Median}} & {\small{0.001185}} & {\small{0.001427}} & {\small{0.000658}} & {\small{0.002574}}\tabularnewline
{\small{SE Mean}} & {\small{0.000721}} & {\small{0.000796}} & {\small{0.000551}} & {\small{0.003025}}\tabularnewline
{\small{Variance}} & {\small{0.000163}} & {\small{0.000199}} & {\small{0.000095}} & {\small{0.002873}}\tabularnewline
{\small{Skewness}} & {\small{-0.185179}} & {\small{0.212642}} & {\small{-0.332898}} & {\small{1.967920}}\tabularnewline
{\small{Excess kurtosis}} & {\small{11.680823}} & {\small{0.584996}} & {\small{0.452510}} & {\small{48.843905}}\tabularnewline
\end{tabular}{\small{\caption{{\small{Basic statistics of the 2007 log-return series}}}
}}
\end{table}
{\small \par}

 

Stylized facts %\vspace-1.5cm


\begin{center}
How non-normal are power prices?\\
 
\par\end{center}

Stylized facts %\vspace-1.5cm


\begin{center}
How non-normal are power prices?\\
 Skewness: 0.07 \hspace{2cm} Excess Kurtosis: 2.31\\
 \includegraphics[width=0.7\textwidth,height=0.7\textheight,bb = 0 0 200 100, draft, type=eps]{../../../pics/DChistogram.pdf} 
\par\end{center}

Stylized facts %\vspace-1.5cm


\begin{center}
How non-normal are power prices?\\
 Skewness: 1.95 \hspace{2cm} Excess Kurtosis: 25.93\\
 \includegraphics[width=0.7\textwidth,height=0.7\textheight,bb = 0 0 200 100, draft, type=eps]{../../../pics/powerhistogram.pdf} 
\par\end{center}



\subsectionMultivariate Parametric Modelling \subsubsectionRequirements for a Joint Distribution
Towards a multivariate distribution -- Requirements %\vspace-1.5cm


Distribution must reflect statistical properties, at least: 

non-normality 

peaked center, heavy tails 

skewness and excess kurtosis 

Distribution must be tractable, i. e. it should allow for: 

Conclusion of portfolio distribution from joint distribution (stable
under convolution) 

Computation of value-at-risk/expected shortfall 



Towards a multivariate distribution -- Requirements

Univariate analyses support the use of NIG (applied to spot power
by Benth \& Saltyte-Benth) and hyperbolic distributions (applied to
power futures by Eberlein \& Stahl, to oil and natural gas by Kat
\& Oomen), Multivariate distributional analysis are not available
thus far.\\


Suggestion: Use a multivariate extension of the distributions above,
i. e. multivariate generalized hyperbolic distributions (GH)

 \subsubsectionGeneralized Hyperbolic Distributions

GH-distributions GH-distribution has a representation as Normal
mean-variance mixture, i. e. $X\sim GH$ (multivariate) if 
$$
X\stackrel{d}{=}\mu+W\gamma+\sqrt{W}AZ
$$
where $Z$ is a $k$-dimensional standard Normal, $W$ Generalized
Inverse Gaussian random variable (mixture variable) independent of
$Z$, $A\in\mathbb{R}^{d\times k}$ and $\mu,\gamma\in\mathbb{R}^{d}$.

GH-distributions %\vspace-2cm
$$
X\stackrel{d}{=}\mu+W\gamma+\sqrt{W}AZ
$$
%\vspace-1cm


Moments of $X$: 

$\E[X]=\mu+\gamma\E[W]$ 

$\Var[X]=AA'\E[W]+\gamma\gamma'\Var[W]$\\[-1.5cm] 

$\mu$ controls mean value\\[-1.5cm] 

$\gamma$ controls skewness ($\gamma=0$: symmetric)\\[-1.5cm] 

$\Sigma:=AA'$ controls covariance\\[-1.5cm] 

$W\sim GIG(\lambda,\chi,\psi)$ controls tail-behaviour 

GH-distributions %\vspace-1.5cm
Further important properties are: %\vspace-0.5cm


GH-distribution has heavy tails. 

Density is explicitly known $\rightarrow$ Maximum-Likelihood estimation. 

Distribution is closed under linear combinations $\rightarrow$ Portfolio
is GH-distributed. 

GH-distribution is infinitely divisible $\rightarrow$ Levy-processes 

GH-distributions

Well-known special cases are: 

NIG ($\lambda=-1/2$, successful in power markets)\\[-1.5cm] 

Hyperbolic ($\lambda=1$, successful in power, gas and oil markets)\\[-1.5cm] 

$t$-distribution ($\lambda=-1/2\nu,\chi=\nu,\psi=0,\gamma=0,\nu>0$) 

 %\vspace-2cm


\begin{center}
\begin{figure}
\includegraphics[width=0.9\textwidth,bb = 0 0 200 100, draft, type=eps]{../../../pics/densitiespower.pdf}
%\includegraphics[width=0.49\textwidth]pics/logdensitiespower.pdf
\end{figure}

\par\end{center}

 %\vspace-2cm


\begin{center}
%\includegraphics[width=\textwidth]pics/densitiespower.pdf
\begin{figure}
\includegraphics[width=0.9\textwidth,bb = 0 0 200 100, draft, type=eps]{../../../pics/logdensitiespower.pdf} 
\end{figure}

\par\end{center}

 Further properties of GH-distributions -- Computation of VaR/ES
%\vspace-1.5cm
Once the joint distribution is specified, we want to analyze energy
portfolios, e. g. compute risk measures. We use the following setting: 

$X_{t+1}^{(i)}:=\frac{S_{t+1}^{(i)}-S_{t}^{(i)}}{S_{t}^{(i)}}$ denotes
relative returns of commodity $i$. 

$X:=-\sum\omega_{i}S_{t}^{(i)}X_{t+1}^{(i)}$ denotes loss of a portfolio
consisting of $\omega_{i}$ shares of commodity $i$ during the period
$[t,t+1]$. 

 Further properties of GH-distributions -- Computation of VaR/ES 

If $X_{t}^{(i)}$ are jointly GH-distributed, then $X$ is univariate
GH-distributed 

Compute value-at-risk (numerically) via $VaR_{\alpha}(X)=F_{X}^{-1}(\alpha)$ 

Expected shortfall is defined (for absolute continuous distributions)
via $ES_{\alpha}(X)=\frac{1}{1-\alpha}\int_{\alpha}^{1}VaR_{u}(X)du$.
In case of GH-distributions this can be computed explicitly: 
\begin{eqnarray*}
ES_{\alpha}(X) & = & \mu+\gamma\BE(W)+\sigma\frac{\varphi(\Phi^{-1}(\alpha))}{1-\alpha}\BE(\sqrt{W})
\end{eqnarray*}




\subsubsectionRisk Management Application Some statistical results
%\vspace-1.5cm
Results are reported for commodity futures/swaps (power, oil, gas,
CO$_{2}$) with delivery during 2007. We worked on time series with
1.5 years of daily prices.

Results are: %\vspace-0.5cm


Mardia's tests reject the use of multivariate normal distribution.\\[-1.3cm] 

Likelihood ratio tests prefer any of the alternative distributions
to the Normal\\[-1.3cm] 

Among all the tested alternatives, we recommend the NIG. 



Coal-fired Power Plant We consider a portfolio that can be viewed
roughly as the year production/consumption of a coal-fired power plant
that has a total output capacity of 1TW: 

$10^{6}$ power contracts (1 contract delivers 1MW during the entire
year), long \\[-1.5cm] 

330,000 t coal, short\\[-1.5cm] 

900,000 t CO$_{2}$, short 

Coal-fired Power Plant %\vspace-2cm
\begin{figure}
%\includegraphics[width=0.49\textwidth]pics/EScoal09ret.pdf


\centering{}\includegraphics[width=0.85\textwidth,bb = 0 0 200 100, draft, type=eps]{../../../pics/VARcoal09ret.pdf} 
\end{figure}


Coal-fired Power Plant %\vspace-2cm
\begin{figure}
\centering{}\includegraphics[width=0.85\textwidth,bb = 0 0 200 100, draft, type=eps]{../../../pics/EScoal09ret.pdf}
%\includegraphics[width=0.49\textwidth]pics/VARcoal09ret.pdf
\end{figure}


Coal-fired Power Plant %\vspace-1cm


\begin{center}
Normal VaR $>$ NIG VaR? 
\par\end{center}

%%\vspace-2cm
95\%-quantile: \hspace{2cm} 0.034 (Normal) \hspace{1cm}\\
 %\vspace-1cm


\begin{center}
\includegraphics[width=0.7\textwidth,height=0.7\textheight,bb = 0 0 200 100, draft, type=eps]{../../../pics/DChistogram.pdf} 
\par\end{center}

Coal-fired Power Plant %\vspace-1cm


\begin{center}
Normal VaR $>$ NIG VaR? 
\par\end{center}

%%\vspace-2cm
95\%-quantile: \hspace{1cm} 0.034 (Normal) \hspace{1cm}$<$ \hspace{1cm}
0.041 (NIG)\\
 %\vspace-1cm


\begin{center}
\includegraphics[width=0.7\textwidth,height=0.7\textheight,bb = 0 0 200 100, draft, type=eps]{../../../pics/DCNIG.pdf} 
\par\end{center}

Coal-fired Power Plant %\vspace-1cm


\begin{center}
Normal VaR $>$ NIG VaR? 
\par\end{center}

95\%-quantile: \hspace{1cm} 0.073 (Normal) \hspace{1cm} \\
 %\vspace-1cm


\begin{center}
\includegraphics[width=0.7\textwidth,height=0.7\textheight,bb = 0 0 200 100, draft, type=eps]{../../../pics/powerhistogram.pdf} 
\par\end{center}

Coal-fired Power Plant %\vspace-1cm


\begin{center}
Normal VaR $>$ NIG VaR? 
\par\end{center}

95\%-quantile: \hspace{1cm} 0.073 (Normal) \hspace{1cm}$>$ \hspace{1cm}
0.053 (NIG) \\
 %\vspace-1cm


\begin{center}
\includegraphics[width=0.7\textwidth,height=0.7\textheight,bb = 0 0 200 100, draft, type=eps]{../../../pics/powerNIG.pdf} 
\par\end{center}

 \subsubsectionDiscussion of multivariate GH-model %\subsectionMarginals
Shortcomings -- Marginals %\vspace-1.5cm


\begin{center}
QQplot using a fitted joint NIG 
\begin{figure}
\centering{}\includegraphics[width=0.4\textwidth,bb = 0 0 200 100, draft, type=eps]{../../../pics/marginalpower.pdf}
\includegraphics[width=0.4\textwidth,bb = 0 0 200 100, draft, type=eps]{../../../pics/marginalcoal.pdf} 
\end{figure}

\par\end{center}

 Shortcomings -- Marginals %\vspace-1.5cm


\begin{center}
QQplot using a fitted joint NIG 
\begin{figure}
\centering{}\includegraphics[width=0.4\textwidth,bb = 0 0 200 100, draft, type=eps]{../../../pics/marginaloil.pdf}
\includegraphics[width=0.4\textwidth,bb = 0 0 200 100, draft, type=eps]{../../../pics/marginalcarbon.pdf} 
\end{figure}

\par\end{center}



Shortcomings -- Marginals %\vspace-1.5cm
%\item Unterschiedliches Tail-Verhalten macht auch die implizite Kalibrierung an Optionen mit unterschiedlichen Lieferperioden/Underlyings instabil.


Same tail-behaviour in all components of 
$$
X\stackrel{d}{=}\mu+W\gamma+\sqrt{W}AZ
$$
since $W$ is the determinant for tail - and common to all marginals. 

If $X$ and $Y$ are two GH-distributed random variables driven by
different $W$s, we loose the convolution property ($X+Y$ is not
GH-distributed). 

Exception: NIG 

 \subsectionCopula-based Modelling Copula Approach Outline %\vspace-1cm


Estimation of marginal distribution of several commodity returns 

Estimation of several Copulas to obtain joint distributions 

Applications: \textbf{Risk Management} (VaR, ES, Portfolio Analysis) 



\subsubsectionMarginal Distributions Marginal Distributions

For each model 

the estimated parameters 

the Akaike Information Criterion (AIC), which is defined to be 
$$
AIC(M_{j})=-2log(L_{j}(\hat{\theta}_{j};X))+2k_{j}\ ,j\in\{1,\ldots,n\}
$$


the log-likelihood 

the p-value of a likelihood ratio test against the asymmetric generalized
hyperbolic model (which is the most general model we consider) 

are provided.



Marginal Distributions - Example: Power



{\tiny{}}
\begin{table}[ht]
{\tiny{\vspace{0.5cm}
 %\begincenter
 }}%
\begin{tabular}{c|c|c|c|c|c|c|c|c}
{\tiny{model}} & {\tiny{$\lambda$}} & {\tiny{$\overline{\alpha}$}} & {\tiny{$\mu$}} & {\tiny{$\sigma$}} & {\tiny{$\gamma$}} & {\tiny{AIC}} & {\tiny{log-likelihood}} & {\tiny{p-value}}\tabularnewline
\hline 
{\tiny{t}} & {\tiny{-1.227624}} & {\tiny{0}} & {\tiny{0.001529}} & {\tiny{0.015459}} & {\tiny{0}} & {\tiny{-1976.120}} & {\tiny{991.0599}} & {\tiny{0.793012}}\tabularnewline
{\tiny{NIG-s}} & {\tiny{-0.5}} & {\tiny{0.316160}} & {\tiny{0.001429}} & {\tiny{0.012340}} & {\tiny{0}} & {\tiny{-1974.458}} & {\tiny{990.2291}} & {\tiny{0.345514}}\tabularnewline
{\tiny{sk- t}} & {\tiny{-1.239572}} & {\tiny{0}} & {\tiny{0.001713}} & {\tiny{0.015189}} & {\tiny{-0.000614}} & {\tiny{-1974.355}} & {\tiny{991.1773}} & {\tiny{0.632315}}\tabularnewline
{\tiny{ghyp-s}} & {\tiny{-1.119172}} & {\tiny{0.129836}} & {\tiny{0.001511}} & {\tiny{0.012967}} & {\tiny{00}} & {\tiny{-1974.338}} & {\tiny{991.1689}} & {\tiny{0.620085}}\tabularnewline
{\tiny{NIG}} & {\tiny{-0.5}} & {\tiny{0.319912}} & {\tiny{0.001582}} & {\tiny{0.012314}} & {\tiny{-0.000346}} & {\tiny{-1972.594}} & {\tiny{990.2971}} & {\tiny{0.158394}}\tabularnewline
{\tiny{ghyp}} & {\tiny{-1.131927}} & {\tiny{0.130704}} & {\tiny{0.001714}} & {\tiny{0.012900}} & {\tiny{-0.000477}} & {\tiny{-1972.584}} & {\tiny{991.2918}} & {\tiny{NA}}\tabularnewline
{\tiny{hyp-s}} & {\tiny{1}} & {\tiny{0.000126}} & {\tiny{0.001184}} & {\tiny{0.011303}} & {\tiny{0}} & {\tiny{-1963.430}} & {\tiny{984.7149}} & {\tiny{0.001392}}\tabularnewline
{\tiny{hyp}} & {\tiny{1}} & {\tiny{0.000003}} & {\tiny{0.001174}} & {\tiny{0.011305}} & {\tiny{0.000057}} & {\tiny{-1961.439}} & {\tiny{984.7197}} & {\tiny{0.000288}}\tabularnewline
{\tiny{N }} & {\tiny{0}} & {\tiny{0}} & {\tiny{0.001237}} & {\tiny{0.000163}} & {\tiny{0}} & {\tiny{-1846.459}} & {\tiny{924.2295}} & {\tiny{0 }}\tabularnewline
\end{tabular}{\tiny{\caption{{\tiny{Marginal models for the Power2007 log-return series}}}
}}{\tiny \par}

{\tiny{%\endcenter
 }}
\end{table}
{\tiny \par}

 \subsubsectionCopulas Copula Modelling

Let $F_{i}(t)$ be a marginal return distribution of asset $i$.

One can use of a copula function $C$ to obtain the multivariate distribution
$$
F(t_{1},\ldots t_{n})=C(F_{1}(t_{1}),\ldots,F_{n}(t_{n})).
$$


Flexibility in the choice of copulae allows to consider several features
of the dependence structure.



 Facts on Copula Functions

A copula $C$ is a multivariate distribution with standard uniform
marginal distributions. That is $C$ is a mapping $[0,1]^{m}\rightarrow[0,1]$
with 

 $C(u_{1},\ldots,u_{m})$ is increasing in each component $u_{i}$ 

 $C(1,\ldots,1,u_{i},1,\ldots,1)=u_{i}$ for all $i\in\{1,\ldots,m\}$,
$u_{i}\in[0,1]$ 

 For all $(a_{1},\ldots,a_{m}),(b_{1},\ldots,b_{m})\in[0,1]^{m}$
with $a_{i}\leq b_{i}$ we have: 
$$
\sum_{i_{1}=1}^{2}\cdots\sum_{i_{d}=1}^{2}(-1)^{i_{1}+\ldots+i_{d}}C(u_{1,i_{1}},\ldots,u_{m,u_{m}})\geq0,
$$
where $u_{j,1}=a_{j}$ and $u_{j,2}=b_{j}$ for all $j\in\{1,\ldots,m\}$. 



 Facts on Copula Functions

Thus a copula function is a multivariate distribution function such
that its marginal distributions are standard uniform. We use the notation
$$
C:[0,1]^{m}\rightarrow[0,1],\A(u_{1},\ldots,u_{m})\rightarrow C(u_{1},\ldots,u_{m}).
$$


If $\bv{X}=(X_{1},\ldots,X_{m})'$ has joint distribution $F$ with
continuous marginals $F_{1},\ldots,F_{m}$, then the distribution
function of the transformed vector 
$$
(F_{1}(X_{1}),\ldots,F_{m}(X_{m}))
$$
is a copula $C$, and 
$$
F(x_{1},\ldots,x_{m})=C(F_{1}(x_{1}),\ldots,F_{m}(x_{m})).
$$




 Examples of Copula Functions

The Gaussian copula is given by: 
$$
C_{\Phi,\Gamma}^{m}(\bv{u})=\bv{\Phi}_{\Gamma}(\Phi^{-1}(u_{1}),\ldots,\Phi^{-1}(u_{m})),
$$
where $\bv{\Phi}_{\Gamma}$ denotes the multivariate Gaussian distribution
function with linear correlation matrix $\Gamma$.

The $t$-copula can be expressed by: 
$$
C_{t,\nu,\Gamma}^{m}(\bv{u})=t_{\nu,\Gamma}^{m}(t_{\nu}^{-1}(u_{1}),\ldots,t_{\nu}^{-1}(u_{m})),
$$
where $t_{\nu,\Gamma}^{m}$ denotes the distribution function of an
m-variate $t$-distributed random vector with parameter $\nu>2$ and
linear correlation matrix $\Gamma$. Furthermore, $t_{\nu}^{-1}$
is the inverse of the univariate $t$-distribution function with parameter
$\nu$.



Examples of Copulas

\begin{center}
\begin{figure}
\includegraphics[width=0.6\textwidth,bb = 0 0 200 100, draft, type=eps]{../../../pics/copula-density.pdf} 
\end{figure}

\par\end{center}



Archimedean copula

An Archimedean copula function $C:[0,1]^{I}\rightarrow[0,1]$ is a
copula function which can be represented in the following form 
$$
C(\bv{x})=\varphi^{-1}\left(\sum_{i=1}^{I}\varphi(x_{i})\right)
$$
with a suitable function $\varphi:[0,1]\rightarrow\mathbb{R}_{t}$
with $\varphi(0)=\infty,\varphi(1)=0$. The function $\varphi$ is
called the generator of the copula.

Examples of this class are 

Clayton: $\varphi(t)=(t^{-\theta}-1);\varphi^{-1}(s)=(1+s)^{-1/\theta},\;\theta\geq0$.\\
 Lower tail dependency, no upper tail dependency. 

Gumbel: $\varphi(t)=(-\log t)^{\theta},\varphi^{-1}(s)=e^{-s1/\theta},\;\theta\geq1$.\\
 Upper tail dependency, no lower tail dependency. 



 Copula Examples

\includegraphics[bb = 0 0 200 100, draft, type=eps]{<}1>{[}height=6cm{]}../../../pics/copulaexamples.pdf



Copulas

We fit 5 copula models (Normal, t, Frank, Gumbel, Clayton) to the
data. 

Fitting is done via the IFM (Inference for margins) method, which
means we fit the marginal models separately, employ the quantile transformation
on each margin and then fit the copula on this transformed dataset. 

For each margin we choose the best (ranked by its AIC) generalized
hyperbolic model as seen in the tables before. 

We rank the copulas by their AIC values.



Copula fit

\begin{table}[ht]
\vspace{0.5cm}
 

\centering{}%
\begin{tabular}{c|c|c|c}
name & %
\parbox[c]{3cm}{%
\centering number of%
} &  & \tabularnewline
fitted parameters & AIC & log-likelihood & \tabularnewline
\hline 
t & 7 & -229.780909 & 121.890455\tabularnewline
normal & 6 & -225.226240 & 118.613120\tabularnewline
clayton & 1 & -92.833737 & 47.416868\tabularnewline
frank & 1 & -67.382373 & 34.691187\tabularnewline
gumbel & 1 & -62.799001 & 32.399500\tabularnewline
\end{tabular}\caption{Copula models for the year 2007 log-return series}
\end{table}




Copula fit -- Results

t and normal copula provide the best fit, because 

they have way more parameters to capture the dependence structure 

they have more flexibility to fit the dependence between the margins
of our models. 



\subsubsectionRisk Management Application Coal-fired Power Plant

We again compute the Value at Risk (VaR) and the one-day Expected
Shortfall (ES).

Now we need to simulate the loss distribution of the portfolio

Generate $N:=100000$ realizations of our multivariate copula based
model; i.e. $N$ realizations of the estimated copula. 

Employ the quantile transform for each margin to get $S_{t}^{(i)}$. 

Transform these samples into realizations of our portfolio loss distribution.

 Coal-fired Power Plant

We consider a coal-fired power plant with an output of $1000000$
MWh of electricity. To produce this amount of electricity one needs
$330000$ tons of coal and as a byproduct the firing process generates
$900000$ tons of CO$_{2}$. So our coal-fired power plant is nothing
else than the following portfolio: 

a long position of $1000000$ MWh of electricity, 

a short position of $330000$ tons of coal, 

and a short position of $900000$ tons of CO$_{2}$. 

 Coal-fired Power Plant -- Margins {\tiny{}}
\begin{table}[ht]
\centering{}{\tiny{}}%
\begin{tabular}{c|c|c|c|c|c|c|c|c}
{\tiny{model}} & {\tiny{$\lambda$}} & {\tiny{$\overline{\alpha}$}} & {\tiny{$\mu$}} & {\tiny{$\sigma$}} & {\tiny{$\gamma$}} & {\tiny{AIC}} & {\tiny{log-likelihood}} & {\tiny{p-value}}\tabularnewline
\hline 
 & \multicolumn{8}{c}{{\tiny{Power2007}}}\tabularnewline
\hline 
{\tiny{t}} & {\tiny{-1.203832}} & {\tiny{0}} & {\tiny{0.001321}} & {\tiny{0.014477}} & {\tiny{0}} & {\tiny{-2561.138}} & {\tiny{1283.569}} & {\tiny{0.843597}}\tabularnewline
\hline 
 & \multicolumn{8}{c}{{\tiny{Coal2007}}}\tabularnewline
\hline 
{\tiny{hyp}} & {\tiny{1}} & {\tiny{3.139811}} & {\tiny{0.003589}} & {\tiny{0.009880}} & {\tiny{-0.003154}} & {\tiny{-2510.830}} & {\tiny{1259.415}} & {\tiny{0.715009}}\tabularnewline
\hline 
 & \multicolumn{8}{c}{{\tiny{Carbon2007}}}\tabularnewline
\hline 
{\tiny{t}} & {\tiny{-1.007697}} & {\tiny{0}} & {\tiny{0.003084}} & {\tiny{0.193450}} & {\tiny{0}} & {\tiny{-1655.236}} & {\tiny{830.6180}} & {\tiny{0.719724}}\tabularnewline
\hline 
 & \multicolumn{8}{c}{{\tiny{Power2008}}}\tabularnewline
\hline 
{\tiny{t}} & {\tiny{-1.535741}} & {\tiny{0}} & {\tiny{0.000637}} & {\tiny{0.009909}} & {\tiny{0}} & {\tiny{-3937.290}} & {\tiny{1971.645}} & {\tiny{0.782617}}\tabularnewline
\hline 
 & \multicolumn{8}{c}{{\tiny{Coal2008}}}\tabularnewline
\hline 
{\tiny{NIG}} & {\tiny{-0.5}} & {\tiny{4.107861}} & {\tiny{0.004122}} & {\tiny{0.009177}} & {\tiny{-0.003702}} & {\tiny{-3804.225}} & {\tiny{1906.112}} & {\tiny{1}}\tabularnewline
\hline 
 & \multicolumn{8}{c}{{\tiny{Carbon2008}}}\tabularnewline
\hline 
{\tiny{sk t}} & {\tiny{-1.481606}} & {\tiny{0}} & {\tiny{0.004414}} & {\tiny{0.032939}} & {\tiny{-0.004633}} & {\tiny{-2548.610}} & {\tiny{1278.305}} & {\tiny{0.352221}}\tabularnewline
\hline 
\end{tabular}{\tiny{\caption{{\tiny{Marginal models for the 3-variate coal-fired power plant data}}}
}}
\end{table}
{\tiny \par}

 Coal-fired Power Plant -- Copula {\small{}}
\begin{table}[ht]
\centering{}{\small{}}%
\begin{tabular}{c|c|c|c}
{\small{name}} & {\small{}}%
\parbox[c]{3cm}{%
\centering number of%
} &  & \tabularnewline
{\small{fitted parameters}} & {\small{AIC}} & {\small{log-likelihood}} & \tabularnewline
\hline 
 & \multicolumn{3}{c}{{\small{2007}}}\tabularnewline
\hline 
{\small{t}} & {\small{4}} & {\small{-238.455625}} & {\small{123.227812}}\tabularnewline
{\small{normal}} & {\small{3}} & {\small{-236.101537}} & {\small{121.050768}}\tabularnewline
{\small{gumbel}} & {\small{1}} & {\small{-115.395942}} & {\small{58.697971}}\tabularnewline
{\small{frank}} & {\small{1}} & {\small{-114.767269}} & {\small{58.383634}}\tabularnewline
{\small{clayton}} & {\small{1}} & {\small{-114.432476}} & {\small{58.216238}}\tabularnewline
\hline 
 & \multicolumn{3}{c}{{\small{2008}}}\tabularnewline
\hline 
{\small{normal}} & {\small{3}} & {\small{-308.304908}} & {\small{157.152454}}\tabularnewline
{\small{t}} & {\small{4}} & {\small{-306.578106}} & {\small{157.289053}}\tabularnewline
{\small{frank}} & {\small{1}} & {\small{-130.467471}} & {\small{66.233736}}\tabularnewline
{\small{clayton}} & {\small{1}} & {\small{-118.717110}} & {\small{60.358555}}\tabularnewline
{\small{gumbel}} & {\small{1}} & {\small{-99.941605}} & {\small{50.970803}}\tabularnewline
\end{tabular}{\small{\caption{{\small{Copula models for the 3-variate coal-fired power plant data}}}
}}
\end{table}
{\small \par}

 One Day Risk Measures (Mio EURO) 

{\tiny{}}
\begin{table}[!ht]
\centering{}{\tiny{}}%
\begin{tabular}{c|c|c|c|c|c|c|c|c|c|c}
 & {\tiny{t}} & {\tiny{normal}} & {\tiny{gumbel}} & {\tiny{clayton}} & {\tiny{frank}} & {\tiny{t}} & {\tiny{normal}} & {\tiny{gumbel}} & {\tiny{clayton}} & {\tiny{frank}}\tabularnewline
\hline 
 & \multicolumn{5}{c||}{{\tiny{2007, 0,9M t CO$_{2}$}}} & \multicolumn{5}{c}{{\tiny{2007, 0,1M t CO$_{2}$}}}\tabularnewline
\hline 
{\tiny{VaR$_{0.9}$}} & {\tiny{0.53}} & {\tiny{0.54}} & {\tiny{0.70}} & {\tiny{0.77}} & {\tiny{0.72}} & {\tiny{0.50}} & {\tiny{0.50}} & {\tiny{0.50}} & {\tiny{0.29}} & {\tiny{0.50}}\tabularnewline
{\tiny{VaR$_{0.95}$}} & {\tiny{0.77}} & {\tiny{0.77}} & {\tiny{1.01}} & {\tiny{1.13}} & {\tiny{1.07}} & {\tiny{0.75}} & {\tiny{0.74}} & {\tiny{0.76}} & {\tiny{0.44}} & {\tiny{0.76}}\tabularnewline
{\tiny{VaR$_{0.99}$}} & {\tiny{1.65}} & {\tiny{1.74}} & {\tiny{2.16}} & {\tiny{2.79}} & {\tiny{2.38}} & {\tiny{1.63}} & {\tiny{1.64}} & {\tiny{1.68}} & {\tiny{0.86}} & {\tiny{1.68}}\tabularnewline
{\tiny{ES$_{0.9}$}} & {\tiny{1.21}} & {\tiny{1.20}} & {\tiny{9.23}} & {\tiny{2.39}} & {\tiny{1.59}} & {\tiny{1.01}} & {\tiny{1.00}} & {\tiny{1.89}} & {\tiny{0.64}} & {\tiny{1.02}}\tabularnewline
{\tiny{ES$_{0.95}$}} & {\tiny{1.79}} & {\tiny{1.77}} & {\tiny{17.63}} & {\tiny{3.86}} & {\tiny{2.32}} & {\tiny{1.41}} & {\tiny{1.40}} & {\tiny{3.17}} & {\tiny{0.91}} & {\tiny{1.44}}\tabularnewline
{\tiny{ES$_{0.99}$}} & {\tiny{4.81}} & {\tiny{4.60}} & {\tiny{82.61}} & {\tiny{12.85}} & {\tiny{5.68}} & {\tiny{2.96}} & {\tiny{2.87}} & {\tiny{11.62}} & {\tiny{2.21}} & {\tiny{2.95}}\tabularnewline
\hline 
 & \multicolumn{5}{c||}{{\tiny{2008, 0,9M t CO$_{2}$}}} & \multicolumn{5}{c}{{\tiny{2008, 0,1M t CO$_{2}$}}}\tabularnewline
\hline 
{\tiny{VaR$_{0.9}$}} & {\tiny{0.53}} & {\tiny{0.53}} & {\tiny{0.71}} & {\tiny{0.79}} & {\tiny{0.72}} & {\tiny{0.49}} & {\tiny{0.49}} & {\tiny{0.49}} & {\tiny{0.28}} & {\tiny{0.48}}\tabularnewline
{\tiny{VaR$_{0.95}$}} & {\tiny{0.71}} & {\tiny{0.71}} & {\tiny{0.97}} & {\tiny{1.04}} & {\tiny{0.98}} & {\tiny{0.69}} & {\tiny{0.69}} & {\tiny{0.71}} & {\tiny{0.40}} & {\tiny{0.69}}\tabularnewline
{\tiny{VaR$_{0.99}$}} & {\tiny{1.18}} & {\tiny{1.20}} & {\tiny{1.65}} & {\tiny{1.73}} & {\tiny{1.73}} & {\tiny{1.28}} & {\tiny{1.29}} & {\tiny{1.35}} & {\tiny{0.67}} & {\tiny{1.35}}\tabularnewline
{\tiny{ES$_{0.9}$}} & {\tiny{0.83}} & {\tiny{0.84}} & {\tiny{1.13}} & {\tiny{1.20}} & {\tiny{1.16}} & {\tiny{0.84}} & {\tiny{0.84}} & {\tiny{0.87}} & {\tiny{0.46}} & {\tiny{0.86}}\tabularnewline
{\tiny{ES$_{0.95}$}} & {\tiny{1.04}} & {\tiny{1.06}} & {\tiny{1.43}} & {\tiny{1.50}} & {\tiny{1.50}} & {\tiny{1.11}} & {\tiny{1.10}} & {\tiny{1.15}} & {\tiny{0.59}} & {\tiny{1.15}}\tabularnewline
{\tiny{ES$_{0.99}$}} & {\tiny{1.72}} & {\tiny{1.74}} & {\tiny{2.35}} & {\tiny{2.38}} & {\tiny{2.52}} & {\tiny{1.99}} & {\tiny{1.95}} & {\tiny{2.06}} & {\tiny{0.92}} & {\tiny{2.11 }}\tabularnewline
\end{tabular}{\tiny{\caption{{\tiny{Risk measures for the coal-fired power plant (with CO$_{2}$)}}}
}}
\end{table}
{\tiny \par}

 Coal-fired Power Plant 
\begin{figure}
%\includegraphics[width=0.49\textwidth]pics/EScoal09ret.pdf


\centering{}\includegraphics[width=7cm,bb = 0 0 200 100, draft, type=eps]{../../../pics/VaRPowerCoalCarbon2007.pdf} 
\end{figure}


Coal-fired Power Plant %\vspace-2cm
\begin{figure}
\centering{}\includegraphics[width=7cm,bb = 0 0 200 100, draft, type=eps]{../../../pics/ESPowerCoalCarbon2007.pdf}
%\includegraphics[width=0.49\textwidth]pics/VARcoal09ret.pdf
\end{figure}




Comparison of Models 

 Copula-based models produce more accurate Risk Numbers 

 Marginal tails seem to be more important than joint tail behavior
(see copulas) 

 Parametric approach allows dynamic modelling 



