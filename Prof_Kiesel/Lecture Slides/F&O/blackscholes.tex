% !TEX root = FuturesOptions_ss13UDE.tex
\section{The Black-Scholes Model}

\begin{frame}
    \frametitle{}
\vspace{0cm}
\begin{center}
\color{beamer@blendedblue}{\large{The Black-Scholes Model}}
\end{center}
\end{frame}

\subsection{The Pricing Formula}

\frame{\frametitle{The Black-Scholes Formula for a European Call}
\begin{center}
\begin{figure}
  \centering
   \subfigure[Myron Scholes]{\includegraphics[height=3cm]{../../../pics/myron_scholes.jpg}}\qquad
   \subfigure[Robert C. Merton]{\includegraphics[height=3cm]{../../../pics/merton.jpg}}\qquad
   \subfigure[Fisher Black]{\includegraphics[height=3cm]{../../../pics/black.jpg}}
\end{figure}
\end{center}
}


%\frame{\frametitle{ Black-Scholes Model}

%\begin{itemize}
%  \item The classical Black-Scholes model is
%\begin{itemize}
%  \item $dB(t) = \DSE r B(t) dt, \A B(0)= 1$
%  \item $dS(t) = \DSE S(t) (b dt + \sigma dW(t) \!, \A &S(0) = p$
%\end{itemize}
%\item with constant coefficients $b \in \setR,\; r,\s \in \setR_+$.
%\end{itemize}
%}


\frame{\frametitle{European Call Price}

For a European call $X = (S(T)-K)^+$ and  we can evaluate the
above expected value

The Black-Scholes price
pro\-cess of a European call is given by
$$
\begin{array}{lll}
C(t) &=&\DSE S(t) \Phi(d_1(S(t), T-t))\\*[12pt]
&&- K e^{-r(T-t)} \Phi(d_2(S(t), T-t)).
\end{array}
$$
The functions $d_1(s,t)$ and $d_2(s,t)$ are given by
$$
\begin{array}{lll}
d_1(s,\tau) &=&\DSE \frac{\log(s/K) + (r +
\frac{\sigma^2}{2})\tau}{\sigma \sqrt{\tau}},\\*[12pt] d_2(s,\tau) &=&\DSE
 \frac{\log(s/K) + (r -
\frac{\sigma^2}{2})\tau}{\sigma \sqrt{\tau}}
\end{array}
$$
}

\subsection{The Greeks}
\frame{\frametitle{Greeks (1)}
\begin{itemize}
\item<1-> We will now analyse the impact of the
underlying parameters in the standard Black-Scholes model on the
prices of call and put options.
\item<2->
The Black-Scholes option values
depend on the
\begin{enumerate}
\item (current) stock price,
\item the volatility,
\item the time to maturity,
\item the interest rate
\item the strike price.
\end{enumerate}
\end{itemize}
}
\frame{\frametitle{Greeks (2)}
\begin{itemize}
\item<1-> The
sensitivities of the option price with respect to the first four
parameters are called the {\it Greeks} and are widely used for
hedging purposes.
\item<1->
We can determine the impact of these parameters
by taking partial derivatives.
\end{itemize}
}
\frame{\frametitle{Greeks}

$$
\begin{array}{lllll}
\Delta &:=&\DSE \frac{\partial C}{\partial S} &=&\DSE \Phi(d_1) >
0,\\*[12pt] 
{\cal V} &:=&\DSE \frac{\partial C}{\partial \s} &=&
\DSE S \sqrt{\tau} \varphi(d_1) >0,\\*[12pt]
 \Theta &:=&\DSE \frac{\partial
C}{\partial \tau} &=& \DSE \frac{S \s}{2 \sqrt{\tau}} \varphi(d_1) + K r
e^{-r\tau} \Phi(d_2) >0,\\*[12pt] 
\rho &:=&\DSE \frac{\partial
C}{\partial r} &=& \DSE \tau K e^{-r\tau} \Phi(d_2) >0,\\*[12pt] \Gamma
&:=&\DSE \frac{\partial^2 C}{\partial S^2} &=& \DSE
\frac{\varphi(d_1)}{S \s \sqrt{\tau}} >0.\\*[12pt]
\end{array}
$$
} \frame{\frametitle{Greeks} From the definitions it is clear that
\begin{itemize}
\item<1->
$\Delta$ -- delta -- measures the change in the value of the
option compared with the change in the value of the underlying
asset,
\item<2-> ${\cal V}$ -- vega -- measures the change of the option
compared with the change in the volatility of the underlying,
\item<3-> similar statements hold for $\Theta$ -- theta -- and $\rho$ -- rho
\end{itemize}
}
\frame{\frametitle{Greeks}
\includegraphics<1>[height=7cm]{../../../pics/Delta.pdf}

} \frame{\frametitle{Greeks}
\includegraphics<1>[height=7cm]{../../../pics/DeltaToZero.pdf}

} \frame{\frametitle{Greeks}
\includegraphics<1>[height=7cm]{../../../pics/gamma.pdf}

} \frame{\frametitle{Greeks}
\includegraphics<1>[height=7cm]{../../../pics/Vega2.pdf}

}


\frame{\frametitle{Greeks}
\begin{itemize}
\item<1->
The Black-Scholes partial differential equation can be used to
obtain the relation between the Greeks,
\item<2->
$$
r C = \frac{1}{2} s^2 \s^2 \Gamma +rs \Delta -\Theta.
$$
\end{itemize}
}

\frame{\frametitle{Vega}
\begin{itemize}
\item<1->
Before we can implement
the Black-Scholes formula to price options, we have to estimate
$\s$.
\item<2->
Because the formula is explicit, we can, determine the ${\cal V}$
-- the partial derivative
$$
{\cal V} = \partial C/\partial \s,
$$
finding
$$
{\cal V} = S \sqrt{T} \Phi(d_1).
$$
\item<3-> Note here is that vega is always positive.
\end{itemize}
}

\subsection{Volatility}
 \frame{\frametitle{Implied Volatility}
\begin{itemize}
\item<1->
Since vega is positive, $C$ is a continuous -- indeed,
differentiable -- strictly increasing function of $\s$.
\item<2-> Turning
this round, $\s$ is a continuous (differentiable) strictly
increasing function of $C$; indeed,
$$
{\cal V} = \frac{\partial C}{\partial \s}, \A \mbox{so} \A
\frac{1}{\cal V} = \frac{\partial \s}{\partial C}.
$$
\item<3->
Thus the value $\s = \s(C)$ corresponding to the actual value $C =
C(\s)$ at which call options are observed to be traded in the
market can be read off.  The value of $\s$ obtained in this way is
called the {\it implied volatility}.
\end{itemize}}
\frame{\frametitle{The Black-Scholes Implied Volatility}
\begin{itemize}
\item<1-> The Black-Scholes model assumes a constant volatility over all maturities and strikes.
\item<2-> One can calculate the implicit volatility given the market price of options.
\item<3-> The implicit volatility is changing over time and maturity, so contradicting the modelling assumptions.
\item<4-> However, it can be used to predict future volatility.
\item<5-> Furthermore, volatility can be made an own "asset class".
\end{itemize}

}

\frame{\frametitle{Volatility Smile}
Volatility Index VDAX-NEW (in percentage points)
\begin{center}
\includegraphics[scale=0.55]{../../../pics/Volatility-Smile-char.pdf}
\end{center}
}
\frame{\frametitle{Volatility Surface}
Volatility Index VDAX-NEW (in percentage points)
$$\includegraphics[scale=0.25]{../../../pics/VolatilitySurface.jpg}$$
}


\frame{\frametitle{VDAX-NEW: Definition}
\begin{itemize}
  \item The volatility index VDAX-NEW was developed by Deutsche B{\"o}rse and Goldman Sachs. It tracks the degree of fluctuation expected by the derivatives market, i.e. the implied volatility, for the DAX index. The index expresses in percentage terms what degree of volatility is to be expected for the following 30 days.
  \item VDAX-NEW started on 20 April 2005 and will replace VDAX in the medium-term.
\end{itemize}
}


\frame{\frametitle{VDAX-NEW: Examples}
Examples:\\
\vspace{0.3cm}
\begin{itemize}
  \item<1-> A DAX of 4,000 and a VDAX-NEW of 10 indicate that the DAX stock index is expected to fluctuate between 3,885 and 4,115 over the next thirty days:
  $$4000\pm4000\times0.1\times\sqrt{\frac{30}{365}}\approx4000\pm115.$$
  \item<2-> A DAX of 4,000 and a VDAX-NEW of 20 indicate that the DAX stock index is expected to fluctuate between 3,770 and 4,230 over the next thirty days:
    $$4000\pm4000\times0.2\times\sqrt{\frac{30}{365}}\approx4000\pm230.$$
\end{itemize}
}


\frame{\frametitle{VDAX-NEW}
Volatility Index VDAX-NEW (in percentage points)
$$\includegraphics[scale=0.6]{../../../pics/VDaxNew.png}$$
}


\frame{\frametitle{VDAX-NEW: Calculation}
\begin{itemize}
  \item <1-> Index is based on 8 sub-indices (option series) which include DAX Options from 2-24 months expiration.
  \item <2-> The main rolling index is calculated 30 days to expiration on linear interpolation of the two sub-indices closest to the 30 days expiration.
  \item <3-> In addition to at-the-money options (VDAX), out-of-the-money options are also considered (VDAX-NEW).
  \item <4-> Calculation frequency: Once a minute on every trading day at Eurex between 8.50am and 5.30pm CET.
\end{itemize}
}

\frame{\frametitle{VDAX-NEW: Features}
\begin{itemize}
  \item <1-> Expresses market expectation of the amplitude of fluctuation in DAX.
  \item <2-> Index is able to react only to changes in volatility.
  \item <3-> Allows better replication for derivatives and structured products.
  \item <4-> Due to ATM and OTM options VDAX-NEW captures more of the volatility skew than VDAX.
  \item <5-> Index establishes volatiliy as a tradable and separate asset class for investors.
\end{itemize}
}

\frame{\frametitle{VDAX-NEW: Use}
\begin{itemize}
  \item <1-> Investment/Trading: Speculation on future levels of volatility.
  \item <2-> Hedging: Cover short volatility positions.
  \item <3-> Diversification of German equity portfolios (correlation to DAX is -0.5689, correlation to MDAX is -0.4668).
  \item <4-> Benchmark for German equity volatility.
\end{itemize}
}

\frame{\frametitle{VDAX-NEW: Diversification}
\begin{itemize}
  \item Portfolio of 80\% DAX and 20\% VDAX-NEW between
1992 and 2004 would have generated a 3.8 percentage
points on average higher yield than investments in
DAX only. And this with lower risk!
$$\includegraphics[scale=0.45]{../../../pics/VDax_div.png}$$
\end{itemize}
}

