% !TEX root = FuturesOptions_ss14UDE.tex
\section{Interest Rates}
\subsection{Basic Interest Rates}

\frame{\frametitle{Basic Interest Rates}
\begin{itemize}
\item<1-> Economic agents have to be rewarded for postponing consumption; in addition, there is a risk premium for the uncertainty of the size of future consumption.
\item<2-> Investors, Firms, banks pay compensation for the willingness to postpone
\item<3-> A common interest rate (equilibrium) emerges which allows to fulfill the aggregate liquidity demand.
\end{itemize}
}

\frame{\frametitle{Yield Curves}

\includegraphics<1>[height=6cm, width=11cm]{../../../pics/Zinsstruktur2013}

}


\frame{\frametitle{Yield Curves}

\includegraphics<1>[height=6cm,width=11cm]{../../../pics/Zinsstruktur2013Bb}

}

\frame{\frametitle{Yield Curves}

\includegraphics<1>[height=6cm,width=11cm]{../../../pics/Zinsstrukturflaeche 1988-2013.png}

{\tiny Daten: Staatsanleihen July 2013.}

}
\frame{\frametitle{Fixed-rate Bond}
With a fixed-rate bond the seller promises the buyer to pay fixed coupons $C$ over time, until the bond matures, and when it matures the seller will repay the principal amount borrowed. The price of a bond is determined by its cashflow and the discount factor ($T=t+n$ maturity, $N$ notional value)
$$
\begin{array}{lll}
p_c(t,T)&=&\DSE\frac{C}{(1+r_1)^1}+\frac{C}{(1+r_2)^2}+\ldots+\frac{C+N}{(1+r_n)^n}\\*[12pt]
&=&\DSE\sum_{i=1}^n \frac{C}{(1+r_i)^i}+\frac{N}{(1+r_n)^n}
\end{array}
$$
}

\frame{\frametitle{Yield to Maturity}
The yield to maturity $y$ can be calculated from the coupon bond prices
$$
p_c(t,T)=\frac{C}{(1+y)^1}+\frac{C}{(1+y)^2}+\ldots+\frac{C+N}{(1+y)^n}
$$
}


%\subsection{Basic Interest Rates -- Continuous Version}
\frame{ \frametitle{Notation}

$p(t,T)$ denotes the price of a risk-free zero-coupon bond at time
$t$ that pays one unit of currency at time $T$.

We will use continuous compounding, i.e. a zero bond with interest
rate $r(t,T)$ maturing at $T$ will have the price
$$p(t,T)=e^{-r(t,T)(T-t)}.$$
}

\frame{ \frametitle{Forward Rates}
Given three dates $t < T_1 <
T_2$ the basic question is: what is the risk-free rate of return,
determined at the contract time $t$, over the interval $[T_1,T_2]$
of an investment of $1$ at time $T_1$?\\

\begin{table}[htbp]
\begin{center}
\begin{tabular}{|c|c|c|c|}
\hline
{\rule[-3mm]{0mm}{8mm} Time }& $t$ & $T_1$ & $T_2$\\
\hline & & & \\*[-2mm]
& Sell $T_1$ bond & Pay out $1$ & \\
& Buy $\frac{p(t,T_1)}{p(t,T_2)}\;$ $T_2$ bonds & & Receive
$\frac{p(t,T_1)}{p(t,T_2)}$\\*[2mm] \hline {\rule[-3mm]{0mm}{8mm}
Net investment} & $0$ & $-1$ &
$+\frac{p(t,T_1)}{p(t,T_2)}$\\*[2mm] \hline
\end{tabular}
\end{center}
\caption{Arbitrage table for forward rates}
\end{table}

}

\frame{ \frametitle{Forward Rates}

To exclude arbitrage opportunities, the equivalent constant rate
of interest $R$ over this period (we pay out $1$ at time $T_1$ and
receive $e^{R(T_2-T_1)}$ at $T_2$) has thus to be given by
$$
e^{R(T_2-T_1)} = \frac{p(t, T_1)}{p(t, T_2)}.
$$
}
\frame{ \frametitle{Various Interest Rates}

\begin{itemize}
\item <1-> The forward rate at time $t$ for time period
$[T_1,T_2]$ is defined as
\[
R(t,T_1,T_2)=\frac{\log(p(t,T_1))-\log(p(t,T_2))}{T_2-T_1}
\]
\item <2->The spot rate for the time period $[T_1,T_2]$ is defined
as
\[
R(T_1,T_2)=R(T_1,T_1,T_2)
\]
\item<3-> The instantaneous forward rate is
\[
f(t,T)=-\frac{\partial \log(p(t,T))}{\partial T}
\]
\item <4-> The instantaneous spot rate is
\[
r(t)=f(t,t)
\]
\end{itemize}

} \frame{ \frametitle{Rates}
\begin{itemize}
\item<1-> The forward rate is the interest rate at which parties
at time $t$ agree to exchange $K$ units of currency at time $T_1$
and give back $Ke^{R(t,T_1,T_2)(T_2-T_1)}$ units at time $T_2$.
This means, one can lock in an interest rate for a future time
period today. \item<2->The spot rate $R(t,T_1)$ is the interest
rate (continuous compounding) at which one can borrow money today
and has to pay it back at $T_1$. \item<3->The instantaneous
forward and spot rate are the corresponding interest rates at
which one can borrow money for an infinitesimal short period of
time.
\end{itemize}
}


\frame{\frametitle{Simple Relations}

The money account process is defined by
$$
B(t) = \exp\left\{ \int_0^t r(s) ds\right\}\!.
$$

The interpretation of the money market account is a strategy of
instantaneously reinvesting at the current short rate.

For $t \leq s \leq T$ we have
$$
p(t,T) = p(t,s) \exp\left\{-\int_s^T f(t,u) du\right\}\!,
$$
and in particular
$$
p(t,T) = \exp\left\{-\int_t^T f(t,s) ds\right\}\!.
$$

}


\subsection{Market Rates}
\frame{\frametitle{Simple Spot Rate}
The simply-compounded spot interest rate prevailing at time $t$ for
the maturity $T$ is denoted by $L(t,T)$ and is the constant rate
at which an investment has to be made to produce an amount of one
unit of currency at maturity, starting from $p(t,T)$ units of
currency at time $t$, when accruing occurs proportionally to the
investment time.
\begin{equation}\label{LIBOR-spot}
L(t,T)=\frac{1-p(t,T)}{\tau(t,T)p(t,T)}
\end{equation}
Here $\tau(t,T)$ is the
daycount for the period $[t,T]$ (typically $T-t$).
}

\frame{\frametitle{Simple Spot Rate}
\begin{itemize}
\item<1-> The bond price can be expressed as
$$
p(t,T)=\frac{1}{1+L(t,T)\tau(t,T)}.
$$
Other 'daycounts' denoted  by $\tau(t,T)$ are possible.
\item<2-> Notation is motivated by LIBOR rates (London InterBank Offered
Rates).
\end{itemize}
}

%\section{Derivative Instruments}


%\subsection{Simply-Compounded Forward Interest Rates}
\frame{\frametitle{Forward Rate Agreements}
In order to introduce simply-compounded forward interest rates we
consider forward-rate agreements (FRA). A FRA involves the current
time $t$, the expiry time $T>t$ and the maturity time $S>T$. The
contract gives its holder an interest-rate payment for the period
between $T$ and $S$. At maturity $S$, a fixed payment based on a
fixed rate $K$ is exchanged against a floating payment based on
the spot rate $L(T,S)$ resetting in $T$ with maturity $S$.
}
\frame{\frametitle{Forward Rate Agreements}

Formally, at time $S$ one receives $\tau(T,S)K\cdot N$ units of
currency and pays the amount $\tau(T,S)L(T,S)\cdot N$, where $N$
is the contract nominal value. The value of the contract is
therefore at $S$
\begin{equation}\label{FRA-1}
N\tau(T,S)(K-L(T,S)).
\end{equation}
Using (\ref{LIBOR-spot})
we write this in terms of bond prices as
$$
N\tau(T,S)\left(K-\frac{1-p(T,S)}{\tau(T,S)p(T,S)}\right)=N\left(K\tau(T,S)-\frac{1}{p(T,S)}+1\right).
$$
}
\frame{\frametitle{Forward Rate Agreements}

Now we discount to obtain the value of this time $S$ cashflow at
$t$
$$
\begin{array}{ll}
& FRA(t,T,S,\tau(T,S),N,K) \\*[12pt] = & Np(t,S)\left(K\tau(T,S)-\frac{p(t,T)}{p(t,T)p(T,S)}+1\right) \\*[12pt]
  = &N(K p(t,S)\tau(T,S)-p(t,T)+p(t,S)).
\end{array}
$$
There is only one value of $K$ that renders the contract value $0$
at $t$. The resulting rate defines the simply-compounded forward
rate.
}
\frame{\frametitle{Simply-Compounded Forward Interest Rate}

The simply-compounded forward interest rate prevailing at time $t$
for the expiry $T>t$ and maturity $S>T$ is denoted by $F(t;T,S)$
and is defined by
\begin{equation}
F(t;T,S):=\frac{1}{\tau(T,S)} \left[\frac{p(t,T)}{p(t,S)}-1\right].
\end{equation}
}
\frame{\frametitle{Simply-Compounded Forward Interest Rate}

\begin{itemize}
\item<1-> $FRA(\ldots)=Np(t,S)\tau(T,S)(K-F(t; T,S))$ is an
equivalent definition.
\item<2-> To value a FRA (typically with a different $K$) replace the LIBOR rate in (\ref{FRA-1}) by
the corresponding forward rate $F(t;T,S)$ and take the present
value of the resulting quantity.
\end{itemize}
}

%\subsection{Interest-Rate Swaps}
%{\bf Proof (i)}
%$$
%Np(t,S)\tau(T,S)(K-F(t;T,S))=N(p(t,S)\tau(T,S)\cdot K-p(t,T)+p(t,S)).
%$$
\frame{\frametitle{Interest-Rate Swap}

A generalisation
of the FRA is the Interest-Rate Swap.(IRS). A Payer (Forward-start)
Interest-Rate Swap (PFS) is a contract that exchanges payments
between two differently indexed legs, starting from a future time
instant. At every instant $T_i$ in a prespecified set of dates
$T_{\alpha+1},\ldots ,T_{\beta}$ the fixed leg pays out the amount
$$
N\tau_i\cdot K
$$
corresponding to a fixed interest rate $K$, a
nominal value $N$, and a year fraction $\tau_i$ between $T_{i-1}$
and $T_i$, whereas the floating leg pays the amount
$$
N\tau_i L(T_{i-1},T_i).
$$
Corresponding to the interest rate
$L(T_{i-1},T)$ resetting at the previous instant $T_{i-1}$ for the
maturity given by the current payment instant $T_i$, with
$T_{\alpha}$ a given date.
}
%%%%%%%%%% Skizze %%%%%%%%%%%%%
\frame{\frametitle{Interest-Rate Swap}

Set
$${\cal T}:=\{T_{\alpha},\ldots ,T_{\beta}\}\quad\mbox{and}\quad
\tau:=\{\tau_{\alpha+1},\ldots,\tau_{\beta}\}.
$$
Payers IRS(PFS):
fixed leg is paid and floating leg is received \\
Receiver IRS (RFS): fixed leg is received and floating leg is
paid.

The discounted payoff at time $t<T_{\alpha}$ of a PFS is
$$
\sum_{i=\alpha+1}^{\beta}D(t,T_i)N\tau_i(L(T_{i-1},T_i)-K)
$$
with $D(t,T)$ the discount factor (typically from bank account).
For a RFS we have
$$
\sum_{i=\alpha+1}^{\beta}D(t,T_i)N\tau_i(K-L(T_{i-1},T_i)).
$$
}
\frame{\frametitle{Interest-Rate Swap}
We
can view the last contract as a portfolio of FRAs and find
$$
\begin{array}{ll}
& \DSE RFS(t,{\cal T},\tau,N,K) \\*[12pt] = & \DSE \sum_{i=\alpha+1}^{\beta}FRA(t,T_{i-1}T_i,\tau_i,N,K) \\*[12pt]
= & \DSE N\sum_{i=\alpha+1}^{\beta}\tau_i p(t,T_i)(K-F(t,T_{i-1},T_i))\\*[12pt]
= & -N   p(t,T_{\alpha})+Np(t,T_{\beta})+N\sum_{i=\alpha+1}^{\beta}\tau_i Kp(t,T_i).
\end{array}
$$
The two legs of an IRS can be viewed as
coupon-bearing bond (fixed leg) and floating rate note (floating
leg).
}
\frame{\frametitle{Interest-Rate Swap}
A floating-rate note is a
contract ensuring the payment at future times
$T_{\alpha+1},\ldots,T_{\beta}$ of the LIBOR rates that reset at the previous instants
$T_{\alpha},\ldots,T_{\beta-1}$. Moreover, the note pays a last
cash flow consisting of the reimbursement of the notational value
of the note at the final time $T_{\beta}$.
}
\frame{\frametitle{Interest-Rate Swap}
We can value the note by changing sign and setting $K=0$ in the
RFS formula and adding it to $Np(t,T_{\beta})$, the present value
of the cash flow $N$ at $T_{\beta}$. So we see
$$
\underbrace{-RFS(t,T,\tau,N,0)+Np(t,T_{\beta})}_{\mbox{value of note}}=
\underbrace{Np(t,T_{\alpha})}_{\mbox{from RFS formula}}.
$$
}
\frame{\frametitle{Interest-Rate Swap}

This implies that the note is always equivalent to $N$ units at
its first reset date $T_{\alpha}$ (the floating note trades at
par). We require the IRS to be fair at time $t$ to obtain the
forward swap rate.

The forward swap rate $S_{\alpha,\beta}(t)$ at time $t$ for the
sets of time $\cal T$ and year fractions $\tau$ is the rate in the
fixed leg of the above IRS that makes the IRS a fair contract at
the present time, i.e. it is the fixed rate $K$ for which
$RFS(t,T,\tau,N,K)=0$. We obtain
\begin{equation}\label{FSR-1}
S_{\alpha,\beta}(t)=\frac{p(t,T_{\alpha})-p(t,T_{\beta})}{\sum_{i=\alpha+1}^{\beta}\tau_ip(t,T_i)}.
\end{equation}
}
\frame{\frametitle{Interest-Rate Swap}

We write (\ref{FSR-1}) in terms of forward rates. First divide numerator
and denominator by $p(t,T_{\alpha})$ and observe that
$$
\frac{p(t,T_k)}{p(t,T_{\alpha})}=
\prod_{j=\alpha+1}^k\frac{p(t,T_j)}{p(t,T_{j-1})}=
\prod_{j=\alpha+1}^k\frac{1}{1+\tau_jF_j(t)}$$
with $F_j(t):=F(t,T_{j-1};T_j)$. So (\ref{FSR-1}) can be written as
\begin{equation}
S_{\alpha,\beta}(t)=
\frac{1-\prod_{j=\alpha+1}^{\beta}\frac{1}{1+\tau_jF_j(t)}}
{\sum_{i=\alpha+1}^{\beta}\tau_i\prod_{j=\alpha+1}^{i}\frac{1}{1+\tau_jF_j(t)}}.
\end{equation}
}
\subsection{Interest Rate Derivatives}
%\subsection{Caps and Floors}
\frame{\frametitle{Caps}
\begin{itemize}
\item<1->
A cap is a contract where the seller of the contract promises to
pay a certain amount of cash to the holder of the contract if the
interest rate exceeds a certain predetermined level (the cap rate)
at a set of future dates.
\item<2->It can be viewed as a payer IRS where
each exchange payment is executed only if it has positive value.
\item<3->The cap discounted payoff is
$$
\sum_{i=\alpha+1}^{\beta}D(t,T_i)N\tau_i(L(T_{i-1},T_i)-K)^+.
$$
\item<4->Each individual term is a caplet.
\end{itemize}
}
\frame{\frametitle{Floors}
\begin{itemize}
\item<1->A floor is
equivalent to a receiver IRS where each exchange is executed only
if it has positive value.
\item<2->
The floor discounted payoff is
$$
\sum_{i=\alpha+1}^{\beta}D(t,T_i)N\tau_i(K-L(T_{i-1},T_i))^+.
$$
\item<3->Each individual term is a floorlet.
\end{itemize}
}
%{\bf Motivation.} Protection against LIBOR increase.


\frame{\frametitle{Simple Properties}
A cap
(floor) is said to be at-the-money (ATM) if and only if
$$
K=K_{ATM}:=S_{\alpha,\beta}(0)=\frac{p(0,T_{\alpha})-p(0,T_{\beta})}{\sum_{i=\alpha+1}^{\beta}\tau_ip(0,T_i)}.
$$
The cap is instead said to be in-the-money (ITM) if $K<K_{ATM}$,
and out-of-the-money (OTM) if $K>K_{ATM}$, with the converse
holding for a floor.
}

\frame{\frametitle{Simple Properties Cap}
\begin{itemize}
\item<1-> Simple protection against rising interest rates, but requires the payment of a premium (of course)
\item<2-> Strike is the maximal interest to be paid
\item<3-> Advantageous only if market expectation becomes true
\end{itemize}
}


%\subsection{Swaptions}
\frame{\frametitle{Swaptions}
\begin{itemize}
\item<1->Swap options or more commonly swaptions are options on an IRS. A
European payer swaption is an option giving the right (and not the
obligation ) to enter a payer IRS at a given future time, the
swaption maturity. Usually the swaption maturity coincides with
the first reset date of the underlying IRS.
\item<2->
The underlying-IRS
length $(T_{\beta}-T_{\alpha})$ is called the tenor of the swap.
\item<3-> The discounted payoff of a payer swaption can be written by
considering the value of the underlying payer IRS at its first
reset date $T_{\alpha}$ (also the maturity of the swaption)
$$
N\sum_{i=\alpha+1}^{\beta} p(T_{\alpha},T_i)\tau_i(F(T_{\alpha};T_{i-1},T_i)-K).
$$

\end{itemize}
}
\frame{\frametitle{Swaptions Payoff}
The option will be exercised only if this
value is positive. So the current value is
$$
ND(t,T_{\alpha})\left(\sum_{i=\alpha+1}^{\beta}p(T_{\alpha},T_i)\tau_i
(F(T_{\alpha};T_{i+1},T_i)-K)\right)^+.
$$
}
\frame{\frametitle{Swaptions Payoff}

Since the positive part operator is a
piece-wise linear and convex function we have
$$\begin{array}{ll}
&\displaystyle
\left(\sum_{i=\alpha+1}^{\beta}p(T_{\alpha},T_i)\tau_i(F(T_{\alpha};T_{i-1},T_i)-K)\right)^+\\*[12pt]
\leq&\displaystyle \sum_{i=\alpha+1}^{\beta}p(T_{\alpha},T_i)\tau_i(F(T_{\alpha};T_{i-1},T_i)-K)^+
\end{array}
$$
with strict inequality in general. Thus an additive decomposition is not
feasible.
}
%\underline{Excercise:} Compare value of payer swaption and
%corresponding cap.

\frame{\frametitle{Swaptions Payoff}
A swaption (either payer or receiver) is said to be at-the-money
(ATM) if and only if
$$
K=K_{ATM}=S_{\alpha,\beta}(0)=\frac{p(0,T_{\alpha})-p(0,T_{\beta})}{\sum_{i=\alpha+1}^{\beta}\tau_ip(0,T_i)}.
$$
The payer swaption is instead said to be in-the-money (ITM) if
$K<K_{ATM}$, and out-of-the-money (OTM) if $K>K_{ATM}$. The
receiver swaption is ITM if $K>K_{ATM}$, and OTM if $K<K_{ATM}$.
}
\subsection{Valuation of Structured Products}
%\subsection{General Principles}
\frame{\frametitle{Product Buyer}
\begin{itemize}
\item<1-> (Institutional) Investor buys product for certain nominal value $N$
\item<2-> Receives coupons at prespecified time points.
\item<3-> At terminal date (maturity) the nominal is paid back.
\item<4-> Investor wants to receive as high as possible coupon payments. Therefore Investor is willing to take a point of view towards market development.
\end{itemize}
}
\frame{\frametitle{Product Buyer}
\begin{itemize}
\item<1-> Investor has a liability and has to pay floating or fixed interest
\item<2-> Investor can enter a swap which pays the liability cash-flow
\item<3-> Investor pays coupons on a structured product in return, which (in case that the market view of the investor becomes true) are cheaper than the original cash flow
\end{itemize}
}
%\subsection{Simple Products}
\frame{\frametitle{Cap}
\begin{itemize}
\item<1-> Simple protection against rising interest rates, but requires the payment of a premium (of course)
\item<2-> Strike is the maximal interest to be paid
\item<3-> Advantageous only if market expectation becomes true
\end{itemize}
}

\frame{\frametitle{Cap}
\includegraphics<1>[height=6cm,width=\textwidth]{../../../pics/cap-structure-german.pdf}
}

\frame{\frametitle{Cap}
\includegraphics<1>[height=6cm,width=\textwidth]{../../../pics/cap-payoff-german.pdf}
}

\frame{\frametitle{Instruments for Interest Rate Management}
\begin{itemize}
\item<1-> Agreements to exchange rates
\begin{itemize}
\item Exchange of rate payments
\item fixed tenor
\item no costs to enter the contract
\item Examples: FRA, Swaps
\end{itemize}
\item<2-> Insurance against rates movements
\begin{itemize}
\item Option on payment
\item fixed maturity
\item buyer pays premium
\item Examples: Caps, Floors, Swaptions
\end{itemize}

\end{itemize}
}

\frame{\frametitle{Interest Rate Swap}
\begin{itemize}
\item<1->  Buyer of a swap
\begin{itemize}
\item receives fixed swap rate $S(0,T)$
\item no initial payment (since this is the fair rate)
\item expects an increase of rates above the swap rate $S(0,T)$ during life-time of the contract
\item to be profitable the increase must be higher than suggested by current swap rate
\end{itemize}
\item<2-> Benefit: Insurance against rising rates, no initial payment
\item<3-> Risk: No participation if interest rates fall
\end{itemize}
}


\frame{\frametitle{Interest Rate Swap}
\includegraphics<1>[height=6cm,width=\textwidth]{../../../pics/swap-structure-german.pdf}
}

\frame{\frametitle{Constant Maturity Swap}
\begin{itemize}
\item<1-> For a constant maturity swap (CMS) one of the reference rates is a variable market rate
\item<2-> Example: 3 -Month Euribor vs 5 year swap rate
\item<3-> For standard term structure CMS has a positive value, so Euribor + spread is paid
\end{itemize}

}

\frame{\frametitle{Constant Maturity Swap}
\begin{itemize}
\item<1-> Advantages: It is possible to take advantage of favourable movements of the term structure; starts with lower costs
\item<2-> Disadvantages: Unfavourable movements of term structure generate losses; losses are potentially unlimited
\end{itemize}

}


%\subsection{Structured Products}

\frame{\frametitle{Interest Rate Swap with Optionality}
\includegraphics<1>[height=6cm,width=\textwidth]{../../../pics/swap-structure-chance-german.pdf}
}

\frame{\frametitle{IRS with Optionality: Risk Profile}
\includegraphics<1>[height=6cm,width=\textwidth]{../../../pics/swap-structure-riskprofile-german.pdf}
}


\frame{\frametitle{Interest Rate Swap with Optionality}
\begin{itemize}
\item<1-> Chance
\begin{itemize}
\item Rates are capped at 5,45 \% for the next 8 years
\item Participation on low rates is still possible
\end{itemize}
\item<2-> Risk
\begin{itemize}
\item Participation on lower than 3,75 \% rates is not possible
\item In case rates are lower than 3,75 \% Rates a high rate (5,45 \%) has to be paid
\end{itemize}
\item<3-> Chance/Risk the swap can be traded (i.e. be sold).
\end{itemize}
}



