% !TEX root = FuturesOptions_ss14UDE.tex
\section{The Black-Scholes Model}

\begin{frame}
    \frametitle{}
\vspace{0cm}
\begin{center}
\color{beamer@blendedblue}{\large{The Black-Scholes Model}}
\end{center}
\end{frame}

\subsection{The Pricing Formula}

\frame{\frametitle{The Black-Scholes Formula for a European Call}
\begin{center}
\begin{figure}
  \centering
   \subfigure[Myron Scholes]{\includegraphics[height=3cm]{../../../pics/myron_scholes.jpg}}\qquad
   \subfigure[Robert C. Merton]{\includegraphics[height=3cm]{../../../pics/merton.jpg}}\qquad
   \subfigure[Fisher Black]{\includegraphics[height=3cm]{../../../pics/black.jpg}}
\end{figure}
\end{center}
}


%\frame{\frametitle{ Black-Scholes Model}

%\begin{itemize}
%  \item The classical Black-Scholes model is
%\begin{itemize}
%  \item $dB(t) = \DSE r B(t) dt, \A B(0)= 1$
%  \item $dS(t) = \DSE S(t) (b dt + \sigma dW(t) \!, \A &S(0) = p$
%\end{itemize}
%\item with constant coefficients $b \in \setR,\; r,\s \in \setR_+$.
%\end{itemize}
%}


\frame{\frametitle{European Call Price}

For a European call $X = (S(T)-K)^+$ and  we can evaluate the
above expected value

The Black-Scholes price
pro\-cess of a European call is given by
$$
\begin{array}{lll}
C(t) &=&\DSE S(t) \Phi(d_1(S(t), T-t))\\*[12pt]
&&- K e^{-r(T-t)} \Phi(d_2(S(t), T-t)).
\end{array}
$$
The functions $d_1(s,t)$ and $d_2(s,t)$ are given by
$$
\begin{array}{lll}
d_1(s,\tau) &=&\DSE \frac{\log(s/K) + (r +
\frac{\sigma^2}{2})\tau}{\sigma \sqrt{\tau}},\\*[12pt] d_2(s,\tau) &=&\DSE
 \frac{\log(s/K) + (r -
\frac{\sigma^2}{2})\tau}{\sigma \sqrt{\tau}}
\end{array}
$$
}

\subsection{The Greeks}
\frame{\frametitle{Greeks (1)}
\begin{itemize}
\item<1-> We will now analyse the impact of the
underlying parameters in the standard Black-Scholes model on the
prices of call and put options.
\item<2->
The Black-Scholes option values
depend on the
\begin{enumerate}
\item (current) stock price,
\item the volatility,
\item the time to maturity,
\item the interest rate
\item the strike price.
\end{enumerate}
\end{itemize}
}
\frame{\frametitle{Greeks (2)}
\begin{itemize}
\item<1-> The
sensitivities of the option price with respect to the first four
parameters are called the {\it Greeks} and are widely used for
hedging purposes.
\item<1->
We can determine the impact of these parameters
by taking partial derivatives.
\end{itemize}
}
\frame{\frametitle{Greeks}

$$
\begin{array}{lllll}
\Delta &:=&\DSE \frac{\partial C}{\partial S} &=&\DSE \Phi(d_1) >
0,\\*[12pt] 
{\cal V} &:=&\DSE \frac{\partial C}{\partial \s} &=&
\DSE S \sqrt{\tau} \varphi(d_1) >0,\\*[12pt]
 \Theta &:=&\DSE \frac{\partial
C}{\partial \tau} &=& \DSE \frac{S \s}{2 \sqrt{\tau}} \varphi(d_1) + K r
e^{-r\tau} \Phi(d_2) >0,\\*[12pt] 
\rho &:=&\DSE \frac{\partial
C}{\partial r} &=& \DSE \tau K e^{-r\tau} \Phi(d_2) >0,\\*[12pt] \Gamma
&:=&\DSE \frac{\partial^2 C}{\partial S^2} &=& \DSE
\frac{\varphi(d_1)}{S \s \sqrt{\tau}} >0.\\*[12pt]
\end{array}
$$
} \frame{\frametitle{Greeks} From the definitions it is clear that
\begin{itemize}
\item<1->
$\Delta$ -- delta -- measures the change in the value of the
option compared with the change in the value of the underlying
asset,
\item<2-> ${\cal V}$ -- vega -- measures the change of the option
compared with the change in the volatility of the underlying,
\item<3-> similar statements hold for $\Theta$ -- theta -- and $\rho$ -- rho
\end{itemize}
}
\frame{\frametitle{Greeks}
\includegraphics<1>[height=7cm]{../../../pics/Delta.pdf}

} \frame{\frametitle{Greeks}
\includegraphics<1>[height=7cm]{../../../pics/DeltaToZero.pdf}

} \frame{\frametitle{Greeks}
\includegraphics<1>[height=7cm]{../../../pics/gamma.pdf}

} \frame{\frametitle{Greeks}
\includegraphics<1>[height=7cm]{../../../pics/Vega2.pdf}

}


\frame{\frametitle{Greeks}
\begin{itemize}
\item<1->
The Black-Scholes partial differential equation can be used to
obtain the relation between the Greeks,
\item<2->
$$
r C = \frac{1}{2} s^2 \s^2 \Gamma +rs \Delta -\Theta.
$$
\end{itemize}
}

\frame{\frametitle{Vega}
\begin{itemize}
\item<1->
Before we can implement
the Black-Scholes formula to price options, we have to estimate
$\s$.
\item<2->
Because the formula is explicit, we can, determine the ${\cal V}$
-- the partial derivative
$$
{\cal V} = \partial C/\partial \s,
$$
finding
$$
{\cal V} = S \sqrt{T} \Phi(d_1).
$$
\item<3-> Note here is that vega is always positive.
\end{itemize}
}

\subsection{Volatility}
 \frame{\frametitle{Implied Volatility}
\begin{itemize}
\item<1->
Since vega is positive, $C$ is a continuous -- indeed,
differentiable -- strictly increasing function of $\s$.
\item<2-> Turning
this round, $\s$ is a continuous (differentiable) strictly
increasing function of $C$; indeed,
$$
{\cal V} = \frac{\partial C}{\partial \s}, \A \mbox{so} \A
\frac{1}{\cal V} = \frac{\partial \s}{\partial C}.
$$
\item<3->
Thus the value $\s = \s(C)$ corresponding to the actual value $C =
C(\s)$ at which call options are observed to be traded in the
market can be read off.  The value of $\s$ obtained in this way is
called the {\it implied volatility}.
\end{itemize}}
\frame{\frametitle{The Black-Scholes Implied Volatility}
\begin{itemize}
\item<1-> The Black-Scholes model assumes a constant volatility over all maturities and strikes.
\item<2-> One can calculate the implicit volatility given the market price of options.
\item<3-> The implicit volatility is changing over time and maturity, so contradicting the modelling assumptions.
\item<4-> However, it can be used to predict future volatility.
\item<5-> Furthermore, volatility can be made an own "asset class".
\end{itemize}

}

\frame{\frametitle{Volatility Smile}
Volatility Index VDAX-NEW (in percentage points)
\begin{center}
\includegraphics[scale=0.55]{../../../pics/Volatility-Smile-char.pdf}
\end{center}
}
\frame{\frametitle{Volatility Surface}
Volatility Index VDAX-NEW (in percentage points)
$$\includegraphics[scale=0.25]{../../../pics/VolatilitySurface.jpg}$$
}


\frame{\frametitle{VDAX-NEW: Definition}
\begin{itemize}
  \item The volatility index VDAX-NEW was developed by Deutsche B{\"o}rse and Goldman Sachs. It tracks the degree of fluctuation expected by the derivatives market, i.e. the implied volatility, for the DAX index. The index expresses in percentage terms what degree of volatility is to be expected for the following 30 days.
  \item VDAX-NEW started on 20 April 2005 and will replace VDAX in the medium-term.
\end{itemize}
}


\frame{\frametitle{VDAX-NEW: Examples}
Examples:\\
\vspace{0.3cm}
\begin{itemize}
  \item<1-> A DAX of 4,000 and a VDAX-NEW of 10 indicate that the DAX stock index is expected to fluctuate between 3,885 and 4,115 over the next thirty days:
  $$4000\pm4000\times0.1\times\sqrt{\frac{30}{365}}\approx4000\pm115.$$
  \item<2-> A DAX of 4,000 and a VDAX-NEW of 20 indicate that the DAX stock index is expected to fluctuate between 3,770 and 4,230 over the next thirty days:
    $$4000\pm4000\times0.2\times\sqrt{\frac{30}{365}}\approx4000\pm230.$$
\end{itemize}
}


\frame{\frametitle{VDAX-NEW}
Volatility Index VDAX-NEW (in percentage points)
$$\includegraphics[scale=0.6]{../../../pics/VDaxNew.png}$$
}


\frame{\frametitle{VDAX-NEW: Calculation}
\begin{itemize}
  \item <1-> Index is based on 8 sub-indices (option series) which include DAX Options from 2-24 months expiration.
  \item <2-> The main rolling index is calculated 30 days to expiration on linear interpolation of the two sub-indices closest to the 30 days expiration.
  \item <3-> In addition to at-the-money options (VDAX), out-of-the-money options are also considered (VDAX-NEW).
  \item <4-> Calculation frequency: Once a minute on every trading day at Eurex between 8.50am and 5.30pm CET.
\end{itemize}
}

\frame{\frametitle{VDAX-NEW: Features}
\begin{itemize}
  \item <1-> Expresses market expectation of the amplitude of fluctuation in DAX.
  \item <2-> Index is able to react only to changes in volatility.
  \item <3-> Allows better replication for derivatives and structured products.
  \item <4-> Due to ATM and OTM options VDAX-NEW captures more of the volatility skew than VDAX.
  \item <5-> Index establishes volatiliy as a tradable and separate asset class for investors.
\end{itemize}
}

\frame{\frametitle{VDAX-NEW: Use}
\begin{itemize}
  \item <1-> Investment/Trading: Speculation on future levels of volatility.
  \item <2-> Hedging: Cover short volatility positions.
  \item <3-> Diversification of German equity portfolios (correlation to DAX is -0.5689, correlation to MDAX is -0.4668).
  \item <4-> Benchmark for German equity volatility.
\end{itemize}
}

\frame{\frametitle{VDAX-NEW: Diversification}
\begin{itemize}
  \item Portfolio of 80\% DAX and 20\% VDAX-NEW between
1992 and 2004 would have generated a 3.8 percentage
points on average higher yield than investments in
DAX only. And this with lower risk!
$$\includegraphics[scale=0.45]{../../../pics/VDax_div.png}$$
\end{itemize}
}

\subsection{Derivatives Risk Management}

%\subsection{Introduction}
\begin{frame}[fragile]
\frametitle{Motivation}
A bank (option seller) tries to control the risk of its short position and has to ask the folkowing questions:
\begin{itemize}
  \item How does the value of the position change if the value of the underlying changes?
  \item How can you protect (i.e., hedge) yourself against those risk?
\end{itemize}
\end{frame}

\begin{frame}[fragile]
\frametitle{Some Terms}
A \emph{hedge} is a trade that reduces the risk of a position.
\begin{itemize}
  \item A \emph{static} hedge is set up once and not touched until maturity of
  the derivative/position that is hedged.
  \item A \emph{dynamic} hedge is set up and adjusted continuously (in practice:
  regularly).
  \item A \emph{semi-static} hedge is set up and adjusted at infrequent
  intervals.
\end{itemize}
A position for which no hedge is in place is called \emph{naked}, a position
that is fully hedged is called \emph{covered} (e.g., sell a forward and buy the
underlying stock).
\end{frame}

\begin{frame}[fragile]
\frametitle{A simple Hedging Strategy: Stop-Loss}
We consider a short position in one European call option with strike $K$. 
\begin{itemize}
  \item A stop-loss hedge involves buying one share of the underlying stock when
  the price moves above $K$ and selling it again when the price falls below $K$.
  \item The strategy usually does not work well in practice because the trader
  cannot know whether the stock price moves above $K$ or falls below $K$ until
  it is too late. Therefore, he buys the stock at $K+\epsilon$ and sells at
  $K-\epsilon$, resulting in a loss.
  \item Also, discounting has to be taken into account.
\end{itemize}
\end{frame}


\begin{frame}[fragile]
\frametitle{Delta Hedging I}
\begin{itemize}
  \item Assume that you have an options position with delta $\Delta$. This means
  that if the price of the underlying moves by a (very) small amount $\epsilon$, the
  price of the option position moves approximately by $\Delta \epsilon$.
  \item A \emph{delta hedge} consists of selling $\Delta$ units of the
  underlying (buying if $\Delta <0$) and gives a portfolio that does not
  change its value if the price of the underlying changes by a marginal amount.
  The portfolio has a delta of zero and is called \emph{delta neutral}.
\end{itemize}
\end{frame}

\begin{frame}[fragile]
\frametitle{Delta Hedging II}
\begin{itemize}
  \item BUT: Delta is not constant, so the portfolio has to be rebalanced.
  \item In the Black-Scholes model, a delta hedge with continuous rebalancing is
  a perfect hedge.
  \item In practice, continuous rebalancing is not possible so that the
  portfolio is not protected against larger market movements. Transaction
  costs and bid-ask spread also result in losses.
\end{itemize}
\end{frame} 
 


\begin{frame}[fragile]
\frametitle{Example: Delta Hedge}
You are long USD $1,000$ in the $104$ call. Interest rate is $5\%$,
stock price today is $99$, time to maturity $1$ month, and implied volatility is
$15.7\%$. 
\begin{itemize}
  \item How can you make your portfolio delta neutral by investing in the stock?
  \item You set up the delta neutral portfolio and the stock price jumps to
  USD$100$ immediately. What is your P/L for the portfolio?
\end{itemize}

\end{frame}

\begin{frame}[fragile]
\frametitle{Example: Delta Hedge}
\begin{itemize}
  \item Compute the price of the call with the BS formula:
  \begin{align*}
    Call_{BS} = 0.3858 \text{USD}.
  \end{align*}
  \item The position consists of $N=1,000/Call_{BS}=2592$ call options.
  \item The delta of each option is 
  \begin{align*}
    \Delta &= \Phi(d_1) =\Phi\left(\frac{\log \left( S/K \right) + (r+\sigma^2/2)(T-t)
    }{\sigma\sqrt{T-t}}\right) \\
    	&= \Phi\left(\frac{\log \left( 99/104 \right) + (0.05+0.157^2/2)(1/12)
    }{0.157\sqrt{1/12}}\right)\\
     	&= 0.1654.
  \end{align*}
\end{itemize}
\end{frame}


\begin{frame}[fragile]
\frametitle{Example: Delta Hedge}
\begin{itemize}
  \item The delta of the position is long $\Delta_P=N\cdot \Delta=428.70$.
  \item The stock has $\Delta=1$.
  \item To make the position delta neutral, you have to enter a short position
  of $428.70$ shares.
\end{itemize}
\end{frame}

\begin{frame}[fragile]
\frametitle{Example: Delta Hedge}
\begin{itemize}
  \item The loss from the short position in the stock is $428.70 \cdot
  1=428.70$.
  \item To compute the gain from the long options position, we have to compute
  the option price for $S=100$. Using the BS formula, we obtain
  \begin{align*}
    Call_{BS}(S=100) = 0.5808.
  \end{align*}
  \item The gain from the options position is $2592\cdot(0.5808-0.3858)=505.37$.
  \item Our profit is $505.37-428.70=76.67$.
\end{itemize}
\end{frame}


\begin{frame}[fragile]
\frametitle{Gamma Neutral Portfolios}
As the delta of a portfolio changes with the price of the underlying, a delta
hedge has to be rebalanced frequently. 
\begin{itemize}
  \item A delta hedge for a portfolio with a high (absolute) gamma has to be
  monitored more carefully than a delta hedge of a portfolio with gamma close to
  zero.
  \item To decrease the hedging error, a trader might want to make a delta
  neutral portfolio gamma neutral.
  \item The gamma cannot be changed by investing in the underlying because its
  gamma is zero.
  \item Strategy: Make the portfolio gamma neutral by investing in an option,
  then make it delta neutral by investing in the underlying.
\end{itemize}
\end{frame}



\begin{frame}[fragile]
\frametitle{Example: Gamma Neutral Portfolio}
You are in the same position as in the previous example. Additionally, you can
trade in the $97$ call. 
\begin{itemize}
  \item Put together a portfolio that is delta and gamma neutral.
  \item What is your P/L if the stock price jumps to $100$?
\end{itemize}
\end{frame}

\begin{frame}[fragile]
\frametitle{Example: Gamma Neutral Portfolio}
\begin{itemize}
  \item The gamma of the $104$ call is 
  \begin{align*}
  \Gamma_{104C} = \frac{\varphi(d_1)}{S\sigma \sqrt{T-t}} = 0.0554.
  \end{align*}
  \item The gamma of the $97$ call is 
  \begin{align*}
  \Gamma_{97C} = \frac{\varphi(d_1)}{S\sigma \sqrt{T-t}} = 0.0758.
  \end{align*}
  \item To make the portfolio gamma neutral, we have to build up a position of
  $n$ $97$ calls so that $2592\cdot 0.0554 +n\cdot 0.0758 = 0$.
  \item We find that $n=-1894.74$ makes the portfolio gamma neutral, i.e., we
  sell $1894.74$ units of the $97$ call.
\end{itemize}
\end{frame}

\begin{frame}[fragile]
\frametitle{Example: Gamma Neutral Portfolio}
\begin{itemize}
  \item We compute the delta of the portfolio consisting of the two positions in
  the calls.
  \item The $97$ call has a delta of $\Delta_{97C}=0.7139$.
  \item The delta of the position is
  \begin{align*}
    \Delta_{P'} = 2592 \cdot 0.1654 - 1894.74 \cdot 0.7139 = -924.02
  \end{align*}
  \item We have to buy $924.02$ units of the underlying to make the portfolio
  delta neutral. By buying the underlying, we do not change the gamma of the
  portfolio, it remains zero.
\end{itemize}
\end{frame}

\begin{frame}[fragile]
\frametitle{Example: Gamma Neutral Portfolio}
\begin{itemize}
  \item The price of the $97$ call for $S=99$ is $Call_{BS,97}(S=99)=3.2235$.
  \item The price of the $97$ call for $S=100$ is $Call_{BS,97}(S=100)=3.9735$.
  \item The P/L from the position in the $97$ call is
  $-1894.97*(3.9735-3.2235)=-1421.11$.
  \item The P/L from the position in the stock is $924.02$.
  \item The portfolio P/L is $924.02-1421.11+505.37=-8.28.$
\end{itemize}
\end{frame}

\subsection{Goldman-Sachs Example}

\frame{\frametitle{Option Portfolio}

$$\includegraphics[scale=0.4]{../../../pics/gs-akademie-OS.pdf}$$
}

\frame{\frametitle{Delta Position}

$$\includegraphics[scale=0.6]{../../../pics/gs-akademie-Delta.pdf}$$
}

\frame{\frametitle{Gamma Position}

$$\includegraphics[scale=1.3]{../../../pics/gs-akademie-Gamma.pdf}$$
}

\frame{\frametitle{Vega Position}

$$\includegraphics[scale=1.2]{../../../pics/gs-akademie-Vega.pdf}$$
}

\frame{\frametitle{Gamma vs Vega Position}

$$\includegraphics[scale=1.3]{../../../pics/gs-akademie-GammaVega.pdf}$$
}

\frame{\frametitle{Total Hedge}

$$\includegraphics[scale=0.91]{../../../pics/gs-akademie-TotalHedge.pdf}$$
}

\frame{\frametitle{Partial Hedges}

$$\includegraphics[scale=1.4]{../../../pics/gs-akademie-PartialHedges.pdf}$$
}


\frame{\frametitle{Hedged Delta Position}

$$\includegraphics[scale=1.2]{../../../pics/gs-akademie-HedgedDelta.pdf}$$
}
\frame{\frametitle{Hedged Gamma Position}

$$\includegraphics[scale=1.2]{../../../pics/gs-akademie-HedgedGamma.pdf}$$
}
\frame{\frametitle{Hedged Vega Position}

$$\includegraphics[scale=1.2]{../../../pics/gs-akademie-HedgedVega.pdf}$$
}

