% !TEX root = FuturesOptions_ss14UDE.tex

\section{Basic Derivatives}

\begin{frame}
    \frametitle{}
\vspace{0cm}
\begin{center}
\color{beamer@blendedblue}{\large{Basic Derivatives}}
\end{center}
\end{frame}


\subsection{Modelling Assumptions}


\frame{\frametitle{Derivative Background (1)}

\begin{itemize}
  \item A derivative security, or contingent claim, is a financial
contract whose value
\begin{itemize}
\item at expiration date $T$ (more briefly, expiry)
is determined exactly \index{contingent claim} by the price (or
prices within a prespecified time-interval) of
\item the underlying
financial assets (or instruments) at time $T$ (within the time
interval $[0,T]$).
\end{itemize}
\end{itemize}
}

\frame{\frametitle{Derivative Background (2)}
\begin{itemize}
  \item Derivative securities can be grouped under three general headings:
{\it Options, Forwards and Futures} and {\it Swaps}.
\item During this lectures we will encounter all this structures and further variants.
\end{itemize}
}

\frame{\frametitle{ Modelling Assumptions (1)}

 We impose the following set of assumptions on the financial
markets:
\begin{itemize}
\item<1-> {\it No market frictions: } No transaction costs, no bid/ask
spread, no taxes,
 no margin requirements, no restrictions on short sales.\\*[12pt]
\item<2-> {\it No default risk:} Implying same interest for borrowing
and lending
\end{itemize}


}

\frame{\frametitle{ Modelling Assumptions (2)}
\begin{itemize}
\item<1-> {\it Competitive markets:}  Market participants
act as price takers, infinite number of participants\\*[12pt]
\item<2-> {\it Rational agents} Market
participants prefer more to less
\end{itemize}
}

\frame{\frametitle{ Arbitrage (1)}
\begin{itemize}
  \item The concept of arbitrage lies at
the centre of the relative pricing theory. All we need to assume additionally is
that economic agents prefer
more to less, or more precisely, an increase in consumption
without any costs will always be accepted.
\end{itemize}
}
\frame{\frametitle{ Arbitrage (2)}
\begin{itemize}
  \item The essence of the technical sense of arbitrage is that it should
not be possible to guarantee a profit without exposure to risk.
Were it possible to do so, arbitrageurs would do so, in unlimited quantity,
using the market as a \lq {money-pump}' to extract arbitrarily
large quantities of riskless profit.
  \item {\it We assume that arbitrage opportunities do not exist!}
\end{itemize}
 }
\frame{\frametitle{Continuous and Discrete Compounding}
\begin{itemize}
  \item <1->Discrete compounding applies when we consider discrete time points, i.e. interest payments at the end of a period.\\
      \begin{itemize}
        \item $S_t = S_0 (1+r)^t$
      \end{itemize}

  \item<2-> Continuous compounding applies when interest payments are done at a constant rate through the time period. \\
      \begin{itemize}
        \item $S_t = S_0e^{rt}$
      \end{itemize}
 \end{itemize}


}

\frame{\frametitle{Derivate Markets}
\begin{itemize}
  \item<1-> Financial derivatives are basically traded in two ways: on organized exchanges and over-the counter (OTC).
  \item<2-> Products at exchanges are standardized contracts that are defined by the exchange. Important exchanges for commodities are the Chicago Mercantile Exchange (CME), the Intercontinental Exchange (ICE) in London or the European Energy Exchange (EEX) in Leipzig.
  \item<3-> OTC trading takes place via computers and phones between various commercials and investment banks. OTC contracts are non-standardized and can be flexible adjusted to the demand of the parties.
\end{itemize}

}

\frame{\frametitle{Underlying Securities}

\begin{itemize}
\item<1-> Stocks (one or several);
\item<2-> Fixed income instruments: T-Bonds, Interest Rates (LIBOR, EURIBOR);
\item<3-> Commodities or Commodity Futures;
\item<4-> Currencies (FX);
\item<5-> Also Derivatives may be used as underlying for compound derivatives (call on call).
\end{itemize}

}


\subsection{Options}
\frame{\frametitle{Options}

\begin{itemize}
 \item<1->  An option is a financial instrument giving one the {\it right, but
not the obligation} to make a specified transaction at (or by) a
specified date at a specified price. 
\item<2->{\it Call} options give one
the right to buy. {\it Put} options give one the right to sell.
\item<3-> {\it European} options give one the right to buy/sell on the
specified date, the expiry date, on which the option expires or
matures. {\it American} options give one the right to buy/sell at any time
prior to or at expiry.
\end{itemize}
}

\frame{\frametitle{Exotic Options}
Many kinds of options now exist, including so-called {\it exotic}
options.  Types include:
\begin{itemize}
  \item<1-> {\it Asian} options, which depend on the
{\it average} price over a period,
\item<2-> {\it barrier} options, which depend on some price level being
attained or not.
\item<3-> {\it lookback} options, which
depend on the  {\it maximum} or {\it minimum} price over a period.
\end{itemize}
}

\frame{\frametitle{Options - Terminology (1)}

\begin{itemize}
\item<1-> The asset to which the option refers is called the {\it underlying
asset} or the {\it underlying}. 
\item<2-> 
The price at which the transaction
to buy/sell the underlying, on/by the expiry date (if exercised),
is made, is called the {\it exercise price} or {\it strike price}.
\item<3-> We shall usually use $K$ for the strike price, time $t = 0$ for
the initial time (when the contract between the buyer and the
seller of the option is struck), time $t = T$ for the expiry or
final time.
\end{itemize}

}


\frame{\frametitle{Options - Terminology (2)}

Consider, say, a European call option, with strike price $K$;
write $S(t)$ for the value (or price) of the underlying at time
$t$.\\*[12pt]
\begin{itemize}
  \item<1-> If $S(t) > K$, the option is {\it in the money},
  \item<2-> if $S(t) = K$, the option is {\it at the money},
  \item<3-> if $S(t) < K$, the option is {\it out of the money}.
\end{itemize}
}

\frame{\frametitle{Options - Payoff}

\begin{itemize}
  \item<1-> The payoff from a call option is $$ S(T) - K \mbox{ if } S(T)
> K\A \mbox{ and }\A 0 \;\; \mbox{otherwise} $$ (more briefly
written as  $(S(T) - K)^+$).
\item<2-> The profit from a call option is the payoff $(S(T) - K)^+$) minus the call premium $c$.
\end{itemize}
%}
}

\frame{\frametitle{Options - Payoff/Profit diagram }

Considering only the option payoff, we obtain the payoff diagram, taking into account the initial payment of an investor one obtains the profit diagram below.\\*[12pt]

\begin{center}
\includegraphics[height=5cm]{../../../pics/payoffprofit.png}
\end{center}
}

\frame{\frametitle{Options - Profit diagrams of vanilla options}
\begin{figure}
  \centering
   \subfigure[long call]{\includegraphics[height=3cm]{../../../pics//longcall.png}}\qquad
   \subfigure[short call]{\includegraphics[height=3cm]{../../../pics//shortcall.png}}
   \subfigure[long put]{\includegraphics[height=3cm]{../../../pics//longput.png}}
   \subfigure[short put]{\includegraphics[height=3cm]{../../../pics//shortput.png}}
\end{figure}
}
\subsection{Forwards and Futures}

\frame{\frametitle{Forwards - Basic Structure}

\begin{itemize}
\item<1->
A {\it forward contract}
is an agreement to buy or sell an asset $S$ at a certain future
date $T$ for a certain price $K$.
\item<2->
The agent who agrees to
buy the underlying asset is said to have a {\it long} position,
the other agent assumes a {\it short} position.
\item<3-> The settlement
date is called {\it delivery date} and the specified price is
referred to as {\it delivery price}.
\end{itemize}
}

\frame{\frametitle{Forwards}
\begin{itemize}
\item<1-> The {\it forward
price} $F(t,T)$ is the delivery price which would make the
contract have zero value at time $t$.
\item<2-> At the time the contract is set up, $t=0$,
the forward price therefore equals the delivery price, hence
$F(0,T) = K$.
\item<3->
The forward prices $F(t,T)$ need not (and will not)
necessarily be equal to the delivery price $K$ during the
life-time of the contract.
\end{itemize}
}

\frame{\frametitle{Forwards}
\begin{itemize}
\item<1->
The payoff from a long position in a forward contract on one unit
of an asset with price $S(T)$ at the maturity of the contract is
$$ S(T)-K.$$
\item<2-> Compared with a call option with the same maturity
and strike price $K$ we see that the investor now faces a downside
risk, too. He has the obligation to buy the asset for price $K$.
\end{itemize}
}

\frame{\frametitle{Spot-Forward Relationship}
Under the no-arbitrage assumption we have

\begin{center}
    \begin{tabular}[ht]{|l|c|c|}
  \hline
  $$ & $t$ & $T$\\
  \hline\hline
  buy stock & $-S(t)$ & delivery\\
  borrow to finance & $S(t)$ & $-S(t)e^{r(T-t)}$\\
  sell forward on S & $$ & $F(t,T)$\\
  \hline
\end{tabular}
\end{center}

All quantities are known at $t$, the time $t$ cashflow is zero, so the cashflow at T needs to be zero so we have $$F(t,T) = S(t)e^{r(T-t)}$$
}



\frame{\frametitle{Futures}
\begin{itemize}
\item<1-> Futures can be defined as standardised forward contracts traded at exchanges where a clearing house acts as a central counterparty for all transactions.
\item<2-> Usually an initial margin is paid as a guarantee.
\item<3-> Each trading day a settlement price is determined and gains or losses are immediately realized at a margin account.
\item<4-> Thus credit risk is eliminated, but there is exposure to interest rate risk.

\end{itemize}


}


\frame{\frametitle{Payoff from a forward/futures contract}
\begin{figure}
  \centering
   \subfigure[long position]{\includegraphics[height=4cm]{../../../pics//forwardlong.png}}\qquad
   \subfigure[short position]{\includegraphics[height=4cm]{../../../pics//forwardshort.png}}
\end{figure}
}

\frame{\frametitle{Swaps}
\begin{itemize}
\item<1->
A {\it swap} is an agreement whereby two parties
undertake to exchange, at known dates in the future, various
financial assets (or cash flows) according to a prearranged
formula that depends on the value of one or more underlying
assets. 
\item<2->
Examples are currency swaps (exchange currencies) and
interest-rate swaps (exchange of fixed for floating set of
interest payments).
\end{itemize}

}