\documentclass[11pt,a4paper,titlepage]{article}
\usepackage{a4,latexsym,amsfonts}
\usepackage{epsfig}
\usepackage{amsmath}
\usepackage{amssymb}
\usepackage{eurosym}
\usepackage{rotating}
\usepackage{url}

%Silbentrennung nach neuer deutscher Rechtschreibung
\usepackage[ngerman]{babel} % Neue Rechtschreibung


%TimesNewRoman Schrift,
%\usepackage{times}

%Zum Einbinden von Grafiken:
\usepackage{graphicx}

\pagestyle{headings}
\pagestyle{plain}

%F|r TeX Zeichen in beliebigen eingebundenen Grafiken:
%\usepackage{psfrag}
\setlength{\textheight}{26.0cm}
\setlength{\textwidth}{15.5cm}

\setlength{\parskip}{6pt}
\setlength{\parindent}{0pt}
\setlength{\oddsidemargin}{-0.0cm}
\setlength{\topmargin}{-1.5cm}
\parindent 0.0cm
\sloppy
\frenchspacing



\newcommand{\sF}{\mathcal{F}}
\newcommand{\sH}{\mathcal{H}}
\newcommand{\sB}{\mathcal{B}}
\newcommand{\sD}{\mathcal{D}}
\newcommand{\sN}{\mathcal{N}}
\newcommand{\sO}{\mathcal{O}}
\newcommand{\mR}{\mathbb{R}}     % blackboard bold R
\newcommand{\mN}{\mathbb{N}}     % blackboard bold N
\newcommand{\mNO}{\mathbb{N}_0}
\newcommand{\mP}{\rm I\hspace{-0.7mm}P}
\newcommand{\mD}{\mathbb{D}}
\newcommand{\mF}{\mathbb{F}}
\newcommand{\mZ}{\mathbb{Z}}
\newcommand{\mQ}{\mathbb{Q}}
\newcommand{\bD}{\mathbf{D}}
\newcommand{\mE}{I\!\!E}
\newcommand{\si}{{(i)}}
\newcommand{\coloneqq}{\mathrel{\mathop:}=}
\newcommand{\eqqcolon}{=\mathrel{\mathop:}}
\newcommand{\norm}[1]{\left\lVert #1 \right\rVert}
\newcommand{\ip}[2]{\left< #1 , #2 \right>}
\newcommand{\Dom}{\text{Dom }}
\newcommand{\indicator}{\mathbf{1}}
\newcommand{\wh}[1]{\widehat{#1}}
% \newcommand{\cn}[1]{\citeasnoun{#1}}

\newcommand{\ind}{\hspace{0.2in}}
\newcommand{\ftext}[1]{\text{\ind\footnotesize #1}}
\newcommand{\comments}[1]{}

\usepackage[latin1]{inputenc} % F¸r Umlaute
%==============================================================================

\input{../def}
\begin{document}

\setlength{\topmargin}{-2.5cm}


\begin{center} {\LARGE \bf Pricing II } \\[5mm]
                {\large Model Risk Question }
\end{center}


\vspace{0.3cm}

\pagestyle{empty}

%-----------hier-beginnen-die-Aufgaben----------------------

\begin{enumerate}

%%%%%%%%%%%%%%%%%%%%%%%%%%%%%%%%%%%%%%%%%%%%%%%%%%%%%%%%%%%%%%%%%%%%%%%%%%%%%%%%%%%%%%
\vspace{0.3cm}

\item[]{\bf Q1} 

\begin{itemize}
\item[(a)]
Describe the difference between a Bayesian Model Averaging and a Utility-based Worst-Case
approach. 
\item[(b)] Assume that you need to price a derivative instrument in an incomplete market situation. What is Cont's suggestion to measure model risk?
\item[(c)] 
Consider in a basic GBM world the pricing of a call option with maturity $T$. 
Assume alternative diffusion models
\begin{equation}
\Q_i\;:\; dS(t) = S(t)(r dt + \sigma_i(t) dW(t))
\end{equation}
where $ \sigma_i: [0,t] \rightarrow [0, \infty[ $ is a bounded deterministic volatility function.
We observe prices of a traded European call (strike $K$, maturity $T$) with prices $C^*$ and
$\Sigma$ the implied Black-Scholes volatility.
\begin{itemize}
\item[(i)]
Formulate the calibration condition for the given models.
\item[(ii)]
Let $0< T_1 <T$ and consider two models. The first on with constant volatility $\sigma_1$ and the second one with a volatility $\sigma_2 < \sigma _1$ for $0 \leq t <T_1$ and a constant volatility for $T_1\leq t \leq T$. Specify  the  volatilities of the models in terms of the implied volatility $\Sigma $.
\item[(iii)]
Consider the $\Delta$ at $t=0$ and $t=T_1$ for both models for the European call with maturity $T$. Which is more sensitive to changes of the underlying at $t=0$ and $t=T_1$? 
\item[(iv)]
Consider  a  European call with strike $K$ and maturity $T_1$. What is Cont's measure of model risk for this option regarding the two models. 
\end{itemize}
\end{itemize}

\item[]{\bf Solution}
\begin{itemize}
\item[(a)]
Describe the difference between a Bayesian Model Averaging and a Utility-based Worst-Case
approach.

{\it Answer:} The Bayesian Model is calculates model dependent quantities by averaging over  expectation of the models. The worst-case approach has its foundations in the {\it MaxMin} approach as a robust version of expected utility. Assume $U$ is a utility function, then
the worst case approach considers 
$\max _X \min_{\prob \in {\cal P}} \EX_{\prob}(U(X))$. 
 
\item[(b)] Assume that you need to price a derivative instrument in an incomplete market situation. What is Cont's suggestion to measure model risk?

{\it Answer:} 
For $X$ a derivative we associate with
$\Gamma(X)$ the ask price and with $-\Gamma(-X)$ its bid price.
Cont's suggestion is
$$
\Gamma^u(X)=\sup_{\Q \in {\cal Q}} \EX_{\Q} \; \mbox{ and } \;
\Gamma^l(X)= -\Gamma^u(-X)=\inf_{\Q \in {\cal Q}} \EX_{\Q}.
$$
Cont measures the model risk as 
$$
\Gamma^u(X)- 
\Gamma^l(X).
$$



\item[(c)] 
Consider in a basic GBM world the pricing of a call option with maturity $T$. 
Assume alternative diffusion models
\begin{equation}
\Q_i\;:\; dS(t) = S(t)(r dt + \sigma_i(t) dW(t))
\end{equation}
where $ \sigma_i: [0,t] \rightarrow [0, \infty[ $ is a bounded deterministic volatility function.
We observe prices of a traded European call (strike $K$, maturity $T$) with prices $C^*$ and
$\Sigma$ the implied Black-Scholes volatility.
\begin{itemize}
\item[(i)] 
Formulate the calibration condition for the given models.

{\it Answer:}  
\begin{equation}
\frac{1}{T} \int_0^T  \sigma_i(s)^2 ds = \Sigma^2,
\end{equation}
\item[(ii)]
Let $0< T_1 <T$ and consider two models. The first one with constant volatility $\sigma_1$ and the second one with a volatility $\sigma_2 < \sigma _1$ for $0 \leq t <T_1$ and a constant volatility for $T_1\leq t \leq T$. Specify  the  volatilities of the models in terms of the implied volatility $\Sigma $.

{\it Answer:}
$$
\sigma_1 = \Sigma,
$$
$$
\sigma_2(t) = \sigma_2 \IF_{[0,T_1]}+ \sqrt{\frac{T\Sigma^2 -T_1\sigma_2^2}{T-T_1}}\IF_{]T_1, T]}.
$$
\item[(iii)]
Consider the $\Delta$ at $t=0$ and $t=T_1$ for both models for the European call with maturity $T$. Which is more sensitive to changes of the underlying at $t=0$ and $t=T_1$? 

{\it Answer:}
We know 
\begin{align*}
  \Delta = \frac{\partial}{\partial S}Call_{BS}(S,K,\sigma,r,t,T) = \Phi(d_1)
\end{align*}
where, as usual, $\Phi$ denotes the c.d.f. of the standard normal distribution and
\begin{align*}
  d_1 = \frac{\log \left( \frac{S}{K} \right) + \left( r + \frac{\sigma^2}{2}
  \right)(T-t)}{\sigma \sqrt{T-t}}.
\end{align*}
At $t=0$ both models have the same $\Delta$ as they are both valued with the implied volatility $\Sigma$.
At $t=T_1$ the  $\Delta$ of model 2 is higher as the remaining volatility is higher. So it is more sensitive.  

\item[(iv)]
Consider  a  European call with strike $K$ and maturity $T_1$. What is Cont's measure of model risk for this option regarding the two models. 

{\it Answer:}
By monotonicity of the BS-formula in terms of volatility the Cont model bounds are
$$
\Gamma^u(X)=C^{BS}(K, T_1; \sigma_1)  \A \Gamma^l(X)=C^{BS}(K, T_1; \sigma_2).$$


\end{itemize}
\end{itemize}



\end{enumerate}

%%%%%%%%%%%%%%%%%%%%%%%%%%%%%%%%%%%%%%%%%%%%%%%%%%%%%%%%%%%%%%%%%%%%%%%%%%%%%%%%%%%%%%%
\vspace{0.5cm}


\end{document} 