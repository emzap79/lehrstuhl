% !TEX root = QCF_ss14UDE.tex
\section{Emission Trading Schemes}

\frame{\frametitle{Example for Emission Trading}
\begin{itemize}
\item<1-> Consider two companies A and B each emitting 100 000 metric tons of $\textnormal{CO}_2$ per year
\item<2-> Each has been allocated 95 000 metric tons under its national allocation plan
\item<3-> Credits are trading at 10\euro\ per metric ton
\item<4-> Company A can cut 10 000 metric tons of emission at 5\euro\ per ton (marginal abatement costs, MAC)
\item<5-> Company B has MAC of 15\euro\ per ton
\item<6-> Company A receives 50 000\euro\ for its surplus and covers the costs of its own reduction
\item<7-> Company B meets the cap at cost 50 000\euro\ instead of 75 000\euro

\end{itemize}
}

\subsection[Overview]{Overview of different ETS}
\begin{frame}
  \frametitle{Overview of different emission trading systems  -- USA}
  \begin{itemize}
  \item<1-> 
  Regional Greenhouse Gas Initiative (launched in 2009)
        The Regional Greenhouse Gas Initiative (RGGI) was the first mandatory cap-and-trade program in the United States to limit carbon dioxide (CO2) from the power sector. It consists of Connecticut, Delaware, Maine, Maryland, Massachusetts, New Hampshire, New York, Rhode Island, and Vermont.
\item<2-> 
   California started an ETS in 2012. 
   
   \url{http://www.arb.ca.gov/cc/capandtrade/capandtrade.htm} 
   
  and  
  
  \url{http://www.c2es.org/us-states-regions/key-legislation/california-cap-trade}        
\end{itemize}      
\end{frame}
          
  \begin{frame}
  \frametitle{Overview of different emission trading systems}
  \begin{itemize}
    \item<1-> 
 Hubei Province became the sixth jurisdiction in China to launch a pilot carbon emissions trading program, joining Shenzhen, Shanghai, Beijing, Tianjin, and Guangdong Province. In the coming months, two additional programs will be introduced in Chongqing and Qingdao.

  \item<2-> \textbf{Other GHG trading systems}
        \begin{itemize}
        \item Australian ETS was supposed to start in 2014. There was a  tax-based first phase. The new government abolished the tax and shelved the plan of an ETS.
        \url{http://www.climatechange.gov.au/}
        \item New Zealand ETS (New Zealand, launched in 2008)
        \url{https://www.climatechange.govt.nz/emissions-trading-scheme}
        \end{itemize}
   \end{itemize}
\end{frame}

%\subsection{The EU ETS}
%8. Folie
\begin{frame}
  \frametitle{Characteristics of EU ETS ($\textnormal{CO}_2$)}
  \begin{itemize}
  \item<1-> EU ETS is split up into three phases
  \begin{itemize}
  \item \textbf{Phase I (2005-07)}
  \item \textbf{Phase II (2008-12)} coinciding with commitment period of Kyoto protocol
  \item \textbf{Phase III (2013-20)} inducing significant changes compared to the two previous periods, according to Directive 2009/29/EC
  \end{itemize}
  \item<2-> Scheme covers approximately 12,000 large emitters in the EU that are responsible for 50\% of total $\textnormal{CO}_2$ emissions. Regulated sectors include energy industry, combustion, cement, etc.
  \item<3->  Emission allowances are traded mostly OTC (approx 60\%), bilateral (approx 10\%) and on eight different exchanges (approx 30\%):
  ECX in London, Nord Pool in Oslo, Powernext in Paris, EEX in Leipzig, The Green Exchange (NYMEX), Sende $\textnormal{CO}_2$, EXAA, New Values Climex.
  \end{itemize}
  \end{frame}
%Sources: DB Research (2011), ETS (Directive 2009.29.EC), http://eur-lex.europa.eu/Notice.do?mode=dbl&lang=en&ihmlang=en&lng1=en,de&lng2=bg,cs,da,de,el,en,es,et,fi,fr,hu,it,lt,nl,pl,pt,ro,sk,sl,&val=463502:cs&page=

  \begin{frame}
  \frametitle{Characteristics of EU ETS ($\textnormal{CO}_2$) - Phases I and II}
  \begin{itemize}
  \item<1-> Process steps concerning the distribution of the allowances according to Phases I and II:
  \begin{itemize}
  \item Each country submits a NAP (National Allocation Plan) to the European Commission (EC)
  \item EC adjusts NAPs if necessary and countries distribute EUAs among regulated firms according to the final NAP as approved by the EC
  \end{itemize}
   %are distributed in each phase 200x EU defines overall target and targets for each country
  %200x Each country allocates EUAs to different sectors and companies via NAP (National Allocation Plan)
  % March Allocation of allowances for the current year (in advance)
  %January Compliance time, i.e. pollutors have to deliver allowances for the last year to the regulator
  \item<2-> At the end of the current phase regulated firms have to pay a fine of 100 Euro for each emitted ton of $\textnormal{CO}_2$ that is not covered by an allowance (excess emissions penalty).
  \end{itemize}
\end{frame}

  \begin{frame}
  \frametitle{Characteristics of EU ETS ($\textnormal{CO}_2$) - Phase III}
  \begin{itemize}
  \item<1-> Process steps concerning the distribution of the allowances according to Phase III:
  \begin{itemize}
  \item NAPs were  abolished
  \item Since 2013  \textbf{Auctioning} has been  introduced as default method of initial allowance allocation
  \item The initial auctioning share  in the power sector was 100\%
  \item For all other sectors the initial auction share was  20\% and is to be increased to 70\% by 2020 (and to 100\% respectively by 2027)
  \item Non-auctioned allowances will be distributed on the basis of benchmarks
  \end{itemize}
   %are distributed in each phase 200x EU defines overall target and targets for each country
  %200x Each country allocates EUAs to different sectors and companies via NAP (National Allocation Plan)
  % March Allocation of allowances for the current year (in advance)
  %January Compliance time, i.e. pollutors have to deliver allowances for the last year to the regulator
  \item<2-> The amount of the excess emissions penalty in the third phase is indexed to the annual inflation rate of the Eurozone.  %Source: DB Research 2011, CMA 2011
  \end{itemize}
\end{frame}

\begin{frame}
  \frametitle{Structural Reform of the EU ETS}
 \begin{itemize}
  \item<1->  At the start of phase III in 2013 the surplus of allowances stood at almost two billion, double its level in early 2012.
  \item<2-> As a short-term measure, the Commission has posted the auctioning of 900 million allowances until 2019-2020 to allow demand to pick up. 
   \item<3-> 
   As back-loading is only a temporary measure, the Commission proposes to establish a market stability reserve at the beginning of the next trading period in 2021.
 \url{http://ec.europa.eu/clima/policies/ets/reform/index_en.htm}
   \end{itemize}
\end{frame} 

\subsection{ETS and Flexibility}

\frame{\frametitle{{\it Where} flexibility}
\begin{itemize}
\item<1-> The ambition should be a global market. However, there are various constraints such as policy differences, differences in the traded good, etc.
\item<2-> Linking different national and regional trading systems can approximate a global market.
\item<3-> Linking markets increases liquidity and thus reduces the cost of trading.
\item<4-> However, different designs of schemes have to be taken into account.
\end{itemize}
}

\frame{\frametitle{{\it Where} flexibility: Gains from Trade}
\begin{center}
\begin{figure}[h!]
\centering
\includegraphics[width=0.9\textwidth, height=0.8\textheight]{../../../pics/carbon-gainsfromtrade.pdf}
%\caption{}
\end{figure}
\end{center}
} %Source: VividEconomics (2008) S.15

\frame{\frametitle{{\it When} flexibility  -- Banking}
\begin{itemize}
\item<1-> effectively increases the depth and liquidity of the market, reducing price volatility by
making current prices a function of a longer time span of activity, rather than being
entirely determined by events today;
\item<2-> creates an incentive for firms to take early action;
\item<3-> firms with banked allowances have a
vested interest in higher prices and the continuation (and success) of the system, to maximise the value of
their allowance assets;
\item<4-> banking can also prevent a price collapse between commitment
periods;
\end{itemize}
}
\frame{\frametitle{{\it When} flexibility -- Borrowing}
\begin{itemize}
\item<1-> the regulator may not be well-equipped to assess the credit worthiness and
solvency of firms who borrow allowances, who thereby become debtors;
\item<2-> borrowing enables firms to delay action if they assume that targets will prove too
onerous and will subsequently be softened;
\item<3-> firms with borrowed allowances have an active interest to lobby for weaker targets,
or even for scrapping emissions trading altogether, so that their debts are cancelled;
\item<4-> the political desire to (be seen to) act early, and potential benefits of early action,
also imply that politicians may prefer to place constraints on borrowing;

\end{itemize}
}
\frame{\frametitle{{\it When} flexibility}
\begin{itemize}
\item<1-> banking is usually allowed between periods (Exemption EU ETS Phase I);
\item<2-> there is typically no borrowing (or only very limited);
\item<3-> when there are limits on borrowing between periods, the length of the commitment period is relevant to 'when' flexibility
and to market efficiency
\begin{itemize}
\item investments to reduce
emissions may require many years for investors to recover their costs
\item in case of short periods, investors have to guess the emissions caps set by
future governments, and attempt to anticipate changes in the underlying structure of the
carbon trading framework
\end{itemize}
\end{itemize}
}


\subsection{ETS and Tax}
\frame{\frametitle{ETS vs Tax: Generalities}
\begin{itemize}
\item<1-> Since compliance costs are uncertain the choice of instrument depends on the relative curvatures
of the marginal benefit curve and the marginal abatement costs curve.
\item<2-> In case of CO2, where damage does not depend on the flow of emissions but on their accumulation in
the atmosphere, scientific results suggest that a carbon tax is more economically efficient under
uncertainty than emissions trading.
\item<3-> In practice, however, the analysis of efficiency
under uncertainty has had little influence on the choice of policy instruments. The preference
for carbon trading over carbon taxes is driven largely by powerful political economy concerns.
Trading systems are easier to implement politically.
  \end{itemize}
}

\frame{\frametitle{ETS vs Tax}
\begin{itemize}
\item<1-> The market for emission
reductions has a demand schedule, which is determined by the marginal abatement costs of
regulated agents, and a supply schedule, which is determined by policy.
\item<2-> Under a pure tax
system, the supply of allowances is infinitely elastic. The market is effectively supplied with as
many allowances as agents wish to buy at a fixed price (the tax rate).
\item<3-> Under a pure allowance
system, supply is completely inelastic as the amount of allowances is exogenously fixed.
\item<4-> Hybrid systems create a supply curve that is neither fully flat (a pure tax) nor fully vertical
(pure cap-and-trade) but (stepwise) upward sloping.
\end{itemize}
}


\frame{\frametitle{Price ceilings and price floors}
\begin{itemize}
\item<1-> a price ceiling and floor provide significantly
greater clarity to investors to deliver dynamic efficiency (in the form of optimal investment
over longer time frames).
\item<2-> the price floor would guarantee a certain minimum return on
investment in low-carbon technologies, reducing the risk faced by innovating firms.
\item<3-> the price ceiling may enhance policy credibility. Because it caps the costs of
compliance, a ceiling reduces the risk of a policy reversal if abatement costs turn out to be
injuriously high.
\end{itemize}
}

\frame{\frametitle{Price ceilings and price floors}
\begin{itemize}
\item<1-> a price ceiling can be established through an unlimited commitment from the regulator to sell allowances onto the market at the price ceiling
\item<2-> drawback: compliance with the emissions cap is sacrificed
\item<3-> a price floor can be established
through an unlimited commitment from the regulator to buy back
allowances from the market at the price floor
\item<4-> drawback: the floor would be achieved at the risk of
imposing a liability on the public balance sheet.
\end{itemize}
}

\frame{\frametitle{Hybrid System with Safety Valve}


\begin{center}
\begin{figure}[h!]
\centering
\includegraphics[width=0.9\textwidth, height=0.8\textheight]{../../../pics/hybridscheme.pdf}
%\caption{}
\end{figure}
\end{center}
%Source: new and Old policies S.12


}

\frame{\frametitle{Ecological Effectiveness}
\begin{itemize}
\item<1-> Emission trading systems sets a cap, which in theory establishes precisely the level of emissions which is desired (100\% effectiveness)
\item<2-> Tax system cannot guarantee an exact amount of emissions as an  outcome. The tax level is set under uncertainty about the marginal abatement costs.
\item<3-> Hybrid systems reach the target value as long as the system is in its trade area. As soon as the trigger level is reached the ecological target becomes diluted due to the additional certificates.
\end{itemize}
}

\frame{\frametitle{Political Feasibility}
Political feasibility is the level of acceptance a policy has in the public. A major factor is the number of people affected.
\begin{itemize}
\item<1-> Emission trading systems have a good political enforceability since costumers feel no direct effect by government action.
\item<2-> Price rises are blamed on companies, especially in case allocation is free. In case of auctioning extra government revenues can be used for redistribution.
\item<3-> Tax systems are generally met with scepticism (especially in the US). Additional costs on emissions will effect consumers more directly via cost increases.
\item<4-> Hybrid systems also generate an extra revenue after the trigger is met, which may be viewed positively.
\end{itemize}
}



\subsection{Multiple Policy Instruments}


\frame{\frametitle{Multiple Instruments I}
\begin{itemize}
\item<1-> Emission regulation is directed at internalizing externalities and economic theory indicates that only one instrument is needed to
internalize one externality.
\item<2-> Policy often involves multiple instruments such as command-and-control regulation, subsidies, taxes, trading schemes, etc.
\item<3-> This process reflects an ad-hoc policy-accretion process driven by the multiplicity of national institutions or ...
\item<4-> the temptation of politician to fix everything.
\end{itemize}
}



\frame{\frametitle{Multiple Policy Instruments II}
\begin{itemize}
\item<1-> Combinations of permit trading schemes, carbon taxes, technology-specific subsidies and regulatory standards
\item<2-> Taxes introduced by Sweden, Norway, Denmark, Ireland
\item<3-> Subsidies for renewable energy by Germany, Spain
\item<4-> Academic literature gives justification for multiple instruments used in a complimentary way, i.e. hybrid systems. Justification in terms of presence of multiple market failures, asymmetric information, principle-agent relation
 \end{itemize}
}


\subsubsection{Symmetric Policy Combinations}

\frame{\frametitle{Tax and Trade}
\begin{itemize}
\item <1-> Carbon tax $t$ (Euro per tonne)
\item <2-> Cap-and-trade scheme with price $p$
\item <3-> Firm must pay tax and procure certificates for emissions
\begin{tabular}{ll}
$e_0$ & baseline emission \\
$e$ & emissions after abatement \\
$a$ & $=e_0 - e$; \\
$c(a)$ & abatement costs, $c' > 0$, $c'' > 0$ \\
\end{tabular} \\
\item <4-> Optimization problem
\begin{align}
\min_{e} & \left\{c(e_0-e)+te+pe\right\} = \min_{e}\left\{f(e)\right\}
\end{align}
\item <5-> First-order-condition (*)
\[
c'(e_0-e^*)=t+p
\]
\end{itemize}
}

							% Folie 5

\frame{\frametitle{Tax and Trade: optimization problem}
\begin{itemize}
\item <1-> Since $c'>0$ we can invert $e^*=e_0-c'^{-1}(t+p)=e^*(t,p)$ \\
\item <2-> Since $f''(e)=c''(e_0-e)>0$ it is a minimum \\
\item <3-> Differentiation of (*) w.r.t. the variable t:
\[
\frac{\partial}{\partial{t}}c'(e_0-e^*(t,p))=\frac{\partial}{\partial{t}}(t+p)
\]
\[
-c''(e-e^*(t,p)) \frac{\partial}{\partial{t}} (e^*(t,p))=1
\]
\[
e^*_t=-\frac{1}{c''} < 0
\]
\item <4-> Thus increases in tax reduce emissions; By symmetry $e^*_p = -\frac{1}{c''} < 0$ increases in price reduce emissions and $e^*_t=e^*_p$
\end{itemize}
}

							% Folie 6

\frame{\frametitle{Tax and Trade: optimization problem}
Assume a cap $E$, $n$ identical firms, such that the aggregate emissions are $E=ne^*$. \\
For constant tax $t$ formally
\[
dE = ne^*_pdp \textnormal{, so } \frac{dp}{dE} = (ne^*_p)^{-1} < 0
\]
that is an increase in the cap reduces the price. \\
For fixed cap we have
\[
ne^*_tdt+ne^*_pdp=0
\]
\[
\frac{dp}{dt}=-\frac{e^*_t}{e^*_p}=-1.
\]
That is ''a small increase in tax results one-for-one in an equivalent reduction of the permit price". \\
}

							% Folie 7

\frame{\frametitle{Tax and Trade: optimization problem}
\begin{center}
\begin{figure}[h!]
\centering
\rotatebox{0}{
\scalebox{0.6}{
\includegraphics[width=1.35\textwidth, height=\textheight]{../../../pics/cost-of-switching.pdf}}}
\caption{Relation of tax and permit price.}
\end{figure}
\end{center}
%MAC = cost of switching,  % marginal abatement cost;
%$a = e_0 - e$
}

							% Folie 8

\frame{\frametitle{Tax and Trade}
\begin{itemize}
\item <1-> MAC to use (for example) wind generation is how much it costs to generate electricity by wind compared with the cheapest alternative. \\
\item <2-> So tax increase reduces MAC, because it reduces the opportunity cost of wind generation. Thus opportunity cost of abatement have decreased since the firm will pay a higher penalty not to abate.
\end{itemize}
}

\frame{\frametitle{Tax and Trade}
\begin{itemize}
\item<1-> Model shows that an increase in tax reduces the permit price
\item<2-> No additional abatement will be achieved
\item<3-> The average carbon price will be reduced and the risk that the price system will collapse is increased
\end{itemize}
}


\frame{\frametitle{Subsidies and Trade}
\begin{itemize}
\item<1-> Model shows that an increase in subsidies reduces the permit price
\item<2-> A higher level of subsidy for an abatement technology reduce the abatement cost at any given level of production
\item<3-> For a given emission cap, demand for permits will be lower and so will be the price.
  \end{itemize}
}


							% Folie 9

\frame{\frametitle{Subsidies and Trade: optimization problem}
Subsidy $s$ applies equally to all firms and technologies provided accordingly to the level of abatement $a = e_0 - e$ \\
\begin{itemize}
\item <1-> Optimization problem
\begin{align}
\min_{e} & \left\{c(e_0-e)-s(e_0-e)+pe\right\} = \min_{e}\left\{f(e)\right\}
\end{align}
\item <2-> First-order-condition
\[
f'(e)=-c'(e_0-e)+ s+p = 0
\]
so
\[
c'(e_0-e)=s+p
\]
and
\[
e^*=e_0-c'^{-1}(s+p)=e^*(s,p)
\]
\end{itemize}
}

							% Folie 10

\frame{\frametitle{Subsidies and Trade: optimization problem}
The same calculation as above shows
\[
\frac{dp}{ds} = -\frac{e^*_s}{e^*_p}=-1.
\]
The higher the subsidy, the lower the permit price.
}

\frame{\frametitle{Trade and Trade}
\begin{itemize}
\item<1-> Two separate trading programs apply upstream to firms that produce electricity and downstream to firms that consume it.
\item<2-> Example: UK with EU ETS for electricity producers and Carbon Reduction Commitment for firms and organizations that are primarily electricity consumers.
\end{itemize}
}

\frame{\frametitle{Trade and Trade}
Compliance upstream
\begin{itemize}
\item <1-> higher electricity price
\item <2-> equivalent to tax on energy consumption (linked to carbon price)
\item <3-> downstream implicit price of carbon
\end{itemize}
Increase in upstream permit prices
\begin{itemize}
\item <1-> increase tax downstream
\item <2-> decrease downstream carbon price (see part 1)
\end{itemize}
}


\subsubsection{Asymmetric Policy Combinations}
\frame{\frametitle{Uniliteral Tax and Trade}
\begin{itemize}
\item<1-> Firms are identical
\item<2-> Tax only affects a fraction $f$ of firms
\item<3-> Uniliteral tax leads to diverging marginal costs and so increased mitigation costs.
\end{itemize}
}


\frame{\frametitle{Unilateral tax and trade}
\begin{itemize}
\item <1-> Firms are identical with optimal emissions $e^*(t,p)$; $e^*_p = e^*_t = -\frac{1}{c''}$
\item <2-> Tax affects only a fraction of firms $f$ (a fraction of the system-wide emissions) $0 < f < 1$ \\
 So
\begin{align}
E = fne^*(t,p)+(1-f)ne^*(p)
\end{align}
Assuming
\[
e^{*f}_p = e^{*(1-f)}_p
\]
We find
\[
dE = fne^*_tdt+ne^*_pdp
\]
\end{itemize}
}

							% Folie 5


\frame{\frametitle{Unilateral tax and trade}
Fix $E$, then the impact of the tax on the permit price is
\[
0 =  fe^*_tdt+e^*_pdp \Rightarrow \frac{dp}{dt} = -f \frac{e^*_t}{e^*_p} = -f
\]
The impact of the tax change on permit prices is diluted and different for the two categories
\[
\frac{de^{*f}}{dt} = e^{*f}_t + e^{*f}_p \frac{dp}{dt} = -\frac{1}{c''} + (-\frac{1}{c''})(-f) = -(1-f)\frac{1}{c''} < 0
\]
\[
\frac{de^{*1-f}}{dt} = e^{*(1-f)}_p\frac{dp}{dt} = \frac{f}{c''} > 0
\]
So the emissions for firms subject to tax fall, but the emissions for firms with no tax increase.
}


							% Folie 6

\frame{\frametitle{Unilateral tax and trade}
The impact on marginal costs for taxed firms is \\
(use $c'(e_0-e^*) = t+p)$)
\[
\frac{d(c^f)'}{dt} = \frac{dt}{dt} + \frac{dp}{dt} = 1-f > 0
\]
and for untaxed firms
\[
\frac{d(c^{1-f})'}{dt} = \frac{dp}{dt} = -f < 0
\]
Diverging marginal costs mean that the gains from trade are, at least in part, reversed. (Remember firms where identical, now mitigation costs increase.)
}

							% Folie 7


\frame{\frametitle{Technology Policies and Trade}
\begin{itemize}
\item<1-> A technology-specific measure affects only part of the MAC curve and will lead to a compositional reorientation of the curve
\item<2-> In EU ETS fuel switching from coal to gas is targeted
\item<3-> Trading price may fall, but mitigation cost will rise in general.
\end{itemize}
}





