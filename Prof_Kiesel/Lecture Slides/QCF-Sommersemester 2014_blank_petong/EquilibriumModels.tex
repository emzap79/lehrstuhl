% !TEX root = QCF_ss14UDE.tex
\section{Modelling Permit Price Dynamics}
\section{Equilibrium Models}
\subsection{Deterministic Equilibrium Model}

% frametitle
{Rubin 1996: Firm i's optimization problem}
Firm i minimizes its cost by buying/selling an optimal quantity of emission permits and by emitting an optimal quantity of emissions, i.e.
\begin{align}
\min_{\theta_i, e_i} & \left\{ \int_0^T e^{-rt} [C_i(e_i(t)) + P(t)\theta_i(t)]dt \right\} \\
\textnormal{subject to }
&\dot{B}_i = S_i(t) - e_i(t) + \theta_i(t) \\
&            B_i(0) = 0 \textnormal{ and } B_i(t) \ge 0 \\
&            e_i(t) \ge 0
\end{align}

{Explanation of variables}
\begin{tiny}
\begin{tabular}{cl}
$e_i(t)$ & quantity of emissions \\
$\theta_i(t)$ & quantity of emission permits bought or sold \\
$S_i(t)$ & endowment of emissions \\
$B_i(t)$ & level of emissions in the bank\\
$C_i(e_i(t))$ & abatement cost function where $C'_i(e_i) < 0$ and $C''_i(e_i) > 0$ \\
$r$ & interest rate \\
\end{tabular}
\end{tiny}

%16. Folie

% frametitle
{Rubin 1996: Market equilibrium}
An intertemporal market equilibrium in emission permits over a T-period horizon consists of \\
$\qquad P^*(t) \ge 0$ (permit price) \\
$\qquad \theta^*(t) = (\theta^*_1(t), \ldots, \theta^*_N(t))$ (vector of optimal trading volumes) \\
$\qquad E^*(t) = (e^*_1(t), \ldots, e^*_N(t))$ (vector of optimal emission levels) \\
such that for a given $P^*(t)$,
$\theta^*(t)$ and $E^*(t)$ minimize each firm's costs subject to each firm's constraints as given in (2) - (4) and \\
the following two conditions hold

% begin itemize



	Market clearing condition on permits \\
$
\qquad \sum_{i=1}^N \theta_i^*(t) = 0
$


	Terminal stock condition \\
$
\qquad P^*(T)\sum_{i=1}^N B^*_i(T) = 0
$

% end itemize

%17. Folie

% frametitle
{Rubin 1996: Joint optimization problem}
A fictitious central planner minimizes total costs by choosing optimal quantities of emissions, i.e.
\begin{align}
\min_{e_1, \ldots, e_N} &\left\{ \int_0^T e^{-rt} \sum_{i=1}^N C_i(e_i(t)) dt \right\} \\
\textnormal{subject to }
&\dot{B}(t) = \sum_{i=1}^N \left(S_i(t) - e_i(t)\right) \\
&            B(0) = 0 \textnormal{ and } B(t) \ge 0 \\
&            e_i(t) \ge 0 \quad \textnormal{ for all } i = 1, \ldots, N
\end{align}
{Explanation of variables}
\begin{tiny}
\begin{tabular}{cl}
$S_i(t)$ & firm i's endowment of emissions \\
$B(t)$ & sum of emissions banked by the firms at time t\\
$C_i(e_i(t))$ & firm i's abatement cost when emitting $e_i(t)$ where $C'_i(e_i) < 0$ and $C''_i(e_i) > 0$ \\
$r$ & interest rate \\
\end{tabular}
\end{tiny}

%18. Folie

% frametitle
{Rubin 1996: Theorem (Market equilibrium and joint optimization problem)}

% begin enumerate



	There exists an intertemporal market equilibrium in emission permits over a T-period horizon


	The market equilibrium solution is at least as inexpensive as the result of the joint cost optimization

% end enumerate

%19. Folie

% frametitle
{Rubin 1996: Theorem (Permit price)}

% begin enumerate



	The permit price equals the marginal abatement costs
\[
P(t) = - C'_i(e_i)
\]


	The permit price

% begin itemize



	follows Hotelling's rule if banking/borrowing are allowed


	grows at a rate less than the interest rate r if there are restrictions on borrowing
\[
\frac{\dot{P}}{P} =
		\left\{ \begin{array}{ll}
		  r  &
			  \mbox{if $\Phi_i = 0$ } \\
			r - \frac{e^{rt}\Phi_i}{P} &
			  \mbox{if $\Phi_i > 0$}
		\end{array}
	  \right.
\]
where $\Phi_i$ is the adjoint variable of the borrowing constraint

% end itemize

% end enumerate

Proof: \\
Follows from evaluating the necessary conditions (hereby use the Hamiltonian)

%20. Folie

% frametitle
{Rubin 1996: Necessary (and sufficient) conditions for firm i's minimization problem}
\begin{tiny}
The \textbf{Hamiltonian} is given by
\[
H_i = e^{-rt} [C_i(e_i(t)) + P(t)\theta_i(t)] + \lambda_i[S_i - e_i + \theta_i]
\]
and the \textbf{generalized Hamiltonian} is given by
$
L_i = H_i  - \Phi_i B_i - \tau_i e_i
$ \\
Hence
\begin{align}
\dot{B}_i &= \frac{\partial L_i}{\partial \lambda_i} = S_i - e_i + \theta_i \\
\dot{\lambda}_i &= -\frac{\partial L_i}{\partial B_i} = \Phi_i \\
B_i &\ge 0, \quad \Phi_i \ge 0, \quad \Phi_i B_i = 0 \\
\frac{\partial L_i}{\partial e_i} &= e^{-rt}C'_i(e_i) - \lambda_i \ge 0 \\
e_i &\ge 0, \quad \tau_i \ge 0, \quad e_i \frac{\partial H_i}{\partial e_i} = 0 \\
B_i(T) &\ge 0, \quad \lambda_i(T) \le 0, \quad B_i(T) \lambda_i(T) = 0
\end{align}
\end{tiny}

\subsection{Stochastic Equilibrium Model}
%\subsubsection{Full Economy Model}

% frametitle
{Carmona et al. 2008: Firm i's optimization problem}
For given forward permit price $A$ and prices of the produced goods $S$ the firm i maximizes its expected terminal wealth by  buying/selling an optimal number of permits and producing an optimal quantity of goods, i.e.
\begin{align}
\sup_{\theta^i, \xi^i} \mathbb{E} \left[ \underbrace{S^i(\xi^i) - C^i(\xi^i)}_{production} + \underbrace{T^i(\theta^i)}_{trading} - \underbrace{\Pi \left(\varepsilon^i + e^i(\xi^i) - \Delta^i - \theta^i_T \right)^+}_{penalty} \right]
\end{align}

% frametitle
{Variables}

\begin{tiny}
\begin{tabular}{ll}
$S^i(\xi^i) =    \sum_{t=0}^{T-1} \sum_{j, k} S^k_t \xi^{i,j,k}_t$ & revenues from selling the produced goods \\
$C^i(\xi^i) =    \sum_{t=0}^{T-1} \sum_{j, k} C^k_t \xi^{i,j,k}_t$ & costs from producing the goods \\
$T^i(\theta^i) = \sum_{t=0}^{T-1} \theta^i_t (A_{t+1} - A_t) - \theta_T^i A_T$ & profit/loss from trading emission permits \\
$e^i(\xi^i) =    \sum_{t=0}^{T-1} \sum_{j, k} S^k_t \xi^{i,j,k}_t$ & firm i's emissions in [0,T] from the production\\
$\Delta^i = \sum_{t=0}^{T-1} \Delta^i_t$ & number of emission permits allocated to firm i in [0,T] \\*[12pt]
\end{tabular}

\begin{tabular}{ll}
$\varepsilon^i$ & quantity of firm i's emissions in [0,T] that cannot be controlled\\
$\theta^i_t$ & number of forward contracts on emission permits held by firm i at time t\\
$\Pi$ & penalty per emission unit \\
$S_t^k$ & price of product k\\
$C_t^{i,j,k}$ & firm i's marginal production costs of product k using production technology j\\
$e_t^{i,j,k}$ & emission factor of firm i, production technology j and product k \\
\end{tabular}
\end{tiny}

%22. Folie

% frametitle
{Market equilibrium}
A market equilibrium in emission permits consists of

% begin itemize



	
$\quad A^*$ (one-dimensional stochastic process for forward price on permits)


	$\quad S^*$ (multi-dim. stochastic process for the prices of the products)


	$\quad \theta^*$ (multi-dim. stochastic process of optimal trading strategies)


	$\quad \xi^*$ (multi-dim. stochastic process of optimal production strategies)

% end itemize

such that for given $A^*$ and $S^*$,
$\theta^*$ and $\xi^*$ lead to a situation where all the firms are satisfied by their strategy.

% frametitle
{Market equilibrium}
Formally
$
\qquad \mathbb{E}\left[ L^{A^*, S^*, i}\left(\theta^{*i},\xi^{*i}\right) \right] \ge \mathbb{E}\left[ L^{A^*, S^*, i}\left(\theta^{i},\xi^{i}\right) \right] \textnormal{ for all } \left(\theta^i, \xi^i \right)
$ \\
and the following two conditions hold

% begin itemize



	Market clearing condition on permits
$$
\sum_{i} \theta^{*i}_t = 0
$$


	Supply meets demand for each good
$$
\sum_{i, j} \xi^{*i,j,k}_t = D_t^k
$$

% end itemize

%23. Folie

% frametitle
{Global optimization problem}
A fictitious central planner minimizes expected total costs by producing an optimal quantity of goods $\xi^*$, i.e. it faces the optimization problem
\begin{align}
\inf_{\xi} \mathbb{E} \left[ \underbrace{C(\xi)}_{production} - \underbrace{\Pi \left(\varepsilon + e(\xi) - \Delta \right)^+}_{penalty} \right]
\end{align}
where \\
\begin{tabular}{ll}
$C(\xi) =  \sum_{i} C^i(\xi^i)$ & total production costs\\
$e(\xi) = \sum_{i} e^i(\xi^i)$ & total emissions from production in [0,T] \\
$\varepsilon = \sum_{i} \varepsilon^i$ & total emissions in [0,T] that are not controllable \\
$\Delta = \sum_{i} \Delta^i$ & total emission certificates handed out by the regulator \\
$\Pi$ & penalty per emission unit \\
\end{tabular}

%24. Folie

% frametitle
{Theorem: Market equilibrium and joint optimization problem}

% begin enumerate



	If $(A^*,S^*)$ is a market equilibrium with associated strategies $(\theta^*,\xi^*)$ then $\xi^*$ is a solution of the global optimization problem


	There exists a solution $\bar{\xi}$ of the global optimization problem


	If $\bar{\xi}$ is a a solution of the global optimization problem then $(\bar{A}, \bar{S})$ is a market equilibrium and the equilibrium allowance price process is almost surely unique

% end enumerate

%25. Folie

% frametitle
{Theorem: Equilibrium prices}
Let $(A^*,S^*)$ be a market equilibrium with associated strategies $(\theta^*,\xi^*)$ then
%\vspace{-0.5cm}

Forward prices on permits are almost surely given by \[
A^*_t = \Pi \cdot \mathbb{E} \left[ \chi_{\left\{ \varepsilon + e(\xi) - \Delta \ge 0 \right\}} | \mathcal{F}_t \right]
\]

% frametitle
{Theorem: Equilibrium prices}
 Spot prices $S^{*k}$ of the goods and the optimal production strategy $\xi^{*i}$ correspond to a merit-order-type equilibrium with adjusted costs $C_t^{i,j,k} + e^{i,j,k} A_t^*$, i.e. at time t and for each good k

% begin itemize



	all the production means of the economy are ranked by increasing adjusted production costs


	demand is met by producing from the cheapest production means


	k's equilibrium spot price is the marginal cost of production of the most expensive production means used to meet demand $D_t^k$

% end itemize

\[
S_t^{*k} = \max_{i,j} \left\{ \left(C_t^{i,j,k} + e^{i,j,k} A_t^* \right) \chi_{\left\{\xi_t^{i,j,k}>0\right\}}\right\}
\]

\subsection{Dynamics of CO2 permit prices}
\subsubsection{Reduced Equilibrium Model}

% frametitle
{Basic Model}

% begin itemize



	Risk-neutral companies with total initial endowment $e_0$


	Total emissions dynamics are
\begin{equation}
dy_t= \mu(t, y_t)dt + \sigma(t, y_t)dW_t
\end{equation}
with deterministic drift and volatility.


	Central planner who minimizes total expected cost over a trading period $[0,T]$ by deciding at any time instant
whether to costly abate some of the CO2 emissions or not.


	At the end of the period actual accumulated emissions and penalty costs are determined.

% end itemize

% frametitle
{Basic Model II}

% begin itemize



	$x_t$ are the total expected emissions over the trading period


	Then
\begin{equation}
x_t=-\int_0^tu_s ds + \EX_t\left[\int_0^T y_s ds \right]
\end{equation}


	$u_t$ is the optimal rate of abatement which is  actively chosen by the central planner.


	So $x_t$ is a controlled stochastic process.

% end itemize

% frametitle
{Total Emissions}

% begin itemize



	$x_T$ are the realized emissions that relate to a potential penalty function


	Without abatement total expected emissions  are
$$
x_0=\EX\left[\int_0^T y_s ds \right]
$$


	The dynamics of the total expected emissions are
\begin{equation}
dx_t=-u_t dt + G(t) dW_t
\end{equation}


	$G(t)$ is the volatility of the uncontrolled part of $x_t$ and depends both on the drift $\mu(t, y_t)$
and the volatility $\sigma(t,y_t)$ of the emission rate.

% end itemize

% frametitle
{Optimisation problem of the central planner I}
\begin{equation}
\max_{u_t} \EX_0\left[\int_0^Te^{-rt}C(t,u_t)dt + e^{-rT}P(x_T)\right]
\end{equation}
with
$$
\begin{array}{lll}
C(t,u_t) &=& - \frac{1}{2}c u_t^2 \\*[12pt]
P(x_T) &=& \min[0,p(e_0-x_T)]
\end{array}
$$

% frametitle
{Optimisation problem of the central planner II}

% begin itemize



	$C(t,u_t)$ are the abatement costs per unit of time. $c$ constant implies no change in technology occurs.
The quadratic form implies linearly increasing marginal abatement costs.


	$P(x_T)$ is the penalty function, with $p$ the penalty including all costs.


	$r$ is the constant interest rate.

% end itemize

% frametitle
{Solution of the control problem}
Let $V(t,x_t)$ be the expected value of the optimal policy given $x_t$. By a standard
Hamilton-Jacobi-Bellman argument we arrive at
\begin{equation}
V_t=-\frac{1}{2}(G(t))^2 V_{xx}-\frac{1}{2c}e^{rt}(V_x)^2
\end{equation}
with boundary condition
$$
V(T, x_T)=e^{-rT}P(x_T)
$$
and optimal control
$$
u_t=-\frac{1}{c} e^{rt}V_x
$$

% frametitle
{Permit Prices}
\begin{center}
\begin{figure}[h!]
\centering
\rotatebox{0}{
\scalebox{0.6}{
\includegraphics[width=1.45\textwidth, height= 1.2\textheight]{../pics/pic1-SUHW.pdf}}}

\end{figure}
\end{center}

% frametitle
{Permit Price Dynamics}

% begin itemize



	Recall that the permit price must equal the marginal abatement costs, so
\begin{equation}
S(t,x_t) = c u_t = -e^{rt} V_x(t,x_t)
\end{equation}


	Using It{\^o}'s formula and the HJB-PDE we find that the discounted permit price is a martingale.


	Its dynamics are
\begin{equation}
dS(t,x_t)= G(t)S_x(t,x_t) dW_t
\end{equation}

% end itemize

% frametitle
{Implied Permit Price Volatility}
\begin{center}
\begin{figure}[h!]
\centering
\rotatebox{0}{
\scalebox{0.6}{
\includegraphics[width=1.45\textwidth, height= 1.2\textheight]{../pics/pic2-SUHW.pdf}}}

\end{figure}
\end{center}

\subsubsection{Central Planner and Equilibrium}

% frametitle
{Individual Company Models}

% begin itemize



	Each individual company has an endowment $e_{i0}$


	Individual emissions dynamics are
\begin{equation}
dy_{it}= \mu(t, y_{it})dt + \sigma(t, y_{it})dW_{it}
\end{equation}
with deterministic drift and volatility.

% end itemize

% frametitle
{Individual Emissions}

% begin itemize



	$x_{it}$ are the total expected emissions of company $i$ over the trading period


	Then
\begin{equation}
x_{it}=-\int_0^tu_{is} ds -\int_0^tz_{is}ds + \EX_t\left[\int_0^T y_{is} ds \right]
\end{equation}


	$u_{it}$ is the individual rate of abatement


	and $z_{it}$ is the instantaneous amount of permits bought or sold.

% end itemize

% frametitle
{Individual Emissions Dynamics}

% begin itemize



	The dynamics of the total expected emissions are
\begin{equation}
dx_{it}=-[u_{it}+z_{it}] dt + G_i(t) dW_{it}
\end{equation}


	$G_i(t)$ is the volatility of the uncontrolled part of $x_{it}$ and depends both on the drift $\mu_i(t, y_{it})$
and the volatility $\sigma_i(t,y_{it})$ of the emission rate.

% end itemize

% frametitle
{Optimisation Problem for the individual Company}
\begin{equation}
\max_{u_{it}, z_{it}} \EX\left[\int_0^Te^{-rt}C_i(t,u_{it})dt - \int_0^T e^{-rt}S(t)z_{it}dt+ e^{-rT}P_i(x_{iT})\right]
\end{equation}
with $S(t)$ the permit price and
$$
\begin{array}{lll}
C_i(t,u_{it}) &=& - \frac{1}{2}c_i u_{it}^2 \\*[12pt]
P_i(x_{iT}) &=& \min[0,p(e_{i0}-x_{iT})]
\end{array}
$$

% frametitle
{Solution of the control problem}
Let $V^i(t,x_{it})$ be the expected value of the optimal policy for company $i$. By a standard
Hamilton-Jacobi-Bellman argument we arrive at
$$
\begin{array}{lll}
0&=\max_{u_{it},z_{it}}&\left[e^{-rt}(C_i(t,u_{it}) - S(t) z_{it})\right.\\*[12pt]
&&+\left.V^i_t -V_x^i(u_{it}+z_{it}) + \frac{1}{2}(G_i(t))^2 V^i_{xx}\right]
\end{array}
$$
with boundary condition
$$
V^i(T, x_{iT})=e^{-rT}P_i(x_{iT}).
$$

% frametitle
{Equilibrium Solution}

% begin itemize



	We solve the HJB for $N$ companies and use the market clearing condition
$$
\sum_{i=1}^N z_{it}^*=0
$$


	The first-order conditions give
$$
\begin{array}{llll}
u_{it}^* &=& -\frac{1}{c_i} e^{rt} V^i_x & i=1, \ldots N \\*[12pt]
S(t) &=& - e^{rt} V^i_{x} & i=1, \ldots N
\end{array}
$$


	So again
$$
S(t) = c_i u_{it}^*, \; i=1, \ldots N.
$$

% end itemize

% frametitle
{Joint Cost Problem I}
Again we image a central planner who has to solve
\begin{equation}
\max_{u_{it}} \EX\left[\int_0^Te^{-rt}\sum_{i=1}^N C_i(t,u_{it})dt + e^{-rT} \sum_{i=1}^N P_i(x_{iT})\right]
\end{equation}
with $C_i$ and $P_i$ as before.

We assume only one source of randomness, i.e. $W_{it}= W_t$, then we have the  joint value function as
$$
V(t, x_{1t}, \ldots, x_{Nt}) = \sum_{i=1}^N V_i(t,x_{it}).
$$

%Here, $z_{it}$ are irrelevant due to the market clearing condition.

% frametitle
{Joint Cost Problem II}
The joint cost problem now is
$$
\begin{array}{lll}
0&=\displaystyle \max_{\{u_{it},i=1, \ldots, N\}}&\displaystyle \left[e^{-rt}\sum_{i=1}^N C_i(t,u_{it})\right.\\*[12pt]
&&+\displaystyle \left.\sum_{i=1}^N (V^i_t -V_x^iu_{it}) + \frac{1}{2}\sum_{i=1}^N(G_i(t))^2 V^i_{xx}\right]
\end{array}
$$
with boundary condition
$$
\sum_{i=1}^N V^i(T, x_{iT})=e^{-rT}\sum_{i=1}^N P_i(x_{iT}).
$$

% frametitle
{Joint Problem Solution}

% begin itemize



	The first-order conditions give
$$
u_{it}^{**}= -\frac{1}{c_i} e^{rt} V^i_x, \; i=1, \ldots N
$$


	Due to linearity we also have
$$
u_{it}^{**}= \argmax \left\{ \max_{u_{it}} \left[e^{-rt} C_i(t,u_{it}) + V^i_t -V_x^iu_{it} + \frac{1}{2}(G_i(t))^2 V^i_{xx}\right]
\right\}
$$


	Again
$$
S(t) = c_i u_{it}^{**}= -e^{rt}V^i_x, \; i=1, \ldots N.
$$

% end itemize

% frametitle
{Equivalence to Equilibrium Solution}
$$
\begin{array}{lll}
&u_{it}^{**}&\\*[12pt]
=&  \argmax \left\{ \displaystyle \max_{u_{it},z_{it}} \right. &\left[e^{-rt} C_i(t,u_{it}) -e^{-rt}S(t)z_{it} +e^{-rt}S(t)z_{it} \right.\\*[12pt]
 &&\left.\left. + V^i_t -V_x^iu_{it} + \frac{1}{2}(G_i(t))^2 V^i_{xx}\right]
\right\}\\*[12pt]
=&  \argmax \left\{ \displaystyle \max_{u_{it},z_{it}} \right. &\left[e^{-rt} (C_i(t,u_{it}) -S(t)z_{it}) \right. \\*[12pt]
&&\left.\left. + V^i_t +V_x^i(-u_{it}-z_{it}) + \frac{1}{2}(G_i(t))^2 V^i_{xx}\right]
\right\}\\*[12pt]
=&u_{it}^*&
\end{array}
$$

\subsubsection{Multi-Period Models}
% \subsection{General Equilibrium model}

% frametitle
{Basic Model}

% begin itemize



	Total emissions dynamics are
\begin{equation}
dy_t= \mu(y_t)dt + \sigma(y_t)dW_t
\end{equation}
with deterministic drift and volatility.


	buy or sell $z_t$  permits in the market


	abate $u_t$ with cost function $C(u_t)$


	pay penalty costs

% end itemize

% frametitle
{Basic Model II}

% begin itemize



	$x_{t,T_k}$ are the total expected emissions in $[0,T_k]$


	Then
\begin{equation}
x_{t,T_k}=-\int_0^tu_s ds  -\int_0^t z_u du + \EX_t\left[\int_0^{T_k} y_s ds \right]
\end{equation}

% end itemize

% frametitle
{Multi-period Framework}

% begin itemize



	Consider $n$ consecutive trading periods $[0,T_1], [T_1, T_2], \ldots [T_{n-1}, T_n]$ with inter-period banking (no borrowing).


	Initial endowment of $e_{T_{k-1}}$ at the beginning of each period $[T_{k-1}, T_k]$.


	Have to pay penalty if emissions $x_{T_k}$ from $0$ to $T_k$ exceed the total allocated permits
$$
e_{T_k}= \sum_{T_j < T_k}e_{T_j}.
$$


	With $R(x_{T_k})=e_{T_k}-x_{T_k}$ penalty cost are
$$
P(x_{T_k}) = P \min\{0, R(x_{T_k})\}
$$

% end itemize

% frametitle
{Multi-period Optimisation problem}
$$
\begin{array}{lll}
\max_{u_t, z_t} & \EX_0& \left[ \displaystyle\int_0^{T_n}e^{-rt}C(u_t)dt - \int_0^{T_n}e^{-rt}S(t) z_tdt \right. \\*[12pt]
& & \displaystyle \left. + \sum_{j=1}^n e^{-rT_j}P(x_{T_j}) + R(X_{T_n}) S_{end}\right]
\end{array}
$$

% frametitle
{Equilibrium Solution}
N companies with

% begin itemize



	$u_{it}$ is the individual rate of abatement


	and $z_{it}$ is the instantaneous amount of permits bought or sold


	solve their individual cost problem


	with market clearing condition
$$
\sum_{i=1}^N z_{it}^*=0 \;\;  \forall t \in [0, T_n]
$$

% end itemize

% frametitle
{Permit Price Process}

% begin itemize



	The general structure is still
$$
S(t) = \sum_{T_j >t} e^{-r(T_t-t)} P \EX_t\left[\IF_{\{R(x_{T_j})<0\}}\right] + e^{rt} S_{end}
$$


	The first-order conditions give
$$
S(t) = c_i u_{it}^*, \; i=1, \ldots N.
$$

% end itemize

% frametitle
{Solution Strategy}

% begin itemize



	Start with the last period

% begin itemize



	find the characteristic PDE with boundary conditions from optimality principle of stochastic control


	solve for strategy value function $V_n$

% end itemize



	step back one period

% begin itemize



	find the characteristic PDE


	solve for strategy value with boundary condition from next periods value

% end itemize



	derive abatement strategy from HJB

% end itemize

