% !TEX root = QCF_ss14UDE.tex
\section{Flexible Mechanisms under the Kyoto Protocol}
\subsection{Clean Development Mechanism (CDM) and Joint Implementation (JI)}

% frametitle
{Project-based Mechanisms}

% begin itemize

	2005:  Emissions Trading (EU ETS) was launched in the European Union as a measure to meet the Kyoto commitment

	Flexible Mechanisms as Clean Development Mechanism (CDM) and Joint Implementation (JI) can be used by countries to meet their Kyoto commitment and also by companies within the framework of EU emissions trading

% end itemize

Objectives of CDM/JI

% begin itemize

	To save the same or a higher amount of CO2 for the same financial effort and by using fewer financial resources (abatement costs are lower)

	Sustainability, emissions reduction, technology transfer, contribution towards economic development
 % end itemize

							% Folie 4

% frametitle
{Cost Advantage}
\begin{figure}[h!]
\centering
\includegraphics[width=0.9\textwidth, height=0.7\textheight]{../../../pics/abatementcosts.pdf}
\caption{Cost advantage due to CDM/JI}
\label{fig:CDM/JI}
\end{figure}

							% Folie 5

% frametitle
{Clean Development Mechanism (CDM)}

% begin itemize

	 Emission reduction projects in a developing or an emerging country (e.g. China, India, Brazil, Malaysia)

	 Accepted technologies (photovoltaic, wind power, biomass, energy efficiency; no nuclear projects!)

	 Projects generate Certified Emission Reductions (CER) in the amount of tonnes of emissions that have been avoided (compared to the 'business-as-usual'-scenario, that means claim for additionality)

% end itemize

\textbf{Secondary CERs:} issued, tradable credits with guaranteed delivery \\
\textbf{Primary CERs:} purchased forward directly from project (or fonds), subject to individual project and delivery risks

							% Folie 6

% frametitle
{Joint Implementation (JI)}

% begin itemize

	 Emission reduction projects in an industrialized or a transition country (e.g. Russia, Ukraine), which has committed to a cap, solely in sectors not covered in the ETS

	 Analogous to CDM, but reduces the emission budget of the country where the project takes place

	 Projects generate ERU (Emissions Reduction Unit) in the amount of the avoided emissions (compared to the 'business-as-usual'-scenario)

	 from 2008 onwards accepted in the EU ETS

% end itemize

CERs and ERUs can be used by \\
1) Countries, for compliance under the Kyoto Protocol \\
2) Companies within the EU ETS (up to 22\%), for compliance

							% Folie 7

% frametitle
{Example of the Project-based Mechanisms}
\begin{figure}[t]
\begin{minipage}[t]{0.475\textwidth}
Construction of a wind farm \\
abroad:
\vspace*{-0.2cm}
\begin{figure}[h!]
\centering
\includegraphics[width=0.6\textwidth, height=0.3\textheight]{../../../pics/windfarm.jpg}
\end{figure}
\vspace*{-0.4cm}

% begin itemize

	annual electricity production: 300 GWh

	no emissions occur during production

% end itemize

\end{minipage}
\begin{minipage}[t]{0.475\textwidth}
Electricity production at a coal \\
power plant:
\vspace*{-0.2cm}
\begin{figure}[h!]
\centering
\includegraphics[width=0.6\textwidth, height=0.3\textheight]{../../../pics/coalpowerplant.jpg}
\end{figure}
\vspace*{-0.4cm}

% begin itemize

	300,000 tonnes of CO2 emissions annually

% end itemize

\end{minipage}
\end{figure}

						% Folie 8

% frametitle
{Example of the Project-based Mechanisms}

% begin itemize

	By the construction of the wind farm, emissions amounting to 300,000 tonnes per year have been avoided

	Therefore, emission certificates amounting to 300,000 tonnes per year will be generated

% end itemize

						% Folie 9

% frametitle
{Distribution of CDM Projects and generated CERs by host country }
\begin{figure}[t]
\begin{minipage}[t]{0.475\textwidth}
Expected average annual CERs \\
from registered projects \\
(total: 605,014,135):
\vspace*{-0.7cm}
\begin{figure}
\centering
\includegraphics[width=1.25\textwidth, height=0.4\textheight]{../pics/ExpectedAverAnnCERs2.png}
%Expected average annual CERs from registered projects by host party. Total: 605,014,135. Source: http:\cdm.unfccc.int (c) 20.06.2012 15:56
\end{figure}
\vspace*{-0.8cm}

% begin itemize

	68\% renewables projects %Source: http://www.cdmpipeline.org/cdm-projects-type.htm

% end itemize

\vspace*{-0.9cm}

\end{minipage}
\begin{minipage}[t]{0.475\textwidth}
Registered project activities\\
(total: 4,248):
\vspace*{-0.7cm}
\begin{figure}
\centering
\includegraphics[width=1.2\textwidth, height=0.4\textheight]{../../../pics/RegisteredProjActivity2.png}
% registered project activities by host party. Total: 4,248. Source: http:\cdm.unfccc.int (c) 20.06.2012 15:56
\end{figure}
\vspace*{-0.8cm}

% begin itemize

	5600  projects worldwide

	2,7 bn CERs expected until end of 2012

% end itemize

\end{minipage}
\end{figure}

% frametitle
{Types of Risk Involved}

% begin itemize

	Market risk

	Volume Risk

	Credit Risk

	Delivery Risk

% end itemize

% frametitle
{CDM/JI Risk Management Example}

% begin itemize

	We combine a CDM project with an ERPA

	Recall an Emission Reduction Purchase Argreement (ERPA) is a  transaction that transfers carbon credits between two parties under the Kyoto Protocol. The buyer pays the seller cash in exchange for carbon credits, thereby allowing the purchaser to emit more carbon dioxide into the atmosphere.

	Assume an ERPA with $10$ \euro/ t CO2

% end itemize

% frametitle
{CDM/JI Risk Management Example II}

% begin itemize

	Sell the expected volume $V_{exp}$ of 50 000 t CO2 forward at $15$ \euro /t CO2

	Now the volume delivered is a random variable $V$ and the certificate spot price $S$ is random.

	So the portfolio value $P$ at delivery is
\begin{equation}\nonumber
P = 15 \times V_{exp} - 10 \times V + (S-15) \times (V_{exp}-V)^+ + S \times (V-V_{exp})^+
\end{equation}

	So we face two risky scenarios

% begin itemize

	Higher volume with low spot

	lower volume with higher spot

% end itemize

% end itemize

\subsection{Performance of CDM/JI scheme}

% frametitle
{Allowance Price Versus CERs}
\begin{figure}[h!]
\centering
\includegraphics[width=0.9\textwidth, height=0.7\textheight]{../../../pics/EUAvsSCER.pdf}
\caption{Price Difference EUA vs SCER}
\label{fig:EUAvsSCER}
\end{figure}

% frametitle
{Allowance Price Versus CERs}

% begin itemize

	Reason for price difference: Limited number of SCER for use in EU

	Intensive discussion of that in Barrieu and Fehr (2010).

% end itemize

% frametitle
{Challenges in CDM Projects}

% begin itemize

	Regulatory  Risk

% begin itemize

	Possible changes of CDM frame after 2012 with an impact on private investments

	Acceptance of project-types and countries (China, Brazil)

% end itemize

	CDM Acceptance

% begin itemize

	Long and bureaucratic

	not transparent, concrete methodic unknown

% end itemize

	Country Risk

% begin itemize

	local (CDM-) infrastructure: people, infrastructure

	local energy infrastructure

	local political risk

% end itemize

% end itemize
