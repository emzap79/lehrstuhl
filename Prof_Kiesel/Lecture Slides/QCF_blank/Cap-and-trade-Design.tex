% !TEX root = QCF_ss13UDE.tex
\section{Emission Trading Schemes}

Example for Emission Trading
	Consider two companies A and B each emitting 100 000 metric tons of $\textnormal{CO}_2$ per year
	Each has been allocated 95 000 metric tons under its national allocation plan
	Credits are trading at 10\euro\ per metric ton
	Company A can cut 10 000 metric tons of emission at 5\euro\ per ton (marginal abatement costs, MAC)
	Company B has MAC of 15\euro\ per ton
	Company A receives 50 000\euro\ for its surplus and covers the costs of its own reduction
	Company B meets the cap at cost 50 000\euro\ instead of 75 000\euro


\subsection[Overview]{Overview of different ETS}

Overview of different emission trading systems
	\textbf{$\textnormal{SO}_2$ (US)} \\
	Acid rain is mostly caused by human emissions of sulfur dioxide (fossil fuel fired power plants). It harms plants,
	aquatic animals and damages infrastructure (buildings, monuments). The $\textnormal{SO}_2$ emission trading system 
	under the framework of the 1990 Clean Air Act Amendments started in 1995 and is expected to reduce 
	$\textnormal{SO}_2$ by 50\% (2010 vs. 1980).

	\textbf{$\textnormal{CO}_2$ (EU)} \\
	A lot of researchers (among them the Intergovernmental Panel on Climate Change, IPCC) conclude that global warming
	is mainly caused by anthropogenic greenhouse gases (GHG) such as $\textnormal{CO}_2$. Global warming will amongst 
	other effects lead to rising sea levels and  more extreme weather events.
  
  
Overview of different emission trading systems
  \textbf{Other GHG trading systems}
		New South Wales Greenhouse Gas Abatement Scheme (Australia, launched in 2003).
    Australian ETS starts in 2014. Currently, tax-based first phase. \url{http://www.climatechange.gov.au/}
    New Zealand ETS (New Zealand, launched in 2008)
    Regional Greenhouse Gas Initiative (US, launched in 2009)
        
  \textbf{GHG trading system in the US (was planned but is now shelved)} \\
		Ambitious plan that would lead to the world's largest $\textnormal{CO}_2$ emission trading system (covering about 80 per cent of US output vs less than half in Europe). However,
		California started an ETS in 2012. \url{http://www.arb.ca.gov/cc/capandtrade/capandtrade.htm}
   
	Overview: \url{http://www.climatechange.gov.au/government/international/global-action-facts-and-fiction/ets-by-country.aspx}
   

%\subsection{The EU ETS}
%8. Folie
Characteristics of EU ETS ($\textnormal{CO}_2$)
	EU ETS is split up into three phases
		\textbf{Phase I (2005-07)}
		\textbf{Phase II (2008-12)} coinciding with commitment period of Kyoto protocol
		\textbf{Phase III (2013-20)} inducing significant changes compared to the two previous periods, according to Directive 2009/29/EC
  
  Scheme covers approximately 12,000 large emitters in the EU that are responsible for 50\% of total $\textnormal{CO}_2$ emissions. 
	Regulated sectors include energy industry, combustion, cement, etc.
  
	Emission allowances are traded mostly OTC (approx 60\%), bilateral (approx 10\%) and on eight different exchanges (approx 30\%):
  ECX in London, Nord Pool in Oslo, Powernext in Paris, EEX in Leipzig, The Green Exchange (NYMEX), Sende $\textnormal{CO}_2$, EXAA, New Values Climex.
  
  
	%Sources: DB Research (2011), ETS (Directive 2009.29.EC), h	H	
	%http://eur-lex.europa.eu/Notice.do?mode=dbl&lang=en&ihmlang=en&lng1=en,de&lng2=bg,cs,da,de,el,en,es,et,fi,fr,hu,it,lt,nl,pl,pt,ro,sk,sl,&val=463502:cs&page=


Characteristics of EU ETS ($\textnormal{CO}_2$) - Phases I and II
	Process steps concerning the distribution of the allowances according to Phases I and II:
		Each country submits a NAP (National Allocation Plan) to the European Commission (EC)
		EC adjusts NAPs if necessary and countries distribute EUAs among regulated firms according to the final NAP as approved by the EC
  
		%are distributed in each phase 200x EU defines overall target and targets for each country
		%200x Each country allocates EUAs to different sectors and companies via NAP (National Allocation Plan)
		% March Allocation of allowances for the current year (in advance)
		%January Compliance time, i.e. pollutors have to deliver allowances for the last year to the regulator
  
	At the end of the current phase regulated firms have to pay a fine of 100 Euro (40 Euro for last phase) for each emitted ton 
	of $\textnormal{CO}_2$ that is not covered by an allowance (excess emissions penalty).
  

Characteristics of EU ETS ($\textnormal{CO}_2$) - Phase III
  Process steps concerning the distribution of the allowances according to Phase III:
		NAPs will be abolished
		From 2013 onwards \textbf{Auctioning} will be introduced as default method of initial allowance allocation
		The initial auctioning share  in the power sector will be 100\%
		For all other sectors the initial auction share will be 20\% and is to be increased to 70\% by 2020 (and to 100\% respectively by 2027)
		Non-auctioned allowances will be distributed on the basis of benchmarks
  
		%are distributed in each phase 200x EU defines overall target and targets for each country
		%200x Each country allocates EUAs to different sectors and companies via NAP (National Allocation Plan)
		% March Allocation of allowances for the current year (in advance)
		%January Compliance time, i.e. pollutors have to deliver allowances for the last year to the regulator
  
	The amount of the excess emissions penalty in the upcoming third phase shall be indexed to the annual inflation rate of the Eurozone.  %Source: DB Research 2011, CMA 2011
  

Allocation and emissions in the EU ETS
	\begin{center}
	\begin{figure}[h!]
	\centering
	\rotatebox{0}{
	\scalebox{0.6}{
	\includegraphics[width=1.45\textwidth, height=1.4\textheight]{../../../pics/Climate-Strategies-2011-ETS-Phase-III.pdf}}} %Source: Climate Strategies 2011
	%\caption{}
	\label{fig:plot1}
	\end{figure}
	\end{center}
	 %Source: Climate Strategies (2011)


\subsection{ETS and Flexibility}

ETS and flexibility: {\it When} and {\it Where}
	\begin{center}
	\begin{figure}[h!]
	\centering
	\includegraphics[width=0.9\textwidth, height=0.7\textheight]{../../../pics/carbon-flexibility.pdf}
	\caption{Cost reductions from allowing 'where' and 'when' flexibility}
	\label{fig:emmissions}
	\end{figure}
	\end{center}
 %Source: VividEconomics (2008) S.10


{\it Where} flexibility}
	The ambition should be a global market. However, there are various constraints such as policy differences, differences in the traded good, etc.
	Linking different national and regional trading systems can approximate a global market.
	Linking markets increases liquidity and thus reduces the cost of trading.
	However, different designs of schemes have to be taken into account.


{\it Where} flexibility: Gains from Trade
	\begin{center}
	\begin{figure}[h!]
	\centering
	\includegraphics[width=0.9\textwidth, height=0.8\textheight]{../../../pics/carbon-gainsfromtrade.pdf}
	%\caption{}
	\end{figure}
	\end{center}
	%Source: VividEconomics (2008) S.15


{\it When} flexibility  -- Banking
	effectively increases the depth and liquidity of the market, reducing price volatility by
	making current prices a function of a longer time span of activity, rather than being
	entirely determined by events today;
	
	creates an incentive for firms to take early action;
	
	firms with banked allowances have a vested interest in higher prices and the continuation 
	(and success) of the system, to maximise the value of their allowance assets;
	
	banking can also prevent a price collapse between commitment periods;


{\it When} flexibility -- Borrowing
	the regulator may not be well-equipped to assess the credit worthiness and
	solvency of firms who borrow allowances, who thereby become debtors;
	
	borrowing enables firms to delay action if they assume that targets will prove too
	onerous and will subsequently be softened;
	
	firms with borrowed allowances have an active interest to lobby for weaker targets,
	or even for scrapping emissions trading altogether, so that their debts are cancelled;
	
	the political desire to (be seen to) act early, and potential benefits of early action,
	also imply that politicians may prefer to place constraints on borrowing;


{\it When} flexibility}
	banking is usually allowed between periods (Exemption EU ETS Phase I);
	
	there is typically no borrowing (or only very limited);
	
	when there are limits on borrowing between periods, the length of the commitment period is relevant to 'when' flexibility
	and to market efficiency
		
		investments to reduce emissions may require many years for investors to recover their costs
		
		in case of short periods, investors have to guess the emissions caps set by
		future governments, and attempt to anticipate changes in the underlying structure of the
		carbon trading framework


Permit price in the EU ETS during the first phases}
	\begin{center}
	\begin{figure}[h!]
	\centering
	\rotatebox{0}{
	\scalebox{0.6}{
	\includegraphics[width=1.25\textwidth, height=1.1\textheight]{../../../pics/EUA-future-prices-2005-2011.pdf}}}
	\caption{EUA - futures prices (1 January 2005 - 15 August 2011).}
	\label{fig:plotCar00-Data}
	\end{figure}
	\end{center}
	 %Source: EEA_GHGtrendsprojections


\subsection{ETS and Tax}

ETS vs Tax: Generalities
	Since compliance costs are uncertain the choice of instrument depends on the relative curvatures
	of the marginal benefit curve and the marginal abatement costs curve.
	
	In case of CO2, where damage does not depend on the flow of emissions but on their accumulation in
	the atmosphere, scientific results suggest that a carbon tax is more economically efficient under
	uncertainty than emissions trading.

	In practice, however, the analysis of efficiency
	under uncertainty has had little influence on the choice of policy instruments. The preference
	for carbon trading over carbon taxes is driven largely by powerful political economy concerns.
	Trading systems are easier to implement politically.
  


ETS vs Tax
	The market for emission reductions has a demand schedule, which is determined by the marginal abatement costs of
	regulated agents, and a supply schedule, which is determined by policy.
	
	Under a pure tax system, the supply of allowances is infinitely elastic. The market is effectively supplied with as
	many allowances as agents wish to buy at a fixed price (the tax rate).
	
	Under a pure allowance system, supply is completely inelastic as the amount of allowances is exogenously fixed.
	
	Hybrid systems create a supply curve that is neither fully flat (a pure tax) nor fully vertical
	(pure cap-and-trade) but (stepwise) upward sloping.


Price ceilings and price floors
	a price ceiling and floor provide significantly
	greater clarity to investors to deliver dynamic efficiency (in the form of optimal investment
	over longer time frames).
	
	the price floor would guarantee a certain minimum return on
	investment in low-carbon technologies, reducing the risk faced by innovating firms.
	
	the price ceiling may enhance policy credibility. Because it caps the costs of
	compliance, a ceiling reduces the risk of a policy reversal if abatement costs turn out to be
	injuriously high.


Price ceilings and price floors
	a price ceiling can be established through an unlimited commitment from the regulator to sell allowances onto the market at the price ceiling
	
	drawback: compliance with the emissions cap is sacrificed
	
	a price floor can be established through an unlimited commitment from the regulator to buy back
	allowances from the market at the price floor
	
	drawback: the floor would be achieved at the risk of imposing a liability on the public balance sheet.


Hybrid System with Safety Valve
	\begin{center}
	\begin{figure}[h!]
	\centering
	\includegraphics[width=0.9\textwidth, height=0.8\textheight]{../../../pics/hybridscheme.pdf}
	%\caption{}
	\end{figure}
	\end{center}
	%Source: new and Old policies S.12


Ecological Effectiveness
	Emission trading systems sets a cap, which in theory establishes precisely the level of emissions which is desired (100\% effectiveness)
	
	Tax system cannot guarantee an exact amount of emissions as an  outcome. The tax level is set under uncertainty 
	about the marginal abatement costs.
	
	Hybrid systems reach the target value as long as the system is in its trade area. As soon as the trigger level 
	is reached the ecological target becomes diluted due to the additional certificates.



Political Feasibility
Political feasibility is the level of acceptance a policy has in the public. A major factor is the number of people affected.
	Emission trading systems have a good political enforceability since costumers feel no direct effect by government action.
	Price rises are blamed on companies, especially in case allocation is free. In case of auctioning extra government revenues can be used for redistribution.
	Tax systems are generally met with scepticism (especially in the US). Additional costs on emissions will effect consumers more directly via cost increases.
	Hybrid systems also generate an extra revenue after the trigger is met, which may be viewed positively.


Financial Impact
Financial impact relates to how consumers in a country with a regulation policy are affected in monetary terms.
	In emission trading systems with free allocation companies have integrated the costs (price) for certificates
	as real costs and prices have increased (pass-through rates of cost between 60\% and 100\% in Germany).
	
	High windfall profits for companies in the power sector (around 5 Mrd Euro).
	
	Tax systems will also increase costumer costs.


\subsection{Multiple Policy Instruments}

Multiple Instruments I
	Emission regulation is directed at internalizing externalities and economic theory indicates that only one instrument is needed to
	internalize one externality.
	
	Policy often involves multiple instruments such as command-and-control regulation, subsidies, taxes, trading schemes, etc.
	
	This process reflects an ad-hoc policy-accretion process driven by the multiplicity of national institutions or ...
	the temptation of politician to fix everything.


Multiple Policy Instruments II
	Combinations of permit trading schemes, carbon taxes, technology-specific subsidies and regulatory standards
	
	Taxes introduced by Sweden, Norway, Denmark, Ireland
	
	Subsidies for renewable energy by Germany, Spain
	
	Academic literature gives justification for multiple instruments used in a complimentary way, i.e. hybrid systems. 
	Justification in terms of presence of multiple market failures, asymmetric information, principle-agent relation
 


\subsubsection{Symmetric Policy Combinations}

Tax and Trade
	Carbon tax $t$ (Euro per tonne)
	
	Cap-and-trade scheme with price $p$
	
	Firm must pay tax and procure certificates for emissions
		\begin{tabular}{ll}
		$e_0$ & baseline emission \\
		$e$ & emissions after abatement \\
		$a$ & $=e_0 - e$; \\
		$c(a)$ & abatement costs, $c' > 0$, $c'' > 0$ \\
		\end{tabular} \\
 Optimization problem
		\begin{align}
		\min_{e} & \left\{c(e_0-e)+te+pe\right\} = \min_{e}\left\{f(e)\right\}
		\end{align}
	First-order-condition (*)
		\[
		c'(e_0-e^*)=t+p
		\]


							% Folie 5

Tax and Trade: optimization problem
	Since $c'>0$ we can invert $e^*=e_0-c'^{-1}(t+p)=e^*(t,p)$ \\
	
	Since $f''(e)=c''(e_0-e)>0$ it is a minimum \\
	
	Differentiation of (*) w.r.t. the variable t:
		\[
		\frac{\partial}{\partial{t}}c'(e_0-e^*(t,p))=\frac{\partial}{\partial{t}}(t+p)
		\]
		\[
		-c''(e-e^*(t,p)) \frac{\partial}{\partial{t}} (e^*(t,p))=1
		\]
		\[
		e^*_t=-\frac{1}{c''} < 0
		\]
 Thus increases in tax reduce emissions; By symmetry $e^*_p = -\frac{1}{c''} < 0$ increases in price reduce emissions and $e^*_t=e^*_p$

	Assume a cap $E$, $n$ identical firms, such that the aggregate emissions are $E=ne^*$. \\
	For constant tax $t$ formally
		\[
		dE = ne^*_pdp \textnormal{, so } \frac{dp}{dE} = (ne^*_p)^{-1} < 0
		\]
	that is an increase in the cap reduces the price. \\

	For fixed cap we have
		\[
		ne^*_tdt+ne^*_pdp=0
		\]
		\[
		\frac{dp}{dt}=-\frac{e^*_t}{e^*_p}=-1.
		\]
	That is ''a small increase in tax results one-for-one in an equivalent reduction of the permit price". \\

	\begin{center}
	\begin{figure}[h!]
	\centering
	\rotatebox{0}{
	\scalebox{0.6}{
	\includegraphics[width=1.35\textwidth, height=\textheight]{../../../pics/cost-of-switching.pdf}}}
	\caption{Relation of tax and permit price.}
	\end{figure}
	\end{center}
	%MAC = cost of switching,  % marginal abatement cost;
	%$a = e_0 - e$

Tax and Trade
	MAC to use (for example) wind generation is how much it costs to generate electricity by wind compared with the cheapest alternative. \\
	
	So tax increase reduces MAC, because it reduces the opportunity cost of wind generation. Thus opportunity cost 
	of abatement have decreased since the firm will pay a higher penalty not to abate.

	Model shows that an increase in tax reduces the permit price
	
	No additional abatement will be achieved
	
	The average carbon price will be reduced and the risk that the price system will collapse is increased


Subsidies and Trade
	Model shows that an increase in subsidies reduces the permit price
	
	A higher level of subsidy for an abatement technology reduce the abatement cost at any given level of production
	
	For a given emission cap, demand for permits will be lower and so will be the price.


Subsidies and Trade: optimization problem
Subsidy $s$ applies equally to all firms and technologies provided accordingly to the level of abatement $a = e_0 - e$ \\
	Optimization problem
		\begin{align}
		\min_{e} & \left\{c(e_0-e)-s(e_0-e)+pe\right\} = \min_{e}\left\{f(e)\right\}
		\end{align}
	First-order-condition
		\[
		f'(e)=-c'(e_0-e)+ s+p = 0
		\]
		so
		\[
		c'(e_0-e)=s+p
		\]
		and
		\[
		e^*=e_0-c'^{-1}(s+p)=e^*(s,p)
		\]
 
	The same calculation as above shows
		\[
		\frac{dp}{ds} = -\frac{e^*_s}{e^*_p}=-1.
		\]
	The higher the subsidy, the lower the permit price.


Trade and Trade
	Two separate trading programs apply upstream to firms that produce electricity and downstream to firms that consume it.
	
	Example: UK with EU ETS for electricity producers and Carbon Reduction Commitment for firms and organizations that are primarily electricity consumers.


Trade and Trade
Compliance upstream
	higher electricity price
	equivalent to tax on energy consumption (linked to carbon price)
	downstream implicit price of carbon


Increase in upstream permit prices
	increase tax downstream
	decrease downstream carbon price (see part 1)


\subsubsection{Asymmetric Policy Combinations}

Uniliteral Tax and Trade
	Firms are identical
	
	Tax only affects a fraction $f$ of firms
	
	Uniliteral tax leads to diverging marginal costs and so increased mitigation costs.
	
	Firms are identical with optimal emissions $e^*(t,p)$; $e^*_p = e^*_t = -\frac{1}{c''}$
	
	Tax affects only a fraction of firms $f$ (a fraction of the system-wide emissions) $0 < f < 1$ \\
		 So
		\begin{align}
		E = fne^*(t,p)+(1-f)ne^*(p)
		\end{align}
	Assuming
		\[
		e^{*f}_p = e^{*(1-f)}_p
		\]
	We find
		\[
		dE = fne^*_tdt+ne^*_pdp
		\]
	Fix $E$, then the impact of the tax on the permit price is
		\[
		0 =  fe^*_tdt+e^*_pdp \Rightarrow \frac{dp}{dt} = -f \frac{e^*_t}{e^*_p} = -f
		\]
	The impact of the tax change on permit prices is diluted and different for the two categories
		\[
		\frac{de^{*f}}{dt} = e^{*f}_t + e^{*f}_p \frac{dp}{dt} = -\frac{1}{c''} + (-\frac{1}{c''})(-f) = -(1-f)\frac{1}{c''} < 0
		\]
		\[
		\frac{de^{*1-f}}{dt} = e^{*(1-f)}_p\frac{dp}{dt} = \frac{f}{c''} > 0
		\]
	So the emissions for firms subject to tax fall, but the emissions for firms with no tax increase.

	The impact on marginal costs for taxed firms is \\
	(use $c'(e_0-e^*) = t+p)$)
		\[
		\frac{d(c^f)'}{dt} = \frac{dt}{dt} + \frac{dp}{dt} = 1-f > 0
		\]
	and for untaxed firms
		\[
		\frac{d(c^{1-f})'}{dt} = \frac{dp}{dt} = -f < 0
		\]
	Diverging marginal costs mean that the gains from trade are, at least in part, reversed. (Remember firms where identical, now mitigation costs increase.)

	\textbf{Tables based on quadratic cost functions} \\
	Mitigation costs rise because more expensive technologies in countries with tax would 
	substitute out more cost-effective emissions reduction technologies in countries without tax.


Impact of Overlapping Tax
	\begin{center}
	\begin{figure}[h!]
	\centering
	\rotatebox{0}{
	\scalebox{0.6}{
	\includegraphics[width=1.45\textwidth, height=1.4\textheight]{../../../pics/ImpactOverlappingTax.pdf}}} %Source: Fankhauser, Hepburn, Park 2010
	%\caption{}
	\label{fig:plot1}
	\end{figure}
	\end{center}


Impact of Overlapping Tax by 2020
	\begin{center}
	\begin{figure}[h!]
	\centering
	\rotatebox{0}{
	\scalebox{0.6}{
	\includegraphics[width=1.45\textwidth, height=1.4\textheight]{../../../pics/ImpactOverlappingTax2020.pdf}}} %Source: Fankhauser, Hepburn, Park 2010
	%\caption{}
	\label{fig:plot1}
	\end{figure}
	\end{center}


Technology Policies and Trade
	A technology-specific measure affects only part of the MAC curve and will lead to a compositional reorientation of the curve
	
	In EU ETS fuel switching from coal to gas is targeted
	
	Trading price may fall, but mitigation cost will rise in general.


Possible Impact of Subsidies I
	\begin{center}
	\begin{figure}[h!]
	\centering
	\rotatebox{0}{
	\scalebox{0.6}{
	\includegraphics[width=1.45\textwidth, height=1.4\textheight]{../../../pics/ImpactSubsidiesB.pdf}}} %Source: Fankhauser, Hepburn, Park 2010
	%\caption{}
	\label{fig:plot1}
	\end{figure}
	\end{center}


Possible Impact of Subsidies II
	\begin{center}
	\begin{figure}[h!]
	\centering
	\rotatebox{0}{
	\scalebox{0.6}{
	\includegraphics[width=1.45\textwidth, height=1.4\textheight]{../../../pics/ImpactSubsidiesE.pdf}}} %Source: Fankhauser, Hepburn, Park 2010
	%\caption{}
	\label{fig:plot1}
	\end{figure}
	\end{center}


Possible Impact of Subsidies III}
	\begin{center}
	\begin{figure}[h!]
	\centering
	\rotatebox{0}{
	\scalebox{0.6}{
	\includegraphics[width=1.45\textwidth, height=1.4\textheight]{../../../pics/ImpactSubsidiesD.pdf}}} %Source: Fankhauser, Hepburn, Park 2010
	%\caption{}
	\label{fig:plot1}
	\end{figure}
	\end{center}
