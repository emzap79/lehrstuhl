% !TEX root = riskmanagement_ws13UDE.tex
\part{Other Types of Risks}
\section{Operational Risk}
\subsection{Motivation and Introduction }

Examples
	Internal fraud: Barings Bank, Societe General
	
	Terror attack (11.September 2001), adverse weather
	
	Software/Hardware problems (Salamon Brothers lost 303 Million USD during a switch of computer technology)


Definition of Operational Risk
	Basel Committee January 2001:\\
	Operational risk is the risk of loss resulting from inadequate or failed internal processes, people, and systems, or from external events.
	
	This includes people risks, technology and processing risks, physical risks, legal risks, etc,
	
	but excludes reputation risk and strategic risk


Types of operational risks
	Internal fraud
	External fraud
	Employment practices and workplace safety
	Clients, products and business practices
	Damage to physical assets
	Business disruption and system failures
	Execution, delivery and process management


\subsection{Calculation of Operational Risk}

Regulatory Framework
	There are three frameworks to calculate the Operational Risk
		Basic Indicator (15\% of annual gross income averaged over the last three years)
		
		Standardized (different percentage for each business line)
		
		Advanced Measurement Approach (AMA)


Standardized Approach
	Activities of a bank are separated in 8 business lines with a beta factor specified for each
	
	the annual gross income averaged over the last three years is multiplied by the beta factor
	
	the regulatory capital is obtained by summing over the business line contributions


Standardized Approach - Business Lines
\begin{tabular}{ll}
Corporate finance & 18\%\\
Trading and sales & 18\%\\
Retail banking & 12\%\\
Commercial banking & 15\%\\
Payment and settlement & 18\%\\
Agency services & 15\%\\
Asset management & 12\%\\
Retail brokerage & 12\%\\
\end{tabular}


Advanced Measurement Approach
	Upon satisfying certain qualitative and quantitative criteria a bank 
	is allowed to use the Standardized approach or the AMA
	
	Using AMA a bank tries to find a probability distribution of the losses 
	from Operational Risks in order to calculate a risk measure (typically VaR).
	
	Banks need to estimate their exposure to each combination of type of risk and business line. 
	Ideally this will lead to $7 \times 8=56$ VaR measures that can be combined into an overall VaR measure.


Tasks in Calculating AMA
	The loss distribution from Operational Risks is generated by (discrete) loss events 
	together with the size of each loss (loss severity).
	
	We need a counting distribution for the number N of events during a time period (typically $T=1$ year). 
	Typically the Poisson distribution is used where
		$$
		\prob(N=k) = e^{-\lambda T}\frac{(\lambda T)^k}{k!}
		$$
	$\lambda$ is the loss intensity and can be estimated using the average number of loss events during a time period.
 
	Loss severity can be based on internal and external historical data. One possibility is to assume a 
	parametric distribution so that we need only estimate the mean and standard deviation of losses.


Monte Carlo Simulation (from Hull)
	\begin{center}
	\includegraphics[height=6.5cm]{../../../pics/OR-MonteCarlo-p1.pdf}
	\end{center}
	

Monte Carlo Algorithm
	Sample from frequency distribution to determine the number of loss events (=k)
	
	Sample k times from the loss severity distribution to determine the loss severity for each loss event $L_1, \ldots, L_k$
	
	Sum loss severities to determine total loss $L^{(i)}= L_1+\ldots+L_k$
	
	Repeat the above M times to get $L^{(i)}$ for $i=1, ...M$ and thus a loss distribution


Data Issues
	Use own (internal) data as much as possible, but typically only limited amount of data is available.
	
	For external data two possibilities exist
		data sharing
		data vendors

	Data from vendors is based on publicly available information and therefore is biased towards large losses, 
	so it can only be used to estimate the relative size of the mean losses and standard deviation  of losses for different risk categories.
	
	Furthermore, a scaling approach has to be used
		$$
		L_A^{est}=L_B^{obs}\times \left(\frac{R_A}{R_B}\right)^\alpha
		$$
	where $L_{A,B}$ are losses, $R_{A,B}$ are revenues for banks A and B and $\alpha=0.23$ is an estimated non-linearity constant.


\section{Liquidity Risk}
\subsection{Liquidity Trading Risk}

Liquidity Trading Risk
	The price received for an asset depends on
		the mid-market price
		how much is to be sold
		how quickly it is to be sold
		the economic environment

	The subprime crisis showed that transparency is factor that affects liquidity massively


Bid-offer spread
	The proportional bid-offer spread $s$ is defined as
		$$
		s= \frac{\mbox{offer price - bid price}}{\mbox{mid-market price}}
		$$
	
	The cost of liquidation in normal markets is
		$$
		\sum _{i=1}^{n} \frac{1}{2} s_i \alpha_i
		$$

	with
		$n$ the number of positions
		$\alpha_i$ the position in the ith instrument
		$s_i$ the proportional bid-offer spread


Liquidation in a stressed market
	To account for changing market conditions we can view the spread $s$ as a random variable.
	
	The standard approach is to assume a normal distribution. Then the 
	cost of liquidation in stressed markets is
		$$
		\sum _{i=1}^{n} \frac{1}{2} (\mu_i+\sigma_i \lambda) \alpha_i
		$$
	
	with
		$\mu_i$ and $\sigma_i$ expectation and standard deviation of the ith spread
		$\lambda$ factor to obtain the required confidence level


Liquidity-adjusted VaR
	We an now simply add the liquidation costs to the standard VaR to obtain liquidity adjusted VaRs.
	
	The liquidity-adjusted VaR is
		$$
		VaR+ \sum _{i=1}^{n} \frac{1}{2} s_i \alpha_i
		$$
	The liquidity-adjusted VaR under stress is
		$$
		VaR+ \sum _{i=1}^{n} \frac{1}{2}  (\mu_i + \lambda \sigma_i) \alpha_i
		$$


Optimal unwinding of a position
	Typically, the bid-ask spread increases with the size of a position. So a trader needs to 
	develop an optimal strategy taking into account that the market might move during a longer time period. 
	
	Suppose the dollar bid-offer spread as a function of units traded is $p(q)$, that the standard 
	deviation of mid-market price changes per day is $\sigma$, that $q_i$ is amount traded 
	on day i and $x_i$ is amount held on day i (so $x_i = x_{i-1}-q_i$)
	
	The trader's objective might be to choose the $q_i$ to minimize the liquidity-adjusted VaR. This is
		$$
		\lambda \sqrt{\sum_{i=1}^n \sigma^2x_i^2} + \sum _{i=1}^{n} \frac{1}{2} q_i p(q_i)
		$$
	such that  $\sum q_i =V$ (the size of the position).


\subsection{Liquidity Funding Risk}

Sources of Liquidity
	liquid assets
	
	ability to liquidate trading positions
	
	wholesale and retail deposits
	
	lines of credit and the ability to borrow at short notice
	
	securitization
	
	central bank borrowing


Liquidity Black Holes
	A liquidity black hole occurs when most market participants want to take one side of the market and liquidity dries up
	
	Examples: Crash of 1987, Long Term Capital Management
	
	Such black holes may be generated (or accentuated) by positive feedback trading where a trader buys after a price increase and sells after a price decrease.
	
	Reasons for positive feedback trading maybe: Computer models incorporating stop-loss trading, dynamic hedging a short option position, margin calls


\section{Model Risk}

Marking Prices of an Instrument to Market
	Use price quoted by market maker (usually financial institutions mark to mid of bid and offer)
	Use price at which financial institution has traded product
	Use interdealer broker prices
	Use interdealer price indications
	Use model (marking to model)


Model Risk can lead to ..
	Incorrect price at time product is bought or sold
	Incorrect hedging


In Finance Models are used to 
	Observe model prices for similar instruments that trade
	Imply model parameters and interpolate as appropriate
	Value new instrument


Products
	Linear products: Very little uncertainty about the right model, but mistakes do happen
	
	Standard Products
		We do not need usually a model to know the price of an actively traded product. The market tells us the price.
		The model is a communication tool (e.g., implied volatilities are quoted for options)
		It is also an interpolation tool (e.g., a tool for interpolating between strike prices and maturities)


Models for Non-Standard Products
	In the case of nonstandard models play a key role in both pricing and hedging
	
	It is a good idea to use more that one model whenever possible


Dangers in Model Building
	Overfitting
	Overparametrization


Detecting Model Problems
	Monitor types of trading a financial institution is doing with other financial institutions
	
	Monitor profits being recorded from trading of different products
	
	Use Model Audit Group
		Check that a model has been implemented correctly
		Examine whether there is a sound rationale for the model
		Compare the model with others that can accomplish the same task
		Specify limitations of model
		Assess uncertainties in prices and hedge parameters given by model