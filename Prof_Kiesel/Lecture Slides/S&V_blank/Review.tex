% !TEX root = StructuringValuation_ws1314UDE.tex
\section{Review -- Markets and Objects}
\subsection{Energy Markets}

Markets
	Since the deregulation of electricity markets in the end of the 1990s,
	power can be traded at exchanges like the Nordpool, http://www.nordpoolspot.com/
	or the European Energy Exchange (EEX), http://www.eex.com/en. All
	exchanges have established spot and futures markets.


%\subsectionSpot market
EEX spot market 
	Trading in Power, Natural Gas and $CO_{2}$ Emission Rights. 
	
	Power day-ahead auctions for Germany, Austria, France and Switzerland 7 days a week, including holidays. 
	The 24 hours of the respective next day can be traded in one-hour intervals or block orders (e.g. Baseload: 1-24h, Peakload: 9-20h, Night: 1-6, Rush Hour: 17-20h, Business: 9-16h, etc.). 
	
	Continuous day-ahead block trading for France 7:30 am to 11:30 am, 7 days a week, including holidays. 
	
	Continuous Power intraday trading for Germany and France until 75 minutes before the beginning of delivery with delivery on the same or the following day in single hours or blocks. 

	participants submit their price offer/bid curves, the EEX system prices are equilibrium prices that clear the market. 

	EEX day prices are the average of the 24-single hours. 

	similar structures can be found on other power exchanges(Nord Pool, APX, etc.). 


EEX Spot Market Price Processes
	\includegraphics[width=\textwidth, height=6.7cm]{../../../pics/PhelixBase2002_12}
	\includegraphics[width=\textwidth, height=6.7cm]{../../../pics/PhelixBase2002_08}
	\includegraphics[width=\textwidth, height=6.7cm]{../../../pics/PhelixBase2008_12}


Electricity is special 
	it is not storable 
	
	it is homogeneous 
	
	it can be produced in different ways 
	
	it has to be produced when it is needed 
	
	there is a high fluctuation in demand 
	
	there is no short-term elasticity in demand 
	
	there is a high price volatility due to renewable in-feed 


%\subsectionFutures market
Futures Market 
	Traded products are 

	Futures contracts for Power, Natural Gas, Emissions and Coal. 

	Phelix Futures on Phelix Baseload or Peakload monthly power index
	for the current month, the next nine months, eleven quarters and six
	years with cash settlement. 

	Baseload and Peakload French/German Power Futures for the current
	month, the next six months, seven quarters and six years with physical
	settlement, obliging for continuous delivery of 1MW during a month,
	quarter or a year. 

	Actively exchange traded are the next 7 months, 5 quarters and 2-3 years. 

	In addition, OTC transactions. 


%Futures Market
	%\includegraphics[width=\textwidth, height=6cm]../../../pics/bbgfutures
	%\includegraphics[width=\textwidth, height=6cm]../../../pics/future
	%\includegraphics[width=\textwidth, height=6cm]../../../pics/future1


Futures Market
	\includegraphics[width=\textwidth, height=6cm]{../../../pics/lec2_2}

	Returns seem to be stationary, no seasonality.
	\includegraphics[width=\textwidth, height=6cm]{../../../pics/lec2_3}

	Forward Selling Activity 
	\includegraphics[width=\textwidth, height=6cm]{../../../pics/RWE-forward-activity}


Characteristics 
 \textbf{Electricity Futures} -- Obligation to buy/sell a specified amount of electricity during a \textbf{delivery period}, typically a month, quarter or year. 

 Futures show a decreasing volatility term structure. 

 Level of volatility depends on length of delivery period. 


Prices of Futures 
	\begin{center}
	\includegraphics[height=6cm, width=10cm]{../../../pics/forwardcurve}
	\end{center}


Simple Futures Market Model 
	We assume for the futures dynamics 
	$$
	df(t)=\mu f(t)dt+\sigma f(t)dW(t).
	$$
	But since it costs nothing to enter a futures position, we can constantly change our futures portfolio. Thus financing for buying a traded asset is not relevant. 


Black's Formula
	We use the usual notation - strike $K$, expiry $T$ as in the spot
	case, and write $\Phi$ for the standard normal distribution function.
	\foreignlanguage{english}{\textit{The arbitrage price $C$ of a European
	futures call option is 
	$$
	C(t)=c(f(t),T-t),
	$$
	where $c(f,t)$ is given by Black's futures options formula: 
	$$
	c(f,\tau):=e^{-r\tau}(f\Phi(\tilde{d}_{1}(f,\tau))-K\Phi(\tilde{d}_{2}(f,\tau))),
	$$
	where 
	$$
	\tilde{d}_{1,2}(f,\tau):=\frac{\log(f/K)\pm\frac{1}{2}{\sigma}^{2}\tau}{\sigma\sqrt{\tau}}.
	$$
	} }

	Observe that the quantities $\tilde{d}_{1}$ and $\tilde{d}_{2}$
	do not depend on the interest rate $r$. This is intuitively clear
	from the classical Black approach: one sets up a replicating risk-free
	portfolio consisting of a position in futures options and an offsetting
	position in the underlying futures contract. The portfolio requires
	no initial investment and therefore should not earn any interest.


\subsection Pricing Relations for Futures 

Storage, Inventory and Convenience Yield 
	The theory of storage aims to explain the differences between spot and Futures (Forward) prices by analyzing why agents hold inventories. 

	Inventories allow to meet unexpected demand, avoid the cost of frequent revisions in the production schedule and eliminate manufacturing disruption. 

	This motivates the concept of convenience yield as a benefit, that accrues to the owner of the physical commodity but not to the holder
	of a forward contract. 

	Thus the convenience yield is comparable to the dividend yield for stocks. 

	A modern view is to view storage (inventory) as a timing option, that allows to put the commodity to the market when prices are high
	and hold it when the prices are low. 


Spot-Forward Relationship in Commodity Markets  
	Under the no-arbitrage assumption we have 
	\begin{equation}
	F(t,T)=S(t)e^{(r-y)(T-t)}\label{SF-rel}
	\end{equation}
	where $r$ is the interest rate at time $t$ for maturity $T$ and $y$ is the convenience yield.

	Observe that (\ref{SF-rel}) implies 
		spot and forward are redundant (one can replace the other) and for a linear relationship (unlike options) 

		with two forward prices we can derive the value of $S(t)$ and $y$ 

		knowledge of $S(t)$ and $y$ allows us to construct the whole forward curve 

		for $r-y<0$ we have backwardation; for $r-y>0$ we have contango. 


Spot-Forward Relationship: Classical theory 
	In a stochastic model we use 
		$$
		F(t,T)=\EX_{\Q}(S(T)|\F_{t})
		$$
		where $\F_{t}$ is the accumulated available market information (in most models the information generated by the spot price). 

	$\Q$ is a risk-neutral probability 
	
	discounted spot price is a $\Q$-martingale 
	
	fixed by calibration to market prices or a market price of risk argument 


Futures Prices and Expectation of Future Spot Prices 
	The rational expectation hypothesis (REH) states that the current futures price $f(t,T)$ for a commodity with 
	delivery a time $T>t$ is the best estimator for the price $S(T)$ of the commodity. 
	
	In mathematical terms 
		\begin{equation}
		f(t,T)=\EX[S(T)|\F_{t}].\label{REH}
		\end{equation}
		where $\F_{t}$ represents the information available at time $t$. The REH has been statistically tested in many studies for a wide range of commodities. 


Futures Prices and Expectation of Future Spot Prices When equality
	in (\ref{REH}) does not hold futures prices are biased estimators of future spot prices. If 

	$>$ holds, then $f(t,T)$ is an up-ward biased estimate, then risk-aversion among market participants is such that buyers are willing to pay more
	than the expected spot price in order to secure access to the commodity at time $T$ (political unrest); 

	$<$ holds, then $f(t,T)$ is an down-ward biased estimate, this may reflect a perception of excess supply in the future. 


Market Risk Premium 
	The \emph{market risk premium} or \emph{forward bias} $\pi(t,T)$ relates forward and expected spot prices. 

	It is defined as the difference, calculated at time $t$, between the forward $F(t,T)$ at time $t$ with delivery at $T$ and expected spot price: 
		\begin{equation}
		\pi(t,T)=F(t,T)-\EX^{\prb}[S(T)|\mathcal{F}_{t}].\label{forward risk premium}
		\end{equation}
	Here $\EX^{\prb}$ is the expectation operator, under the historical measure $\prb$, with information up until time $t$ and $S(T)$ is
	the spot price at time $T$. 


\subsection[Mean-Reversion Model]A Mean-Reversion Diffusion Model
%\subsectionA simple Valuation Framework


Definition of the model 
	We assume a simple market model in which the price of the underlying commodity, $S_{t}$, follows a stochastic
	process which can be described as follows: Let $X_{t}=\ln S_{t}$ and 
		\begin{align*}
		dX_{t}=\kappa(\ln\theta-X_{t})dt+\sigma dW_{t}~~,~~X_{0}=\ln(S_{0})
		\end{align*}
	Thus, the logarithm of the prices follow a mean reverting diffusionCprocess, the so-called \textcolor{red}{Ornstein-Uhlenbeck-Process}.\\
		$\kappa$ - Speed of mean reversion 
		$\theta$ - Level of mean reversion 
		$\sigma$ - Volatility of the process 
		$dW_{t}$ - Brownian increments 


Sample paths of the model 
	We simulate $S_{t}$ (with $S_{0}=20,\theta=20,\kappa=1,\sigma=0.2$) and get sample paths 
		\begin{figure}
			\centering
				\includegraphics[width=1.00\textwidth]{../../../pics/samplepaths}
			\label{fig:samplepahts}
		\end{figure}


Why this model? 
	The \textcolor{red}{mean-reversion} process has been considered the natural choice for commodities. Basic microeconomics theory tells
	that, in the long run, the price of a commodity ought to be tied to its long-run marginal production cost or, \textquotedbl{}in case of
	a cartelized commodity like oil, the long-run profit-maximizing price sought by cartel managers\textquotedbl{} (Laughton \& Jacoby, 1995, p.188). 

	The model was introduced by Eduardo Schwartz (\textquotedbl{}The Stochastic Behavior of Commodity Prices: Implications for Valuation
	and Hedging\textquotedbl{}, Journal of Finance, vol.52, n. 3, July 1997, pp.923-973) and is often used in academic studies (and for practical pricing?). 


Properties of the model 
	Mean reverting 
	Bounded volatility 
	Continuous paths 
	Relative price changes are normally distributed 
	\textcolor{red}{Analytic results} for the forward-curve and option prices exist 
	\textcolor{red}{Calibration} easily possible 


%\subsectionPrices in the Model


Spot prices in the model 
	The spot price at any time $t$ is 
		\begin{align*}
		S_{t}=\exp\left(e^{-\kappa t}\ln S_{0}+(1-e^{-\kappa t})\ln\theta+\int_{0}^{t}{\sigma e^{-\kappa(t-s)}}dW_{s}\right).
		\end{align*}


Forward prices in the model 
	Using that the forward price is the expected value of future spot prices, %$F(t,T) = \EX^Q[S_T | \Ft]$,
	we get the formula for the forward price at time $t$ for the forward expiring in $T$ as 
		\begin{align*}
			& F(t,T)=\\
			& \exp\left(e^{-\kappa(T-t)}\ln S_{t}+(1-e^{-\kappa(T-t)})\ln\theta+\frac{\sigma^{2}}{4\kappa}(1-e^{-2\kappa(T-t)})\right).
		\end{align*}
	We can see that the forward prices converge to the spot price with time to maturity tending to zero. 


Forward curves in the model 
	Assuming the same parameters as before and varying $S_{t}$ we get the forward curves: 
		\begin{figure}
			\centering
				\includegraphics[width=1.00\textwidth]{../../../pics/forwardcurves}
			\label{fig:forwardcurves}
		\end{figure}


Properties of the forward curve in the model 
	$F(t,T)\rightarrow\exp(\ln\theta+\frac{\sigma^{2}}{4\kappa})=\theta\exp{(\frac{\sigma^{2}}{4\kappa})}$ as $T\rightarrow\infty$. 
	If the spot price is low compared to the long term mean, the forward curve is upward sloping (contango). 
	If the spot price is high compared to the long term mean, the forward curve is downward sloping (backwardation). 
	If the spot price is close to the long term mean, the forward curve might be humped-shaped. 


\subsectionA structural model -- Barlow (2002)

Set-up 
	$u_{t}(x)$ is supply at time $t$, if price is $x$; an increasing function. 
	$d_{t}(x)$ is demand at time $t$, if price is $x$; a decreasing function. 
	The electricity price at time $t$ is the unique number $S_{t}$ such that 
		$$
		u_{t}(S_{t})=d_{t}(S_{t})
		$$
	Need to specify $u_{t}$ and $d_{t}$. 

 
Barlow (2002) specification 
	Supply is non-random, independent of $t$ 
		$$
		u_{t}(x)=g(x).
		$$
		
	Demand is inelastic, independent of $x$ 
		$$
		d_{t}(x)=D_{t}
		$$
		a random process. 

	With 
		$$
		f_{\alpha}(x)=(1+\alpha x)^{\frac{1}{\alpha}},\;\alpha\not=0,\; f_{0}(x)=\exp(x)
		$$
		and 
		$$
		X_{t}=-\lambda(X_{t}-a)dt+\sigma dW_{t}
		$$
		
	Barlow (2002) motivates the model 
		$$
			S_{t}=\left\{ \begin{array}{ll}
			{\displaystyle 
			f_{\alpha}(X_{t})} & 1+\alpha X_{t}>\epsilon_{0}\\* {}[12pt]
			\epsilon_{0}^{\frac{1}{\alpha}} & 1+\alpha X_{t}\leq\epsilon_{0}
			\end{array}
			\right.
		$$




