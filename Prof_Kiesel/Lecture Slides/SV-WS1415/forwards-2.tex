\subsection{An Equilibrium Approach}

\frame{\frametitle{Equilibrium Approach -- Players}

\begin{itemize}
\item<1->
The main
motivation for players to engage in forward contracts is that of
risk diversification.
\item<2->
Producers have made large investments with the
aim of recouping them over a long period of time as well as making a
return on them.
\item<3->
Retailers (which might be intermediaries and/or use the commodity in
their production process) also have an incentive to hedge their
positions in the market by contracting forwards that help diversify
their risks.
\item<4->
Exposure to the market will differ both between producers and
retailers as well as within their own group.
So the need for risk-diversification has a temporal dimension.
\end{itemize}
}


\frame{\frametitle{Market Risk Premium}

\begin{itemize}
\item<1-> These differences in the
desire to hedge positions are employed to explain the market risk premium and
its sign.
\item<2-> Retailers are less incentivized to contract commodity forwards
the further out we look into the market.
\item<3-> In contrast, on the producers' side the need to hedge in the long-term
does not fade away as quickly.
\end{itemize}

}

\frame{\frametitle{Market Risk Premium}
\begin{itemize}
\item<1-> We associate situations where
$\pi(t,T)>0$ with the fact that retailers' desire to cover their
positions `outweighs' those of the producers, resulting in a
positive market risk premium.
\item<2-> The mirror image is therefore one
where the producers' desire to hedge their positions outweighs that
of the retailers resulting in a negative market risk premium.

\end{itemize}
}



\frame{\frametitle{Representative Agents}

\begin{itemize}
\item<1-> We describe producers' and retailers'
preferences via the utility function of two representative agents.
\item<2-> Agents
must decide how to manage their exposure to the spot and forward
markets for every future date $T$.
\item<3->
A key question for the producer
is how much of his future production, which cannot be predicted with
total certainty, will he wish to sell on the forward market or, when
the time comes, sell it on the spot market.
\item<4-> Similarly, the retailer
must decide how much of her future needs, which cannot be predicted
with full certainty either, will be acquired via the forward markets
and how much on the spot.
\end{itemize}
}

\frame{\frametitle{Representative Agents}
We approach this financial decision and
equilibrium price formation in two steps.

\begin{itemize}
\item<1-> First, we determine the
forward price that makes the agents indifferent between the forward
and spot market.
\item<2-> Second, we discuss how the relative willingness
of producers and retailers to hedge their exposures determines
market clearing prices.
\end{itemize}


}

\frame{\frametitle{Representative Agents}

We assume that the risk preferences of the representative agents are
expressed in terms of an exponential utility function parameterized
by the risk aversion constant $\gamma>0$;
$$
U(x)=1-\exp(-\gamma x)\,.
$$
We let $\gamma:=\gamma_p$ for the producer and $\gamma:=\gamma_c$
for the consumer.

}


\frame{\frametitle{The Model}
%Let $(\Omega,\mathcal{F},P)$ be a probability space equipped with a
%filtration $\mathcal{F}_t$.

We assume that the electricity spot price follows a
mean-reverting multi-factor additive process
\begin{equation}\label{equation for additive stock price}
S_t=\Lambda(t)+\sum_{i=1}^mX_i(t)+\sum_{j=1}^nY_j(t)
\end{equation}
where $\Lambda(t)$ is the deterministic seasonal spot price level,
while $X_i(t)$ and $Y_j(t)$ are the solutions to the stochastic
differential equations
\begin{equation}
dX_i(t)=-\alpha_i X_i(t)\,dt+\sigma_i(t)\,dB_i(t)
\end{equation}
and
\begin{equation}
dY_j(t)=-\beta_j Y_j(t)\,dt+dL_j(t).
\end{equation}
$B_i(t)$, $i=1,\ldots,m$, are standard independent Brownian
motions, $\sigma_i(t)$ are deterministic volatility functions
and $L_j(t)$, $j=1,\ldots,n$ are independent L\'evy
processes.
}


\frame{\frametitle{The Model}

The processes $Y_j(t)$ are
zero-mean reverting processes responsible for the spikes or large
deviations which revert at a fast rate $\beta_j>0$.\\*[12pt]

$X_i(t)$
are zero-mean reverting processes that account for the normal
variations in the spot price evolution with lower degree of
mean-reversion $\alpha_i>0$.

%It is thus natural from a market point
%of view to assume that $\max_i\alpha_i<\min_j\beta_j$, although this
%is not necessary in what follows.

}



\frame{\frametitle{Indifference Prices}

Assume that the producer will deliver the spot over the time
interval $[T_1,T_2]$.\\*[12pt]

He has the choice to deliver the production in
the spot market, where he faces uncertainty in the prices over the
delivery period, or to sell a forward contract with delivery over
the same period.\\*[12pt]

The producer takes this decision at time $t\leq
T_1$.


}
\frame{\frametitle{Indifference Prices}
We determine the forward price that makes the producer indifferent
between the two alternatives: denote by $F_{\hbox{pr}}(t,T_1,T_2)$
the forward price derived from the equation
$$
\begin{array}{ll}


& 1-\E^P\left[\exp\left(-\gamma_p\int_{T_1}^{T_2}S(u)\,du\right)\,|\,
\mathcal{F}_t\right]\\*[12pt]
= & 1-\E^P\left[\exp\left(-\gamma_p(T_2-T_1)F_{\hbox{pr}}(t,T_1,T_2)\right)\,|\,
\mathcal{F}_t\right]
\end{array}
$$
}
\frame{\frametitle{Indifference Prices}
Equivalently,
\begin{equation}
\label{def-producer}
F_{\hbox{pr}}(t,T_1,T_2)=-\frac1{\gamma_p}\frac1{T_2-T_1}\ln\E^P
\left[\exp\left(-\gamma_p\int_{T_1}^{T_2}S(u)\,du\right)\,|\,
\mathcal{F}_t\right]\,,
\end{equation}
where for simplicity we have assumed that the risk-free interest
rate is zero.

$\int_{T_1}^{T_2}S(u)\,du$ is what the
producer collects from selling the commodity on the spot market
over the delivery period $[T_1,T_2]$, while he receives
$(T_2-T_1)F_{\hbox{pr}}(t,T_1,T_2)$ from selling it on the forward
market.

}
\frame{\frametitle{Notation}
For $i=1,\ldots,m$ and $j=1,\ldots,n$,
\begin{equation}\label{alpha bar}
\bar{\alpha}_i(s,T_1,T_2)=\left\{\begin{array}{lll}
\frac1{\alpha_i}\left(\e^{-\alpha_i(T_1-s)}-\e^{-\alpha_i(T_2-s)}\right) &
, &
s\leq T_1\,, \\
\frac1{\alpha_i}\left(1-\e^{-\alpha_i(T_2-s)}\right) & , & s\geq T_1\,.
\end{array}\right.
\end{equation}
and
\begin{equation}\label{beta bar}
\bar{\beta}_j(s,T_1,T_2)=\left\{\begin{array}{lll}
\frac1{\beta_j}\left(\e^{-\beta_j(T_1-s)}-\e^{-\beta_j(T_2-s)}\right) & ,
&
s\leq T_1\,, \\
\frac1{\beta_j}\left(1-\e^{-\beta_j(T_2-s)}\right) & , & s\geq T_1\,.
\end{array}\right.
\end{equation}



}
\frame{\frametitle{Indifference Prices}
The price for which the producer is indifferent between the forward
and spot market is given by
\begin{eqnarray*}
F_{\text{pr}}(t,T_1,T_2)&=&\frac{1}{T_2-T_1}\int_{T_1}^{T_2}\Lambda(u)\,du\\
&&+\sum_{i=1}^m\frac{\bar{\alpha}_i(t,T_1,T_2)}{T_2-T_1}X_i(t)+\sum_{j=1}^n
\frac{\bar{\beta}_j(t,T_1,T_2)}{T_2-T_1}Y_j(t) \\
&&-\frac{\gamma_p}{2(T_2-T_1)}\int_t^{T_2}\sum_{i=1}^m
\sigma_i^2(s)\bar{\alpha}_i^2(s,T_1,T_2)\,ds
\\&&-\frac1{\gamma_p}\frac1{T_2-T_1}\int_t^{T_2}\sum_{j=1}^n
\phi_j\left(-\gamma_p\bar{\beta}_j(s,T_1,T_2)\right)\,ds\,,
\end{eqnarray*}
where $ \bar{\alpha}_i$ and $\bar{\beta}_j$ are given by
\eqref{alpha bar} and \eqref{beta bar} respectively.
}



\frame{\frametitle{Indifference Price -- Consumer}

The consumer will derive the indifference price from the incurred expenses
in the spot or forward market, which entails
$$
\begin{array}{ll}
&1-\E^P\left[\exp\left(-\gamma_c\left(-\int_{T_1}^{T_2}S(u)\,du\right)\right)
\,|\,\mathcal{F}_t\right]\\
= &1-\E^P\left[\exp\left(-\gamma_c(-(T_2-T_1)F_{\text{c}}(t,T_1,T_2)\right))\,|\,
\mathcal{F}_t\right]\,,
\end{array}
$$
or,
\begin{equation}
F_{\text{c}}(t,T_1,T_2)=\frac1{\gamma_c}\frac1{T_2-T_1}\ln\E^P\left[
\exp\left(\gamma_c\int_{T_1}^{T_2}S(u)\,du\right)\,|\,\mathcal{F}_t\right]\,.
\end{equation}

}
\frame{\frametitle{Indifference Price -- Consumer}
The price that makes the consumer indifferent between the forward
and the spot market is given by
\begin{align*}
F_{\text{c}}(t,T_1,T_2)&=\frac1{T_2-T_1}\int_{T_1}^{T_2}\Lambda(u)\,du
+\sum_{i=1}^m\frac{\bar{\alpha}_i(t,T_1,T_2)}{T_2-T_1}X_i(t)\\
&\qquad+\sum_{j=1}^n\frac{\bar{\beta}_j(t,T_1,T_2)}{T_2-T_1}Y_j(t) \\
&\qquad+\frac{\gamma_c}{2(T_2-T_1)}\int_t^{T_2}
\sum_{i=1}^m\sigma_i^2(s)\bar{\alpha}_i^2(s,T_1,T_2)\,ds \\
&\qquad+\frac1{\gamma_c}\frac1{T_2-T_1}\int_t^{T_2}\sum_{j=1}^n
\phi_j\left(\gamma_c\bar{\beta}_j(s,T_1,T_2)\right)\,ds\,.
\end{align*}

}
\frame{\frametitle{Indifference Price -- Bounds}
Note that the producer prefers to sell his production in the forward
market as long as the market forward price $F(t,T_1,T_2)$ is higher
than $F_{\text{pr}}(t,T_1,T_2)$. On the other hand, the consumer
prefers the spot market if the market forward price is more
expensive than his indifference price $F_{\text{c}}(t,T_1,T_2)$.
Thus, we have the bounds
\begin{equation}\label{bounds for forward}
F_{\text{pr}}(t,T_1,T_2)\leq F(t,T_1,T_2)\leq
F_{\text{c}}(t,T_1,T_2)\,.
\end{equation}
}

\frame{\frametitle{Market Power}
\begin{itemize}
\item<1-> We introduce the deterministic function $p(t,T_1,T_2)\in[0,1]$
describing the \emph{market power of the representative producer}.
\item<2-> For $p(t,T_1,T_2)=1$ the
producer has full market power and can charge the maximum price possible in the forward market (short-term positions),
namely $F_{\text{c}}(t,T_1,T_2)$.
\item<3-> If the
consumer has full power, ie $p(t,T_1,T_2)=0$ (long-term positions), she will drive the
forward price as far down as possible which corresponds to
$F_{\text{pr}}(t,T_1,T_2)$.

\end{itemize}





}
\frame{\frametitle{Market Power}

For any market power $0<p(t,T_1,T_2)<1$,\\
the forward price $F^p(t,T_1,T_2)$ is defined to be
\begin{eqnarray}
\nonumber
F^p(t,T_1,T_2)&=&p(t,T_1,T_2)F_{\text{c}}(t,T_1,T_2)\\*[12pt]
&&+(1-p(t,T_1,T_2))
F_{\text{pr}}(t,T_1,T_2).
\end{eqnarray}

}
\frame{\frametitle{Market Power}
For $0\leq t\leq T_1<T_2$ the forward prices are
$$\begin{array}{ll}
& F^p(t,T_1,T_2)\\
&=\frac1{T_2-T_1}\int_{T_1}^{T_2}\Lambda(u)\,du
+\sum_{i=1}^m\frac{\bar{\alpha}_i(t,T_1,T_2)}{T_2-T_1}X_i(t)+
\sum_{j=1}^n\frac{\bar{\beta}_j(t,T_1,T_2)}{T_2-T_1}Y_j(t) \\*[12pt]
&\qquad+\frac{p(t,T_1,T_2)(\gamma_{\text{pr}}+
\gamma_{\text{c}})-\gamma_{\text{pr}}}{2(T_2-T_1)}\int_t^{T_2}
\sum_{i=1}^m\sigma_i^2(s)\bar{\alpha}_i^2(s,T_1,T_2)\,ds \\*[12pt]
&\qquad+\frac{p(t,T_1,T_2)}{\gamma_{\text{c}}(T_2-T_1)}\int_t^{T_2}
\sum_{j=1}^n\phi_j(\gamma_{\text{c}}\bar{\beta}_j(s,T_1,T_2))\,ds \\*[12pt]
&\qquad-\frac{1-p(t,T_1,T_2)}{\gamma_{\text{pr}}(T_2-T_1)}
\int_t^{T_2}\sum_{j=1}^n\phi_j(-\gamma_{\text{pr}}\bar{\beta}_j(s,T_1,T_2))
\,ds\,,
\end{array}
$$

}



\frame{\frametitle{Constant Market Power and Poisson jumps}
We consider a forward market consisting of 52 contracts with weekly delivery.
The market power is supposed to be constant $p(t,T_1,T_2)=p\in[0,1]$.
Assume that the spot model has $m=52$ diffusion components $X_i(t)$, and
one ($n=1$) jump component $Y(t)$. Suppose that the seasonal
function is
$$
\Lambda(t)=150+20\cos(2\pi t/365)\,,
$$
and the mean-reversion parameters for the diffusion components are
$\alpha_i=0.067/i$, with volatility $\sigma_i=0.3/\sqrt{i}$, for
$i=1,\ldots,52$.
}
\frame{\frametitle{Model Specification}

\begin{itemize}
\item<1-> We mimic here a sequence of mean-reverting
processes with decreasing speeds of mean reversion and with
decreasing volatility.
\item<2-> The speed of mean reversion equal to
$0.067$ means that a shock will be halved over 10 days.
\item<3-> The jump
process is driven by $L(t)=\eta N(t)$, where $N(t)$ is a Poisson
process with intensity $\lambda$ and the jump
 size is constant, equal to $\eta$.
\end{itemize}
}
\frame{\frametitle{Model Specification}

\begin{itemize}
\item<1-> The mean-reversion for the jump component is $\beta=0.5$, meaning
that a jump will, on average, revert back in two days.
\item<2-> We have
a combination of slow mean reverting normal variations and fast mean
reverting spikes in the spot market.
\item<3-> The frequency of spikes is set
to $\lambda=2/365$, i.e. two spikes, on average, per year.

\end{itemize}
}
\frame{\frametitle{Model Specification}
\begin{itemize}
\item<1-> Time $t=0$ corresponds to January 1, and we assume that the initial spot
price is $S(0)=172$.
\item<2-> We let
$X_1(0)=2$, and $X_i(0)=Y(0)=0$ for $i=2,\ldots,52$ to achieve this.
\item<3-> The risk aversion coefficients of the producer and consumer are set
equal to $\gamma_c=\gamma_{pr}=0.5$.
\item<4-> We derive
forward curves for weakly settled forward contracts over a year.

\end{itemize}
}
\frame{\frametitle{Indifference price with forward
curves for positive jumps}
%The indifference price curves together with the forward
%curves for market powers equal to $p=0.25, 0.5$ and $p=0.75$, in
%increasing order. The forecasted curve is depicted `+'. The jumps
%are positive of size 10.
\begin{center}
\includegraphics[width=10cm,height=6cm]{../../../pics/posjump}
\end{center}

}
\frame{\frametitle{Market risk premium -- positive jumps}
\begin{itemize}
\item<1-> Market clearing
forward prices are increasing with increasing market power, since
the producer will command higher prices with more power.
\item<2-> For a low market power of 0.25, we observe that the
forecasted price curve is below the forward curve in the shorter
end, while in the medium to long end we see the opposite.
\item<3-> This
corresponds to a positive market risk premium in the shorter end,
whereas it becomes negative in the medium and longer end.
\item<4-> The consumer wishes to avoid upward jumps in the price and is, even for a weak producer,
willing to accept a positive market risk premium in
the short end. In the long end, the effect of jumps vanish as a
consequence of mean reversion, so the consumer will have more power.
\end{itemize}

}
\frame{\frametitle{Market risk premium -- positive jumps}

To illustrate this particular example we have plotted
the difference of the forward curve with market power 0.25 and the
forecasted curve. For the contracts with
delivery up to approximately week 20, the market premium is
positive. The premium decreases with time to delivery, and becomes
negative in the medium and long end.

}
\frame{\frametitle{Market risk premium -- positive jumps}

%The market risk premium given by the difference of the forward
%curve with market power 0.25 and the forecasted curve.
\begin{center}
\includegraphics[width=10cm,height=6cm]{../../../pics/posjumpmp}
\end{center}

}

\frame{\frametitle{Empirical Example: German Market}

\begin{itemize}
\item  We need to estimate the physical parameters
of our two-factor model.
\item From forward market data, denoted by
$F(t,T_1,T_2)$, we estimate the risk-aversion coefficients for both
producers and retailers and estimate the producer's market power.
\end{itemize}
}
\frame{\frametitle{Data used}
\begin{itemize}
\item Spot prices: Phelix base load traded at the EEX.
\item Forward contract prices with delivery periods: monthly, quarterly and yearly.
\item Period covered: January 2 2002 to January 1 2006 with 1461 spot price
observations.
\item Forward data: $108$ contracts with
monthly delivery, $35$ contracts with quarterly delivery and $12$
contracts with yearly delivery.
\end{itemize}
}
\frame{\frametitle{Spot model specification}

We apply the model to
\[S(t)=\Lambda(t)+X(t)+Y(t)\]
where, $\Lambda(t)$ is the seasonal component,
\begin{equation}
dX(t)=-\alpha X(t)dt+\sigma dB(t)
\end{equation}
where $\alpha\geq 0$,  $\sigma \geq 0$ and $B(t)$ is a
standard Brownian motion,

}
\frame{\frametitle{Spot model specification}

\begin{equation}
dY(t)=-\beta Y(t)dt+dL(t)
\end{equation}
with $\beta \geq 0$ and
\begin{equation}
L(t)=\sum^{N(t)}_iJ_i
\end{equation}
is a compound Poisson process.

$N(t)$ is a homogeneous Poisson
process with intensity $\lambda$ and $J_i$'s are i.i.d. with
exponential density function
\[f(j)=p\lambda_1e^{-\lambda_1j}\textbf{1}_{j>0}+(1-p)\lambda_2e^{-\lambda_2|j|}\textbf{1}_{j<0},\]
where $\lambda_1>0$ and $\lambda_2>0$ are responsible for the
decay of the tails for the distribution.

We assume that $N(t)$, $J$ and $B(t)$ are independent.

}
\frame{\frametitle{Spot model specification}

For the seasonal component we assume
\begin{eqnarray*}
\Lambda(t) & = & a_0+a_1\textbf{1}_{\left\{t=Su\right\}}+a_2\textbf{1}_{\left\{t=Mo\right\}}+a_3\textbf{1}_{\left\{t=Tu,We,Th\right\}}+a_4\textbf{1}_{\left\{t=Sa\right\}} \\
& & +a_5\cos\left[\frac{6\pi}{365}\left(t+a_6\right)\right]+a_7t,
\end{eqnarray*}
where the indicator function is acting on the different days of the
week.

}

\frame{\frametitle{Risk aversion coefficients}

Recall that $F_c(t,T_1,T_2)$ (upper bound) and
$F_{pr}(t,T_1,T_2)$ (lower bound) depend on the choice of
$\gamma_{c}$ and $\gamma_{pr}$, we estimate $\gamma_{pr}$ and
$\gamma_{c}$ by minimizing the distance between $F_c(t,T_1,T_2)$,
$F_{pr}(t,T_1,T_2)$ and the market prices of forwards
$F(t,T_1,T_2)$, respectively, in the following way.

}
\frame{\frametitle{Risk aversion coefficients}

\begin{itemize}
\item For all trading days $t\in[1,1461]$,  we determine all values of $\gamma_{pr}$ and $\gamma_c$
such that
\begin{equation}\label{Ebounds2}
F_{pr}(t,T_1,T_2)\leq F(t,T_1,T_2)\leq F_c(t,T_1,T_2).
\end{equation}
\item
We define the intervals
$I_{pr}^t$ and $I_{c}^t$ containing values for $\gamma_{pr}$ and
$\gamma_{c}$ by guaranteeing that (\ref{Ebounds2}) holds.
\item
For the intersection of all these interval no forward prices $F(t,T_1,T_2)$ will
lay outside the bounds $F_{pr}(t,T_1,T_2)$ and $F_{c}(t,T_1,T_2)$.
\item
We find that $\gamma_{pr}\in  [0.421,\infty)$
and $\gamma_{c}\in [0.701,\infty)$.
\item
Thus we choose $\gamma_{pr}=0.421$ and $\gamma_{c}=0.701$.
\end{itemize}

}
\frame{\frametitle{Market power and market risk }

Recall
\[p(t,T_1,T_2)=\frac{F(t,T_1,T_2)-F_{pr}(t,T_1,T_2)}{F_c(t,T_1,T_2)-F_{pr}(t,T_1,T_2)}\]
 and
 \[\pi(t,T_1,T_2)=F(t,T_1,T_2)-\mathbb{E}^{P}\left[\frac{1}{T_2-T_1}\int_{T_1}^{T_2}S(u)du|\mathcal{F}_t\right].\]

}
\frame{\frametitle{Market power and market risk }

We consider three periods

\begin{table}[h]
\begin{center}
\begin{tabular}{c c c c c}
$t$ & Type & \# Contracts & Delivery Periods &
$F(t,T_1,T_2)$\\ \hline
01/Jan/2002 & monthly & 18 & Jan 2002 - May 2003 & $F(2,T_1,T_2)$\\
01/Jan/2002 & quarterly & 7 & 2nd qtr 2002 - 4th qtr 2003 & $F(2,T_1,T_2)$\\
01/Jan/2002 & yearly & 3 & 2003 - 2005 & $F(2,T_1,T_2)$\\
\hline
03/Mar/2003 & monthly & 7 & Feb 2003 - Aug 2003 & $F(400,T_1,T_2)$\\
03/Mar/2003 & quarterly & 7 & 2nd qtr 2003 - 4th qtr 2004 & $F(400,T_1,T_2)$\\
03/Mar/2003 & yearly & 3 & 2004 - 2006 & $F(400,T_1,T_2)$\\
\hline
04/Oct/2005 & monthly & 7 & Oct 2005 - Apr 2006 & $F(1373,T_1,T_2)$\\
04/Oct/2005 & quarterly & 7 & 1st qtr 2006 - 3rd qtr 2007 & $F(1373,T_1,T_2)$\\
04/Oct/2005 & yearly & 6 & 2006 - 2011 & $F(1373,T_1,T_2)$\\
\end{tabular}
\end{center}
\caption{Forward contracts}\label{table with the 3 lots of forward
contracts}
\end{table}

} \frame{\frametitle{Producer's market power and market risk
premium, 18 monthly contracts with $t=$ January 2 2002}

\begin{figure}[htbp]
\includegraphics[width=10cm,height=6cm]{../../../pics/picFmonth1}
%\caption{Producer's market power and market risk premium, 18
%monthly contracts with $t=$ January 2 2002} \label{figure market
%power monthly forwards 2002}
\end{figure}


} \frame{\frametitle{Producer's market power and market risk
premium, 7 quarterly contracts with $t=$ second quarter 2002}

\begin{figure}
\includegraphics[width=10cm,height=6cm]{../../../pics/picFquarter1}
%\caption{Producer's market power and market risk premium, 7
%quarterly contracts with $t=$ second quarter 2002} \label{figure
%market power quarterly forwards 2002}
\end{figure}

} \frame{\frametitle{Producer's market power and market risk
premium, 3 yearly contracts with $t=$ 2002}


\begin{figure}[htbp]
\includegraphics[width=10cm,height=6cm]{../../../pics/picFyear1}
%\caption{Producer's market power and market risk premium, 3 yearly
%contracts with $t=$ 2002} \label{figure market power yearly
%forwards 2002}
\end{figure}

}



\subsection{Information Approach}


\frame{\frametitle{Information Approach}
\begin{itemize}
\item<1-> As electricity is non-storable future predictions about the market will not affect the current spot price, but will affect forward prices.
\item<2-> Stylized example: planned outage of a power plant in one month
\item<3-> Market example: in 2007 the market knew that in 2008 CO$_2$ emission costs will be introduced; this had a clearly observable effect on the forward prices!
\item<4-> German moratorium 2011: shut-down 7 nuclear power plants for 3 months with possible complete shut-down.
\end{itemize}
}


\begin{frame}
  \frametitle{German Moratorium I}
%\vspace{-0.5cm}

\begin{figure}[htbp]

  \includegraphics[width=0.8\textwidth]{../../../pics/spotdata.pdf}
    \caption{EEX spot prices}
\end{figure}

\end{frame}

\begin{frame}
  \frametitle{German Moratorium II}
%\vspace{-0.5cm}

\begin{figure}[htbp]

  \includegraphics[width=0.8\textwidth]{../../../pics/Mai2011graph.pdf}
    \caption{EEX forward prices delivery May 2011}
\end{figure}

\end{frame}
\begin{frame}
  \frametitle{German Moratorium III}
%\vspace{-0.5cm}

\begin{figure}[htbp]

  \includegraphics[width=0.8\textwidth]{../../../pics/August2011graph.pdf}
    \caption{EEX forward prices delivery August 2011}
\end{figure}

\end{frame}


\begin{frame}
  \frametitle{Example: 2008 CO$_2$ Emission Costs}
\vspace{-0.5cm}

\begin{figure}[htbp]
  \centering
  \subfigure{
    \includegraphics[width=0.47\textwidth]{../../../pics/forward3.pdf}
  }
  \subfigure{
    \includegraphics[width=0.47\textwidth]{../../../pics/forward2.pdf}
  }
  \caption{EEX Forward prices observed on 01/10/06 (left) and 01/10/07 (right)}
\end{figure}

\begin{itemize}
\item Typical winter and bank holidays behaviour in both graphs
\item General upward shift in 2008 \\ \vspace{0.2cm}
\textcolor{red}{$~~~~~~ \Rightarrow$ 2nd phase of $CO_2$ certificates}
\end{itemize}
\end{frame}



\begin{frame}
 \frametitle{Information Approach}
\vspace{-0.5cm}

\begin{itemize}
\item<1-> Future information is incorporated in the forward price
\item<2-> ... but not necessarily in the spot price due to \textcolor{red}{non-storability}
\item<3-> ... buy-and-hold strategy does not work
%\item Thus:
\end{itemize}
%\pause
%\begin{block}{Efficient Markets Hypothesis (semi-strong)}
%The price (spot) now reflects all publicly available information.
%\end{block}
%\begin{itemize}
%\item ... is not valid on electricity markets!
%\end{itemize}
\end{frame}


\begin{frame}
  \frametitle{Information Approach}
\vspace{-0.5cm}

\begin{itemize}
\item The usual pricing relation between spot and forward:
\begin{align*}
F(t,T)=\Ef^\mathbb{Q}[S_T|\mathcal{F}_t]
\end{align*}
\item Not sufficient: natural filtration $\mathcal{F}_t = \sigma(S_s, s\leq t)$
\vspace{0.5cm}
\pause
\item Idea: \textcolor{red}{enlarge the filtration!}
\vspace{0.5cm}
\pause
\item ... by information about the spot at some future time $T_\Upsilon$
\item Info could be that spot will be in certain interval...
\item ... or the value of a correlated process (temperature)
\end{itemize}
\end{frame}


\begin{frame}
  \frametitle{The Information Premium}
\vspace{-0.5cm}

\begin{itemize}
\item Quantify the influence of future information using:
\end{itemize}
\begin{block}{Information Premium}
The information premium is defined to be
\begin{align*}
I(t,T) = \Ef[S_T | \mathcal{G}_t] - \Ef[S_T | \mathcal{F}_t]
\end{align*}
i.e. the difference between the prices of the forward under $\mathcal{G}$ and $\mathcal{F}$.
\end{block}
\end{frame}





