% !TEX root = StructuringValuation_ws1415UDE.tex
\part{Forward Markets}
\section{Basic Definitions and Models}
\subsection{Basic Pricing Relations for Futures}

% frametitle
{Storage, Inventory and Convenience Yield}

% begin itemize

	The theory of storage aims to explain the differences between spot
and Futures (Forward) prices by analyzing why agents hold inventories.

	Inventories
allow to meet unexpected demand, avoid the cost of frequent revisions in
the production schedule and eliminate manufacturing disruption.

	This
motivates the concept of convenience yield as a benefit, that accrues to the
owner of the physical commodity but not to the holder of a forward contract.

	Thus the convenience yield is comparable to the dividend yield for stocks.

	A modern view is to view storage (inventory) as a timing option, that
allows to put the commodity to the market when prices are high
and hold it when the prices are low.

% end itemize

% frametitle
{Spot-Forward Relationship in Commodity Markets }
Under the no-arbitrage assumption we have
\begin{equation}\label{SF-rel}
F(t,T)=S(t)e^{(r-y)(T-t)}
\end{equation}
where $r$ is the interest rate at time $t$ for maturity $T$ and $y$ is the convenience yield.

% frametitle
{Spot-Forward Relationship in Commodity Markets }

Observe that (\ref{SF-rel}) implies

% begin itemize

	spot and forward are redundant (one can replace the other) and form a
linear relationship (unlike options)

	with two forward prices we can derive the value of $S(t)$ and $y$

	knowledge of $S(t)$ and $y$ allows us to construct the whole forward curve

	for $r-y <0$ we have backwardation; for $r-y>0$ we have contango.

% end itemize

% frametitle
{Spot-Forward Relationship: Classical theory}

% begin itemize

	In a stochastic model we use
$$
F(t,T)=\EX_{\Q}(S(T)|\F_t)
$$
where $\F_t$ is the accumulated available market information (in most models the information generated by the spot price).

	$\Q$ is a risk-neutral probability

% begin itemize

	discounted spot price is a $\Q$-martingale

	fixed by calibration to market prices or a market price of risk argument

% end itemize

% end itemize

% frametitle
{Market Risk Premium}

% begin itemize

	The \emph{market risk premium} or \emph{forward bias} $\pi (t,T)$
relates forward and expected spot prices.

	It is defined as the difference, calculated at time $t$, between
the forward $F(t,T)$ at time $t$ with delivery at $T$ and expected
spot price:
\begin{equation}\label{forward risk premium}
\pi(t,T)= F(t,T)-\EX^\prb[S(T)|\mathcal F_t].
\end{equation}
Here $\EX^\prb$ is the expectation operator, under the
historical measure $\prb$, with information up until time $t$ and
$S(T)$ is the spot price at time $T$.

% end itemize

\subsection{Black's Formula}

% frametitle
{Simple Futures Market Model}

% begin itemize

	We assume for the futures dynamics
$$
df(t) = \mu f(t) dt + \sigma f(t) dW(t).
$$

	But since it costs nothing to enter a futures position, we can constantly change our futures portfolio.
Thus financing for buying a traded asset is not relevant.

% end itemize

% frametitle
{Black's Formula}

% begin itemize

	We use the usual notation -
strike $K$, expiry $T$, $\tau=T-t$ for time to maturity as in the spot case, and write $\Phi$ for the
standard normal distribution function.

	The arbitrage price $C$ of a European futures call option is
\begin{equation}\label{Black}
C(t)= e^{-r\tau} (f \Phi(d_1 (f,\tau)) - K \Phi(d_2 (f,\tau))),
\end{equation}
where
$$
d_{1,2} (f,\tau) := \frac{\log (f/K) \pm \frac{1}{2} {\sigma}^2 \tau}{
\sigma \sqrt{\tau}}.
$$

% end itemize

% frametitle
{Black's Formula}

% begin itemize

	Observe that the quantities $d_1$ and $d_2$ do not depend on the
interest rate $r$.

	This is intuitively clear from the classical
Black approach: one sets up a replicating risk-free portfolio
consisting of a position in futures options and an offsetting
position in the underlying futures contract. The portfolio
requires no initial investment and therefore should not earn any
interest.

% end itemize

\section{Bessembinder - Lemon model}

% frametitle
{Bessembinder- Lemon Model specification}

% begin itemize

	One-period model

	Power companies are able to forecast demand in the immediate future with precision

	$N_P$ identical producers; $N_R$ identical retailers that buy power in the wholesale market and sell it to final consumers at fixed unit price

	$P_R$ fixed unit price that consumers pay

	$Q_{R_i}$ an exogenous random variable that denotes the realized demand for retailer $i$

% end itemize

% frametitle
{The cost function}

% begin itemize

	Each producer $i$ has cost function
$$
TC_i=F+\frac{a}{c}(Q_{P_i})^c,
$$
where $F$ are fixed costs, $Q_{P_i}$ is the output of producer $i$, and $c\geq 2$.

	The cost function implies that the marginal production costs increase with output.

	If $c>2$ marginal costs increase at an increasing rate with output.

	Moreover, the distribution of power prices will be positively skewed even when the distribution of power demand is symmetric.

% end itemize

% frametitle
{Clearing prices}

% begin itemize

	First, assume that forward prices are gives

	Obtain optimal behaviour in the spot market

	Work back and find optimal positions in the forward market.

% end itemize

% frametitle
{The wholesale spot market}

% begin itemize

	Producers sell to retailers who in turn distribute to power consumers

	$P_W$ denotes the wholesale spot price, $Q_{P_i}^W$ quantity sold by producer $i$ in the wholesale spot market, $Q_{P_i}^F$ quantity
that producer $i$ has agreed to deliver (purchase if negative) in the forward market at the fixed forward price $P_F$.

	The ex-post profit of producer $i$ is given by
$$
\pi_{P_i}=P_W Q_{P_i}^W + P_FQ_{P_i}^F
-F-\frac{a}{c}(Q_{P_i})^c,$$
where each producer's physical production, $Q_{P_i}$, is the sum of its spot and forward sales $Q_{P_i}^W+Q_{P_i}^F$.

% end itemize

% frametitle
{The wholesale spot market}

% begin itemize

	Retailers buy in the real-time wholesale market the difference between realised retail demand and their forward positions

	$Q_{R_j}^F$ quantity sold (purchased if negative) forward by retailer $j$, $P_R$ fixed retail price per unit

	The ex-post profit for each retailer is
$$
\pi_{R_j}=P_R Q_{R_j} + P_FQ_{R_j}^F - P_W (Q_{R_j}+Q_{R_j}^F),$$

	The profit maximising quantity for producer $i$ is (FOC wrt $Q_{P_i}^W$)
$$Q_{P_i}^W=\left(\frac{P_W}{a}\right)^{x}-Q_{P_i}^F$$
with $x=1/(c-1)$

% end itemize

% frametitle
{The wholesale spot market}

% begin itemize

	The equilibrium total retail demand is equal to total production and forward contracts are in zero net supply

	Hence we must have that summing over all producers production must equal total demand from retailers
$$
N_P\left(\frac{P_W}{a}\right)^{x}=\sum_{i=1}^{N_R}Q_{R_i}^F
$$

% end itemize

% frametitle
{The wholesale spot market}

% begin itemize

	Therefore the market-clearing wholesale price is
$$
P_W=a \left(\frac{Q^D}{N_P}\right)^{c-1},$$
where $Q^D=\sum_{j=1}^{N_R}Q_{R_j}$ is total system demand. We see that when $c>2$ an increase in demand has a disproportionate effect on power prices.

	Each producers sale in the wholesale market is
$$
Q_{P_i}^W= \frac{Q^D}{N_P}-Q_{P_i}^F.
$$

% end itemize

% frametitle
{Demand for forward positions}

% begin itemize

	Producers profit (with no forwards) is
$$
\rho_{P_i}=P_W\frac{Q^D}{N_P}-F-\frac{a}{c}\left(\frac{Q^D}{N_P}\right)^{c}.$$

	Retailers profit (with no forwards) is
$$
\rho_{R_j}=P_RQ_{R_j}-P_WQ_{R_j}.
$$

% end itemize

% frametitle
{Mean-Variance Analysis for optimal forward position}
Assume that market players
$$
\max_{Q^F_{\{P_i,R_j\}}}\EX[\pi_{\{P_i,R_j\}}]-\frac{A}{2}\var[\pi_{\{P_i,R_j\}}]
$$
where, for example, producers have the profit function
$$
\pi_{P_i}=\rho_{P_i}+ P^FQ^F-P_WQ^F.
$$
FOCs imply
$$
Q^F_{\{P_i,R_j\}}= \frac{P^F-\EX[P_W]}{A\var[P_W]}+\frac{\Cov[\rho_{\{P_i,R_j\}},P_W]}{\var[P_W]}.
$$

% frametitle
{Mean-Variance Analysis for optimal forward position}

% begin itemize

	The optimal forward position contains two components

% begin itemize

	The first term reflects the position taken in response to the bias $P^F-\EX[P_W]$

	The second term is the quantity sold or bought forward to minimize the variance of profits

% end itemize

	Forward hedging can reduce risk precisely because the covariance term is generally non-zero.

% end itemize

% frametitle
{The equilibrium forward price}

% begin itemize

	One can show that
$$
P_F=\EX[P_W]-\frac{N_P}{Nca^x}\left[cP_R\Cov[P_W^{x}, P_W]-\Cov[P_W^{x+1}, P_W]\right],
$$
where $N=(N_R+N_P)/A$ reflects the number of firms in the industry and the degree to which they are concerned with risk.

	The forward price will be less than the expected wholesale price, if the first term in brackets, which reflects retail risk, is
larger than the second term, which reflects production cost risk.

	The equilibrium forward price will depend only on the risks borne by the industry as a whole: variability in retail revenue and in %production costs

% end itemize

% !TEX root = StructuringValuation_ws1415UDE.tex
\section{A Dynamic Equilibrium Approach}

% frametitle
{Equilibrium Approach -- Players}

% begin itemize
\item<1->
The main
motivation for players to engage in forward contracts is that of
risk diversification.
\item<2->
Producers have made large investments with the
aim of recouping them over a long period of time as well as making a
return on them.
\item<3->
Retailers (which might be intermediaries and/or use the commodity in
their production process) also have an incentive to hedge their
positions in the market by contracting forwards that help diversify
their risks.
\item<4->
Exposure to the market will differ both between producers and
retailers as well as within their own group.
So the need for risk-diversification has a temporal dimension.
% end itemize

% frametitle
{Market Risk Premium}

% begin itemize
\item<1-> These differences in the
desire to hedge positions are employed to explain the market risk premium and
its sign.
\item<2-> Retailers are less incentivized to contract commodity forwards
the further out we look into the market. We associate situations where
$\pi(t,T)>0$ with the fact that retailers' desire to cover their
positions `outweighs' those of the producers, resulting in a
positive market risk premium.
\item<3-> In contrast, on the producers' side the need to hedge in the long-term
does not fade away as quickly. Now the producers' desire to hedge their positions outweighs that of the retailers resulting in a negative market risk premium.
% end itemize

% frametitle
{Representative Agents}

% begin itemize
\item<1-> We describe producers' and retailers'
preferences via the utility function of two representative agents.
\item<2-> Agents
must decide how to manage their exposure to the spot and forward
markets for every future date $T$.
\item<3->
A key question for the producer
is how much of his future production, which cannot be predicted with
total certainty, will he wish to sell on the forward market or, when
the time comes, sell it on the spot market.
\item<4-> Similarly, the retailer
must decide how much of her future needs, which cannot be predicted
with full certainty either, will be acquired via the forward markets
and how much on the spot.
% end itemize

% frametitle
{Representative Agents}
We approach this financial decision and
equilibrium price formation in two steps.

% begin itemize
\item<1-> First, we determine the
forward price that makes the agents indifferent between the forward
and spot market.
\item<2-> Second, we discuss how the relative willingness
of producers and retailers to hedge their exposures determines
market clearing prices.
% end itemize

% frametitle
{Representative Agents}

We assume that the risk preferences of the representative agents are
expressed in terms of an exponential utility function parameterized
by the risk aversion constant $\gamma>0$;
$$
U(x)=1-\exp(-\gamma x)\,.
$$
We let $\gamma:=\gamma_p$ for the producer and $\gamma:=\gamma_r$
for the retailer.

% frametitle
{The Model}
%Let $(\Omega,\mathcal{F},P)$ be a probability space equipped with a
%filtration $\mathcal{F}_t$.
% begin itemize
\item<1->
We assume that the electricity spot price follows a
mean-reverting multi-factor additive process
\begin{equation}\label{equation for additive stock price}
S_t=\Lambda(t)+\sum_{i=1}^mX_i(t)+\sum_{j=1}^nY_j(t)
\end{equation}
\item<2->
where
% begin itemize
\item $\Lambda(t)$ is the deterministic seasonal spot price level,
\item $X_i(t)$ are zero-mean reverting processes that account for the normal
variations in the spot price evolution with lower degree of
mean-reversion.
\item $Y_j(t)$ are
zero-mean reverting processes responsible for the spikes or large
deviations which revert at a fast rate.
% end itemize
% end itemize

% frametitle
{Indifference Prices}

Assume that the producer will deliver the spot over the time
interval $[T_1,T_2]$.\\*[12pt]

He has the choice to deliver the production in
the spot market, where he faces uncertainty in the prices over the
delivery period, or to sell a forward contract with delivery over
the same period.\\*[12pt]

The producer takes this decision at time $t\leq
T_1$.

% frametitle
{Indifference Prices}
We determine the forward price that makes the producer indifferent
between the two alternatives: denote by $F_{\hbox{pr}}(t,T_1,T_2)$
the forward price derived from the equation
$$
\begin{array}{ll}

& 1-\E^P\left[\exp\left(-\gamma_p\int_{T_1}^{T_2}S(u)\,du\right)\,|\,
\mathcal{F}_t\right]\\*[12pt]
= & 1-\E^P\left[\exp\left(-\gamma_p(T_2-T_1)F_{\hbox{pr}}(t,T_1,T_2)\right)\,|\,
\mathcal{F}_t\right]
\end{array}
$$

% frametitle
{Indifference Price -- Producer}
Equivalently,
\begin{equation}
\label{def-producer}
F_{\hbox{pr}}(t,T_1,T_2)=-\frac1{\gamma_p}\frac1{T_2-T_1}\ln\E^P
\left[\exp\left(-\gamma_p\int_{T_1}^{T_2}S(u)\,du\right)\,|\,
\mathcal{F}_t\right]\,,
\end{equation}
where for simplicity we have assumed that the risk-free interest
rate is zero.

$\int_{T_1}^{T_2}S(u)\,du$ is what the
producer collects from selling the commodity on the spot market
over the delivery period $[T_1,T_2]$, while he receives
$(T_2-T_1)F_{\hbox{pr}}(t,T_1,T_2)$ from selling it on the forward
market.

% frametitle
{Indifference Price -- Retailer}

The retailer will derive the indifference price from the incurred expenses
in the spot or forward market, which entails
$$
\begin{array}{ll}
&1-\E^P\left[\exp\left(-\gamma_r\left(-\int_{T_1}^{T_2}S(u)\,du\right)\right)
\,|\,\mathcal{F}_t\right]\\
= &1-\E^P\left[\exp\left(-\gamma_r(-(T_2-T_1)F_{\text{r}}(t,T_1,T_2)\right))\,|\,
\mathcal{F}_t\right]\,,
\end{array}
$$
or,
\begin{equation}
F_{\text{r}}(t,T_1,T_2)=\frac1{\gamma_r}\frac1{T_2-T_1}\ln\E^P\left[
\exp\left(\gamma_r\int_{T_1}^{T_2}S(u)\,du\right)\,|\,\mathcal{F}_t\right]\,.
\end{equation}

% frametitle
{Indifference Price -- Bounds}
Note that the producer prefers to sell his production in the forward
market as long as the market forward price $F(t,T_1,T_2)$ is higher
than $F_{\text{pr}}(t,T_1,T_2)$. On the other hand, the retailer
prefers the spot market if the market forward price is more
expensive than his indifference price $F_{\text{r}}(t,T_1,T_2)$.
Thus, we have the bounds
\begin{equation}\label{bounds for forward}
F_{\text{pr}}(t,T_1,T_2)\leq F(t,T_1,T_2)\leq
F_{\text{r}}(t,T_1,T_2)\,.
\end{equation}

% frametitle
{Market Power}
% begin itemize
\item<1-> We introduce the deterministic function $p(t,T_1,T_2)\in[0,1]$
describing the \emph{market power of the representative producer}.
\item<2-> For $p(t,T_1,T_2)=1$ the
producer has full market power and can charge the maximum price possible in the forward market (short-term positions),
namely $F_{\text{r}}(t,T_1,T_2)$.
\item<3-> If the
retailer has full power, ie $p(t,T_1,T_2)=0$ (long-term positions), she will drive the
forward price as far down as possible which corresponds to
$F_{\text{pr}}(t,T_1,T_2)$.

% end itemize

% frametitle
{Market Power -- Forward Price}

For any market power $0<p(t,T_1,T_2)<1$,\\
the forward price $F^p(t,T_1,T_2)$ is defined to be
\begin{eqnarray}
\nonumber
F^p(t,T_1,T_2)&=&p(t,T_1,T_2)F_{\text{r}}(t,T_1,T_2)\\*[12pt]
&&+(1-p(t,T_1,T_2))
F_{\text{pr}}(t,T_1,T_2).
\end{eqnarray}

% frametitle
{figure}[htbp]
\includegraphics[width=10cm,height=6cm]{../../../pics/picFmonth1}
%\caption{Producer's market power and market risk premium, 18
%monthly contracts with $t=$ January 2 2002} \label{figure market
%power monthly forwards 2002}
\end{figure}

% frametitle
{figure}
\includegraphics[width=10cm,height=6cm]{../../../pics/picFquarter1}
%\caption{Producer's market power and market risk premium, 7
%quarterly contracts with $t=$ second quarter 2002} \label{figure
%market power quarterly forwards 2002}
\end{figure}

% frametitle
{figure}[htbp]
\includegraphics[width=10cm,height=6cm]{../../../pics/picFyear1}
%\caption{Producer's market power and market risk premium, 3 yearly
%contracts with $t=$ 2002} \label{figure market power yearly
%forwards 2002}
\end{figure}

\section{Information Approach}

% frametitle
{Information Approach}
% begin itemize
\item<1-> As electricity is non-storable future predictions about the market will not affect the current spot price, but will affect forward prices.
\item<2-> Stylized example: planned outage of a power plant in one month
\item<3-> Market example: in 2007 the market knew that in 2008 CO$_2$ emission costs will be introduced; this had a clearly observable effect on the forward prices!
\item<4-> German moratorium 2011: shut-down 7 nuclear power plants for 3 months with possible complete shut-down.
% end itemize

% frametitle
{German Moratorium I}
%\vspace{-0.5cm}

\begin{figure}[htbp]

  \includegraphics[width=0.8\textwidth]{../../../pics/spotdata.pdf}
    \caption{EEX spot prices}
\end{figure}

\end{frame}

% frametitle
{German Moratorium II}
%\vspace{-0.5cm}

\begin{figure}[htbp]

  \includegraphics[width=0.8\textwidth]{../../../pics/Mai2011graph.pdf}
    \caption{EEX forward prices delivery May 2011}
\end{figure}

\end{frame}

% frametitle
{German Moratorium III}
%\vspace{-0.5cm}

\begin{figure}[htbp]

  \includegraphics[width=0.8\textwidth]{../../../pics/August2011graph.pdf}
    \caption{EEX forward prices delivery August 2011}
\end{figure}

\end{frame}

% frametitle
{Example: 2008 CO$_2$ Emission Costs}
\vspace{-0.5cm}

\begin{figure}[htbp]
  \centering
  \subfigure{
    \includegraphics[width=0.47\textwidth]{../../../pics/forward3.pdf}
  }
  \subfigure{
    \includegraphics[width=0.47\textwidth]{../../../pics/forward2.pdf}
  }
  \caption{EEX Forward prices observed on 01/10/06 (left) and 01/10/07 (right)}
\end{figure}

% begin itemize
\item Typical winter and bank holidays behaviour in both graphs
\item General upward shift in 2008 \\ \vspace{0.2cm}
\textcolor{red}{$~~~~~~ \Rightarrow$ 2nd phase of $CO_2$ certificates}
% end itemize
\end{frame}

% frametitle
{Information Approach}
\vspace{-0.5cm}

% begin itemize
\item<1-> Future information is incorporated in the forward price
\item<2-> ... but not necessarily in the spot price due to \textcolor{red}{non-storability}
\item<3-> ... buy-and-hold strategy does not work
%\item Thus:
% end itemize
%\pause
%\begin{block}{Efficient Markets Hypothesis (semi-strong)}
%The price (spot) now reflects all publicly available information.
%\end{block}
%% begin itemize
%\item ... is not valid on electricity markets!
%% end itemize
\end{frame}

% frametitle
{Information Approach}
\vspace{-0.5cm}

% begin itemize
\item The usual pricing relation between spot and forward:
\begin{align*}
F(t,T)=\Ef^\mathbb{Q}[S_T|\mathcal{F}_t]
\end{align*}
\item Not sufficient: natural filtration $\mathcal{F}_t = \sigma(S_s, s\leq t)$
\vspace{0.5cm}
\pause
\item Idea: \textcolor{red}{enlarge the information set!}
\vspace{0.5cm}
\pause
\item ... by information about the spot at some future time $T_\Upsilon$
\item Info could be that spot will be in certain interval...
\item ... or the value of a correlated process (temperature)
% end itemize
\end{frame}

% frametitle
{The Information Premium}
\vspace{-0.5cm}

% begin itemize
\item Quantify the influence of future information using:
% end itemize
\begin{block}{Information Premium}
The information premium is defined to be
\begin{align*}
I(t,T) = \Ef[S_T | \mathcal{G}_t] - \Ef[S_T | \mathcal{F}_t]
\end{align*}
i.e. the difference between the prices of the forward under $\mathcal{G}$ and $\mathcal{F}$.
\end{block}
\end{frame}


