% !TEX root = QCF_ss14UDE.tex
\section{Modelling Permit Price Dynamics}
\subsection{Deterministic Equilibrium Models}
\begin{frame}
\frametitle{Rubin 1996: Firm i's optimization problem}
Firm i minimizes its cost by buying/selling an optimal quantity of emission permits and by emitting an optimal quantity of emissions, i.e.
\begin{align}
\min_{\theta_i, e_i} & \left\{ \int_0^T e^{-rt} [C_i(e_i(t)) + P(t)\theta_i(t)]dt \right\} \\
\textnormal{subject to }
&\dot{B}_i = S_i(t) - e_i(t) + \theta_i(t) \\
&            B_i(0) = 0 \textnormal{ and } B_i(t) \ge 0 \\
&            e_i(t) \ge 0
\end{align}


{Explanation of variables}
\begin{tiny}
\begin{tabular}{cl}
$e_i(t)$ & quantity of emissions \\
$\theta_i(t)$ & quantity of emission permits bought or sold \\
$S_i(t)$ & endowment of emissions \\
$B_i(t)$ & level of emissions in the bank\\
$C_i(e_i(t))$ & abatement cost function where $C'_i(e_i) < 0$ and $C''_i(e_i) > 0$ \\
$r$ & interest rate \\
\end{tabular}
\end{tiny}
\end{frame}

%16. Folie
\begin{frame}
\frametitle{Rubin 1996: Market equilibrium}
An intertemporal market equilibrium in emission permits over a T-period horizon consists of \\
$\qquad P^*(t) \ge 0$ (permit price) \\
$\qquad \theta^*(t) = (\theta^*_1(t), \ldots, \theta^*_N(t))$ (vector of optimal trading volumes) \\
$\qquad E^*(t) = (e^*_1(t), \ldots, e^*_N(t))$ (vector of optimal emission levels) \\
such that for a given $P^*(t)$,
$\theta^*(t)$ and $E^*(t)$ minimize each firm's costs subject to each firm's constraints as given in (2) - (4) and \\
the following two conditions hold
\begin{itemize}
\item Market clearing condition on permits \\
$
\qquad \sum_{i=1}^N \theta_i^*(t) = 0
$
\item Terminal stock condition \\
$
\qquad P^*(T)\sum_{i=1}^N B^*_i(T) = 0
$
\end{itemize}
\end{frame}

%17. Folie
\begin{frame}
\frametitle{Rubin 1996: Joint optimization problem}
A fictitious central planner minimizes total costs by choosing optimal quantities of emissions, i.e.
\begin{align}
\min_{e_1, \ldots, e_N} &\left\{ \int_0^T e^{-rt} \sum_{i=1}^N C_i(e_i(t)) dt \right\} \\
\textnormal{subject to }
&\dot{B}(t) = \sum_{i=1}^N \left(S_i(t) - e_i(t)\right) \\
&            B(0) = 0 \textnormal{ and } B(t) \ge 0 \\
&            e_i(t) \ge 0 \quad \textnormal{ for all } i = 1, \ldots, N
\end{align}
{Explanation of variables}
\begin{tiny}
\begin{tabular}{cl}
$S_i(t)$ & firm i's endowment of emissions \\
$B(t)$ & sum of emissions banked by the firms at time t\\
$C_i(e_i(t))$ & firm i's abatement cost when emitting $e_i(t)$ where $C'_i(e_i) < 0$ and $C''_i(e_i) > 0$ \\
$r$ & interest rate \\
\end{tabular}
\end{tiny}
\end{frame}

%18. Folie
\begin{frame}
\frametitle{Rubin 1996: Theorem (Market equilibrium and joint optimization problem)}
\begin{enumerate}[(a)]
\item There exists an intertemporal market equilibrium in emission permits over a T-period horizon
\item The market equilibrium solution is at least as inexpensive as the result of the joint cost optimization
\item The permit price equals the marginal abatement costs
\[
P(t) = - C'_i(e_i)
\]

\end{enumerate}
\end{frame}




\subsection{Stochastic Equilibrium Model}
%\subsubsection{Full Economy Model}
\begin{frame}
\frametitle{Carmona et al. 2008: Firm i's optimization problem}
For given forward permit price $A$ and prices of the produced goods $S$ the firm i maximizes its expected terminal wealth by  buying/selling an optimal number of permits and producing an optimal quantity of goods, i.e.
\begin{align}
\sup_{\theta^i, \xi^i} \mathbb{E} \left[ \underbrace{S^i(\xi^i) - C^i(\xi^i)}_{production} + \underbrace{T^i(\theta^i)}_{trading} - \underbrace{\Pi \left(\varepsilon^i + e^i(\xi^i) - \Delta^i - \theta^i_T \right)^+}_{penalty} \right]
\end{align}
\end{frame}

\begin{frame}
\frametitle{Variables}

\begin{tiny}
\begin{tabular}{ll}
$S^i(\xi^i) =    \sum_{t=0}^{T-1} \sum_{j, k} S^k_t \xi^{i,j,k}_t$ & revenues from selling the produced goods \\
$C^i(\xi^i) =    \sum_{t=0}^{T-1} \sum_{j, k} C^k_t \xi^{i,j,k}_t$ & costs from producing the goods \\
$T^i(\theta^i) = \sum_{t=0}^{T-1} \theta^i_t (A_{t+1} - A_t) - \theta_T^i A_T$ & profit/loss from trading emission permits \\
$e^i(\xi^i) =    \sum_{t=0}^{T-1} \sum_{j, k} S^k_t \xi^{i,j,k}_t$ & firm i's emissions in [0,T] from the production\\
$\Delta^i = \sum_{t=0}^{T-1} \Delta^i_t$ & number of emission permits allocated to firm i in [0,T] \\*[12pt]
\end{tabular}

\begin{tabular}{ll}
$\varepsilon^i$ & quantity of firm i's emissions in [0,T] that cannot be controlled\\
$\theta^i_t$ & number of forward contracts on emission permits held by firm i at time t\\
$\Pi$ & penalty per emission unit \\
$S_t^k$ & price of product k\\
$C_t^{i,j,k}$ & firm i's marginal production costs of product k using production technology j\\
$e_t^{i,j,k}$ & emission factor of firm i, production technology j and product k \\
\end{tabular}
\end{tiny}
\end{frame}

%22. Folie
\begin{frame}
\frametitle{Market equilibrium}
A market equilibrium in emission permits consists of
\begin{itemize}
\item<1->
$\quad A^*$ (one-dimensional stochastic process for forward price on permits)
\item<2-> $\quad S^*$ (multi-dim. stochastic process for the prices of the products)
\item<3-> $\quad \theta^*$ (multi-dim. stochastic process of optimal trading strategies)
\item<4-> $\quad \xi^*$ (multi-dim. stochastic process of optimal production strategies)
\end{itemize}
such that for given $A^*$ and $S^*$,
$\theta^*$ and $\xi^*$ lead to a situation where all the firms are satisfied by their strategy.
\end{frame}

\begin{frame}
\frametitle{Market equilibrium}
Formally
$
\qquad \mathbb{E}\left[ L^{A^*, S^*, i}\left(\theta^{*i},\xi^{*i}\right) \right] \ge \mathbb{E}\left[ L^{A^*, S^*, i}\left(\theta^{i},\xi^{i}\right) \right] \textnormal{ for all } \left(\theta^i, \xi^i \right)
$ \\
and the following two conditions hold
\begin{itemize}
\item<1-> Market clearing condition on permits
$$
\sum_{i} \theta^{*i}_t = 0
$$
\item<2-> Supply meets demand for each good
$$
\sum_{i, j} \xi^{*i,j,k}_t = D_t^k
$$
\end{itemize}
\end{frame}


%23. Folie
\begin{frame}
\frametitle{Global optimization problem}
A fictitious central planner minimizes expected total costs by producing an optimal quantity of goods $\xi^*$, i.e. it faces the optimization problem
\begin{align}
\inf_{\xi} \mathbb{E} \left[ \underbrace{C(\xi)}_{production} - \underbrace{\Pi \left(\varepsilon + e(\xi) - \Delta \right)^+}_{penalty} \right]
\end{align}
where \\
\begin{tabular}{ll}
$C(\xi) =  \sum_{i} C^i(\xi^i)$ & total production costs\\
$e(\xi) = \sum_{i} e^i(\xi^i)$ & total emissions from production in [0,T] \\
$\varepsilon = \sum_{i} \varepsilon^i$ & total emissions in [0,T] that are not controllable \\
$\Delta = \sum_{i} \Delta^i$ & total emission certificates handed out by the regulator \\
$\Pi$ & penalty per emission unit \\
\end{tabular}

\end{frame}


%24. Folie
\begin{frame}
\frametitle{Theorem: Market equilibrium and joint optimization problem}
\begin{enumerate}[(a)]
\item If $(A^*,S^*)$ is a market equilibrium with associated strategies $(\theta^*,\xi^*)$ then $\xi^*$ is a solution of the global optimization problem
\item There exists a solution $\bar{\xi}$ of the global optimization problem
\item If $\bar{\xi}$ is a a solution of the global optimization problem then $(\bar{A}, \bar{S})$ is a market equilibrium and the equilibrium allowance price process is almost surely unique
\end{enumerate}
\end{frame}


%25. Folie
\begin{frame}\frametitle{Theorem: Equilibrium prices}
Let $(A^*,S^*)$ be a market equilibrium with associated strategies $(\theta^*,\xi^*)$ then
%\vspace{-0.5cm}

Forward prices on permits are almost surely given by \[
A^*_t = \Pi \cdot \mathbb{E} \left[ \IF_{\left\{ \varepsilon + e(\xi) - \Delta \ge 0 \right\}} | \mathcal{F}_t \right]
\]

\end{frame}


\begin{frame}\frametitle{Theorem: Equilibrium prices}
 Spot prices $S^{*k}$ of the goods and the optimal production strategy $\xi^{*i}$ correspond to a merit-order-type equilibrium with adjusted costs $C_t^{i,j,k} + e^{i,j,k} A_t^*$, i.e. at time t and for each good k
\begin{itemize}
\item all the production means of the economy are ranked by increasing adjusted production costs
\item demand is met by producing from the cheapest production means
\item k's equilibrium spot price is the marginal cost of production of the most expensive production means used to meet demand $D_t^k$
\end{itemize}
\[
S_t^{*k} = \max_{i,j} \left\{ \left(C_t^{i,j,k} + e^{i,j,k} A_t^* \right) \IF_{\left\{\xi_t^{i,j,k}>0\right\}}\right\}
\]
\end{frame}

\subsection{Dynamics of CO2 permit prices}
\frame{\frametitle{Basic Model}
\begin{itemize}
\item<1-> Risk-neutral companies with total initial endowment $e_0$
\item<2-> Total emissions dynamics are
\begin{equation}
dy_t= \mu(t, y_t)dt + \sigma(t, y_t)dW_t
\end{equation}
with deterministic drift and volatility.
\item<3-> Central planner who minimizes total expected cost over a trading period $[0,T]$ by deciding at any time instant
whether to costly abate some of the CO2 emissions or not.
\item<4-> At the end of the period actual accumulated emissions and penalty costs are determined.
\end{itemize}
}

\frame{\frametitle{Basic Model II}
\begin{itemize}
\item<1-> $x_t$ are the total expected emissions over the trading period
\item<2-> Then
\begin{equation}
x_t=-\int_0^tu_s ds + \EX_t\left[\int_0^T y_s ds \right]
\end{equation}
\item<3-> $u_t$ is the optimal rate of abatement which is  actively chosen by the central planner.
\item<4-> So $x_t$ is a controlled stochastic process.
\end{itemize}
}

\frame{\frametitle{Total Emissions}
\begin{itemize}
\item<1-> $x_T$ are the realized emissions that relate to a potential penalty function
\item<2-> Without abatement total expected emissions  are
$$
x_0=\EX\left[\int_0^T y_s ds \right]
$$
\item<3-> The dynamics of the total expected emissions are
\begin{equation}
dx_t=-u_t dt + G(t) dW_t
\end{equation}
\item<4-> $G(t)$ is the volatility of the uncontrolled part of $x_t$ and depends both on the drift $\mu(t, y_t)$
and the volatility $\sigma(t,y_t)$ of the emission rate.
\end{itemize}
}

\frame{\frametitle{Optimisation problem of the central planner I}
\begin{equation}
\max_{u_t} \EX_0\left[\int_0^Te^{-rt}C(t,u_t)dt + e^{-rT}P(x_T)\right]
\end{equation}
with
$$
\begin{array}{lll}
C(t,u_t) &=& - \frac{1}{2}c u_t^2 \\*[12pt]
P(x_T) &=& \min[0,p(e_0-x_T)]
\end{array}
$$
}

\frame{\frametitle{Optimisation problem of the central planner II}
\begin{itemize}
\item<1-> $C(t,u_t)$ are the abatement costs per unit of time. $c$ constant implies no change in technology occurs.
The quadratic form implies linearly increasing marginal abatement costs.
\item<2-> $P(x_T)$ is the penalty function, with $p$ the penalty including all costs.
\item<3-> $r$ is the constant interest rate.
\end{itemize}
}

\frame{\frametitle{Solution of the control problem}
Let $V(t,x_t)$ be the expected value of the optimal policy given $x_t$. By a standard
Hamilton-Jacobi-Bellman argument we arrive at
\begin{equation}
V_t=-\frac{1}{2}(G(t))^2 V_{xx}-\frac{1}{2c}e^{rt}(V_x)^2
\end{equation}
with boundary condition
$$
V(T, x_T)=e^{-rT}P(x_T)
$$
and optimal control
$$
u_t=-\frac{1}{c} e^{rt}V_x
$$
}
\begin{frame}
  \frametitle{Permit Prices}
\begin{center}
\begin{figure}[h!]
\centering
\rotatebox{0}{
\scalebox{0.6}{
\includegraphics[width=1.45\textwidth, height= 1.2\textheight]{../pics/pic1-SUHW.pdf}}}

\end{figure}
\end{center}
\end{frame}

\frame{\frametitle{Permit Price Dynamics}
\begin{itemize}
\item<1-> Recall that the permit price must equal the marginal abatement costs, so
\begin{equation}
S(t,x_t) = c u_t = -e^{rt} V_x(t,x_t)
\end{equation}
\item<2-> Using It{\^o}'s formula and the HJB-PDE we find that the discounted permit price is a martingale.
\item<3-> Its dynamics are
\begin{equation}
dS(t,x_t)= G(t)S_x(t,x_t) dW_t
\end{equation}
\end{itemize}
}

\begin{frame}
  \frametitle{Implied Permit Price Volatility}
\begin{center}
\begin{figure}[h!]
\centering
\rotatebox{0}{
\scalebox{0.6}{
\includegraphics[width=1.45\textwidth, height= 1.2\textheight]{../pics/pic2-SUHW.pdf}}}

\end{figure}
\end{center}
\end{frame}

\subsubsection{Central Planner and Equilibrium}
\frame{\frametitle{Individual Company Models}
\begin{itemize}
\item<1-> Each individual company has an endowment $e_{i0}$
\item<2-> Individual emissions dynamics are
\begin{equation}
dy_{it}= \mu(t, y_{it})dt + \sigma(t, y_{it})dW_{it}
\end{equation}
with deterministic drift and volatility.
\end{itemize}
}

\frame{\frametitle{Individual Emissions}
\begin{itemize}
\item<1-> $x_{it}$ are the total expected emissions of company $i$ over the trading period
\item<2-> Then
\begin{equation}
x_{it}=-\int_0^tu_{is} ds -\int_0^tz_{is}ds + \EX_t\left[\int_0^T y_{is} ds \right]
\end{equation}
\item<3-> $u_{it}$ is the individual rate of abatement
\item<4-> and $z_{it}$ is the instantaneous amount of permits bought or sold.
\end{itemize}
}
\frame{\frametitle{Individual Emissions Dynamics}
\begin{itemize}
\item<1-> The dynamics of the total expected emissions are
\begin{equation}
dx_{it}=-[u_{it}+z_{it}] dt + G_i(t) dW_{it}
\end{equation}
\item<2-> $G_i(t)$ is the volatility of the uncontrolled part of $x_{it}$ and depends both on the drift $\mu_i(t, y_{it})$
and the volatility $\sigma_i(t,y_{it})$ of the emission rate.
\end{itemize}
}

\frame{\frametitle{Optimisation Problem for the individual Company}
\begin{equation}
\max_{u_{it}, z_{it}} \EX\left[\int_0^Te^{-rt}C_i(t,u_{it})dt - \int_0^T e^{-rt}S(t)z_{it}dt+ e^{-rT}P_i(x_{iT})\right]
\end{equation}
with $S(t)$ the permit price and
$$
\begin{array}{lll}
C_i(t,u_{it}) &=& - \frac{1}{2}c_i u_{it}^2 \\*[12pt]
P_i(x_{iT}) &=& \min[0,p(e_{i0}-x_{iT})]
\end{array}
$$
}

\frame{\frametitle{Solution of the control problem}
Let $V^i(t,x_{it})$ be the expected value of the optimal policy for company $i$. By a standard
Hamilton-Jacobi-Bellman argument we arrive at
$$
\begin{array}{lll}
0&=\max_{u_{it},z_{it}}&\left[e^{-rt}(C_i(t,u_{it}) - S(t) z_{it})\right.\\*[12pt]
&&+\left.V^i_t -V_x^i(u_{it}+z_{it}) + \frac{1}{2}(G_i(t))^2 V^i_{xx}\right]
\end{array}
$$
with boundary condition
$$
V^i(T, x_{iT})=e^{-rT}P_i(x_{iT}).
$$
}

\frame{\frametitle{Equilibrium Solution}
\begin{itemize}
\item<1-> We solve the HJB for $N$ companies and use the market clearing condition
$$
\sum_{i=1}^N z_{it}^*=0
$$
\item<2-> The first-order conditions give
$$
\begin{array}{llll}
u_{it}^* &=& -\frac{1}{c_i} e^{rt} V^i_x & i=1, \ldots N \\*[12pt]
S(t) &=& - e^{rt} V^i_{x} & i=1, \ldots N
\end{array}
$$
\item<3-> So again
$$
S(t) = c_i u_{it}^*, \; i=1, \ldots N.
$$
\end{itemize}
}
\frame{\frametitle{Joint Cost Problem I}
Again we image a central planner who has to solve
\begin{equation}
\max_{u_{it}} \EX\left[\int_0^Te^{-rt}\sum_{i=1}^N C_i(t,u_{it})dt + e^{-rT} \sum_{i=1}^N P_i(x_{iT})\right]
\end{equation}
with $C_i$ and $P_i$ as before.

We assume only one source of randomness, i.e. $W_{it}= W_t$, then we have the  joint value function as
$$
V(t, x_{1t}, \ldots, x_{Nt}) = \sum_{i=1}^N V_i(t,x_{it}).
$$

%Here, $z_{it}$ are irrelevant due to the market clearing condition.

}
\frame{\frametitle{Joint Cost Problem II}
The joint cost problem now is
$$
\begin{array}{lll}
0&=\displaystyle \max_{\{u_{it},i=1, \ldots, N\}}&\displaystyle \left[e^{-rt}\sum_{i=1}^N C_i(t,u_{it})\right.\\*[12pt]
&&+\displaystyle \left.\sum_{i=1}^N (V^i_t -V_x^iu_{it}) + \frac{1}{2}\sum_{i=1}^N(G_i(t))^2 V^i_{xx}\right]
\end{array}
$$
with boundary condition
$$
\sum_{i=1}^N V^i(T, x_{iT})=e^{-rT}\sum_{i=1}^N P_i(x_{iT}).
$$
}

\frame{\frametitle{Joint Problem Solution}
\begin{itemize}
\item<1-> The first-order conditions give
$$
u_{it}^{**}= -\frac{1}{c_i} e^{rt} V^i_x, \; i=1, \ldots N
$$
\item<2-> Due to linearity we also have
$$
u_{it}^{**}= \argmax \left\{ \max_{u_{it}} \left[e^{-rt} C_i(t,u_{it}) + V^i_t -V_x^iu_{it} + \frac{1}{2}(G_i(t))^2 V^i_{xx}\right]
\right\}
$$
\item<3-> Again
$$
S(t) = c_i u_{it}^{**}= -e^{rt}V^i_x, \; i=1, \ldots N.
$$
\end{itemize}
}

\frame{\frametitle{Equivalence to Equilibrium Solution}
$$
\begin{array}{lll}
&u_{it}^{**}&\\*[12pt]
=&  \argmax \left\{ \displaystyle \max_{u_{it},z_{it}} \right. &\left[e^{-rt} C_i(t,u_{it}) -e^{-rt}S(t)z_{it} +e^{-rt}S(t)z_{it} \right.\\*[12pt]
 &&\left.\left. + V^i_t -V_x^iu_{it} + \frac{1}{2}(G_i(t))^2 V^i_{xx}\right]
\right\}\\*[12pt]
=&  \argmax \left\{ \displaystyle \max_{u_{it},z_{it}} \right. &\left[e^{-rt} (C_i(t,u_{it}) -S(t)z_{it}) \right. \\*[12pt]
&&\left.\left. + V^i_t +V_x^i(-u_{it}-z_{it}) + \frac{1}{2}(G_i(t))^2 V^i_{xx}\right]
\right\}\\*[12pt]
=&u_{it}^*&
\end{array}
$$}



\subsection{Dynamic Reduced Form Models}
\begin{frame}
\frametitle{Permit Prices}
\begin{itemize}
\item<1-> Model  the emission rate as
$$
dQ_t = Q_t(\mu dt + \sigma dW_t)
$$
\item<2->
The cumulative emissions are
$$
q_{[0,t]} = \int_0^t Q_s ds
$$
\item<3->
The futures permit price is given as
$$
F(t,T) = P \prb\left[q_{[0,T]}>N |\F_t\right]
$$
\end{itemize}
\end{frame}

\begin{frame}
\frametitle{Approximative Pricing}
\begin{itemize}
\item<1-> Linear approximation approach
$$
\begin{array}{lll}
q_{[t_1,t_2]} &\approx& \tilde{q}^{Lin}_{[t_1,t_2]} = Q_{t_2} (t_2 - t_1) \\*[12pt]
&=&\displaystyle   Q_{t_1} e^{\left\{\log (t_2 - t_1) + \left(\mu-\frac{\sigma^2}{2}\right)(t_2-t_1)+\int_{t_1}^{t_2}\sigma dW_t\right\}}
\end{array}
$$
\item<2-> Moment matching
$$
\begin{array}{lll}
q_{[t_1,t_2]} &\approx& \tilde{q}^{Log}_{[t_1,t_2]}\\*[12pt]
&=& Q_{t_1} \exp\left\{ \int_{t_1}^{t_2}\mu_t dt + \int_{t_1}^{t_2} \sigma_t dW_t\right\}
\end{array}
$$
where the functions $\mu_t$ and $\sigma_t$ are defined by the functions $\alpha_t, \beta_t$ from the moment matching.
\end{itemize}
\end{frame}
%\subsection[Reduced Form Dynamics]{Dynamics of the permit process in the reduced form model}

\begin{frame}
\frametitle{Carmona-Hinz Approach}
\begin{itemize}
\item<1-> Use a lognormal process
$$
\Gamma_{T}= \Gamma_0  \exp{\left\{\int_{0}^{T}\sigma_t dW_t -\frac{1}{2}\int_0^T \sigma^2_t dt\right\}}
$$
with $\Gamma_0 >0$ and $\sigma(.)$ a deterministic square-integrable function.
\item<2-> Define the futures price under a risk-neutral measure $\Q$ as
$$
F(t,T) = P \Q\left[\Gamma_T>1 |\F_t\right]
$$
\end{itemize}
\end{frame}

\frame{\frametitle{Reduced-Form Dynamics}
The martingale
$$
a_t = \EX^\Q\left[\IF_{\{\Gamma_T>1\}} |\F_t\right]
$$
is given by
$$
a_t= \Phi \left[\frac{\Phi^{-1}(a_0) \sqrt{\int_0^T \sigma^2_s ds}+\int_0^t \sigma_s dW_s}{\sqrt{\int_t^T \sigma^2_s ds}}\right]
$$
and solves the stochastic differential equation
$$
da_t = \Phi'\left(\Phi^{-1}(a_t)\right)\sqrt{z_t}dW_t
$$
with
$$
z_t=\frac{\sigma_t^2}{\int_t^T \sigma^2_u du}
$$
}
\frame{\frametitle{Reduced-Form Dynamics -- Proof}
\begin{itemize}
\item<1-> $a_t$ formula is straightforward calculation
\item<2-> For dynamics use that
$$
a_t = \Phi(\xi_t)
$$
with
$$
\xi_t = \frac{\xi_{0,T}+\int_0^t\sigma_s dW_s}{\sqrt{\int_t^T\sigma_s^2ds}}\; \mbox{and}\;  \xi_{0,T}=\log \Gamma_0 - \frac{1}{2} \int_0^T\sigma_s^2ds.
$$
Starting with the dynamics of $\xi_t$ an application of It{\^o}'s formula gives the result.
\end{itemize}
}

%\subsection{Historical Calibration}
\frame{\frametitle{Model Parametrization}
\begin{itemize}
\item<1-> For constant $\sigma$ we find $z_t=(T-t)^{-1}$, so a richer specification is needed.
\item<2-> A standard model is
$$
da_t = \Phi'\left(\Phi^{-1}(a_t)\right)\sqrt{\beta(T-t)^{-\alpha}}dW_t
$$
which specifies a family $\sigma_s(\alpha,\beta)$.
\item<3->
So $z_t(\alpha, \beta) = \beta(T-t)^{-\alpha}$ and
$$
\begin{array}{lll}
\sigma_t^2(\alpha,\beta)&=& \displaystyle z_t(\alpha, \beta) \exp\left\{-\int_0^t z_s(\alpha, \beta) ds \right\}\\*[12pt]
&=&\displaystyle
\left\{
\begin{array}{ll}
\beta(T-t)^{-\alpha} e^{-\frac{\beta}{1-\alpha}[T^{1-\alpha}-(T-1)^{1-\alpha}]} &\alpha \not=1\\
\beta(T-t)^{\beta-1}T^{-\beta} &\alpha=1.
\end{array}
\right.
\end{array}
$$
\end{itemize}
}
\frame{\frametitle{Objective Measure}
\begin{itemize}
\item<1-> We do a historical calibration and change measure to the objective measure.
\item<2-> The standard change of measure gives
$$
\frac{d\prb}{d\Q} = \exp\left\{\int_0^T H_s dW_s -\frac{1}{2} \int_0^T H_s^2ds \right\}).
$$
\item<3->
Under constant market price of risk $H_t \equiv h$ and by Girsanov's theorem
$$
\tilde{W}_t = W_t - ht
$$
is a $\prb$ Brownian motion.
\end{itemize}
}
\frame{\frametitle{Objective Measure}
\begin{itemize}
\item<1->
Under $\prb$
$$
d\xi_t = \left(\frac{1}{2} z_t \xi_t + h \sqrt{z_t} \right)dt + \sqrt{z_t} d\tilde{W}_t,
$$
so $\xi_{\tau}$ given $\xi_t$ is Gaussian.
\item<2-> So we can invert permit prices to obtain $\xi$ values and calculate the log-likelihood to obtain
estimates for $\alpha$ and $\beta$.
\end{itemize}
}


%\subsection{Option Pricing}
\frame{\frametitle{Pricing Formula}
For a European call with strike $K$ and maturity $\tau$ the option price is
$$
C_t = e^{-\int_t^\tau r_s ds} \int_{-\infty}^\infty (P\Phi(x)-K)^+ \Phi_{\mu_{t,\tau}, \sigma_{t,\tau}}(dx)
$$
with
$$
\mu_{t,\tau}=
\left\{
\begin{array}{ll}
\xi_t \left(\frac{T-t}{T-\tau}\right)^{\frac{\beta}{2}} & \alpha =1\\
\xi_t \exp\left\{\frac{\beta}{2(1-\alpha)}[(T-t)^{1-\alpha}-(T-\tau)^{1-\alpha}]\right\} & \alpha \not= 1.
\end{array}
\right.
$$
and
$$
\sigma^2_{t,\tau}=
\left\{
\begin{array}{ll}
\left(\frac{T-t}{T-\tau}\right)^\beta-1 & \alpha =1\\
\exp\left\{\frac{\beta}{1-\alpha}[(T-t)^{1-\alpha}-(T-\tau)^{1-\alpha}]\right\}-1 & \alpha \not= 1.
\end{array}
\right.
$$

}



%\subsection{Reduced Form Option Pricing in Multi-period Models}
\frame{\frametitle{Two-period Model}
\begin{itemize}
\item<1-> We consider a two-period model, $[0,T]$ and $[T,T']$, with banking and withdrawal
\item<2-> Let $\Q$ be a martingale measure and $(A_t)_{t\in[0,T]}$ and $(A'_t)_{t\in[0,T']}$
be the futures contracts which are $\Q$ martingales.
\item<3-> Let $N\in \F_T$ resp. $N' \in \F_{T'}$ be non-compliance in the first resp. second period.
\end{itemize}
}
\frame{\frametitle{(Non-) Compliance at $T$}
\begin{itemize}
\item<1-> In case of compliance, i.e. event $\Omega-N$, since $A_T$ is the spot price at $T$ and the permit can be banked,  we have
$$
A_T\IF_{\{\Omega-N\}}= \kappa A'_T \IF_{\{\Omega-N\}}
$$
with $\kappa= \exp\{-\int_T^{T'}r_s ds\}$ a discount factor.
\item<2-> In case of non-compliance
$$
A_T\IF_{N}= \kappa A'_T \IF_{N} + P\IF_N
$$
\item<3-> Thus
$$
A_t-\kappa A'_t = \EX_t^\Q[A_T-\kappa A'_T]= \EX^\Q_t[P \IF_N]
$$
\end{itemize}
}

\frame{\frametitle{Reduced-Form Model}
\begin{itemize}
\item<1-> As in the one period case,  we set
$$
A_t-\kappa A'_t = P \Phi(\xi_t^1)
$$
with $\xi^1_t$ a Gaussian process driven by a Brownian motion $W^1_t$.
\item<2-> Assume that the ETS ends after the second period, then
$$
A'_t = P \Phi(\xi^2_t)
$$
with $\xi^2_t$ a Gaussian process driven by a Brownian motion $W^2_t$.
\end{itemize}
}
\frame{\frametitle{Pricing Formula}
For a European call written on a futures in the first period with strike $K$ and maturity $\tau$ we decompose the payoff
$$
(A_\tau-K)^+= (A_\tau - \kappa A'_\tau + \kappa A'_\tau -K )^+= (P\Phi(\xi^1_t) + \kappa P \Phi(\xi_t^2) -K)^+
$$
We obtain for the option price
$$
C_t = e^{-\int_t^\tau r_s ds} \int_{\setR^2} (P\Phi(x_1)+\kappa P \Phi(x_2) -K)^+ \Phi_{\mu_\tau, \sigma_\tau}(dx_1dx_2)
$$
where the parameters of the two-dimensional Gaussian distribution depend on the individual drift and volatility terms and the correlation
of $\xi^1$ and $\xi^2$.

}

\frame{\frametitle{Parameters of the Pricing Formula}
The means are
\begin{eqnarray}\nonumber
\mu^1_{t,\tau} &=& \Phi^{-1}\left(\frac{A_t - \kappa A'_t}{P}\right) \sqrt{\left(\frac{T-t}{T-\tau}\right)\beta_1}\\\nonumber
\mu^2_{t,\tau} &=& \Phi^{-1}\left(\frac{\kappa A'_t}{P}\right) \sqrt{\left(\frac{T'-t}{T'-\tau}\right)\beta_2}\\\nonumber
\end{eqnarray}
and the covariance matrix is
\begin{eqnarray}\nonumber
\nu^{1,1}_{t,\tau} &=& \var(\xi_\tau^1) =  \left(\frac{T-t}{T-\tau}\right)^{\beta_1}-1\\\nonumber
\nu^{2,2}_{t,\tau} &=& \var(\xi_\tau^2) =  \left(\frac{T'-t}{T'-\tau}\right)^{\beta_2}-1\\\nonumber
\nu^{1,2}_{t,\tau}= \nu^{2,1}_{t,\tau} &=& \beta_1^\frac{1}{2}\beta_2^\frac{1}{2}
\frac{\int_t^\tau (T-u)^\frac{\beta_1-1}{2} (T'-u)^\frac{\beta_2-1}{2} \rho du}{(T-\tau)^\frac{\beta_1}{2} (T'-\tau)^\frac{\beta_2}{2}} \\\nonumber
\end{eqnarray}
}

