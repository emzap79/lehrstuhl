% !TEX root = EnergyTrading_ss14UDE.tex



\subsection{Energy Trading}

\frame{\frametitle{Organisation of the Power System}
\begin{center}
\includegraphics[heigth=0.8\textheight, width=0.8 \textwidth]{../../../pics/OrganisationElectricityMarket}
%\caption{Source: Slides Cornelusse}
\end{center}


}


\frame{\frametitle{System Balancing}
\begin{itemize}
\item<1-> The transmission system operator (TSO) has the task to match demand and supply (to balance the system).
\item<2-> The TSO defines a balancing period (15 minutes in Germany), which is the granularity of the measured electric energy supply and during which a constant power supply takes place (by the energy merchant).
\item<3->  The power balancing during the balancing period (not smaller than the granularity, in Germany an hour) is the task of the TSO.
\item<4-> The TSO usually has no own generation capacities and has to act on the reserve market to compensate imbalances.
\end{itemize}
}

\frame{\frametitle{Balancing and Reserve Markets}
We use the following definitions
\begin{itemize}
\item<1-> Reserve Market: allows the TSO to purchase the products needed for compensating imbalances between supply and demand
\item<2->  Balancing Market: allows merchants to purchase or sell additional energy for balancing their accounting grid. Typically, the only market partner is the TSO.
\end{itemize}
}

\frame{\frametitle{Balancing and Reserve Markets}
In Europe, the {\it European Network of Transmission System Operators for Electricity,(ENTSO-E)} coordinates overarching grid topics. The main task of a TSO is to ensure a constant power frequency in the transmission system. The following control actions are applied 
\begin{itemize}
\item<1->{\it Primary Reserve}   starts within seconds as a joined action of all TSOs in the system. 
\item<2-> {\it Secondary Reserve} replaces the primary reserve after a few minutes and is put into action by the responsible TSOs only. 
\item<3->{\it Tertiary Reserve} frees secondary reserves by rescheduling generation by the responsible TSOs.
\end{itemize}
The TSO tenders the required products for fulfilling these functions. Reserve products may involve payments for the availability of the reserved capacity.
}

\frame{\frametitle{Market Coupling}
\begin{itemize}
\item<1-> Neighbouring electricity markets are typically coupled via transmission capacities owned by the TSOs. 
\item<2->  Transmission capacities can be integrated in the price finding algorithm of cooperating exchanges via implicit auctioning. 
\item<3-> With implicit auctions  players do not  receive allocations of cross-border capacity themselves but bid for energy on their Exchange. The Exchanges then use the available cross-border transmission capacity to minimize the price difference between two or more areas.
\item<4->  Currently, the Central Western Europe (CWE) initiative couples Belgium, France, the Netherlands, Germany and Luxemburg.
\end{itemize}

}

\frame{\frametitle{Market Coupling}
\begin{center}
\includegraphics[heigth=0.6\textheight, width=0.6 \textwidth]{../../../pics/ENTSO-E-flow}
\end{center}


}

\frame{\frametitle{Electricity Markets}

A centralized platform where participants can exchange electricity transparently
according to the price they are will to pay or receive, and according to the capacity of
the electrical network.


\begin{itemize}
\item<1-> Fixed Gate Auction 
\begin{itemize}
\item Participants submit sell or buy orders for several areas, several hours,
\item the submissions are closed at a pre-specified time (closure)
\item the market is cleared. 
\item Example: day-ahead.
\end{itemize}
\item<2-> Continuous-time Auction 
\begin{itemize}
\item Participants continuously submit orders. Orders are stored,
\item Each time a deal is feasible, it is executed,
\item Example: intra-day.
\end{itemize}

\end{itemize}
}


\frame{\frametitle{Electricity Exchanges}

Electricity related contracts  can be traded at exchanges such as 
\begin{itemize}
\item<1-> the Nord Pool, mainly Northern European countries, \url{http://www.nordpoolspot.com/}  
\item<2-> the European Energy Exchange (EEX), \url{http://www.eex.com/en} 
\item<3-> EPEX, located in Paris, founded by EEX and Powernext (French Energy Exchange);
Electricity spot market for Germany, Austria, France and Switzerland;   
\url{http://www.epexspot.com/en/}
\item<4-> Amsterdam Power Exchange (APX), covers the Netherlands, Belgium and the UK, \url{http://www.apxgroup.com}
\end{itemize}
}

\frame{\frametitle{EPEX -- traded products}
\begin{itemize}
\item<1-> Auction day-ahead and continuous intra-day market.
\item<2-> Products are individual hours, baseload, peakload, blocks of contiguous hours.
\item<3-> Intraday market is open 24 hours a day, 7 days a week and products can be traded until 45 minutes before delivery.
\item<4-> in Germany 15 minutes contracts can be traded.
\end{itemize}
}

\frame{\frametitle{EEX -- traded products}
\begin{itemize}
\item<1-> Futures contracts for Germany and France with delivery periods week, month, quarter, year. 
\item<2-> For Germany single days and weekends are available. 
\item<3-> European style options on futures.
\end{itemize}
}


\frame{\frametitle{Auction EPEX}
\begin{center}
\includegraphics[heigth=0.9 \textheight, width=0.9 \textwidth]{../../../pics/auction-epex07042014}
\end{center}


}

\frame{\frametitle{Spot price EPEX}
\begin{center}
\includegraphics[heigth=0.9 \textheight, width=0.9 \textwidth]{../../../pics/pricevolume-epex07042014}
\end{center}


}


\frame{\frametitle{Spot prices}
\begin{center}
\includegraphics[heigth=0.9 \textheight, width=0.9 \textwidth]{../../../pics/phelixBase2002_12.pdf}
\end{center}


}

\frame{\frametitle{Spot prices}
\begin{center}
\includegraphics[heigth=0.9 \textheight, width=0.9 \textwidth]{../../../pics/phelixBase2002_08.pdf}
\end{center}


}
\frame{\frametitle{Spot prices}
\begin{center}
\includegraphics[heigth=0.9 \textheight, width=0.9 \textwidth]{../../../pics/phelixBase2008_12.pdf}
\end{center}


}







