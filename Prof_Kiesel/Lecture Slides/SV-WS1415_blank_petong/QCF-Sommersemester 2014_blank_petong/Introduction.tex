% !TEX root = QCF_ss14UDE.tex
\section{Economics of Climate Change}
\subsection[Emissions]{$\textnormal{CO}_2$ emissions and global warming}

% frametitle
{United Nations Framework Convention on Climate Change}

Background from United Nations Framework Convention on Climate Change can be found\\*[12pt]

\url{http://unfccc.int/essential_background/items/6031.php}

\vspace{12pt}
The Fifth Assessment Report (AR5) of the International Panel on Climate Change (IPCC) provides a clear and most up to date view of the current state of scientific knowledge relevant to climate change.

\url{http://www.ipcc.ch}

% frametitle
{Greenhouse Gas Emissions and Climate Change}



% begin itemize



	Greenhouse Gas Emissions (GHG) are seen as a major reason for global climate change. Since ${CO}_2$ emissions are the main source of GHG emissions, GHG are measured in terms of ${CO}_2$ equivalents, CO2e.

	In 2010 we observed for the first time in many years an increase in GHG emissions, which makes it increasingly unlikely to met the World Climate Conference (Canc{\'u}n 2010) target of a $2^o$ Celcius temperature increase by 2050.



% end itemize



% frametitle
{Greenhouse Gas Emissions and Climate Change}



% begin itemize



	The climate change will lead to an increase of the probability and frequency of storms, floods, draughts according to the Intergovernmental Panel on Climate Change (IPCC).

	The overall stock of CO2e in the atmosphere is relevant. In 2007 CO2e were around 430 parts per million (ppm) rising around 2,5 ppm per year.

	Targets may be a certain temperature increase or a certain stock (with different flow paths)



% end itemize



% frametitle
{Probabilities of Temperature Increases}
\begin{center}
\begin{figure}[h!]
\centering
\rotatebox{0}{
\scalebox{0.6}{
\includegraphics[width=1.4\textwidth]{../../../pics/temperature-probabilities.pdf}}} %Source: Stern 2007
\caption{Probability of Temperature Increase higher than}
\label{fig:temperature}
\end{figure}
\end{center}

% frametitle
{Carbon Target}
\begin{center}
\begin{figure}[h!]
\centering
\rotatebox{0}{
\scalebox{0.6}{
\includegraphics[width= 1.4 \textwidth]{../../../pics/IPCC2014-GlobalWarming}}} %Source: IPCC
\caption{IPCC Scenarios}
\label{fig:emissions}
\end{figure}
\end{center}

% frametitle
{Carbon Target}
\begin{center}
\begin{figure}[h!]
\centering
\rotatebox{0}{
\scalebox{0.6}{
\includegraphics[width= 1.4 \textwidth]{../../../pics/IPCC2014-EffectsOfWarming}}} %Source: IPCC
\caption{IPCC Scenarios}
\label{fig:emissions}
\end{figure}
\end{center}

% frametitle
{Consequences of Climate Change - High-Emission Scenario}
%1) No change in policy towards climate change



% begin itemize



	Global temperatures could rise by 5-7 Celsius this century - well above the 2 Celsius scientists regard as the safety limit

	Sea level rises will overwhelm 1-2bn people living in low-lying areas

	4bn people could be put at risk of water shortages

	The ice caps will melt entirely

	The Amazon rainforest may die off



% end itemize


 %Source: McKinsey & Co (2004(?))

%\begin{frame}
 %\frametitle{Consequences of Climate Change - Scenario 2}
%2) Developed world takes the lead, with USD 350bn per year investment by 2030
%% begin itemize
%

	 Sea levels will rise and low-lying land such as Tarawa, Kiribati will be at risk
%

	 Hunger will increase, but more slowly
%

	 Northern areas such as Canada and northern Europe will become more agriculturally productive
%

	 Substantial increase in 'extreme weather events': more droughts, more heatwaves, more floods and more intense storms
%% end itemize
%\end{frame} %Source: McKinsey & Co (2004(?))

% frametitle
{Consequences of Climate Change - Low-Emission Scenario }
%3) Global action, with USD 565bn per year investment by 2030



% begin itemize



	World will warm by no more than 2 Celsius by mid-century and thereafter temperatures may start to decline

	Hottest parts of the world will suffer serious declines in crop yields, but increase in fertility in other areas will offset this

	Ice at the poles will diminish, but some reduced ice cover could remain

	Increase in floods, droughts and storms, but damage manageable

	Tropical diseases will spread, but not too far



% end itemize


 %Source: McKinsey & Co (2004(?))

\subsection{Assessment via Environmental Economics}

%\subsection{GHGs as Externalities}

% frametitle
{GHG Emissions are Externalities}



% begin itemize



	GHGs are global in origin and impact

	Some effects are long-term and governed by a flow-stock process

	There are great uncertainties in most steps of the scientific chain

	Failure to act may have large, possibly irreversible effect.



% end itemize



% frametitle
{Challenges}



% begin itemize



	Economics of risk and uncertainty have to be used

	Links between economics and ethics have to be considered

	International economic policy plays an important role



% end itemize



%\subsection{Risks and Costs}

% frametitle
{Risks}



% begin itemize



	Relation between stock of GHG and temperature increase has to be considered: climate sensitivity

	General Circulation Models (GCM) of climate science produce via Monte Carlo Analysis probability distributions of outcome

	By design: high sensitivity to parameter values



% end itemize



% frametitle
{Abatement Costs}
\begin{center}
\begin{figure}[h!]
\centering
\rotatebox{0}{
\scalebox{0.45}{
\includegraphics[width=1.4\textwidth]{../../../pics/costs-of-abatement.pdf}}} %Source: McKinsey 2007
\caption{Abatement Costs}
\label{fig:abatement}
\end{figure}
\end{center}

% frametitle
{Cost Comparison}



% begin itemize



	Compare cost of abatement with social cost of carbon (SCC)

	Calculation SCC in a time interval $[0,t]$



% begin itemize



	marginal social utility of consumption at time $\tau \in [0,t]$

	impact on consumption at $\tau$ on all relevant preceding temperature changes

	impact on relevant temperature increases of increases in preceding carbon stock

	the impact of all relevant stocks of an increase in carbon emissions in $\tau$



% end itemize





% end itemize



%\subsection{Ethics}

% frametitle
{Ethics}



% begin itemize



	How to value benefits accruing to different people at different times?



% begin itemize



	intratemporal distribution (between different people at the same time)

	intertemporal distribution (between generations)



% end itemize



	The discussion focuses on appropriate discount factors and utility functions to capture values

	Utility functions need to take environment, health, type of consumption into account

	Technological progress has to be taken into account.



% end itemize



\subsection{Policy Instruments}
%\subsection{Types of Instruments}

% frametitle
{Instruments}



% begin itemize



	Price for GHG emissions (externalities are market failures)

	Technology and acceleration of its development

	Energy efficiency (in terms of information and transaction costs)

	International framework and collaboration



% end itemize



% frametitle
{EU Roadmap 2050}
Headline Target for the EU to be achieved by 2020 relating to energy and climate change aims are



% begin itemize



	reducing greenhouse gas emissions (GHG) by 20\%,

	increasing the share of renewables in the EU's energy mix to 20\%,

	achieving the 20\% energy efficiency target.



% end itemize


 %Source: EC (2011) _ EU Roadmap for 2050

% frametitle
{Emission target}



% begin itemize



	To keep climate change below 2 Celcius, the European Council reconfirmed in February
2011 the EU objective of reducing greenhouse gas emissions by 80-95\% by 2050 compared to
1990.

	The EU Emission Trading System (ETS) is supposed to play a key role by generating a sufficient carbon price,
which is long-term predictable.

	The EU  considers taxation and technological support as additional measures.



% end itemize



% frametitle
{CO2 Emissions Deutschland}
\begin{center}
\begin{figure}[h!]
\centering
\rotatebox{0}{
\scalebox{0.55}{
\includegraphics[width=1.4\textwidth]{../../../pics/CO2-EmissionsDeutschland2014.pdf}}}
%\caption{Emissions 2014}
\label{fig:abatement}
\end{figure}
\end{center}

%\subsection{Pricing Policy Approaches}

% frametitle
{Putting a Price on Carbon}
   % begin itemize

	taxation

	carbon trading on the basis of allocation and auctioning

	implicit, via regulation and standards.
  % end itemize

% frametitle
{Possible policy responses I}
  % begin itemize

	\textbf{Emission standards ("Command-and-Control")} \\
        Legal limit on the amount of the pollutant an individual source is allowed to emit. %\\
        %Example: Compulsory Filter %\\
        Problem: Standards ensure the required reduction but in practice it is not achieved in a cost-effective way (sources are usually allocated an equal reduction).

	\textbf{Taxes: Emission charges} \\
        Pollutor has to pay a fee on each unit of pollutant emitted. %\\
        %Example: xxx %\\
        Problem: Does not necessarily lead to a lower pollution level.
  % end itemize

% frametitle
{Possible policy responses II}
  % begin itemize

	\textbf{Taxes: Product charges} \\
        Control authority taxes the commodity that is responsible for the pollution instead of the pollutant. %\\
        %Example: Tax on fuel % \\
        Problem: They are easy to administer. However, not every unit of the taxed product may have the same impact on the environment. \\

	\textbf{Emission trading} \\
        All sources are allocated allowances to emit either on the basis of some criterion such as historic emissions or by auctioning the allowances off to the highest bidder.
        The control authority issues exactly the number of allowances needed to produce the desired aggregate emission level.
        The allowances are freely tradeable.
        Advantage: Leads to a cost-effective allocation.
  % end itemize
