% !TEX root = StructuringValuation_ws1415UDE.tex
\section{Review -- Markets and Objects}
\subsection{Energy Markets}

% frametitle
{Markets}

Since the deregulation of electricity markets in the end of the
1990s, power can be traded at exchanges like the Nordpool, http://www.nordpoolspot.com/  or the
European Energy Exchange (EEX), http://www.eex.com/en. All exchanges have established
spot and futures markets.

%\subsection{Spot market}

% frametitle
{EEX spot market}



% begin itemize



	Trading in Power, Natural Gas and $CO_2$ Emission Rights.

	Power day-ahead auctions for Germany, Austria, France and Switzerland 7 days a week, including holidays. The 24 hours of the respective next day can be traded in one-hour intervals or block orders (e.g. Baseload: 1-24h, Peakload: 9-20h, Night: 1-6, Rush Hour: 17-20h, Business: 9-16h, etc.).

	Continuous day-ahead block trading for France 7:30 am to 11:30 am, 7 days a week, including holidays.

	Continuous Power intraday trading for Germany and France until 75 minutes before the beginning of delivery with delivery on the same or the following day in single hours or blocks.



% end itemize



% frametitle
{EEX spot market}



% begin itemize



	participants submit their price offer/bid curves, the EEX system prices are equilibrium prices that clear the market.

	EEX day prices are the average of the 24-single hours.

	similar structures can be found on other power exchanges(Nord Pool, APX, etc.).



% end itemize



% frametitle
{EEX Spot Market Price Processes}

\includegraphics[width=\textwidth, height=6.7cm]{../../../pics/PhelixBase2002_12}

% frametitle
{EEX Spot Market Price Processes}

\includegraphics[width=\textwidth, height=6.7cm]{../../../pics/PhelixBase2002_08}

% frametitle
{EEX Spot Market Price Processes}

\includegraphics[width=\textwidth, height=6.7cm]{../../../pics/PhelixBase2008_12}

% frametitle
{Electricity is special}



% begin itemize



	it is not storable

	it is homogeneous

	it can be produced in different ways

	it has to be produced when it is needed

	there is a high fluctuation in demand

	there is no short-term elasticity in demand

	there is a high price volatility due to renewable in-feed



% end itemize



%\subsection{Futures market}

% frametitle
{Futures Market} Traded products are



% begin itemize



	Futures contracts for Power, Natural Gas, Emissions and Coal.

	Phelix Futures on Phelix Baseload or Peakload monthly power index for the current month, the next nine months, eleven quarters and six years with cash settlement.

	Baseload and Peakload French/German Power Futures for the current month, the next six months, seven quarters and six years with physical settlement, obliging for continuous delivery of 1MW during a month, quarter or a year.

	Actively exchange traded are the next 7 months, 5 quarters and 2-3 years.

	In addition, OTC transactions.



% end itemize



% frametitle
{Futures Market}

%\includegraphics[width=\textwidth, height=6cm]{../../../pics/bbgfutures}

%}

% frametitle
{Futures Market}

%\includegraphics[width=\textwidth, height=6cm]{../../../pics/future}

} \frame{\frametitle{Futures Market}

%\includegraphics[width=\textwidth, height=6cm]{../../../pics/future1}

%}

% frametitle
{Futures Market}

\includegraphics[width=\textwidth, height=6cm]{../../../pics/lec2_2}

% frametitle
{Futures Market}
Returns seem to be stationary, no seasonality.
\includegraphics[width=\textwidth, height=6cm]{../../../pics/lec2_3}

% frametitle
{Forward Selling Activity}
\includegraphics[width=\textwidth, height=6cm]{../../../pics/RWE-forward-activity}

% frametitle
{Characteristics}



% begin itemize



	\textbf{Electricity Futures} -- Obligation to buy/sell a specified amount of electricity during a \textbf{delivery period}, typically a month, quarter or year.

	Futures show a decreasing volatility term structure.

	Level of volatility depends on length of delivery period.



% end itemize



% frametitle
{Prices of Futures}
\begin{center}
\includegraphics[height=6cm, width=10cm]{../../../pics/forwardcurve}
\end{center}

% frametitle
{Simple Futures Market Model}



% begin itemize



	We assume for the futures dynamics
$$
df(t) = \mu f(t) dt + \sigma f(t) dW(t).
$$

	But since it costs nothing to enter a futures position, we can constantly change our futures portfolio.
Thus financing for buying a traded asset is not relevant.



% end itemize



% frametitle
{Black's Formula}

We use the usual notation -
strike $K$, expiry $T$ as in the spot case, and write $\Phi$ for the
standard normal distribution function.
{\it
The arbitrage price $C$ of a European futures call option is
$$
C(t)= c(f(t), T-t),
$$
where $c(f,t)$ is given by Black's futures options formula:
$$
c(f,\tau) := e^{-r\tau} (f \Phi(\tilde{d}_1 (f,\tau)) - K \Phi(\tilde{d}_2 (f,\tau))),
$$
where
$$
\tilde{d}_{1,2} (f,\tau) := \frac{\log (f/K) \pm \frac{1}{2} {\sigma}^2 \tau}{
\sigma \sqrt{\tau}}.
$$
}

% frametitle
{Black's Formula}

Observe that the quantities $\tilde{d}_1$ and $\tilde{d}_2$ do not depend on the
interest rate $r$. This is intuitively clear from the classical
Black approach: one sets up a replicating risk-free portfolio
consisting of a position in futures options and an offsetting
position in the underlying futures contract. The portfolio
requires no initial investment and therefore should not earn any
interest.

\subsection{Basic Pricing Relations for Futures}

% frametitle
{Storage, Inventory and Convenience Yield}



% begin itemize



	The theory of storage aims to explain the differences between spot
and Futures (Forward) prices by analyzing why agents hold inventories.

	Inventories
allow to meet unexpected demand, avoid the cost of frequent revisions in
the production schedule and eliminate manufacturing disruption.

	This
motivates the concept of convenience yield as a benefit, that accrues to the
owner of the physical commodity but not to the holder of a forward contract.

	Thus the convenience yield is comparable to the dividend yield for stocks.

	A modern view is to view storage (inventory) as a timing option, that
allows to put the commodity to the market when prices are high
and hold it when the prices are low.



% end itemize



% frametitle
{Spot-Forward Relationship in Commodity Markets }
Under the no-arbitrage assumption we have
\begin{equation}\label{SF-rel}
F(t,T)=S(t)e^{(r-y)(T-t)}
\end{equation}
where $r$ is the interest rate at time $t$ for maturity $T$ and $y$ is the convenience yield.

% frametitle
{Spot-Forward Relationship in Commodity Markets }

Observe that (\ref{SF-rel}) implies



% begin itemize



	spot and forward are redundant (one can replace the other) and form a
linear relationship (unlike options)

	with two forward prices we can derive the value of $S(t)$ and $y$

	knowledge of $S(t)$ and $y$ allows us to construct the whole forward curve

	for $r-y <0$ we have backwardation; for $r-y>0$ we have contango.



% end itemize



% frametitle
{Spot-Forward Relationship: Classical theory}



% begin itemize



	In a stochastic model we use
$$
F(t,T)=\EX_{\Q}(S(T)|\F_t)
$$
where $\F_t$ is the accumulated available market information (in most models the information generated by the spot price).

	$\Q$ is a risk-neutral probability



% begin itemize



	discounted spot price is a $\Q$-martingale

	fixed by calibration to market prices or a market price of risk argument



% end itemize





% end itemize



% frametitle
{Futures Prices and Expectation of Future Spot Prices}
The rational expectation hypothesis (REH) states that the current futures price $f(t,T)$ for a commodity with
delivery a time $T>t$ is the best estimator for the price $S(T)$ of the commodity.
In mathematical terms
\begin{equation}\label{REH}
f(t,T) = \EX[S(T) |\F_t].
\end{equation}
where $\F_t$ represents the information available at time $t$. The REH has been statistically
tested in many studies for a wide range of commodities.

% frametitle
{Futures Prices and Expectation of Future Spot Prices}
When equality in (\ref{REH}) does not hold futures prices are biased estimators of
future spot prices. If



% begin itemize



	holds, then $f(t,T)$ is an up-ward biased estimate, then risk-aversion
among market participants is such that buyers are willing to pay more than the expected
spot price in order to secure access to the commodity at time $T$ (political unrest);

	holds, then $f(t,T)$ is an down-ward biased estimate, this may reflect a
perception of excess supply in the future.



% end itemize



% frametitle
{Market Risk Premium}



% begin itemize



	The \emph{market risk premium} or \emph{forward bias} $\pi (t,T)$
relates forward and expected spot prices.

	It is defined as the difference, calculated at time $t$, between
the forward $F(t,T)$ at time $t$ with delivery at $T$ and expected
spot price:
\begin{equation}\label{forward risk premium}
\pi(t,T)= F(t,T)-\EX^\prb[S(T)|\mathcal F_t].
\end{equation}
Here $\EX^\prb$ is the expectation operator, under the
historical measure $\prb$, with information up until time $t$ and
$S(T)$ is the spot price at time $T$.



% end itemize



\subsection[Mean-Reversion Model]{A Mean-Reversion Diffusion Model}
%\subsection{A simple Valuation Framework}

% frametitle
{Definition of the model}
We assume a simple market model in which the price of the underlying commodity, $S_t$, follows a stochastic process which can be described as follows:
Let $X_t = \ln S_t$ and
\begin{align*}
	dX_t = \kappa (\ln \theta - X_t)dt + \sigma dW_t~~,~~X_0 = \ln(S_0)
\end{align*}
Thus, the logarithm of the prices follow a mean reverting diffusion process, the so-called \textcolor{red}{Ornstein-Uhlenbeck-Process}.\\



% begin itemize



	$\kappa$ - Speed of mean reversion

	$\theta$ - Level of mean reversion

	$\sigma$ - Volatility of the process

	$dW_t$ - Brownian increments



% end itemize



% frametitle
{Sample paths of the model}
We simulate $S_t$ (with $S_0=20, \theta = 20, \kappa = 1, \sigma = 0.2$) and get sample paths
\begin{figure}
	\centering
		\includegraphics[width=1.00\textwidth]{../../../pics/samplepaths}
	\label{fig:samplepahts}
\end{figure}

% frametitle
{Why this model?}



% begin itemize



	The \textcolor{red}{mean-reversion} process has been considered the natural choice for commodities. Basic microeconomics theory tells that, in the long run, the price of a commodity ought to be tied to its long-run marginal production cost or, "in case of a cartelized commodity like oil, the long-run profit-maximizing price sought by cartel managers" (Laughton \& Jacoby, 1995, p.188).

	The model was introduced by Eduardo Schwartz ("The Stochastic Behavior of Commodity Prices: Implications for Valuation and Hedging", Journal of Finance, vol.52, n. 3, July 1997, pp.923-973) and is often used in academic studies (and for practical pricing?).



% end itemize



% frametitle
{Properties of the model}



% begin itemize



	Mean reverting

	Bounded volatility

	Continuous paths

	Relative price changes are normally distributed

	\textcolor{red}{Analytic results} for the forward-curve and option prices exist

	\textcolor{red}{Calibration} easily possible



% end itemize



%\subsection{Prices in the Model}

% frametitle
{Spot prices in the model}
The spot price at any time $t$ is
\begin{align*}
	S_t = \exp \left( e^{-\kappa t} \ln S_0 + (1-e^{-\kappa t}) \ln \theta + \int_0^t{\sigma e^{-\kappa (t-s)}}dW_s \right).
\end{align*}

% frametitle
{Forward prices in the model}
Using that the forward price is the expected value of future spot prices,
%$F(t,T) = \EX^{Q}[S_T | \Ft]$,
we get the formula for the forward price at time $t$ for the forward expiring in $T$ as
\begin{align*}
	&F(t,T) = \\
	&\exp \left( e^{-\kappa(T-t)} \ln S_t + (1-e^{-\kappa(T-t)}) \ln \theta + \frac{\sigma^2}{4\kappa}(1-e^{-2\kappa(T-t)}) \right).
\end{align*}
We can see that the forward prices converge to the spot price with time to maturity tending to zero.

% frametitle
{Forward curves in the model}
Assuming the same parameters as before and varying $S_t$ we get the forward curves:
\begin{figure}
	\centering
		\includegraphics[width=1.00\textwidth]{../../../pics/forwardcurves}
	\label{fig:forwardcurves}
\end{figure}

% frametitle
{Properties of the forward curve in the model}



% begin itemize



	$F(t,T) \rightarrow \exp(\ln \theta + \frac{\sigma^2}{4 \kappa}) = \theta \exp{(\frac{\sigma^2}{4\kappa})}$ as $T \rightarrow \infty$.

	If the spot price is low compared to the long term mean, the forward curve is upward sloping (contango).

	If the spot price is high compared to the long term mean, the forward curve is downward sloping (backwardation).

	If the spot price is close to the long term mean, the forward curve might be humped-shaped.



% end itemize



\subsection{A structural model -- Barlow (2002)}

% frametitle
{Set-up}



% begin itemize



	$u_t(x)$ is supply at time $t$, if price is $x$; an increasing function.

	$d_t(x)$ is demand at time $t$, if price is $x$; a decreasing function.

	The electricity price at time $t$ is the unique number $S_t$ such that
$$
u_t(S_t)=d_t(S_t)
$$

	Need to specify $u_t$ and $d_t$.



% end itemize



% frametitle
{Barlow (2002) specification}



% begin itemize



	Supply is non-random, independent of $t$
$$u_t(x)=g(x).$$

	Demand is inelastic, independent of $x$
$$d_t(x)=D_t$$
a random process.

	With
$$
f_\alpha(x) = (1+\alpha x)^\frac{1}{\alpha}, \;\alpha \not=0, \; f_0(x)=\exp(x)
$$
and
$$
X_t= -\lambda (X_t-a)dt +\sigma dW_t
$$
Barlow (2002) motivates the model
$$
S_t= \left \{ \begin{array}{ll}
\displaystyle
f_\alpha(X_t) & 1+\alpha  X_t> \epsilon_0 \\*[12pt]
\epsilon_0^\frac{1}{\alpha} & 1+\alpha X_t \leq \epsilon_0
\end{array}
\right.
$$



% end itemize


