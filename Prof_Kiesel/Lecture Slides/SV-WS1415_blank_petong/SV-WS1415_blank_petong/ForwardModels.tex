\section{HJM-models with delivery period}
\subsection{Classical Model Specification}

% frametitle
{ Heath-Jarrow-Morton (HJM) model}

The Heath-Jarrow-Morton model uses the entire forward rate curve as
(infinite-dimensional) state variable. The dynamics of the forward rates $f(t,T)$ are {\it exogenously} given by
$$
%\label{forward rate dynamics}
df(t,T) = \alpha(t,T) dt + \s(t,T) dW(t).
$$
For any fixed maturity $T$, the
initial condition of the stochastic differential equation
is determined by the current value
of the empirical (observed) forward rate for the future date $T$
which prevails at time $0$.

% frametitle
{One-Factor GBM Specification}
Here the volatility is
$$
\sigma_1(t,T)=e^{-\kappa (T-t)}\sigma
$$
and
$$
dF(t,T)=F(t,T)\sigma_1(t,T)dW(t)
$$

% frametitle
{Two-Factor GBM Specification}
Here the volatilities are
$$
\sigma_1(t,T)=e^{-\kappa (T-t)}\sigma_1 \; \mbox{ and } \; \sigma_2>0
$$
and
$$
\frac{dF(t,T)}{F(t,T)}=\sigma_1(t,T)dW_1(t)+\sigma_2dW_2(t)
$$

\subsection{HJM-type Forward Models}

% frametitle
{Modelling Approach}



% begin itemize



	We use the HJM-framework to model the forward dynamics directly.

	We distinguish between forward contracts with a fixed time delivery and forward contracts with a delivery period, called \emph{swaps}.

	A typical lognormal dynamics of the swap price is,
\begin{equation}
dF(t,T_1,T_2)=\Sigma(t,T_1,T_2)F(t,T_1,T_2)\, dW(t). \label{eqn: lognormal dynamics}
\end{equation}
The only parameter in this model is the volatility function $\Sigma$ which has to capture all movements of the swap price and especially the time to maturity effect.



% end itemize



% frametitle
{Volatility Functions}
We assume that the swap price dynamics for all swaps is given by (\ref{eqn: lognormal dynamics})
where $\Sigma(t,T_1,T_2)$ is a continuously differentiable and positive function.

Starting out with a given volatility function for a fixed time forward contract the volatility function $\Sigma$ for the swap contract is given by
\begin{equation}
\Sigma(t,T_1,T_2)=\int_{T_1}^{T_2} \hat{w}(u,T_1,T_2) \sigma(t,u) \, du. \label{eqn: swap volatility creation}
\end{equation}

% frametitle
{Schwartz Volatility}
For the related volatility function of the forward we obtain
\begin{equation}\label{vol-schwartz}
\sigma(t,u)=a e^{-b(u-t)}
\end{equation}
where $a,b >0 $ are constant.

% frametitle
{Schwartz Volatility}
The time to maturity effect is modeled by a negative exponential function.



% begin itemize



	When the time to maturity tends to infinity the volatility function converges to zero.

	The exponential function causes that the volatility increases as the time to maturity decreases which leads to an increased volatility when the contract approaches the maturity.



% end itemize



% frametitle
{Schwartz Volatility}

Applying this forward volatility to (\ref{eqn: swap volatility creation}) the swap volatility is:
\begin{align}
\Sigma(t,T_1,T_2)&=a\,\varphi(T_1,T_2)
\end{align}
where
\begin{align}
\varphi(T_1,T_2)= \frac{e^{-b(T_1-t)}-e^{-b(T_2-t)}}{b(T_2-T_1)}
\label{volatility function varphi}
\end{align}
The Black-76 specification of the forward volatility can be obtained if $\varphi(T_1,T_2) =1$, that is $b=0$
in (\ref{vol-schwartz}).

The associated swap price volatility is then given by $\Sigma(t,T_1,T_2)=a$.

\subsection{The Brownian Factor Model}

% frametitle
{The Model Framework}



% begin itemize



	Use observable products, e.g. month futures as building blocks,

	Under a risk-neutral measure month forward prices $F(t,T,T+m)=F(t,T)$ have to be martingales,

	Assume the dynamics
$$dF(t,T)=\sigma(t,T)F(t,T)dW(t),$$
where $\sigma(t,T)$ is an adapted $d$-dimensional deterministic function and
$W(t)$ a $d$-dimensional Brownian motion.

	Initial value of this SDE is the initial forward curve observed at the market.



% end itemize



% frametitle
{Options on Building Blocks}
A European call option on $F(t,T)$ with maturity $T_0$ and strike
$K$ can be easily evaluated by the Black-formula
\begin{eqnarray}\label{eq:month-option}
V^{option}(0)=e^{-rT_0}\left(F(0,T)\mathcal{N}(d_1)-K\mathcal{N} (d_2)\right),
\end{eqnarray}
where $\mathcal{N}$ denotes the normal distribution, $\Sigma(T_0,T)=\int_0^{T_0}||\sigma(s,T)||^2ds$ and
\begin{eqnarray*}
d_1& = & \frac{\log \frac{F(0,T)}{K}+\frac{1}{2}\Sigma(T_0,T)}{\sqrt{\Sigma(T_0,T)}}\\
d_2 & = & d_1 - \sqrt{\Sigma(T_0,T)}
\end{eqnarray*}

% frametitle
{The Model Framework -- $n$-period futures}



% begin itemize



	Use observable products, e.g. month futures as building blocks,

	Express an $n$-period delivery futures as
$$Y_{T_1, \ldots, T_n}(t)=\frac{\sum_{i=1}^n e^{-r(T_i-t)}F(t,T_i)}{\sum_{i=1}^n e^{-r(T_i-t)}}.$$
(Compare modelling of forward swap rates in terms of forward LIBOR rates)

	In case of 1-year-futures, the swap rate is the forward price of the 1-year-futures,
which can be also observed in the market.



% end itemize



% frametitle
{Implied Prices of Futures}
\begin{center}
\includegraphics[height=6cm, width=10cm]{../../../pics/forwardcurve2}
\end{center}

% frametitle
{Options on $n$-period futures}



% begin itemize



	We need to compute
$$e^{-rT_0} \EX \left[\left(Y(T_0)-K\right)^+\right],$$
where the distribution of $Y$ as a sum of lognormals is unknown.

We approximate $Y$ by a random variable $\hat{Y}$,
which is lognormal and matches $Y$ in mean and variance.

Then,
$$\log \hat{Y} \sim \mathcal{N}(m,s)$$
with $s^2$ depending on the choice of the volatility functions
$\sigma(t,T_i)$.

An analysis of the goodness of the approximation
can be found in Brigo-Mercurio (2003).



% end itemize



% frametitle
{Options on $n$-period futures}

Using this approximation, it
is possible to apply a Black-Option formula again to obtain the
option value as
\begin{eqnarray}\label{eq:approx-option}
V^{option} & = & e^{-rT_0} \EX \left[\left(Y(T_0)-K\right)^+\right] \nonumber \\
& \approx & e^{-rT_0} \EX \left[\left(\hat{Y}(T_0)-K\right)^+\right]\nonumber \\
& = &  e^{-rT_0} \left(Y(0)\mathcal{N}(d_1)-K\mathcal{N}(d_2)\right)
\end{eqnarray}
with
\begin{eqnarray*}
d_1& = & \frac{\log \frac{Y(0)}{K}+\frac{1}{2}s^2}{s}\\
d_2 & = & d_1 - s
\end{eqnarray*}

\subsection{A two-factor model}

% frametitle
{A Two-Factor Model}



% begin itemize



	For a fixed delivery start $T$ and delivery period 1 month, let the dynamics of a Future $F_{t,T}$ be given by the two factor model:
\begin{eqnarray*}
F(t,T)& =&F(0,T)\exp\left\{\mu(t,T)  +\int_0^t\hat{\sigma_1}(s,T)dW_s^{(1)}+\sigma_2W_t^{(2)}\right\}
\end{eqnarray*}

	$W^{(1)}$ and $W^{(2)}$ independent Brownian motions

	$\hat{\sigma_1}(s,T)=\sigma_1e^{-\kappa(T-s)}$

	$\sigma_1$, $\sigma_2$, $\kappa>0$ constants

	$\mu(t,T)$ being the risk-neutral martingale drift



% end itemize



% frametitle
{Model Parameters}
$\sigma_1$ affects the level at the short end of the volatility curve

\begin{center}
\includegraphics[height=6cm, width=10cm]{../../../pics/sigma1}
\end{center}

% frametitle
{Model Parameters}
$\kappa$ affects the slope of the volatility curve at the short end

\begin{center}
\includegraphics[height=6cm, width=10cm]{../../../pics/kappa}
\end{center}

% frametitle
{Model Parameters}
$\sigma_2$ affects the level at the long end of the volatility curve

\begin{center}
\includegraphics[height=6cm, width=10cm]{../../../pics/sigma2}
\end{center}

% frametitle
{Pricing of Futures}



% begin itemize



	In this model, all products are expressed using Month-Futures

	Prices of quarterly and yearly Futures are given as an
average of the  $n$ corresponding monthly Futures.

	$Y_{t,T_1, \ldots T_n}=Y=\frac{\sum e^{-rT_i}F_{t,T_i}}{\sum e^{-rT_i}}$
is the forward price of a $n$-month forward quoted in the market (cp. swap rate)



% end itemize



% frametitle
{Pricing of Options on Month-Futures}



% begin itemize



	At time $t=0$, the price of a  Call-Option with strike $K$ and maturity $T_0$ on a Month-Future $F_{t,T}$ is given by
$$e^{-rT_0}\EX\left[\left(F_{T_0,T}-K\right)^+\right]$$

	Within the model,  $F_{T_0,T}$ is log-normally distributed with known variance
$$
\Sigma(T_0,T) =   \frac{\sigma_1^2}{2\kappa}(e^{-2\kappa (T-T_0)}-e^{-2\kappa T})+\sigma_2^2T_0
$$

	Thus the option's value is given by the formula (Black 76):
\begin{eqnarray*}
e^{-rT_0}\EX\left[\left(F_{T_0,T}-K\right)^+\right] & = &
e^{-rT_0}\left(F_{0,T}\mathcal{N}(d_1)-K\mathcal{N}(d_2)\right)
\end{eqnarray*}with $d_{1,2}$ depending on the parameters $\sigma_1, \sigma_2, \kappa$.



% end itemize



% frametitle
{Pricing of Options on quarterly and yearly Futures}



% begin itemize



	At time $t=0$, the price of a  Call-Option with strike $K$ and maturity $T_0$ on a $n$-Month-Future
$Y$ is given by
$$
e^{-rT_0}\EX\left[\left(Y-K\right)^+\right]=e^{-rT_0}\EX
\left[\left(\frac{\sum e^{-rT_i}F_{t,T_i}}{\sum e^{-rT_i}}-K\right)^+\right]$$

	The distribution of the sum is not known within the model. There is no explicit solution to this integral.

	Approximate the random variable $Y$ by a log-normal random variable $\hat{Y}$
with same mean and variance (depending on the model parameters)



% end itemize



% frametitle
{Matching the Variance}
Using the moment-generating function of a normal random variable, we get
$$
\exp (s^2) = \frac{\textrm{Var}(Y)}{\left(\EX (Y)\right)^2} + 1  = \frac{\EX(Y^2)}{\EX(Y)^2}
$$
From the martingale property
$\EX (F_{T_0,T_i}) =  F_{0,T_i} $ and
$$ \EX (Y_{T_1, \ldots T_n}(T_0))  =
\frac{\sum e^{-r(T_i-T_0)} F_{0,T_i}}{\sum e^{-r(T_i-T_0)}}.$$

% frametitle
{Matching the Variance}

So
$$
\EX (Y_{T_1, \ldots T_n}(T_0)^2)
=  \frac{\sum_{i, j}e^{-r(T_i+T_j-2T_0)} F_{0,T_i}F_{0,T_j}\cdot \exp Cov_{ij}}{\left(\sum e^{-r(T_i-T_0)}\right)^2}
$$ with
$Cov_{ij} = \textrm{Cov}(\log F(T_0,T_i), \log F(T_0,T_j))$.

The covariance can be computed directly from the explicit solution of the SDE
$$\begin{array}{ll}
& \DSE \textrm{Cov}(\log F(T_0,T_i), \log F(T_0,T_j)) \\*[12pt]
 = & \DSE
 e^{-\kappa(T_i+T_j-2T_0)}\frac{\sigma_1^2}{2\kappa}(1-e^{-2\kappa T_0})+\sigma_2^2 T_0
\end{array}
$$

% frametitle
{Pricing of Options on quarterly and yearly Futures}
The option value can be computed by Black's formula
\begin{eqnarray*}
e^{-rT_0}\EX\left[\left(Y-K\right)^+\right]&\approx & e^{-rT_0}\EX\left[\left(\hat{Y}-K\right)^+\right]\\
& = &e^{-rT_0}\left(Y(0)\mathcal{N}(d_1) - K\mathcal{N}(d_2)\right)
\end{eqnarray*}
with $d_{1,2}$ depending on the parameters $\sigma_1, \sigma_2, \kappa$.

% frametitle
{Parameter Estimation}



% begin itemize



	Use the approximating Black-formula
    \begin{eqnarray*}
        \textrm{Option value}&=&    e^{-rT_0}\left(Y(0)\mathcal{N} (d_1)-K\mathcal{N} (d_2) \right)
\\
        d_{1,2} & = & d_{1,2}\left(Y(0),K,Var(\log \hat{Y}(T_0))\right)
    \end{eqnarray*}
        Only the variance $Var(\log \hat{Y}(T_0))$ depends on the unknown parameters

	Compute the variances $Var(\log\hat{Y}(T_0))$ for products observable in the market

	Choose parameter $\sigma_1$, $\sigma_2$ and $\kappa$ to minimize the distance of the model variances
to the market variances in a given metric (in the least-square sense)



% end itemize



% frametitle
{Data}
{\small \begin{table}[btp]
\begin{center}
\begin{tabular}{ccrrrr}
Product     & Delivery Start    & Strike    &   Forward & Market Price  &Implied Vola \\
\hline
Month   &October 05     &48 &48.90&2.023&43.80\%\\
Month   &November 05        &49 &50.00&3.064&37.66\%\\
Month   &December 05        &49 &49.45&3.244&34.72\%\\
\hline
Quarter & October 05        &48 &49.44&2.086&35.15\%\\
Quarter & January 06        &47 &48.59&3.637&28.43\%\\
Quarter & April 06          &40 &40.71&3.421&26.84\%\\
Quarter & July 06           &42 &41.80&3.758&27.19\%\\
Quarter & October 06        &43 &43.71&4.566&25.35\%\\
\hline
Year    &January 06     &44 &43.68&1.521&20.19\%\\
Year    &January 07     &43 &42.62&3.228&19.14\%\\
Year    &January 08     &42 &42.70&4.286&17.46\%\\
\end{tabular}
\caption{ATM calls and implied Black-volatility, Sep 14}
\label{fig:data}
\end{center}
\end{table}
}

% frametitle
{Parameter Estimates}
 \begin{table}[btp]
 \begin{center}
\begin{tabular}{p{4cm}cccrrrrcc}
Method  &Constraints    &$\sigma_1$ &$\sigma_2$ &$\kappa$   &Time&\\
\hline
Function calls and numerical gradient       &yes            &0.37       &0.15       &1.40       &$<$1min&\\
Least Square Algorithm  &no&0.37&0.15&1.41      &$<$1min&\\
\end{tabular}
\caption{Parameter estimates with different optimizers, market data as of Sep 14}
\label{fig:estimates18}
\end{center}
\end{table}

Options, which are far away from maturity, will have a volatility of about $15\%$,
which can add up to more than $50\%$, when time to maturity decreases.

A $\kappa$ value of 1.40 indicates, that disturbances in the futures market
halve in $\frac{1}{\kappa}\cdot \log 2 \approx 0.69$ years.
