% !TEX root = StructuringValuation_ws1314UDE.tex
\section{Energy Derivatives}
\subsection{Review: Spread Options}

% frametitle
{Spread Options to Manage Market Risk}
Spread options can be used by owners of corresponding plants to
manage the market risk. Instead of spread trading with futures the owner of a power plant can directly purchase/sell a spread option.\\
\vspace{0.2cm}
The pay off of a typical spread is
$$C_{\mbox{spread}}^{(T)}=\max(S_1(T)-S_2(T)-K,0)$$ with $S_i$ the
underlyings, $K$ the strike.

% frametitle
{Spread Options to Manage Market Risk }
P$\text{\&}$L diagram of a typical spread
\begin{figure}\label{payoffeurocall}
\unitlength1cm \thicklines
\begin{picture}(10,7)
\put(1,2){\vector(1,0){7}} \put(6,1.5){$S_1(T)-S_2(T)$} \put(4,2){$K$}
\put(2,1){\vector(0,1){5}} \put(1.4,6.5){P$\text{\&}$L}
\put(4,1.5){\line(1,1){4}} \put(2,1.5){\line(1,0){2}}
\end{picture}
\caption{Profit diagram for a European call}
\end{figure}

%\subsection{Valuation of Spread Options}

% frametitle
{Valuation of Spread Options}

For $K=0$ (exchange option) there is an analytic formula due to
Margrabe (1978).
$$\begin{array}{lll}
 C_{\mbox{spread}}(t) & = & (S_1(t)\Phi(d_1)-S_2(t)\Phi(d_2))
 \\*[12pt]
 P_{\mbox{spread}}(t) & = & (S_2(t)\Phi(-d_2)-S_1(t)\Phi(-d_1))
 \\*[12pt]
 \mbox{where}\quad d_1 & = & \frac{\log(S_1(t)/S_2(t))+\sigma^{2}(T-t)/2}{\sqrt{\sigma^{2}(T-t)}},\quad d_2=d_1-\sqrt{\sigma^{2}(T-t)}
 \\*[12pt]
 \mbox{and}\quad \sigma & = & \sqrt{\sigma_1^2-2\rho\sigma_1\sigma_2+\sigma_2^2}
\end{array}$$
where $\rho$ is the correlation between the two underlyings.

For $K\neq 0$ no easy analytic formula is available.

% frametitle
{Valuation of Spread Options - Price}
In this case, the price of the option depending on the underlying prices has the following structure:
\begin{figure}
	\centering
		\includegraphics[width=.80\textwidth]{../../../pics/spreadprice}
	\label{fig:spreadprice}
\end{figure}

% frametitle
{Spread Option Value and Correlation}
The value of a spread option depends strongly on the correlation between the two underlyings.
$$\tiny\text{$S_1=S_2=100$, $T=3$, $r=0.02$, $\sigma_1=0.6$, $\sigma_2=0.4$.}$$
\vspace{-0.76cm}
$$\includegraphics[scale=0.3]{../../../pics/corr}$$

% begin itemize

\vspace{-1cm}

	The higher the correlation between the two underlyings the lower is the volatility of the spread and hence the value of the spread option.

% end itemize

\subsection{Caps and  Floors}

% frametitle
{Caps}

% begin itemize

	A Cap gives the option holder the right (but not the
obligation) to buy a certain amount of energy at stipulated times
$t_1,\ldots,t_N$ during the delivery period at a fixed strike
price $K$.

	It can be viewed as a strip of
independent call options, for each time $t_i$ the holder of the cap holds call options with maturity $t_i$ and Strike $K$.

	Typically, the price of a cap is quoted as price per delivery hours to make
different delivery periods comparable. In this case we get a price
per MWh. The formula is
$$U_c(t)=\frac{1}{N}\sum_{i=1}^Ne^{-r(t_i-t)}\EX[\max(S(t_i)-K,0)].$$

% end itemize

% frametitle
{Cap - Payoff}
\begin{figure}
	\centering
		\includegraphics[width=.80\textwidth]{../../../pics/Cap2}
	\label{fig:Cap2}
\end{figure}

% frametitle
{Caps - Hedging}
The strike price $K$ secures a maximum price for which the option holder is able to buy energy. A cap is used to cover a short position in the underlying (energy) against
increasing market prices not only at a certain point in time but over the whole period covered by the exercising times $t_1,\ldots,t_N$.
On the other hand, the option holder is still able to profit from low energy prices as he has the right but not the obligation to exercise the option at each time point.

% frametitle
{Floors}

% begin itemize

	A floor gives the holder the right (but not the
obligation) to sell a certain amount of energy at stipulated times
$t_1,\ldots,t_N$ during the delivery period at a fixed strike
price $K$.

	It can be viewed as a strip of
independent put options, for each time $t_i$ the holder of the floor holds put options with maturity $t_i$ and Strike $K$.

	Similar to the case of a cap, the pricing formula is
$$U_f(t)=\frac{1}{N}\sum_{i=1}^Ne^{-r(t_i-t)}\EX[\max(K-S(t_i),0)].$$
As with the cap, the price is quoted in Euro/MWh.

% end itemize

% frametitle
{Floor - Payoff}
\begin{figure}
	\centering
		\includegraphics[width=.80\textwidth]{../../../pics/Floor}
	\label{fig:Floor}
\end{figure}

% frametitle
{Collars}
A collar is a combination of a cap and a floor such that variable prices are limited to a certain corridor. A long collar position consists of long one cap (with high strike $K_2$) and short one floor (with low strike $K_1$) - a short collar position is short one cap and long one floor. As long as the price of the underlying is between $K_1$ and $K_2$ at one of the dates $t_i$, no cash flows are exchanged. If the underlying is above $K_2$, the holder of the long collar position receives the difference of the actual price and $K_2$. If the underlying is below $K_1$, the short collar position receives the difference between $K_1$ and the actual price.

% frametitle
{Collar - Payoff}
As a long collar position is a strip of call options minus a strip of put options, the payoff of a collar at each time point $t_i$ is the following:
\begin{figure}
	\centering
		\includegraphics[width=.80\textwidth]{../../../pics/Collar}
	\label{fig:Collar}
\end{figure}

% frametitle
{Collar - Pricing}
Collars might be seen as a strip of bear/bull spreads, or as a strip of call options minus a strip of put options in the case of a long collar position. Consequently, the pricing formula is just the combination of the formulas for the cap and the floor:
\begin{align*}
	U^{K_1, K_2}_{collar}(t)&=U^{K_2}_{cap}(t)-U^{K_1}_{floor}(t)\\
	&=\frac{1}{N}\sum_{i=1}^Ne^{-r(t_i-t)}\EX[(S(t_i)-K_2)^+ - (K_1-S(t_i))^+]
\end{align*}
The price of a collar might be positive or negative - or even zero. In case the price is zero, the collar is called zero-cost collar.

% frametitle
{Collars - Hedging}
The holder of a long position in a collar is protected against increases in the underlying price above $K_2$, but does not profit from falling underlying prices below $K_1$. Thus he is protected against rising prices with limited participation on downside prices. Having a short position in the underlying, a long collar ensures the ability to cover the short position for prices in the range of $[K_1, K_2]$.
A short collar protects against falling prices. At the same time, the ability to participate on rising prices is limited to $K_2$. Having a long position in the underlying, a short collar ensures that the position can be closed for prices in the range of $[K_1, K_2]$.

% frametitle
{Collars - 3-way-collars}
A long collar is short one floor with strike $K_1$, long one cap with higher strike $K_2$. A possible extension is to include a short position in one cap with strike $K_3 >> K_2$ in order to reduce the cost of the collar. This extension is called 3-way-collar.
The price of a 3-way-collar is thus:
\begin{align*}
	U^{K_1, K_2, K_3}_{3-way}(t)&=U^{K_2}_{cap}(t)-U^{K_3}_{cap}(t)-U^{K_1}_{floor}(t)&\\
	&=\frac{1}{N}\sum_{i=1}^Ne^{-r(t_i-t)}\EX[(S(t_i)-K_2)^+ &\\
	 &- (S(t_i)-K_3)^+ - (K_1-S(t_i))^+]&
\end{align*}

% frametitle
{3-Way-Collar - Payoff}
The holder of the 3-way-collar is protected against increases in the underlying price above $K_2$, but only till $K_3$. Afterwards, no protection exists anymore. This strategy might be a good choice if one wants to protect its buying costs but is able to stop its business if prices rally unexpectedly high (above $K_3$).
\begin{figure}
	\centering
		\includegraphics[width=.80\textwidth]{../../../pics/collar3way}
	\label{fig:collar3way}
\end{figure}

\subsection{Swing Options}

% frametitle
{Swing Options Definition}

% begin itemize

	Swing options are contracts that allow the option holder buy a predetermined quantity of energy at a predetermined price while having some flexibility in the amount purchased and the price paid.

	A swing option contract states the least and most energy an option holder can buy (or "take") per day and per month, how much that energy will cost (its strike price) and how many times during the month the option holder can change (or "swing") the daily quantity of energy purchased.

% end itemize

% frametitle
{Swing Options Terminology}

% begin itemize

	Swing direction: The holder of a call swing option has the right to purchase energy. The holder of a put swing option has the right to sell energy.

	Swing rights: The contract defines the number of exercise opportunities and the exercise period.

	Quantity limits: The contract defines the maximum quantity for both the individual exercise as well as for the overall quantity.

	Delivery: Might be for the next day or for the remainder of the term of the option.

% end itemize

% frametitle
{Swing Options}
A swing option is similar to a cap or floor except that we have
additional restrictions on the number of option exercises. Let
$\phi_i\in\{0,1\}$ be the decision whether to exercise
$(\phi_i=1)$ or not to exercise $(\phi_i=0)$ the option at time
$t_i$. The option's payoff at time $t_i$ is given by
$$\phi_i(S(t_i)-K)\quad\mbox{call resp.}\quad\phi_i(K-S(t_i))\quad\mbox{put}.$$
We now require that the number of exercises is between $E_{\min}$
and $E_{\max}$.

% frametitle
{Swing Options}
To determine the swing option value, we have to find an optimal exercise
strategy $\Phi=(\phi_1,\ldots,\phi_N)$ maximising the expected
payoff
$$\sum_{i=1}^Ne^{-r(t_i-t)}\EX[{\phi_i(S(t_i)-K)}]\quad\rightarrow\max$$
subject to $$E_{\min}\leq\sum_{i=1}^N\phi_i\leq E_{\max}.$$

To calculate the option value various mathematical techniques are used.

% frametitle
{Bounds for Swing Options}

\underline{Strategy}\\
For deterministic spot prices, we

% begin itemize

	Calculate the discounted payoffs
  $P(t_i)=e^{-r(t_i-t)}(S(t_i)-K)$.

	Sort the discounted payoffs $P(t_i)$ in descending order.

	Take the first $E_{\min}$ payoffs regardless of their
  value and subsequent payoffs up to $E_{\max}$ until their sign
  become negative.

% end itemize

% frametitle
{Bounds for Swing Options}

For stochastic spot prices the MC-approach gives an upper bound,
since information on the whole path is used, but in reality only
information up to time $t$ is available when deciding at time $t$.

A lower bound is given by the intrinsic value
$$\sum_{i=1}^Ne^{-r(t_i-t)} \phi_i^F (F(t,t_i)-K) \quad\rightarrow\max$$
subject to $$E_{\min}\leq\sum_{i=1}^N\phi^F_i\leq E_{\max}$$
where $\phi_i^F=\IF_{\{F(t,t_i)>K\}}$, unless the restriction on $E_{\min}$ is in force.

\subsection{Indexed Supply Contracts}

% frametitle
{Indexed Supply Contracts}

% begin itemize

	Total or partial price linkage between energy products (electricity, natural gas) and retail prices for commodities or industrial end-products.

	The price linkage might be linear (indexation to the price of an output product), inverse (indexation to the price of an input product) or captures individual price dependency structures.

	Example: Price linkage between electricity and metals like aluminium, zinc, copper, nickel (products with liquid futures market).

	Example: An aluminium supplier has an interest to purchase power at a lower price if aluminium price falls.

% end itemize

% frametitle
{Indexed Supply Contracts}
Indexed supply contracts have two components:

% begin itemize

	Energy supply contract for the physical delivery.

	Financially settled contract of the price linkage.

% end itemize

\vspace{0.5cm}
Advantages:

% begin itemize

	Help to mitigate price risks, risk of input or output price fluctuations.

	Help to lock in profit margins.

% end itemize

% frametitle
{Zinc Indexed Power Supply Contract with Cap}
Consider a 1-year zinc indexed power supply contract with floor strike $K_1=1\;800$ €/mt, cap strike $K_2=2\;400$ €/mt, slope $\frac{1.7\text{€/MWh}}{100\text{€/mt}}$ and monthly settlement.
%\vspace{-0.4cm}
$$\includegraphics[scale=0.29]{../../../pics/index1.png}$$

% frametitle
{Zinc Indexed Power Supply Contract with Cap}
Assume that the market price for July base power contract is \textcolor[rgb]{1.00,0.00,0.00}{55€/MWh} (no zinc linkage). July base power price $P$ with zinc linkage:\\
\vspace{0.2cm}

% begin itemize

	If the zinc price (monthly average of the LME spot price) $Zn=1\;700$€/mt, then \textcolor[rgb]{0.00,0.00,1.00}{$P=52$€/MWh}.

	If the zinc price $Zn=2\;200$€/mt, then \textcolor[rgb]{0.00,0.00,1.00}{$P=58.8$€/MWh}:
  $$P=52\frac{\text{€}}{\text{MWh}}+(2\;200-1\;800)\frac{\text{€}}{\text{mt}}\times\frac{1.7\text{€/MWh}}{100\text{€/mt}}=58.8\frac{\text{€}}{\text{MWh}}.$$

	If the zinc price $Zn=2\;600$€/mt, then \textcolor[rgb]{0.00,0.00,1.00}{$P=62.2$€/MWh}.

% end itemize

% frametitle
{Zinc Indexed Power Supply Contract with Cap}

% begin itemize

	Zinc linkage is only possible in connection with a power supply contract.

	Slope can be adjusted according to the production costs of the customer.

	Floor and cap can be adjusted.

	Equivalent (from the customer point of view) to a short collar position in zinc and a long swap position.

% end itemize

% frametitle
{Zinc Indexed Power Supply Contract without Cap}
Consider a 1-year zinc indexed power supply contract with floor strike $K_1=2\;200$€/mt, without cap, with slope $\frac{1.2\text{€/MWh}}{100\text{€/mt}}$ above the floor and monthly settlement.
%\vspace{-0.4cm}
$$\includegraphics[scale=0.29]{../../../pics/index2.png}$$

% frametitle
{Zinc Indexed Power Supply Contract with Cap}
Assume that the market price for April base power contract is \textcolor[rgb]{1.00,0.00,0.00}{55€/MWh} (no zinc linkage). April base power price $P$ with zinc linkage embedded in the electricity supply contract:
\vspace{0.4cm}

% begin itemize

	If the zinc price (e.g. monthly average of the LME spot price) $Zn=2\;000$€/mt, then \textcolor[rgb]{0.00,0.00,1.00}{$P=52$€/MWh}.

	If the zinc price $Zn=2\;700$€/mt, then \textcolor[rgb]{0.00,0.00,1.00}{$P=58$€/MWh}:
  $$P=52\frac{\text{€}}{\text{MWh}}+(2\;700-2\;200)\frac{\text{€}}{\text{mt}}\times\frac{1.2\text{€/MWh}}{100\text{€/mt}}=58\frac{\text{€}}{\text{MWh}}.$$

% end itemize

% frametitle
{Aluminium Indexed Gas Supply Contract}
Consider a 1-gas-business-year (1. October 06:00:00 - 1. October 05:59:59) aluminium indexed gas supply contract with floor strike $K_1=1\;950$€/mt, without cap, with slope $\frac{1.5\text{€/MWh}}{100\text{€/mt}}$ above the floor and monthly settlement.
%\vspace{-0.4cm}
$$\includegraphics[scale=0.27]{../../../pics/index3.png}$$

% frametitle
{Zinc Indexed Power Supply Contract with Cap}
Assume that the gas price for a gas business year is fixed to \textcolor[rgb]{1.00,0.00,0.00}{42€/MWh} (no aluminium linkage). The gas price $G$ with aluminium linkage embedded in the supply contract:
\vspace{0.4cm}

% begin itemize

	If the aluminium price (e.g. monthly average of the LME spot price) $Al=1\;750$€/mt, then \textcolor[rgb]{0.00,0.00,1.00}{$G=36$€/MWh}.

	If the aluminium price $Al=2\;400$€/mt, then \textcolor[rgb]{0.00,0.00,1.00}{$G=42.75$€/MWh}:
  $$G=36\frac{\text{€}}{\text{MWh}}+(2\;400-1\;750)\frac{\text{€}}{\text{mt}}\times\frac{1.5\text{€/MWh}}{100\text{€/mt}}=42.75\frac{\text{€}}{\text{MWh}}.$$

% end itemize

% frametitle
{Aluminium Indexed Gas Supply Contract with inverse price dependency}
Inverse indexation of the price of natural gas on the monthly price development of aluminium with slope $\frac{1.4375\text{Euro/MWh}}{100\text{Euro/mt}}$.
$$\includegraphics[scale=0.28]{../../../pics/index4.png}$$

% frametitle
{Aluminium Indexed Gas Supply Contract with inverse price dependency}

% begin itemize

	Increasing aluminium prices will decrease the price of natural gas.

	Decreasing aluminium prices will increase the price of natural gas.

	No cap, no floor.

% end itemize

% frametitle
{Indexed Contracts}

% begin itemize

	The energy price is not fixed all at ones - diversification.

	The risk of buying the energy at the wrong moment when prices are near a local maximum is reduced.

% end itemize

% frametitle
{Indexed Contracts}
Indexed energy price of the contract with $n$ fixings is described by
$$P_x=(P+R+M)\frac{\sum_{k=1}^n{I(t_k)}}{nI(T_0)},\;t_k\in[T_0,T_1].$$
Where $P$ denotes the basic energy price, $R$ the risk premium, $M$ the retail margin, $n$ the number of possible fixings,
$T_0$ the closing date of the contract, $I(t)$ the value of the index at $t$ and $[T_0,T_1]$ the delivery period.

% frametitle
{Example: Indexed Contract with four fixings}
$$\includegraphics[scale=0.6]{../../../pics/index}$$
The buyer has the possibility of fixing the energy price 4-times for $1/4$ of the total quantity each (alternatively there is an automatic fixing at specified dates prior to delivery).

% frametitle
{Indexed Contracts}

% begin itemize

	Index as a convex combination of two or more forwards or futures contracts:
  $$I(t)=(1-a)F_{Base}(t,T_1,T_2)+aF_{Peak}(t,T_1,T_2),\;a\in[0,1].$$

	Primary energy index, e.g. Euro-based coal index:
  $$P_m=(P+R+M)\left(1+c\left(\frac{X_mC_m}{X_0C_0}-1\right)\right),$$
  where $P_m$ denotes the monthly adapted power price valid for delivery month $m$, $C_m$ the monthly average of the coal index for the month $m$, $X_m$ the average value of one Dollar in Euros for the month $m$ and $c$ the coal price ratio.

% end itemize

%The coal price ratio determines the influence of coal prices to the power supply price. If one tonne coal is needed for generating three MWh electricity, a value for the ratio $c=33\%$ can be calculated.

\subsection{Pricing Volume Flexibility}

% frametitle
{Example: Gas Delivery with Tolerance Band}

% begin itemize

	Consider a customer who needs around 1000 MWh Gas per day, depending on the temperature and other factors sometimes 600 MWh, sometimes up to 1400 MWh.

	The customer wants to enter a delivery contract with maturity end of this year which should supply the amount of gas needed.

	The relevant gas market for the customer is the GTS system, i.e. the customer has access to the TTF.

	The customer is able to trade gas spot at TTF market prices, i.e. we have to assume that the customer uses his contract in a rational way.

% end itemize

% frametitle
{Example: Gas Delivery}
Possible contracts are:

% begin itemize

	C1: The customer enters into a contract which delivers him 1000 MWh/day for a fixed and constant price without any flexibility. That means that he has to buy gas at the spot market if he needs more or has to sell gas at the spot market if he has too much. The gas is paid at delivery.

	C2: This contract delivers 1000 MWh/day with an option for 400 MWh/day more if needed (for fixed price).

	C3: This contract gives him the right to get between 600 MWh and 1400 WMh delivered at a fixed price.

% end itemize

% frametitle
{Definition of the model}
We assume a simple market model in which the price of the underlying commodity, $S_t$, follows a stochastic process which can be described as follows:
Let $X_t = \ln S_t$ and
\begin{align*}
	dX_t = \kappa (\ln \theta - X_t)dt + \sigma dW_t~~,~~X_0 = \ln(S_0)
\end{align*}
Thus, the logarithm of the prices follow a mean reverting diffusion process, the so-called \textcolor{red}{Ornstein-Uhlenbeck-Process}.\\

% begin itemize

	$\kappa$ - Speed of mean reversion

	$\theta$ - Level of mean reversion

	$\sigma$ - Volatility of the process

	$dW_t$ - Brownian increments

% end itemize

% frametitle
{Example: The market}
We assume the market parameters to be:

% begin itemize

	$S_0 = 18.95 EUR/MWh, \kappa = 0.5, \sigma = 0.8, r= 5\%$

% end itemize

The forward curve has the following shape:
\begin{figure}
	\centering
		\includegraphics[width=.80\textwidth]{../../../pics/exampleforward}
	\label{fig:exampleforward}
\end{figure}

% frametitle
{Contract C1}

% begin itemize

	The contract consists of 191 deliveries of 1000 MWh of TTF gas.

	For each delivery, it is clear how much gas is needed.

	Thus, the gas can be bought today at the forward market in order to offset any risk.

% end itemize

% frametitle
{Contract C1}

We get the following prices:
\begin{tabular}{rrc}
   Forward &      Price (EUR/MWh)& Present value of 1000 MWh \\
\hline
$S_0$ = F(0,0) &         18.95 &      18950.0 \\

 F(0,1/365) &    19.36 &    19361.4 \\

 F(0,2/365) &    19.28 &    19276.1 \\

 F(0,3/365) &    19.22 &    19223.9 \\
...&...&...\\
\end{tabular}  \\
Summing up all the discounted prices of each delivery of 1000 MWh of gas we see that the present value of the gas sums up to $3,797,167.66$ EUR.

% frametitle
{Contract C1 Price}

% begin itemize

	The fair price of the contract is the fixed and constant price which makes the payments be worth today the same as the gas.

	Let $P_1$ be the fixed and constant price for one MWh of gas under contract 1.

	The customer will pay $1000 \cdot P_1$ at each delivery day.

% end itemize

% frametitle
{Contract C1 Price}
Discounting those payments and summing them up leads to the present value of:
$$
	PV = \sum_{k=0}^{190}{\frac{1000 \cdot P_1}{(1 + r)^{k/365}}} = 1000 P_1 \sum_{k=0}^{190} (1.05^{1/365})^{-k} = 187,594.98 P_1
$$
Which has to be equal to the present value of the gas, $3,797,167.66$ EUR. Solving this equation leads to
$$
	P_1 = 20.24 EUR/MWh
$$
This is the fair price for contract C1.

%\subsection{2. Contract}

% frametitle
{Contract C2}

% begin itemize

	Contract C2 gives the customer exactly the same rights as the first one and \textcolor{red}{additionally} the right to buy up to 400 MWh/day more, if it is advantageous for the customer.

	This right can be seen as an \textcolor{red}{option on the gas spot price at each delivery day}.

	Thus, the contract consists of the forward portfolio as before \textcolor{red}{and} a strip of call options on the gas spot price.

	The strip has length 191 days and a volume of 400 MWh per delivery day. The strike price is the price $P_2$ which was agreed as a fixed and constant price for all gas deliveries during the lifetime of contract 2.

% end itemize

% frametitle
{Contract C2 - Value of the Option}
Depending on the strike price we get the following prices for call options in the model:
\begin{figure}
	\centering
		\includegraphics[width=.90\textwidth]{../../../pics/examplecall}
	\label{fig:examplecall}
\end{figure}

% frametitle
{C2 - Fair Price}

% begin itemize

Writing $C(T,K)$ for the price of a call with maturity $T$ and strike $K$ we get the present value of the oil delivery side of the contract:
$$
	PV = \sum_{k=0}^{190} \frac{1000}{(1+r)^{k/365}} \cdot F(0,\frac k {365}) + \sum_{k=0}^{190} 400 \cdot C(\frac k {365},P_2)
$$
with $P_2$ to be determined.

	The present value of the payments is as before $187,594.98 \cdot P_2$ EUR.

% end itemize

% frametitle
{C2 - Fair Price}

% begin itemize

Equating those two values leads to an equation which can be solved for $P_2$ leading to the fair price of
$$
	P_2 = 21.26 EUR/MWh.
$$

\textcolor{red}{The premium which makes contract 2 more expensive as contract 1 is exactly the premium of the call options.}

% end itemize

% frametitle
{Contract C2 - Fair Price depending on Volume}

% begin itemize

	We could imagine to have to add to each forward contract price $(400 [optional]/1000 [fixed]) = \frac 2 5$-th of a call option price.

	Thus, it is obvious that the premium would be a lot higher, if only a delivery of e.g. 200 MWh fix and 400 optional would have been agreed.

	In this case, one would have to add to each forward the price of two whole call options. However, the premium would not be 5 times as high as before, as the call would be written on a higher strike.

% end itemize

%\subsection{3. Contract}

% frametitle
{Contract C3}

% begin itemize

	Contract C3 gives the customer exactly the same rights as the second and \textcolor{red}{additionally} the right to sell up to 400 MWh/day at the fixed price, if it is advantageous for the customer, i.e. to buy only 600 MWh/day.

	This right - which does not imply any obligation - can be seen as a put option on the gas spot price at each delivery day.

	Thus, the contract consists of the forward portfolio \textcolor{red}{and} the call option strip as before \textcolor{red}{and} a strip of put options on the oil spot price.

	The strip has length 191 days and a volume of 400 MWh per delivery day. The strike price is the price $P_3$ which was agreed as a fixed and constant price for all oil deliveries during the lifetime of contract C3.

% end itemize

% frametitle
{Contract C3  - Put Option Value}
Depending on the strike price we get the following prices for put options in the model:
\begin{figure}
	\centering
		\includegraphics[width=0.9\textwidth]{../../../pics/exampleput}
	\label{fig:exampleput}
\end{figure}

% frametitle
{Contract C3 - Fair Price}

% begin itemize

	Writing $P(T,K)$ for the price of a put with maturity $T$ and strike $K$ we get the present value of the oil delivery side of the contract:
\begin{align*}
	PV &= \sum_{k=0}^{190} \frac{1000}{(1+r)^{k/365}} \cdot F(0,\frac k {365}) + \sum_{k=0}^{190} 400 \cdot C(\frac k {365},P_3)\\
	 &+ \sum_{k=0}^{190} 400 \cdot P(\frac k {365},P_3)
\end{align*}
with $P_3$ to be determined.

	The present value of the payments is as before $187,594.98 \cdot P_3$ EUR.

% end itemize

% frametitle
{3. Contract - Fair Price}

% begin itemize

	Equating those two values leads to an equation which can be solved for $P_3$ leading to the fair price of
$$
	P_3 = 24.22 EUR
$$
per MWh.

There is an other way of seeing this contract. One could argue that the company has to buy 600 MWh/day for sure and has the additional right of buying up to 800 MWh/day if it wants. Indeed, both approaches are equivalent and thus the prices calculated coincide.

% end itemize

%\subsection{Generalization}

% frametitle
{Ratio fixed vs. optional}

% begin itemize

	We are able to generalize the above results. Assume any volume flexibility between the minimum volume $K_{min}$ and the maximum volume $K_{max}$ (per day).

	The price one has to pay for a contract with this flexibility only depends on the ratio
$$
	\rho = \frac{K_{max}}{K_{min}}
$$
in the following way:

% end itemize

% frametitle
{Ratio fixed vs. optional}
\begin{figure}
	\centering
		\includegraphics[width=1.0\textwidth]{../../../pics/volumeflex}
	\label{fig:volumeflex}
\end{figure}

% frametitle
{Summary}

% begin itemize

	Optionalities increase the price.

	If it is possible to act financially-driven, optionalities have to be priced accordingly.

	If the customer acts demand driven, it might be possible to offer flexibility cheaper.

	In this case, monetarization of optionalities has to be excluded legally.

	The size of flexibility in a contract is one parameter to influence the price.

% end itemize
